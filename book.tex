\documentclass[10pt]{book}
\usepackage{ulem}
\usepackage{a4wide}
\usepackage[dvipsnames,svgnames]{xcolor}
\usepackage[pdftex]{graphicx}
\usepackage[utf8]{inputenc}
\usepackage[hidelinks]{hyperref}
\usepackage{geometry}
\geometry{
  papersize={108mm,139.5mm},
  left=10mm,
  right=10mm,
  bottom=15mm,
  top=10mm
}
\usepackage{titlesec}
%en vez de renewcommand*\@makechapterhead
%\titlespacing*{\chapter}{0pt}{-40pt}{20pt}
%\titleformat{\chapter}[display]{\normalfont\huge\bfseries}{\chaptertitlename\ \thechapter}{20pt}{\Huge}
\titleformat*{\section}{\normalfont\normalsize\bfseries}
\titleformat*{\subsection}{\normalfont\medium\bfseries}
\titleformat*{\subsubsection}{\normalfont\footnotesize\bfseries}
\titleformat*{\paragraph}{\normalfont\scriptsize\bfseries}
%espacio despues de los paragraph
\makeatletter
\renewcommand\paragraph{%
   \@startsection{paragraph}{4}{0mm}%
      {-\baselineskip}%
      {.5\baselineskip}%
      {\normalfont\scriptsize\bfseries}}
\makeatother
\usepackage{verbatim}
\newbox\examplebox
\newenvironment{code}{%
  \scriptsize
    \verbatim
}{%
    \endverbatim
    \newline
}
\usepackage{calc}%para ancho tablas

%capítulos sin espacio vertical
\makeatletter
\renewcommand*\@makechapterhead[1]{%
  %\vspace*{50\p@}%
  {\parindent \z@ \raggedright \normalfont
    \ifnum \c@secnumdepth >\m@ne
        \huge\bfseries \@chapapp\space \thechapter
        \par\nobreak
        \vskip 20\p@
    \fi
    \interlinepenalty\@M
    \Huge \bfseries #1\par\nobreak
    \vskip 40\p@
  }}
%ídem contents,tablas,etc.
\renewcommand*\@makeschapterhead[1]{%
  %\vspace*{50\p@}%
  {\parindent \z@ \raggedright
    \normalfont
    \interlinepenalty\@M
    \Huge \bfseries  #1\par\nobreak
    \vskip 40\p@
  }}
\makeatother

%http://en.wikibooks.org/wiki/LaTeX/Page_Layout#Top_margin_above_Chapter
\usepackage{fancyhdr}
\setlength{\headheight}{15.2pt}
\pagestyle{fancy}
%headers en todas las páginas
\fancyhf{}%limpia
\chead[\thechapter.- \leftmark]{\rightmark} 
\cfoot[\thepage]{\thepage} 	
%capítulos en todas las páginas no son mayúsculas
\renewcommand{\chaptermark}[1]{\markboth{#1}{}}
\renewcommand{\sectionmark}[1]{\markright{#1}{}}

\begin{document}
\footnotesize

\chapter*{Preface}
\addcontentsline{toc}{chapter}{Preface}
This book is a \LaTeX port of Ruby on Rails' documentation available at http://guides.rubyonrails.org, so all credit should go to those guys.

It was extracted from the page itself using \textbf{html2latex} and then bulk edited with \textbf{vim} and \textbf{LaTeXila} (on Debian \textit{Wheezy}).

As of September 2012, the book spans the basic chapters from \textit{Getting Started\ldots} to \textit{Rails Routing\ldots}, through the \textbf{MVC} architecture.
They were treated (\& read) one-at-a-time so their editions are slighty different and tend toward better style as chapter numbers grow -- see code display, for example. The idea is to standarize it to best practices whenever time allows it.

It is a \texttt{pocket} book. It's dimensions are a quarter of Chilean Oficio (216x279\emph{mm}, a bit shorter than U.S. Office, if not deceiving myself). This is because it's the cheapest paper around, that's all. Anyone willing to take this to an A6 or whatever harmonic ratio of \sqrt{2}, please do fork! I'm eager to see how it looks. Be aware that tables will give work, though.

\chapter{Getting Started with Rails}

This guide covers getting up and running with Ruby on Rails. After reading it, you should be familiar with:
\begin{itemize}
	\item Installing Rails, creating a new Rails application, and connecting your application to a database
	\item The general layout of a Rails application
	\item The basic principles of MVC (Model, View Controller) and RESTful design
	\item How to quickly generate the starting pieces of a Rails application
\end{itemize}


This Guide is based on Rails 3.2. Some of the code shown here will not work in earlier versions of Rails.

\section{ Guide Assumptions}

This guide is designed for beginners who want to get started with a Rails application from scratch. It does not assume that you have any prior experience with Rails. However, to get the most out of it, you need to have some prerequisites installed:
\begin{itemize}
	\item The \href{http://www.ruby-lang.org/en/downloads}{Ruby} language version 1.8.7 or higher
\end{itemize}

Note that Ruby 1.8.7 p248 and p249 have marshaling bugs that crash Rails 3.0 and above. Ruby Enterprise Edition have these fixed since release 1.8.7-2010.02 though. On the 1.9 front, Ruby 1.9.1 is not usable because it outright segfaults on Rails 3.0 and above, so if you want to use Rails 3.0 or above with 1.9.x jump on 1.9.2 for smooth sailing.
\begin{itemize}
	\item The \href{http://rubyforge.org/frs/?group_id=126}{RubyGems} packaging system  
\begin{itemize}
	\item If you want to learn more about RubyGems, please read the \href{http://docs.rubygems.org/read/book/1}{RubyGems User Guide}
\end{itemize}
	\item A working installation of the \href{http://www.sqlite.org/}{SQLite3 Database}
\end{itemize}

Rails is a web application framework running on the Ruby programming language. If you have no prior experience with Ruby, you will find a very steep learning curve diving straight into Rails. There are some good free resources on the internet for learning Ruby, including:
\begin{itemize}
	\item \href{http://www.humblelittlerubybook.com/}{Mr. Neighborly’s Humble Little Ruby Book}
	\item \href{http://www.ruby-doc.org/docs/ProgrammingRuby/}{Programming Ruby}
	\item \href{http://mislav.uniqpath.com/poignant-guide/}{Why’s (Poignant) Guide to Ruby}
\end{itemize}

Also, the example code for this guide is available in the rails github:

https://github.com/rails/rails repository in rails/railties/guides/code/getting\_started.

\section{ What is Rails?}

This section goes into the background and philosophy of the Rails framework in detail. You can safely skip this section and come back to it at a later time. Section 3 starts you on the path to creating your first Rails application.

Rails is a web application development framework written in the Ruby language. It is designed to make programming web applications easier by making assumptions about what every developer needs to get started. It allows you to write less code while accomplishing more than many other languages and frameworks. Experienced Rails developers also report that it makes web application development more fun.

Rails is opinionated software. It makes the assumption that there is a “best” way to do things, and it’s designed to encourage that way – and in some cases to discourage alternatives. If you learn “The Rails Way” you’ll probably discover a tremendous increase in productivity. If you persist in bringing old habits from other languages to your Rails development, and trying to use patterns you learned elsewhere, you may have a less happy experience.

The Rails philosophy includes several guiding principles:
\begin{itemize}
	\item DRY – “Don’t Repeat Yourself” – suggests that writing the same code over and over again is a bad thing.
	\item Convention Over Configuration – means that Rails makes assumptions about what you want to do and how you’re going to do it, rather than requiring you to specify every little thing through endless configuration files.
	\item REST is the best pattern for web applications – organizing your application around resources and standard HTTP verbs is the fastest way to go.
\end{itemize}

\subsection{ The MVC Architecture}

At the core of Rails is the Model, View, Controller architecture, usually just called MVC. MVC benefits include:
\begin{itemize}
	\item Isolation of business logic from the user interface
	\item Ease of keeping code DRY
	\item Making it clear where different types of code belong for easier maintenance
\end{itemize}

\subsubsection{ Models}

A model represents the information (data) of the application and the rules to manipulate that data. In the case of Rails, models are primarily used for managing the rules of interaction with a corresponding database table. In most cases, each table in your database will correspond to one model in your application. The bulk of your application’s business logic will be concentrated in the models.

\subsubsection{ Views}

Views represent the user interface of your application. In Rails, views are often HTML files with embedded Ruby code that perform tasks related solely to the presentation of the data. Views handle the job of providing data to the web browser or other tool that is used to make requests from your application.

\subsubsection{ Controllers}

Controllers provide the “glue” between models and views. In Rails, controllers are responsible for processing the incoming requests from the web browser, interrogating the models for data, and passing that data on to the views for presentation.

\subsection{ The Components of Rails}

Rails ships as many individual components.  Each of these components are briefly explained below.  If you are new to Rails, as you read this section, don’t get hung up on the details of each component, as they will be explained in further detail later.  For instance, we will bring up Rack applications, but you don’t need to know anything about them to continue with this guide.
\begin{itemize}
	\item Action Pack  
\begin{itemize}
	\item Action Controller
	\item Action Dispatch
	\item Action View
\end{itemize}
	\item Action Mailer
	\item Active Model
	\item Active Record
	\item Active Resource
	\item Active Support
	\item Railties
\end{itemize}

\subsubsection{ Action Pack}

Action Pack is a single gem that contains Action Controller, Action View and Action Dispatch. The “VC” part of “MVC”.

\paragraph{ Action Controller}

Action Controller is the component that manages the controllers in a Rails application. The Action Controller framework processes incoming requests to a Rails application, extracts parameters, and dispatches them to the intended action.  Services provided by Action Controller include session management, template rendering, and redirect management.

\paragraph{ Action View}

Action View manages the views of your Rails application. It can create both HTML and XML output by default. Action View manages rendering templates, including nested and partial templates, and includes built-in AJAX support.  View templates are covered in more detail in another guide called \href{http://guides.rubyonrails.org/layouts_and_rendering.html}{Layouts and Rendering}.

\paragraph{ Action Dispatch}

Action Dispatch handles routing of web requests and dispatches them as you want, either to your application or any other Rack application.  Rack applications are a more advanced topic and are covered in a separate guide called \href{http://guides.rubyonrails.org/rails_on_rack.html}{Rails on Rack}.

\subsubsection{ Action Mailer}

Action Mailer is a framework for building e-mail services. You can use Action Mailer to receive and process incoming email and send simple plain text or complex multipart emails based on flexible templates.

\subsubsection{ Active Model}

Active Model provides a defined interface between the Action Pack gem services and Object Relationship Mapping gems such as Active Record. Active Model allows Rails to utilize other ORM frameworks in place of Active Record if your application needs this.

\subsubsection{ Active Record}

Active Record is the base for the models in a Rails application. It provides database independence, basic CRUD functionality, advanced finding capabilities, and the ability to relate models to one another, among other services.

\subsubsection{ Active Resource}

Active Resource provides a framework for managing the connection between business objects and RESTful web services. It implements a way to map web-based resources to local objects with CRUD semantics.

\subsubsection{ Active Support}

Active Support is an extensive collection of utility classes and standard Ruby library extensions that are used in Rails, both by the core code and by your applications.

\subsubsection{ Railties}

Railties is the core Rails code that builds new Rails applications and glues the various frameworks and plugins together in any Rails application.

\subsection{ REST}

Rest stands for Representational State Transfer and is the foundation of the RESTful architecture. This is generally considered to be Roy Fielding’s doctoral thesis, \href{http://www.ics.uci.edu/%7Efielding/pubs/dissertation/top.htm}{Architectural Styles and the Design of Network-based Software Architectures}. While you can read through the thesis, REST in terms of Rails boils down to two main principles:
\begin{itemize}
	\item Using resource identifiers such as URLs to represent resources.
	\item Transferring representations of the state of that resource between system components.
\end{itemize}

\begin{minipage}{\textwidth}
For example, the following HTTP request:
\\ \\
\verb+	DELETE /photos/17+
\\ \\
would be understood to refer to a photo resource with the ID of 17, and to indicate a desired action – deleting that resource. REST is a natural style for the architecture of web applications, and Rails hooks into this shielding you from many of the RESTful complexities and browser quirks.
\end{minipage}


If you’d like more details on REST as an architectural style, these resources are more approachable than Fielding’s thesis:
\begin{itemize}
	\item \href{http://www.infoq.com/articles/rest-introduction}{A Brief Introduction to REST} by Stefan Tilkov
	\item \href{http://bitworking.org/news/373/An-Introduction-to-REST}{An Introduction to REST} (video tutorial) by Joe Gregorio
	\item \href{http://en.wikipedia.org/wiki/Representational_State_Transfer}{Representational State Transfer} article in Wikipedia
	\item \href{http://www.infoq.com/articles/webber-rest-workflow}{How to GET a Cup of Coffee} by Jim Webber, Savas Parastatidis \& Ian Robinson
\end{itemize}

\section{ Creating a New Rails Project}

The best way to use this guide is to follow each step as it happens, no code or step needed to make this example application has been left out, so you can literally follow along step by step. You can get the complete code \href{https://github.com/lifo/docrails/tree/master/railties/guides/code/getting_started}{here}.

By following along with this guide, you’ll create a Rails project called \texttt{blog}, a (very) simple weblog. Before you can start building the application, you need to make sure that you have Rails itself installed.

The examples below use \# and \$ to denote terminal  prompts. If you are using Windows, your prompt will look something like  c:$\backslash$source\_code$>$

\subsection{ Installing Rails}

In most cases, the easiest way to install Rails is to take advantage of RubyGems:

Usually run this as the root user:
\begin{verbatim}
# gem install rails
\end{verbatim}

If you’re working on Windows, you can quickly install Ruby and Rails with \href{http://railsinstaller.org/}{Rails Installer}.

To verify that you have everything installed correctly, you should be able to run the following:
\begin{verbatim}
$ rails --version
\end{verbatim}

If it says something like “Rails 3.2.3” you are ready to continue.

\subsection{ Creating the Blog Application}

To begin, open a terminal, navigate to a folder where you have rights to create files, and type:
\begin{verbatim}
$ rails new blog
\end{verbatim}

This will create a Rails application called Blog in a directory called blog.

You can see all of the switches that the Rails application builder accepts by running

\begin{verbatim}
$ rails new -h
\end{verbatim}

After you create the blog application, switch to its folder to continue work directly in that application:
\begin{verbatim}
$ cd blog
\end{verbatim}

The ‘rails new blog’ command we ran above created a folder in your working directory called \texttt{blog}. The \texttt{blog} folder has a number of auto-generated folders that make up the structure of a Rails application. Most of the work in this tutorial will happen in the \texttt{app/} folder, but here’s a basic rundown on the function of each of the files and folders that Rails created by default:
\\ \\
\begin{tabular}{r|p{350}}
\hline 
\textbf{File/Folder} & \textbf{Purpose} \\ 
\hline 
app/ & Contains the controllers, models, views and assets for your  application. You’ll focus on this folder for the remainder of this  guide. \\ 
config/ & Configure your application’s runtime rules, routes, database, and more.  This is covered in more detail in \href{http://guides.rubyonrails.org/configuring.html}{Configuring Rails Applications} \\ 
config.ru & Rack configuration for Rack based servers used to start the application. \\ 
db/ & Contains your current database schema, as well as the database migrations. \\ 
doc/ & In-depth documentation for your application. \\ 
Gemfile
\\Gemfile.lock & These files allow you to specify what gem dependencies are needed for your Rails application. \\ 
lib/ & Extended modules for your application. \\ 
log/ & Application log files. \\ 
public/ & The only folder seen to the world as-is. Contains the static files and compiled assets. \\ 
Rakefile & This file locates and loads tasks that can be run from the command  line. The task definitions are defined throughout the components of  Rails. Rather than changing Rakefile, you should add your own tasks by  adding files to the lib/tasks directory of your application. \\ 
README.rdoc & This is a brief instruction manual for your application. You  should edit this file to tell others what your application does, how to  set it up, and so on. \\ 
script/ & Contains the rails script that starts your app and can contain other scripts you use to deploy or run your application. \\ 
test/ & Unit tests, fixtures, and other test apparatus. These are covered in \href{http://guides.rubyonrails.org/testing.html}{Testing Rails Applications} \\ 
tmp/ & Temporary files \\ 
vendor/ & A place for all third-party code. In a typical Rails application,  this includes Ruby Gems, the Rails source code (if you optionally  install it into your project) and plugins containing additional  prepackaged functionality.
\end{tabular}


\subsection{ Configuring a Database}

Just about every Rails application will interact with a database. The database to use is specified in a configuration file, \texttt{config/database.yml}.  If you open this file in a new Rails application, you’ll see a default database configured to use SQLite3. The file contains sections for three different environments in which Rails can run by default:
\begin{itemize}
	\item The \texttt{development} environment is used on your development/local computer as you interact manually with the application.
	\item The \texttt{test} environment is used when running automated tests.
	\item The \texttt{production} environment is used when you deploy your application for the world to use.
\end{itemize}

You don’t have to update the database configurations manually. If you look at the options of the application generator, you will see that one of the options is named \texttt{—database}. This option allows you to choose an adapter from a list of the most used relational databases. You can even run the generator repeatedly: \texttt{cd .. \&\& rails new blog —database=mysql}. When you confirm the overwriting  of the \texttt{config/database.yml} file, your application will be configured for MySQL instead of SQLite.  Detailed examples of the common database connections are below.

\subsubsection{ Configuring an SQLite3 Database}

Rails comes with built-in support for \href{http://www.sqlite.org/}{SQLite3}, which is a lightweight serverless database application. While a busy production environment may overload SQLite, it works well for development and testing. Rails defaults to using an SQLite database when creating a new project, but you can always change it later.

Here’s the section of the default configuration file (\texttt{config/database.yml}) with connection information for the development environment:
\\ \\
\begin{tabular}{ll}
development:\\
\hline
adapter: & sqlite3\\
database: & db/development.sqlite3\\
pool: & 5\\
timeout: & 5000\\
\end{tabular}
\\ \\

In this guide we are using an SQLite3 database for data storage, because it is a zero configuration database that just works. Rails also supports MySQL and PostgreSQL “out of the box”, and has plugins for many database systems. If you are using a database in a production environment Rails most likely has an adapter for it.

\subsubsection{ Configuring a MySQL Database}

If you choose to use MySQL instead of the shipped SQLite3 database, your \texttt{config/database.yml} will look a little different. Here’s the development section:
\\ \\
\begin{tabular}{ll}
development:\\
\hline
adapter: & mysql2 \\
encoding: & utf8 \\
database: & blog\_development \\
pool: & 5 \\
username: & root\\
password:\\
socket: & /tmp/mysql.sock
\end{tabular}
\\ \\
If your development computer’s MySQL installation includes a root user with an empty password, this configuration should work for you. Otherwise, change the username and password in the \texttt{development} section as appropriate.

\subsubsection{ Configuring a PostgreSQL Database}

If you choose to use PostgreSQL, your \texttt{config/database.yml} will be customized to use PostgreSQL databases:
\\ \\
\begin{tabular}{ll}
development:\\
\hline
adapter: & postgresql\\
encoding: & unicode\\
database: & blog\_development\\
pool: & 5\\
username: & blog\\
password:
\end{tabular}

\subsubsection{ Configuring an SQLite3 Database for JRuby Platform}

If you choose to use SQLite3 and are using JRuby, your \texttt{config/database.yml} will look a little different. Here’s the development section:
\\ \\
\begin{tabular}{ll}
development:\\
\hline
adapter: & jdbcsqlite3\\
database: & db/development.sqlite3\\
\end{tabular}

\subsubsection{ Configuring a MySQL Database for JRuby Platform}

If you choose to use MySQL and are using JRuby, your \texttt{config/database.yml} will look a little different. Here’s the development section:
\\ \\
\begin{tabular}{ll}
development:\\
\hline
adapter: & jdbcmysql\\
database: & blog\_development\\
username: & root\\
password:
\end{tabular}

\subsubsection{ Configuring a PostgreSQL Database for JRuby Platform}

Finally if you choose to use PostgreSQL and are using JRuby, your \texttt{config/database.yml} will look a little different. Here’s the development section:
\\ \\
\begin{tabular}{ll}
development:\\
\hline
adapter: & jdbcpostgresql\\
encoding: & unicode\\
database: & blog\_development\\
username: & blog\\
password:
\end{tabular}
\\

Change the username and password in the \texttt{development} section as appropriate.

\subsection{ Creating the Database}

Now that you have your database configured, it’s time to have Rails create an empty database for you. You can do this by running a rake command:

\begin{verbatim}
$ rake db:create
\end{verbatim}

This will create your development and test SQLite3 databases inside the \texttt{db/} folder.

Rake is a general-purpose command-runner that Rails uses for many things. You can see the list of available rake commands in your application by running \texttt{rake -T}.

\section{ Hello, Rails!}

One of the traditional places to start with a new language is by getting some text up on screen quickly. To do this, you need to get your Rails application server running.

\subsection{ Starting up the Web Server}

You actually have a functional Rails application already. To see it, you need to start a web server on your development machine. You can do this by running:

\begin{verbatim}
$ rails server
\end{verbatim}

Compiling CoffeeScript to JavaScript requires a JavaScript runtime and the absence of a runtime will give you an \texttt{execjs} error. Usually Mac OS X and Windows come with a JavaScript runtime installed. Rails adds the \texttt{therubyracer} gem to Gemfile in a commented line for new apps and you can uncomment if you need it. \texttt{therubyrhino} is the recommended runtime for JRuby users and is added by default to Gemfile in apps generated under JRuby. You can investigate about all the supported runtimes at \href{https://github.com/sstephenson/execjs#readme}{ExecJS}.

This will fire up an instance of the WEBrick web server by default (Rails can also use several other web servers). To see your application in action, open a browser window and navigate to \verb+http://localhost:3000/+. You should see Rails’ default information page:

\\ \\
\includegraphics[width=\textwidth]{../rails_welcome.png}
\\ \\

To stop the web server, hit Ctrl+C in the terminal window where it’s running. In development mode, Rails does not generally require you to stop the server; changes you make in files will be automatically picked up by the server.

The “Welcome Aboard” page is the \emph{smoke test} for a new Rails application: it makes sure that you have your software configured correctly enough to serve a page. You can also click on the \emph{About your application’s environment} link to see a summary of your application’s environment.

\subsection{ Say “Hello”, Rails}

To get Rails saying “Hello”, you need to create at minimum a controller and a view. Fortunately, you can do that in a single command. Enter this command in your terminal:
\begin{verbatim}
$ rails generate controller home index
\end{verbatim}

If you get a command not found error when running this command, you need to explicitly pass Rails \texttt{rails} commands to Ruby: \texttt{ruby $\backslash$path$\backslash$to$\backslash$your$\backslash$application$\backslash$script$\backslash$rails generate controller home index}.

Rails will create several files for you, including \texttt{app/views/home/index.html.erb}. This is the template that will be used to display the results of the \texttt{index} action (method) in the \texttt{home} controller. Open this file in your text editor and edit it to contain a single line of code:

\verb+		<h1>Hello, Rails!</h1>+

\subsection{ Setting the Application Home Page}

Now that we have made the controller and view, we need to tell Rails when we want “Hello Rails!” to show up. In our case, we want it to show up when we navigate to the root URL of our site, \verb+http://localhost:3000/+, instead of the “Welcome Aboard” smoke test.

The first step to doing this is to delete the default page from your application:
\begin{verbatim}
$ rm public/index.html
\end{verbatim}

We need to do this as Rails will deliver any static file in the \texttt{public} directory in preference to any dynamic content we generate from the controllers.

Now, you have to tell Rails where your actual home page is located. Open the file \texttt{config/routes.rb} in your editor. This is your application’s \emph{routing file} which holds entries in a special DSL (domain-specific language) that tells Rails how to connect incoming requests to controllers and actions. This file contains many sample routes on commented lines, and one of them actually shows you how to connect the root of your site to a specific controller and action. Find the line beginning with \texttt{root :to} and uncomment it. It should look something like the following:


\begin{verbatim}
Blog::Application.routes.draw do
	#...
	# You can have the root of your site routed with "root"
	# just remember to delete public/index.html.
	  root :to => "home#index"
\end{verbatim}

The \texttt{root :to =$>$ "home\#index"} tells Rails to map the root action to the home controller’s index action.

Now if you navigate to \href{http://localhost:3000/}{http://localhost:3000} in your browser, you’ll see \texttt{Hello, Rails!}.

For more information about routing, refer to \href{http://guides.rubyonrails.org/routing.html}{Rails Routing from the Outside In}.

\section{ Getting Up and Running Quickly with Scaffolding}

Rails \emph{scaffolding} is a quick way to generate some of the major pieces of an application. If you want to create the models, views, and controllers for a new resource in a single operation, scaffolding is the tool for the job.

\section{ Creating a Resource}

In the case of the blog application, you can start by generating a scaffold for the Post resource: this will represent a single blog posting. To do this, enter this command in your terminal:
\begin{verbatim}
$ rails generate scaffold Post name:string title:string content:text
\end{verbatim}

The scaffold generator will build several files in your application, along with some folders, and edit \texttt{config/routes.rb}. Here’s a quick overview of what it creates:

\begin{tabular}{p{210}|p{210}}
\hline
\textbf{File} & \textbf{Purpose} \\ 
\hline
db/migrate/20100207214725\_create\_posts.rb     & Migration to create the posts table in your database (your name will include a different timestamp) \\ 
app/models/post.rb                            & The Post model \\ 
test/unit/post\_test.rb                        & Unit testing harness for the posts model \\ 
test/fixtures/posts.yml                       & Sample posts for use in testing \\ 
config/routes.rb                              & Edited to include routing information for posts \\ 
app/controllers/posts\_controller.rb           & The Posts controller \\ 
app/views/posts/index.html.erb                & A view to display an index of all posts  \\ 
app/views/posts/edit.html.erb                 & A view to edit an existing post \\ 
app/views/posts/show.html.erb                 & A view to display a single post \\ 
app/views/posts/new.html.erb                  & A view to create a new post \\ 
app/views/posts/\_form.html.erb                & A partial to control the overall look and feel of the form used in edit and new views \\ 
test/functional/posts\_controller\_test.rb      & Functional testing harness for the posts controller \\ 
app/helpers/posts\_helper.rb                   & Helper functions to be used from the post views \\ 
test/unit/helpers/posts\_helper\_test.rb        & Unit testing harness for the posts helper \\ 
app/assets/javascripts/posts.js.coffee        & CoffeeScript for the posts controller \\ 
app/assets/stylesheets/posts.css.scss         & Cascading style sheet for the posts controller \\ 
app/assets/stylesheets/scaffolds.css.scss     & Cascading style sheet to make the scaffolded views look better
\end{tabular}
\\ \\

While scaffolding will get you up and running quickly, the code it generates is unlikely to be a perfect fit for your application. You’ll most probably want to customize the generated code. Many experienced Rails developers avoid scaffolding entirely, preferring to write all or most of their source code from scratch. Rails, however, makes it really simple to customize templates for generated models, controllers, views and other source files. You’ll find more information in the \href{http://guides.rubyonrails.org/generators.html}{Creating and Customizing Rails Generators \& Templates} guide.

\subsection{ Running a Migration}

One of the products of the \texttt{rails generate scaffold} command is a \emph{database migration}. Migrations are Ruby classes that are designed to make it simple to create and modify database tables. Rails uses rake commands to run migrations, and it’s possible to undo a migration after it’s been applied to your database. Migration filenames include a timestamp to ensure that they’re processed in the order that they were created.

If you look in the \texttt{db/migrate/20100207214725\_create\_posts.rb} file (remember, yours will have a slightly different name), here’s what you’ll find:

\begin{minipage}{\textwidth}
\begin{verbatim}
class CreatePosts < ActiveRecord::Migration
  def change
    create_table :posts do |t|
      t.string :name
      t.string :title
      t.text :content
 
      t.timestamps
    end
  end
end
\end{verbatim}
\end{minipage}
\\ \\

The above migration creates a method named \texttt{change} which will be called when you run this migration. The action defined in this method is also reversible, which means Rails knows how to reverse the change made by this migration, in case you want to reverse it later. When you run this migration it will create a \texttt{posts} table with two string columns and a text column. It also creates two timestamp fields to allow Rails to track post creation and update times. More information about Rails migrations can be found in the \href{http://guides.rubyonrails.org/migrations.html}{Rails Database Migrations} guide.

At this point, you can use a rake command to run the migration:


\begin{verbatim}
$ rake db:migrate
\end{verbatim}

Rails will execute this migration command and tell you it created the Posts table.


\begin{verbatim}
==  CreatePosts: migrating ====================================================
-- create_table(:posts)
   -> 0.0019s
==  CreatePosts: migrated (0.0020s) ===========================================
\end{verbatim}

Because you’re working in the development environment by default, this command will apply to the database defined in the \texttt{development} section of your \texttt{config/database.yml} file. If you would like to execute migrations in another environment, for instance in production, you must explicitly pass it when invoking the command: \texttt{rake db:migrate RAILS\_ENV=production}.

\subsection{ Adding a Link}

To hook the posts up to the home page you’ve already created, you can add a link to the home page. Open \texttt{app/views/home/index.html.erb} and modify it as follows:


\begin{verbatim}
<h1>Hello, Rails!</h1>
<%= link_to "My Blog", posts_path %>
\end{verbatim}

The \texttt{link\_to} method is one of Rails’ built-in view helpers. It creates a hyperlink based on text to display and where to go – in this case, to the path for posts.

\subsection{ Working with Posts in the Browser}

Now you’re ready to start working with posts. To do that, navigate to \href{http://localhost:3000/}{http://localhost:3000} and then click the “My Blog” link:

\\
\includegraphics{../posts_index.png}
\\

This is the result of Rails rendering the \texttt{index} view of your posts. There aren’t currently any posts in the database, but if you click the \texttt{New Post} link you can create one. After that, you’ll find that you can edit posts, look at their details, or destroy them. All of the logic and HTML to handle this was built by the single \texttt{rails generate scaffold} command.

In development mode (which is what you’re working in by default), Rails reloads your application with every browser request, so there’s no need to stop and restart the web server.

Congratulations, you’re riding the rails! Now it’s time to see how it all works.

\subsection{ The Model}

The model file, \texttt{app/models/post.rb} is about as simple as it can get:
\begin{verbatim}
class Post < ActiveRecord::Base
  attr_accessible :content, :name, :title
end
\end{verbatim}

There isn’t much to this file – but note that the \texttt{Post} class inherits from \texttt{ActiveRecord::Base}. Active Record supplies a great deal of functionality to your Rails models for free, including basic database CRUD (Create, Read, Update, Destroy) operations, data validation, as well as sophisticated search support and the ability to relate multiple models to one another. Another important part of this file is \texttt{attr\_accessible}. It specifies a whitelist of attributes that are allowed to be updated in bulk (via \texttt{update\_attributes} for instance).

\subsection{ Adding Some Validation}

Rails includes methods to help you validate the data that you send to models. Open the \texttt{app/models/post.rb} file and edit it:


\begin{verbatim}
class Post < ActiveRecord::Base
  attr_accessible :content, :name, :title
 
  validates :name,  :presence => true
  validates :title, :presence => true,
                    :length => { :minimum => 5 }
end
\end{verbatim}

These changes will ensure that all posts have a name and a title, and that the title is at least five characters long. Rails can validate a variety of conditions in a model, including the presence or uniqueness of columns, their format, and the existence of associated objects. Validations are covered in detail in \href{http://guides.rubyonrails.org/active_record_validations_callbacks.html#validations-overview}{Active Record Validations and Callbacks}

\subsection{ Using the Console}

To see your validations in action, you can use the console. The console is a command-line tool that lets you execute Ruby code in the context of your application:


\begin{verbatim}
$ rails console
\end{verbatim}

The default console will make changes to your database. You can instead open a console that will roll back any changes you make by using \texttt{rails console —sandbox}.

After the console loads, you can use it to work with your application’s models:

\begin{verbatim}
>> p = Post.new(:content => "A new post")
=> #<Post id: nil, name: nil, title: nil,
     content: "A new post", created_at: nil,
     updated_at: nil>
>> p.save
=> false
>> p.errors.full_messages
=> ["Name can't be blank", "Title can't be blank", "Title is too short (minimum is 5 characters)"]
\end{verbatim}

This code shows creating a new \texttt{Post} instance, attempting to save it and getting \texttt{false} for a return value (indicating that the save failed), and inspecting the \texttt{errors} of the post.

When you’re finished, type \texttt{exit} and hit \texttt{return} to exit the console.

Unlike the development web server, the console does not automatically load your code afresh for each line. If you make changes to your models (in your editor) while the console is open, type \texttt{reload!} at the console prompt to load them.

\subsection{ Listing All Posts}

Let’s dive into the Rails code a little deeper to see how the application is showing us the list of Posts. Open the file \texttt{app/controllers/posts\_controller.rb} and look at the \texttt{index} action:
\begin{verbatim}
def index
  @posts = Post.all
 
  respond_to do |format|
    format.html  # index.html.erb
    format.json  { render :json => @posts }
  end
end
\end{verbatim}

\texttt{Post.all} returns all of the posts currently in the database as an array of \texttt{Post} records that we store in an instance variable called \texttt{@posts}.

For more information on finding records with Active Record, see \href{http://guides.rubyonrails.org/active_record_querying.html}{Active Record Query Interface}.

The \texttt{respond\_to} block handles both HTML and JSON calls to this action. If you browse to \href{http://localhost:3000/posts.json}{http://localhost:3000/posts.json}, you’ll see a JSON containing all of the posts. The HTML format looks for a view in \texttt{app/views/posts/} with a name that corresponds to the action name. Rails makes all of the instance variables from the action available to the view. Here’s \texttt{app/views/posts/index.html.erb}:


\begin{verbatim}
<h1>Listing posts</h1>
 
<table>
  <tr>
    <th>Name</th>
    <th>Title</th>
    <th>Content</th>
    <th></th>
    <th></th>
    <th></th>
  </tr>
 
<% @posts.each do |post| %>
  <tr>
    <td><%= post.name %></td>
    <td><%= post.title %></td>
    <td><%= post.content %></td>
    <td><%= link_to 'Show', post %></td>
    <td><%= link_to 'Edit', edit_post_path(post) %></td>
    <td><%= link_to 'Destroy', post, :confirm => 'Are you sure?',
                                     :method => :delete %></td>
  </tr>
<% end %>
</table>
 
<br />
 
<%= link_to 'New post', new_post_path %>
\end{verbatim}

This view iterates over the contents of the \texttt{@posts} array to display content and links. A few things to note in the view:
\begin{itemize}
	\item \texttt{link\_to} builds a hyperlink to a particular destination
	\item \texttt{edit\_post\_path} and \texttt{new\_post\_path} are helpers that  Rails provides as part of RESTful routing. You’ll see a variety of  these helpers for the different actions that the controller includes.
\end{itemize}

In previous versions of Rails, you had to use \texttt{$<$\%=h post.name \%$>$} so that any HTML would be escaped before being inserted into the page. In Rails 3 and above, this is now the default. To get unescaped HTML, you now use \texttt{$<$\%= raw post.name \%$>$}.

For more details on the rendering process, see \href{http://guides.rubyonrails.org/layouts_and_rendering.html}{Layouts and Rendering in Rails}.

\subsection{ Customizing the Layout}

The view is only part of the story of how HTML is displayed in your web browser. Rails also has the concept of \texttt{layouts}, which are containers for views. When Rails renders a view to the browser, it does so by putting the view’s HTML into a layout’s HTML. In previous versions of Rails, the \texttt{rails generate scaffold} command would automatically create a controller specific layout, like \texttt{app/views/layouts/posts.html.erb}, for the posts controller. However this has been changed in Rails 3. An application specific \texttt{layout} is used for all the controllers and can be found in \texttt{app/views/layouts/application.html.erb}. Open this layout in your editor and modify the \texttt{body} tag to include the style directive below:


\begin{verbatim}
<!DOCTYPE html>
<html>
<head>
  <title>Blog</title>
  <%= stylesheet_link_tag "application" %>
  <%= javascript_include_tag "application" %>
  <%= csrf_meta_tags %>
</head>
<body style="background: #EEEEEE;">
 
<%= yield %>
 
</body>
</html>
\end{verbatim}

Now when you refresh the \texttt{/posts} page, you’ll see a gray background to the page. This same gray background will be used throughout all the views for posts.

\subsection{ Creating New Posts}

Creating a new post involves two actions. The first is the \texttt{new} action, which instantiates an empty \texttt{Post} object:


\begin{verbatim}
def new
  @post = Post.new
 
  respond_to do |format|
    format.html  # new.html.erb
    format.json  { render :json => @post }
  end
end
\end{verbatim}

The \texttt{new.html.erb} view displays this empty Post to the user:


\begin{verbatim}
<h1>New post</h1>
 
<%= render 'form' %>
 
<%= link_to 'Back', posts_path %>
\end{verbatim}

The \texttt{$<$\%= render 'form' \%$>$} line is our first introduction to \emph{partials} in Rails. A partial is a snippet of HTML and Ruby code that can be reused in multiple locations. In this case, the form used to make a new post is basically identical to the form used to edit a post, both having text fields for the name and title, a text area for the content, and a button to create the new post or to update the existing one.

If you take a look at \texttt{views/posts/\_form.html.erb} file, you will see the following:
\begin{verbatim}
<%= form_for(@post) do |f| %>
  <% if @post.errors.any? %>
  <div id="errorExplanation">
    <h2><%= pluralize(@post.errors.count, "error") %> prohibited
        this post from being saved:</h2>
    <ul>
    <% @post.errors.full_messages.each do |msg| %>
      <li><%= msg %></li>
    <% end %>
    </ul>
  </div>
  <% end %>
 
  <div class="field">
    <%= f.label :name %><br />
    <%= f.text_field :name %>
  </div>
  <div class="field">
    <%= f.label :title %><br />
    <%= f.text_field :title %>
  </div>
  <div class="field">
    <%= f.label :content %><br />
    <%= f.text_area :content %>
  </div>
  <div class="actions">
    <%= f.submit %>
  </div>
<% end %>
\end{verbatim}

This partial receives all the instance variables defined in the calling view file. In this case, the controller assigned the new \texttt{Post} object to \texttt{@post}, which will thus be available in both the view and the partial as \texttt{@post}.

For more information on partials, refer to the \href{http://guides.rubyonrails.org/layouts_and_rendering.html#using-partials}{Layouts and Rendering in Rails} guide.

The \texttt{form\_for} block is used to create an HTML form. Within this block, you have access to methods to build various controls on the form. For example, \texttt{f.text\_field :name} tells Rails to create a text input on the form and to hook it up to the \texttt{name} attribute of the instance being displayed. You can only use these methods with attributes of the model that the form is based on (in this case \texttt{name}, \texttt{title}, and \texttt{content}). Rails uses \texttt{form\_for} in preference to having you write raw HTML because the code is more succinct, and because it explicitly ties the form to a particular model instance.

The \texttt{form\_for} block is also smart enough to work out if you are doing a \emph{New Post} or an \emph{Edit Post} action, and will set the form \texttt{action} tags and submit button names appropriately in the HTML output.

If you need to create an HTML form that displays arbitrary fields, not tied to a model, you should use the \texttt{form\_tag} method, which provides shortcuts for building forms that are not necessarily tied to a model instance.

When the user clicks the \texttt{Create Post} button on this form, the browser will send information back to the \texttt{create} action of the controller (Rails knows to call the \texttt{create} action because the form is sent with an HTTPPOST request; that’s one of the conventions that were mentioned earlier):


\begin{verbatim}
def create
  @post = Post.new(params[:post])
 
  respond_to do |format|
    if @post.save
      format.html  { redirect_to(@post,
                    :notice => 'Post was successfully created.') }
      format.json  { render :json => @post,
                    :status => :created, :location => @post }
    else
      format.html  { render :action => "new" }
      format.json  { render :json => @post.errors,
                    :status => :unprocessable_entity }
    end
  end
end
\end{verbatim}

The \texttt{create} action instantiates a new Post object from the data supplied by the user on the form, which Rails makes available in the \texttt{params} hash. After successfully saving the new post, \texttt{create} returns the appropriate format that the user has requested (HTML in our case). It then redirects the user to the resulting post \texttt{show} action and sets a notice to the user that the Post was successfully created.

If the post was not successfully saved, due to a validation error, then the controller returns the user back to the \texttt{new} action with any error messages so that the user has the chance to fix the error and try again.

The “Post was successfully created.” message is stored in the Rails \texttt{flash} hash (usually just called \emph{the flash}), so that messages can be carried over to another action, providing the user with useful information on the status of their request. In the case of \texttt{create}, the user never actually sees any page rendered during the post creation process, because it immediately redirects to the new \texttt{Post} as soon as Rails saves the record. The Flash carries over a message to the next action, so that when the user is redirected back to the \texttt{show} action, they are presented with a message saying “Post was successfully created.”

\subsection{ Showing an Individual Post}

When you click the \texttt{show} link for a post on the index page, it will bring you to a URL like \texttt{http://localhost:3000/posts/1}. Rails interprets this as a call to the \texttt{show} action for the resource, and passes in \texttt{1} as the \texttt{:id} parameter. Here’s the \texttt{show} action:

\begin{minipage}{\textwidth}
\begin{verbatim}
def show
  @post = Post.find(params[:id])
 
  respond_to do |format|
    format.html  # show.html.erb
    format.json  { render :json => @post }
  end
end
\end{verbatim}
\end{minipage}
\\  \\

The \texttt{show} action uses \texttt{Post.find} to search for a single record in the database by its id value. After finding the record, Rails displays it by using \texttt{app/views/posts/show.html.erb}:


\begin{verbatim}
<p id="notice"><%= notice %></p>
 
<p>
  <b>Name:</b>
  <%= @post.name %>
</p>
 
<p>
  <b>Title:</b>
  <%= @post.title %>
</p>
 
<p>
  <b>Content:</b>
  <%= @post.content %>
</p>
 
 
<%= link_to 'Edit', edit_post_path(@post) %> |
<%= link_to 'Back', posts_path %>
\end{verbatim}

\subsection{ Editing Posts}

Like creating a new post, editing a post is a two-part process. The first step is a request to \texttt{edit\_post\_path(@post)} with a particular post. This calls the \texttt{edit} action in the controller:


\begin{verbatim}
def edit
  @post = Post.find(params[:id])
end
\end{verbatim}

After finding the requested post, Rails uses the \texttt{edit.html.erb} view to display it:
\begin{verbatim}
<h1>Editing post</h1>
 
<%= render 'form' %>
 
<%= link_to 'Show', @post %> |
<%= link_to 'Back', posts_path %>
\end{verbatim}

Again, as with the \texttt{new} action, the \texttt{edit} action is using the \texttt{form} partial. This time, however, the form will do a PUT action to the \texttt{PostsController} and the submit button will display “Update Post”.

Submitting the form created by this view will invoke the \texttt{update} action within the controller:


\begin{verbatim}
def update
  @post = Post.find(params[:id])
 
  respond_to do |format|
    if @post.update_attributes(params[:post])
      format.html  { redirect_to(@post,
                    :notice => 'Post was successfully updated.') }
      format.json  { head :no_content }
    else
      format.html  { render :action => "edit" }
      format.json  { render :json => @post.errors,
                    :status => :unprocessable_entity }
    end
  end
end
\end{verbatim}

In the \texttt{update} action, Rails first uses the \texttt{:id} parameter passed back from the edit view to locate the database record that’s being edited. The \texttt{update\_attributes} call then takes the \texttt{post} parameter (a hash) from the request and applies it to this record. If all goes well, the user is redirected to the post’s \texttt{show} action. If there are any problems, it redirects back to the \texttt{edit} action to correct them.

\subsection{ Destroying a Post}

Finally, clicking one of the \texttt{destroy} links sends the associated id to the \texttt{destroy} action:


\begin{verbatim}
def destroy
  @post = Post.find(params[:id])
  @post.destroy
 
  respond_to do |format|
    format.html { redirect_to posts_url }
    format.json { head :no_content }
  end
end
\end{verbatim}

The \texttt{destroy} method of an Active Record model instance removes the corresponding record from the database. After that’s done, there isn’t any record to display, so Rails redirects the user’s browser to the index action of the controller.

\section{ Adding a Second Model}

Now that you’ve seen what a model built with scaffolding looks like, it’s time to add a second model to the application. The second model will handle comments on blog posts.

\subsection{ Generating a Model}

Models in Rails use a singular name, and their corresponding database tables use a plural name. For the model to hold comments, the convention is to use the name \texttt{Comment}. Even if you don’t want to use the entire apparatus set up by scaffolding, most Rails developers still use generators to make things like models and controllers. To create the new model, run this command in your terminal:


\begin{verbatim}
$ rails generate model Comment commenter:string body:text post:references
\end{verbatim}

This command will generate four files:
\\ \\
\begin{tabular}{p{230}p{200}}
\hline
\textbf{File} & \textbf{Purpose} \\ 
\hline
db/migrate/20100207235629\_create\_comments.rb  &  Migration to create the comments table in your database (your name will include a different timestamp)  \\ 
 app/models/comment.rb                        &  The Comment model  \\ 
 test/unit/comment\_test.rb                    &  Unit testing harness for the comments model  \\ 
 test/fixtures/comments.yml                   &  Sample comments for use in testing 
\end{tabular}
\\ \\

First, take a look at \texttt{comment.rb}:

\begin{verbatim}
class Comment < ActiveRecord::Base
  belongs_to :post
end
\end{verbatim}

This is very similar to the \texttt{post.rb} model that you saw earlier. The difference is the line \texttt{belongs\_to :post}, which sets up an Active Record \emph{association}. You’ll learn a little about associations in the next section of this guide.

In addition to the model, Rails has also made a migration to create the corresponding database table:


\begin{verbatim}
class CreateComments < ActiveRecord::Migration
  def change
    create_table :comments do |t|
      t.string :commenter
      t.text :body
      t.references :post
 
      t.timestamps
    end
 
    add_index :comments, :post_id
  end
end
\end{verbatim}

The \texttt{t.references} line sets up a foreign key column for the association between the two models. And the \texttt{add\_index} line sets up an index for this association column. Go ahead and run the migration:


\begin{verbatim}
$ rake db:migrate
\end{verbatim}

Rails is smart enough to only execute the migrations that have not already been run against the current database, so in this case you will just see:

\begin{verbatim}
==  CreateComments: migrating =================================================
-- create_table(:comments)
   -> 0.0008s
-- add_index(:comments, :post_id)
   -> 0.0003s
==  CreateComments: migrated (0.0012s) ========================================
\end{verbatim}

\subsection{ Associating Models}

Active Record associations let you easily declare the relationship between two models. In the case of comments and posts, you could write out the relationships this way:

\begin{itemize}
	\item Each comment belongs to one post.
	\item One post can have many comments.
\end{itemize}

In fact, this is very close to the syntax that Rails uses to declare this association. You’ve already seen the line of code inside the Comment model that makes each comment belong to a Post:

\begin{verbatim}
class Comment < ActiveRecord::Base
  belongs_to :post
end
\end{verbatim}

You’ll need to edit the \texttt{post.rb} file to add the other side of the association:

\begin{verbatim}
class Post < ActiveRecord::Base
  attr_accessible :content, :name, :title
 
  validates :name,  :presence => true
  validates :title, :presence => true,
                    :length => { :minimum => 5 }
 
  has_many :comments
end
\end{verbatim}

These two declarations enable a good bit of automatic behavior. For example, if you have an instance variable \texttt{@post} containing a post, you can retrieve all the comments belonging to that post as an array using \texttt{@post.comments}.

For more information on Active Record associations, see the \href{http://guides.rubyonrails.org/association_basics.html}{Active Record Associations} guide.

\subsection{ Adding a Route for Comments}

As with the \texttt{home} controller, we will need to add a route so that Rails knows where we would like to navigate to see \texttt{comments}. Open up the \texttt{config/routes.rb} file again. Near the top, you will see the entry for \texttt{posts} that was added automatically by the scaffold generator: \texttt{resources :posts}. Edit it as follows:

\begin{verbatim}
resources :posts do
  resources :comments
end
\end{verbatim}

This creates \texttt{comments} as a \emph{nested resource} within \texttt{posts}. This is another part of capturing the hierarchical relationship that exists between posts and comments.

For more information on routing, see the \href{http://guides.rubyonrails.org/routing.html}{Rails Routing from the Outside In} guide.

\subsection{ Generating a Controller}


With the model in hand, you can turn your attention to creating a matching controller. Again, there’s a generator for this:
\begin{verbatim}
$ rails generate controller Comments
\end{verbatim}

This creates six files and one empty directory:
\\ \\
\begin{tabular}{ll}
\hline
\textbf{File/Directory} & \textbf{Purpose} \\ 
\hline
 app/controllers/comments\_controller.rb       &  The Comments controller                   \\ 
 app/views/comments/                          &  Views of the controller are stored here   \\ 
 test/functional/comments\_controller\_test.rb  &  The functional tests for the controller   \\ 
 app/helpers/comments\_helper.rb               &  A view helper file                        \\ 
 test/unit/helpers/comments\_helper\_test.rb    &  The unit tests for the helper             \\ 
 app/assets/javascripts/comment.js.coffee     &  CoffeeScript for the controller           \\ 
 app/assets/stylesheets/comment.css.scss      &  Cascading style sheet for the controller 
\end{tabular}
\\ \\

Like with any blog, our readers will create their comments directly after reading the post, and once they have added their comment, will be sent back to the post show page to see their comment now listed. Due to this, our \texttt{CommentsController} is there to provide a method to create comments and delete spam comments when they arrive.

So first, we’ll wire up the Post show template (\texttt{/app/views/posts/show.html.erb}) to let us make a new comment:

\begin{minipage}{\textwidth}
\begin{verbatim}
<p id="notice"><%= notice %></p>
 
<p>
  <b>Name:</b>
  <%= @post.name %>
</p>
 
<p>
  <b>Title:</b>
  <%= @post.title %>
</p>
 
<p>
  <b>Content:</b>
  <%= @post.content %>
</p>
 
<h2>Add a comment:</h2>
<%= form_for([@post, @post.comments.build]) do |f| %>
  <div class="field">
    <%= f.label :commenter %><br />
    <%= f.text_field :commenter %>
  </div>
  <div class="field">
    <%= f.label :body %><br />
    <%= f.text_area :body %>
  </div>
  <div class="actions">
    <%= f.submit %>
  </div>
<% end %>
 
<%= link_to 'Edit Post', edit_post_path(@post) %> |
<%= link_to 'Back to Posts', posts_path %> |
\end{verbatim}
\end{minipage}
\\ \\

This adds a form on the \texttt{Post} show page that creates a new comment by calling the \texttt{CommentsController}\texttt{create} action. Let’s wire that up:


\begin{verbatim}
class CommentsController < ApplicationController
  def create
    @post = Post.find(params[:post_id])
    @comment = @post.comments.create(params[:comment])
    redirect_to post_path(@post)
  end
end
\end{verbatim}

You’ll see a bit more complexity here than you did in the controller for posts. That’s a side-effect of the nesting that you’ve set up. Each request for a comment has to keep track of the post to which the comment is attached, thus the initial call to the \texttt{find} method of the \texttt{Post} model to get the post in question.

In addition, the code takes advantage of some of the methods available for an association. We use the \texttt{create} method on \texttt{@post.comments} to create and save the comment. This will automatically link the comment so that it belongs to that particular post.

Once we have made the new comment, we send the user back to the original post using the \texttt{post\_path(@post)} helper. As we have already seen, this calls the \texttt{show} action of the \texttt{PostsController} which in turn renders the \texttt{show.html.erb} template. This is where we want the comment to show, so let’s add that to the \texttt{app/views/posts/show.html.erb}.

\begin{minipage}{\textwidth}
\begin{verbatim}
<p id="notice"><%= notice %></p>
 
<p>
  <b>Name:</b>
  <%= @post.name %>
</p>
 
<p>
  <b>Title:</b>
  <%= @post.title %>
</p>
 
<p>
  <b>Content:</b>
  <%= @post.content %>
</p>
 
<h2>Comments</h2>
<% @post.comments.each do |comment| %>
  <p>
    <b>Commenter:</b>
    <%= comment.commenter %>
  </p>
 
  <p>
    <b>Comment:</b>
    <%= comment.body %>
  </p>
<% end %>
 
<h2>Add a comment:</h2>
<%= form_for([@post, @post.comments.build]) do |f| %>
  <div class="field">
    <%= f.label :commenter %><br />
    <%= f.text_field :commenter %>
  </div>
  <div class="field">
    <%= f.label :body %><br />
    <%= f.text_area :body %>
  </div>
  <div class="actions">
    <%= f.submit %>
  </div>
<% end %>
 
<br />
 
<%= link_to 'Edit Post', edit_post_path(@post) %> |
<%= link_to 'Back to Posts', posts_path %> |
\end{verbatim}
\end{minipage}
\\ \\

Now you can add posts and comments to your blog and have them show up in the right places.

\section{ Refactoring}

Now that we have posts and comments working, take a look at the \texttt{app/views/posts/show.html.erb} template. It is getting long and awkward. We can use partials to clean it up.

\subsection{ Rendering Partial Collections}

First we will make a comment partial to extract showing all the comments for the post. Create the file \texttt{app/views/comments/\_comment.html.erb} and put the following into it:


\begin{verbatim}
<p>
  <b>Commenter:</b>
  <%= comment.commenter %>
</p>
 
<p>
  <b>Comment:</b>
  <%= comment.body %>
</p>
\end{verbatim}

Then you can change \texttt{app/views/posts/show.html.erb} to look like the following:

\begin{minipage}{\textwidth}
\begin{verbatim}
<p id="notice"><%= notice %></p>
 
<p>
  <b>Name:</b>
  <%= @post.name %>
</p>
 
<p>
  <b>Title:</b>
  <%= @post.title %>
</p>
 
<p>
  <b>Content:</b>
  <%= @post.content %>
</p>
 
<h2>Comments</h2>
<%= render @post.comments %>
 
<h2>Add a comment:</h2>
<%= form_for([@post, @post.comments.build]) do |f| %>
  <div class="field">
    <%= f.label :commenter %><br />
    <%= f.text_field :commenter %>
  </div>
  <div class="field">
    <%= f.label :body %><br />
    <%= f.text_area :body %>
  </div>
  <div class="actions">
    <%= f.submit %>
  </div>
<% end %>
 
<br />
 
<%= link_to 'Edit Post', edit_post_path(@post) %> |
<%= link_to 'Back to Posts', posts_path %> |
\end{verbatim}
\end{minipage}
\\ \\

This will now render the partial in \texttt{app/views/comments/\_comment.html.erb} once for each comment that is in the \texttt{@post.comments} collection. As the \texttt{render} method iterates over the \texttt{@post.comments} collection, it assigns each comment to a local variable named the same as the partial, in this case \texttt{comment} which is then available in the partial for us to show.

\subsection{ Rendering a Partial Form}

Let us also move that new comment section out to its own partial. Again, you create a file \texttt{app/views/comments/\_form.html.erb} containing:


\begin{verbatim}
<%= form_for([@post, @post.comments.build]) do |f| %>
  <div class="field">
    <%= f.label :commenter %><br />
    <%= f.text_field :commenter %>
  </div>
  <div class="field">
    <%= f.label :body %><br />
    <%= f.text_area :body %>
  </div>
  <div class="actions">
    <%= f.submit %>
  </div>
<% end %>
\end{verbatim}

Then you make the \texttt{app/views/posts/show.html.erb} look like the following:


\begin{verbatim}
<p id="notice"><%= notice %></p>
 
<p>
  <b>Name:</b>
  <%= @post.name %>
</p>
 
<p>
  <b>Title:</b>
  <%= @post.title %>
</p>
 
<p>
  <b>Content:</b>
  <%= @post.content %>
</p>
 
<h2>Comments</h2>
<%= render @post.comments %>
 
<h2>Add a comment:</h2>
<%= render "comments/form" %>
 
<br />
 
<%= link_to 'Edit Post', edit_post_path(@post) %> |
<%= link_to 'Back to Posts', posts_path %> |
\end{verbatim}

The second render just defines the partial template we want to render, \texttt{comments/form}. Rails is smart enough to spot the forward slash in that string and realize that you want to render the \texttt{\_form.html.erb} file in the \texttt{app/views/comments} directory.

The \texttt{@post} object is available to any partials rendered in the view because we defined it as an instance variable.

\section{ Deleting Comments}

Another important feature of a blog is being able to delete spam comments. To do this, we need to implement a link of some sort in the view and a \texttt{DELETE} action in the \texttt{CommentsController}.

So first, let’s add the delete link in the \texttt{app/views/comments/\_comment.html.erb} partial:


\begin{verbatim}
<p>
  <b>Commenter:</b>
  <%= comment.commenter %>
</p>
 
<p>
  <b>Comment:</b>
  <%= comment.body %>
</p>
 
<p>
  <%= link_to 'Destroy Comment', [comment.post, comment],
               :confirm => 'Are you sure?',
               :method => :delete %>
</p>
\end{verbatim}

Clicking this new “Destroy Comment” link will fire off a \texttt{DELETE /posts/:id/comments/:id} to our \texttt{CommentsController}, which can then use this to find the comment we want to delete, so let’s add a destroy action to our controller:


\begin{verbatim}
class CommentsController < ApplicationController
 
  def create
    @post = Post.find(params[:post_id])
    @comment = @post.comments.create(params[:comment])
    redirect_to post_path(@post)
  end
 
  def destroy
    @post = Post.find(params[:post_id])
    @comment = @post.comments.find(params[:id])
    @comment.destroy
    redirect_to post_path(@post)
  end
 
end
\end{verbatim}

The \texttt{destroy} action will find the post we are looking at, locate the comment within the \texttt{@post.comments} collection, and then remove it from the database and send us back to the show action for the post.

\subsection{ Deleting Associated Objects}

If you delete a post then its associated comments will also need to be deleted. Otherwise they would simply occupy space in the database. Rails allows you to use the \texttt{dependent} option of an association to achieve this. Modify the Post model, \texttt{app/models/post.rb}, as follows:


\begin{verbatim}
class Post < ActiveRecord::Base
  attr_accessible :content, :name, :title
 
  validates :name,  :presence => true
  validates :title, :presence => true,
                    :length => { :minimum => 5 }
  has_many :comments, :dependent => :destroy
end
\end{verbatim}

\section{ Security}

If you were to publish your blog online, anybody would be able to add, edit and delete posts or delete comments.

Rails provides a very simple HTTP authentication system that will work nicely in this situation.

In the \texttt{PostsController} we need to have a way to block access to the various actions if the person is not authenticated, here we can use the Rails \texttt{http\_basic\_authenticate\_with} method, allowing access to the requested action if that method allows it.

To use the authentication system, we specify it at the top of our \texttt{PostsController}, in this case, we want the user to be authenticated on every action, except for \texttt{index} and \texttt{show}, so we write that:


\begin{verbatim}
class PostsController < ApplicationController
 
  http_basic_authenticate_with :name => "dhh", :password => "secret", :except => [:index, :show]
 
  # GET /posts
  # GET /posts.json
  def index
    @posts = Post.all
    respond_to do |format|
# snipped for brevity
\end{verbatim}

We also only want to allow authenticated users to delete comments, so in the \texttt{CommentsController} we write:

\begin{minipage}{\textwidth}
\begin{verbatim}
class CommentsController < ApplicationController
 
  http_basic_authenticate_with :name => "dhh", :password => "secret", :only => :destroy
 
  def create
    @post = Post.find(params[:post_id])
# snipped for brevity
\end{verbatim}
\end{minipage}
\\ \\

Now if you try to create a new post, you will be greeted with a basic HTTP Authentication challenge


\includegraphics[width=\textwidth]{../challenge.png}

\section{ Building a Multi-Model Form}

Another feature of your average blog is the ability to tag posts. To implement this feature your application needs to interact with more than one model on a single form. Rails offers support for nested forms.

To demonstrate this, we will add support for giving each post multiple tags, right in the form where you create the post. First, create a new model to hold the tags:


\begin{verbatim}
$ rails generate model tag name:string post:references
\end{verbatim}

Again, run the migration to create the database table:


\begin{verbatim}
$ rake db:migrate
\end{verbatim}

Next, edit the \texttt{post.rb} file to create the other side of the association, and to tell Rails (via the \texttt{accepts\_nested\_attributes\_for} macro) that you intend to edit tags via posts:


\begin{verbatim}
class Post < ActiveRecord::Base
  attr_accessible :content, :name, :title, :tags_attributes
 
  validates :name,  :presence => true
  validates :title, :presence => true,
                    :length => { :minimum => 5 }
 
  has_many :comments, :dependent => :destroy
  has_many :tags
 
  accepts_nested_attributes_for :tags, :allow_destroy => :true,
    :reject_if => proc { |attrs| attrs.all? { |k, v| v.blank? } }
end
\end{verbatim}

The \texttt{:allow\_destroy} option tells Rails to enable destroying tags through the nested attributes (you’ll handle that by displaying a “remove” checkbox on the view that you’ll build shortly). The \texttt{:reject\_if} option prevents saving new tags that do not have any attributes filled in.

Also note we had to add \texttt{:tags\_attributes} to the \texttt{attr\_accessible} list. If we didn’t do this there would be a \texttt{MassAssignmentSecurity} exception when we try to update tags through our posts model.

We will modify \texttt{views/posts/\_form.html.erb} to render a partial to make a tag:

\begin{minipage}{\textwidth}
\begin{verbatim}
<% @post.tags.build %>
<%= form_for(@post) do |post_form| %>
  <% if @post.errors.any? %>
  <div id="errorExplanation">
    <h2><%= pluralize(@post.errors.count, "error") %> prohibited this post from being saved:</h2>
    <ul>
    <% @post.errors.full_messages.each do |msg| %>
      <li><%= msg %></li>
    <% end %>
    </ul>
  </div>
  <% end %>
 
  <div class="field">
    <%= post_form.label :name %><br />
    <%= post_form.text_field :name %>
  </div>
  <div class="field">
    <%= post_form.label :title %><br />
    <%= post_form.text_field :title %>
  </div>
  <div class="field">
    <%= post_form.label :content %><br />
    <%= post_form.text_area :content %>
  </div>
  <h2>Tags</h2>
  <%= render :partial => 'tags/form',
             :locals => {:form => post_form} %>
  <div class="actions">
    <%= post_form.submit %>
  </div>
<% end %>
\end{verbatim}
\end{minipage}
\\ \\

Note that we have changed the \texttt{f} in \texttt{form\_for(@post) do |f|} to \texttt{post\_form} to make it easier to understand what is going on.

This example shows another option of the render helper, being able to pass in local variables, in this case, we want the local variable \texttt{form} in the partial to refer to the \texttt{post\_form} object.

We also add a \texttt{@post.tags.build} at the top of this form. This is to make sure there is a new tag ready to have its name filled in by the user. If you do not build the new tag, then the form will not appear as there is no new Tag object ready to create.

Now create the folder \texttt{app/views/tags} and make a file in there called \texttt{\_form.html.erb} which contains the form for the tag:

\begin{minipage}{\textwidth}
\begin{verbatim}
<%= form.fields_for :tags do |tag_form| %>
  <div class="field">
    <%= tag_form.label :name, 'Tag:' %>
    <%= tag_form.text_field :name %>
  </div>
  <% unless tag_form.object.nil? || tag_form.object.new_record? %>
    <div class="field">
      <%= tag_form.label :_destroy, 'Remove:' %>
      <%= tag_form.check_box :_destroy %>
    </div>
  <% end %>
<% end %>
\end{verbatim}
\end{minipage}
\\ \\

Finally, we will edit the \texttt{app/views/posts/show.html.erb} template to show our tags.


\begin{verbatim}
<p id="notice"><%= notice %></p>
 
<p>
  <b>Name:</b>
  <%= @post.name %>
</p>
 
<p>
  <b>Title:</b>
  <%= @post.title %>
</p>
 
<p>
  <b>Content:</b>
  <%= @post.content %>
</p>
 
<p>
  <b>Tags:</b>
  <%= @post.tags.map { |t| t.name }.join(", ") %>
</p>
 
<h2>Comments</h2>
<%= render @post.comments %>
 
<h2>Add a comment:</h2>
<%= render "comments/form" %>
 
 
<%= link_to 'Edit Post', edit_post_path(@post) %> |
<%= link_to 'Back to Posts', posts_path %> |
\end{verbatim}

With these changes in place, you’ll find that you can edit a post and its tags directly on the same view.

However, that method call \texttt{@post.tags.map \{ |t| t.name \}.join(", ")} is awkward, we could handle this by making a helper method.

\section{ View Helpers}

View Helpers live in \texttt{app/helpers} and provide small snippets of reusable code for views. In our case, we want a method that strings a bunch of objects together using their name attribute and joining them with a comma. As this is for the Post show template, we put it in the PostsHelper.

Open up \texttt{app/helpers/posts\_helper.rb} and add the following:


\begin{verbatim}
module PostsHelper
  def join_tags(post)
    post.tags.map { |t| t.name }.join(", ")
  end
end
\end{verbatim}

Now you can edit the view in \texttt{app/views/posts/show.html.erb} to look like this:

\begin{minipage}{\textwidth}
\begin{verbatim}
<p id="notice"><%= notice %></p>
 
<p>
  <b>Name:</b>
  <%= @post.name %>
</p>
 
<p>
  <b>Title:</b>
  <%= @post.title %>
</p>
 
<p>
  <b>Content:</b>
  <%= @post.content %>
</p>
 
<p>
  <b>Tags:</b>
  <%= join_tags(@post) %>
</p>
 
<h2>Comments</h2>
<%= render @post.comments %>
 
<h2>Add a comment:</h2>
<%= render "comments/form" %>
 
 
<%= link_to 'Edit Post', edit_post_path(@post) %> |
<%= link_to 'Back to Posts', posts_path %> |
\end{verbatim}
\end{minipage}
\\ \\

\section{ What’s Next?}

Now that you’ve seen your first Rails application, you should feel free to update it and experiment on your own. But you don’t have to do everything without help. As you need assistance getting up and running with Rails, feel free to consult these support resources:
\begin{itemize}
	\item The \href{http://guides.rubyonrails.org/index.html}{Ruby on Rails guides}
	\item The \href{http://railstutorial.org/book}{Ruby on Rails Tutorial}
	\item The \href{http://groups.google.com/group/rubyonrails-talk}{Ruby on Rails mailing list}
	\item The \href{irc://irc.freenode.net/#rubyonrails}{\#rubyonrails} channel on irc.freenode.net
\end{itemize}

Rails also comes with built-in help that you can generate using the rake command-line utility:
\begin{itemize}
	\item Running \texttt{rake doc:guides} will put a full copy of the Rails Guides in the \texttt{doc/guides} folder of your application. Open \texttt{doc/guides/index.html} in your web browser to explore the Guides.
	\item Running \texttt{rake doc:rails} will put a full copy of the API documentation for Rails in the \texttt{doc/api} folder of your application. Open \texttt{doc/api/index.html} in your web browser to explore the API documentation.
\end{itemize}

\section{ Configuration Gotchas}

The easiest way to work with Rails is to store all external data as UTF-8. If you don’t, Ruby libraries and Rails will often be able to convert your native data into UTF-8, but this doesn’t always work reliably, so you’re better off ensuring that all external data is UTF-8.

If you have made a mistake in this area, the most common symptom is a black diamond with a question mark inside appearing in the browser. Another common symptom is characters like “ü” appearing instead of “ü”. Rails takes a number of internal steps to mitigate common causes of these problems that can be automatically detected and corrected. However, if you have external data that is not stored as UTF-8, it can occasionally result in these kinds of issues that cannot be automatically detected by Rails and corrected.

Two very common sources of data that are not UTF-8:
\begin{itemize}
	\item Your text editor: Most text editors (such as Textmate), default to saving files as   UTF-8. If your text editor does not, this can result in special characters that you   enter in your templates (such as é) to appear as a diamond with a question mark inside   in the browser. This also applies to your I18N translation files.   Most editors that do not already default to UTF-8 (such as some versions of   Dreamweaver) offer a way to change the default to UTF-8. Do so.
	\item Your database. Rails defaults to converting data from your database into UTF-8 at   the boundary. However, if your database is not using UTF-8 internally, it may not   be able to store all characters that your users enter. For instance, if your database   is using Latin-1 internally, and your user enters a Russian, Hebrew, or Japanese   character, the data will be lost forever once it enters the database. If possible,   use UTF-8 as the internal storage of your database.
\end{itemize}

\chapter{Migrations}

Migrations are a convenient way for you to alter your database in a structured and organized manner. You could edit fragments of SQL by hand but you would then be responsible for telling other developers that they need to go and run them. You’d also have to keep track of which changes need to be run against the production machines next time you deploy.

Active Record tracks which migrations have already been run so all you have to do is update your source and run \texttt{rake db:migrate}. Active Record will work out which migrations should be run. It will also update your \texttt{db/schema.rb} file to match the structure of your database.

Migrations also allow you to describe these transformations using Ruby. The great thing about this is that (like most of Active Record’s functionality) it is database independent: you don’t need to worry about the precise syntax of \texttt{CREATE TABLE} any more than you worry about variations on \texttt{SELECT *} (you can drop down to raw SQL for database specific features). For example you could use SQLite3 in development, but MySQL in production.

In this guide, you’ll learn all about migrations including:
\begin{itemize}
	\item The generators you can use to create them
	\item The methods Active Record provides to manipulate your database
	\item The Rake tasks that manipulate them
	\item How they relate to \texttt{schema.rb}
\end{itemize}

\newpage

\section{ Anatomy of a Migration}
Before we dive into the details of a migration, here are a few examples of the sorts of things you can do:
\\ \\
\begin{minipage}{\textwidth}
\begin{verbatim}
class CreateProducts < ActiveRecord::Migration
  def up
    create_table :products do |t|
      t.string :name
      t.text :description

      t.timestamps
    end
  end

  def down
    drop_table :products
  end
end
\end{verbatim}
\end{minipage}
\\ \\


This migration adds a table called \texttt{products} with a string column called \texttt{name} and a text column called \texttt{description}. A primary key column called \texttt{id} will also be added, however since this is the default we do not need to ask for this. The timestamp columns \texttt{created\_at} and \texttt{updated\_at} which Active Record populates automatically will also be added. Reversing this migration is as simple as dropping the table.

Migrations are not limited to changing the schema. You can also use them to fix bad data in the database or populate new fields:
\\ \\
\begin{minipage}{\textwidth}
\begin{verbatim}
class AddReceiveNewsletterToUsers < ActiveRecord::Migration
  def up
    change_table :users do |t|
      t.boolean :receive_newsletter, :default => false
    end
    User.update_all ["receive_newsletter = ?", true]
  end
 
  def down
    remove_column :users, :receive_newsletter
  end
end
\end{verbatim}
\end{minipage}
\\ \\


Some \hyperlink{using-models-in-your-migrations}{caveats} apply to using models in your migrations.

This migration adds a \texttt{receive\_newsletter} column to the \texttt{users} table. We want it to default to \texttt{false} for new users, but existing users are considered to have already opted in, so we use the User model to set the flag to \texttt{true} for existing users.

Rails 3.1 makes migrations smarter by providing a new \texttt{change} method. This method is preferred for writing constructive migrations (adding columns or tables). The migration knows how to migrate your database and reverse it when the migration is rolled back without the need to write a separate \texttt{down} method.
\\ \\
\begin{minipage}{\textwidth}
\begin{verbatim}
class CreateProducts < ActiveRecord::Migration
  def change
    create_table :products do |t|
      t.string :name
      t.text :description
 
      t.timestamps
    end
  end
end
\end{verbatim}
\end{minipage}
\\ \\

\subsection{ Migrations are Classes}

A migration is a subclass of \texttt{ActiveRecord::Migration} that implements two methods: \texttt{up} (perform the required transformations) and \texttt{down} (revert them).

Active Record provides methods that perform common data definition tasks in a database independent way (you’ll read about them in detail later):
\begin{itemize}
	\item \texttt{add\_column}
	\item \texttt{add\_index}
	\item \texttt{change\_column}
	\item \texttt{change\_table}
	\item \texttt{create\_table}
	\item \texttt{drop\_table}
	\item \texttt{remove\_column}
	\item \texttt{remove\_index}
	\item \texttt{rename\_column}
\end{itemize}

If you need to perform tasks specific to your database (for example create a \hyperlink{active-record-and-referential-integrity}{foreign key} constraint) then the \texttt{execute} method allows you to execute arbitrary SQL. A migration is just a regular Ruby class so you’re not limited to these functions. For example after adding a column you could write code to set the value of that column for existing records (if necessary using your models).

On databases that support transactions with statements that change the schema (such as PostgreSQL or SQLite3), migrations are wrapped in a transaction. If the database does not support this (for example MySQL) then when a migration fails the parts of it that succeeded will not be rolled back. You will have to rollback the changes that were made by hand.

\subsection{ What’s in a Name}

Migrations are stored as files in the \texttt{db/migrate} directory, one for each migration class. The name of the file is of the form \texttt{YYYYMMDDHHMMSS\_create\_products.rb}, that is to say a UTC timestamp identifying the migration followed by an underscore followed by the name of the migration. The name of the migration class (CamelCased version) should match the latter part of the file name. For example \texttt{20080906120000\_create\_products.rb} should define class \texttt{CreateProducts} and \texttt{20080906120001\_add\_details\_to\_products.rb} should define \texttt{AddDetailsToProducts}. If you do feel the need to change the file name then you \emph{have to} update the name of the class inside or Rails will complain about a missing class.

Internally Rails only uses the migration’s number (the timestamp) to identify them. Prior to Rails 2.1 the migration number started at 1 and was incremented each time a migration was generated. With multiple developers it was easy for these to clash requiring you to rollback migrations and renumber them. With Rails 2.1+ this is largely avoided by using the creation time of the migration to identify them. You can revert to the old numbering scheme by adding the following line to \texttt{config/application.rb}.

\begin{verbatim}
config.active_record.timestamped_migrations = false
\end{verbatim}


The combination of timestamps and recording which migrations have been run allows Rails to handle common situations that occur with multiple developers.

For example Alice adds migrations \texttt{20080906120000} and \texttt{20080906123000} and Bob adds \texttt{20080906124500} and runs it. Alice finishes her changes and checks in her migrations and Bob pulls down the latest changes. When Bob runs \texttt{rake db:migrate}, Rails knows that it has not run Alice’s two migrations so it executes the \texttt{up} method for each migration.

Of course this is no substitution for communication within the team. For example, if Alice’s migration removed a table that Bob’s migration assumed to exist, then trouble would certainly strike.

\subsection{ Changing Migrations}

Occasionally you will make a mistake when writing a migration. If you have already run the migration then you cannot just edit the migration and run the migration again: Rails thinks it has already run the migration and so will do nothing when you run \texttt{rake db:migrate}. You must rollback the migration (for example with \texttt{rake db:rollback}), edit your migration and then run \texttt{rake db:migrate} to run the corrected version.

In general editing existing migrations is not a good idea: you will be creating extra work for yourself and your co-workers and cause major headaches if the existing version of the migration has already been run on production machines. Instead, you should write a new migration that performs the changes you require. Editing a freshly generated migration that has not yet been committed to source control (or, more generally, which has not been propagated beyond your development machine) is relatively harmless.

\subsection{ Supported Types}

Active Record supports the following database column types:
\begin{itemize}
	\item \texttt{:binary}
	\item \texttt{:boolean}
	\item \texttt{:date}
	\item \texttt{:datetime}
	\item \texttt{:decimal}
	\item \texttt{:float}
	\item \texttt{:integer}
	\item \texttt{:primary\_key}
	\item \texttt{:string}
	\item \texttt{:text}
	\item \texttt{:time}
	\item \texttt{:timestamp}
\end{itemize}

These will be mapped onto an appropriate underlying database type. For example, with MySQL the type \texttt{:string} is mapped to \texttt{VARCHAR(255)}. You can create columns of types not supported by Active Record when using the non-sexy syntax, for example
\begin{verbatim}
create_table :products do |t|
  t.column :name, 'polygon', :null => false
end
\end{verbatim}


This may however hinder portability to other databases.

\section{ Creating a Migration}

\subsection{ Creating a Model}

The model and scaffold generators will create migrations appropriate for adding a new model. This migration will already contain instructions for creating the relevant table. If you tell Rails what columns you want, then statements for adding these columns will also be created. For example, running

\begin{verbatim}
$ rails generate model Product name:string description:text
\end{verbatim}


will create a migration that looks like this
\\ \\
\begin{minipage}{\textwidth}
\begin{verbatim}
class CreateProducts < ActiveRecord::Migration
  def change
    create_table :products do |t|
      t.string :name
      t.text :description
 
      t.timestamps
    end
  end
end
\end{verbatim}
\end{minipage}
\\ \\


You can append as many column name/type pairs as you want. By default, the generated migration will include \texttt{t.timestamps} (which creates the \texttt{updated\_at} and \texttt{created\_at} columns that are automatically populated by Active Record).

\subsection{ Creating a Standalone Migration}

If you are creating migrations for other purposes (for example to add a column to an existing table) then you can also use the migration generator:
\begin{verbatim}
$ rails generate migration AddPartNumberToProducts
\end{verbatim}


This will create an empty but appropriately named migration:

\begin{verbatim}
class AddPartNumberToProducts < ActiveRecord::Migration
  def change
  end
end
\end{verbatim}


If the migration name is of the form “AddXXXToYYY” or “RemoveXXXFromYYY” and is followed by a list of column names and types then a migration containing the appropriate \texttt{add\_column} and \texttt{remove\_column} statements will be created.
\begin{verbatim}
$ rails generate migration AddPartNumberToProducts part_number:string
\end{verbatim}


will generate
\begin{verbatim}
class AddPartNumberToProducts < ActiveRecord::Migration
  def change
    add_column :products, :part_number, :string
  end
end
\end{verbatim}

Similarly,
\begin{verbatim}
$ rails generate migration RemovePartNumberFromProducts part_number:string
\end{verbatim}


generates
\\ \\
\begin{minipage}{\textwidth}
\begin{verbatim}
class RemovePartNumberFromProducts < ActiveRecord::Migration
  def up
    remove_column :products, :part_number
  end
 
  def down
    add_column :products, :part_number, :string
  end
end
\end{verbatim}
\end{minipage}
\\ \\

You are not limited to one magically generated column, for example
\begin{verbatim}
$ rails generate migration AddDetailsToProducts part_number:string price:decimal
\end{verbatim}

generates
\begin{verbatim}
class AddDetailsToProducts < ActiveRecord::Migration
  def change
    add_column :products, :part_number, :string
    add_column :products, :price, :decimal
  end
end
\end{verbatim}

As always, what has been generated for you is just a starting point. You can add or remove from it as you see fit by editing the db/migrate/YYYYMMDDHHMMSS\_add\_details\_to\_products.rb file.

The generated migration file for destructive migrations will still be old-style using the \texttt{up} and \texttt{down} methods. This is because Rails needs to know the original data types defined when you made the original changes.

\section{ Writing a Migration}

Once you have created your migration using one of the generators it’s time to get to work!

\subsection{ Creating a Table}

Migration method \texttt{create\_table} will be one of your workhorses. A typical use would be
\\ \\
\begin{minipage}{\textwidth}
\begin{verbatim}
create_table :products do |t|
  t.string :name
end
\end{verbatim}
\end{minipage}
\\ \\


which creates a \texttt{products} table with a column called \texttt{name} (and as discussed below, an implicit \texttt{id} column).

The object yielded to the block allows you to create columns on the table. There are two ways of doing it. The first (traditional) form looks like
\begin{verbatim}
create_table :products do |t|
  t.column :name, :string, :null => false
end
\end{verbatim}

The second form, the so called “sexy” migration, drops the somewhat redundant \texttt{column} method. Instead, the \texttt{string}, \texttt{integer}, etc. methods create a column of that type. Subsequent parameters are the same.
\begin{verbatim}
create_table :products do |t|
  t.string :name, :null => false
end
\end{verbatim}


By default, \texttt{create\_table} will create a primary key called \texttt{id}. You can change the name of the primary key with the \texttt{:primary\_key} option (don’t forget to update the corresponding model) or, if you don’t want a primary key at all (for example for a HABTM join table), you can pass the option \texttt{:id =$>$ false}. If you need to pass database specific options you can place an SQL fragment in the \texttt{:options} option. For example,
\begin{verbatim}
create_table :products, :options => "ENGINE=BLACKHOLE" do |t|
  t.string :name, :null => false
end
\end{verbatim}

will append \texttt{ENGINE=BLACKHOLE} to the SQL statement used to create the table (when using MySQL, the default is \texttt{ENGINE=InnoDB}).

\subsection{ Changing Tables}

A close cousin of \texttt{create\_table} is \texttt{change\_table}, used for changing existing tables. It is used in a similar fashion to \texttt{create\_table} but the object yielded to the block knows more tricks. For example
\begin{verbatim}
change_table :products do |t|
  t.remove :description, :name
  t.string :part_number
  t.index :part_number
  t.rename :upccode, :upc_code
end
\end{verbatim}

removes the \texttt{description} and \texttt{name} columns, creates a \texttt{part\_number} string column and adds an index on it. Finally it renames the \texttt{upccode} column.

\subsection{ Special Helpers}

Active Record provides some shortcuts for common functionality. It is for example very common to add both the \texttt{created\_at} and \texttt{updated\_at} columns and so there is a method that does exactly that:
\begin{verbatim}
create_table :products do |t|
  t.timestamps
end
\end{verbatim}

will create a new products table with those two columns (plus the \texttt{id} column) whereas
\begin{verbatim}
change_table :products do |t|
  t.timestamps
end
\end{verbatim}

adds those columns to an existing table.

Another helper is called \texttt{references} (also available as \texttt{belongs\_to}). In its simplest form it just adds some readability.
\begin{verbatim}
create_table :products do |t|
  t.references :category
end
\end{verbatim}

will create a \texttt{category\_id} column of the appropriate type. Note that you pass the model name, not the column name. Active Record adds the \texttt{\_id} for you. If you have polymorphic \texttt{belongs\_to} associations then \texttt{references} will add both of the columns required:
\begin{verbatim}
create_table :products do |t|
  t.references :attachment, :polymorphic => {:default => 'Photo'}
end
\end{verbatim}

will add an \texttt{attachment\_id} column and a string \texttt{attachment\_type} column with a default value of ‘Photo’.

The \texttt{references} helper does not actually create foreign key constraints for you. You will need to use \texttt{execute} or a plugin that adds \hyperlink{active-record-and-referential-integrity}{foreign key support}.

If the helpers provided by Active Record aren’t enough you can use the \texttt{execute} method to execute arbitrary SQL.

For more details and examples of individual methods, check the API documentation, in particular the documentation for \href{http://api.rubyonrails.org/classes/ActiveRecord/ConnectionAdapters/SchemaStatements.html}{\texttt{ActiveRecord::ConnectionAdapters::SchemaStatements}} (which provides the methods available in the \texttt{up} and \texttt{down} methods), \href{http://api.rubyonrails.org/classes/ActiveRecord/ConnectionAdapters/TableDefinition.html}{\texttt{ActiveRecord::ConnectionAdapters::TableDefinition}} (which provides the methods available on the object yielded by \texttt{create\_table}) and \href{http://api.rubyonrails.org/classes/ActiveRecord/ConnectionAdapters/Table.html}{\texttt{ActiveRecord::ConnectionAdapters::Table}} (which provides the methods available on the object yielded by \texttt{change\_table}).

\subsection{ Using the \texttt{change} Method}

The \texttt{change} method removes the need to write both \texttt{up} and \texttt{down} methods in those cases that Rails know how to revert the changes automatically. Currently, the \texttt{change} method supports only these migration definitions:
\begin{itemize}
	\item \texttt{add\_column}
	\item \texttt{add\_index}
	\item \texttt{add\_timestamps}
	\item \texttt{create\_table}
	\item \texttt{remove\_timestamps}
	\item \texttt{rename\_column}
	\item \texttt{rename\_index}
	\item \texttt{rename\_table}
\end{itemize}

If you’re going to need to use any other methods, you’ll have to write the \texttt{up} and \texttt{down} methods instead of using the \texttt{change} method.

\subsection{ Using the \texttt{up}/\texttt{down} Methods}

The \texttt{down} method of your migration should revert the transformations done by the \texttt{up} method. In other words, the database schema should be unchanged if you do an \texttt{up} followed by a \texttt{down}. For example, if you create a table in the \texttt{up} method, you should drop it in the \texttt{down} method. It is wise to reverse the transformations in precisely the reverse order they were made in the \texttt{up} method. For example,
\\ \\
\begin{minipage}{\textwidth}
\begin{verbatim}
class ExampleMigration < ActiveRecord::Migration
  def up
    create_table :products do |t|
      t.references :category
    end
    #add a foreign key
    execute <<-SQL
      ALTER TABLE products
        ADD CONSTRAINT fk_products_categories
        FOREIGN KEY (category_id)
        REFERENCES categories(id)
    SQL
    add_column :users, :home_page_url, :string
    rename_column :users, :email, :email_address
  end
 
  def down
    rename_column :users, :email_address, :email
    remove_column :users, :home_page_url
    execute <<-SQL
      ALTER TABLE products
        DROP FOREIGN KEY fk_products_categories
    SQL
    drop_table :products
  end
end
\end{verbatim}
\end{minipage}
\\ \\

Sometimes your migration will do something which is just plain irreversible; for example, it might destroy some data. In such cases, you can raise \texttt{ActiveRecord::IrreversibleMigration} from your \texttt{down} method. If someone tries to revert your migration, an error message will be displayed saying that it can’t be done.

\section{ Running Migrations}

Rails provides a set of rake tasks to work with migrations which boil down to running certain sets of migrations.

The very first migration related rake task you will use will probably be \texttt{rake db:migrate}. In its most basic form it just runs the \texttt{up} or \texttt{change} method for all the migrations that have not yet been run. If there are no such migrations, it exits. It will run these migrations in order based on the date of the migration.

Note that running the \texttt{db:migrate} also invokes the \texttt{db:schema:dump} task, which will update your db/schema.rb file to match the structure of your database.

If you specify a target version, Active Record will run the required migrations (up, down or change) until it has reached the specified version. The version is the numerical prefix on the migration’s filename. For example, to migrate to version 20080906120000 run
\begin{verbatim}
$ rake db:migrate VERSION=20080906120000
\end{verbatim}

If version 20080906120000 is greater than the current version (i.e., it is migrating upwards), this will run the \texttt{up} method on all migrations up to and including 20080906120000, and will not execute any later migrations. If migrating downwards, this will run the \texttt{down} method on all the migrations down to, but not including, 20080906120000.

\subsection{ Rolling Back}

A common task is to rollback the last migration, for example if you made a mistake in it and wish to correct it. Rather than tracking down the version number associated with the previous migration you can run
\begin{verbatim}
$ rake db:rollback
\end{verbatim}

This will run the \texttt{down} method from the latest migration. If you need to undo several migrations you can provide a \texttt{STEP} parameter:

\begin{verbatim}
$ rake db:rollback STEP=3
\end{verbatim}

will run the \texttt{down} method from the last 3 migrations.

The \texttt{db:migrate:redo} task is a shortcut for doing a rollback and then migrating back up again. As with the \texttt{db:rollback} task, you can use the \texttt{STEP} parameter if you need to go more than one version back, for example
\begin{verbatim}
$ rake db:migrate:redo STEP=3
\end{verbatim}

Neither of these Rake tasks do anything you could not do with \texttt{db:migrate}. They are simply more convenient, since you do not need to explicitly specify the version to migrate to.

\subsection{ Resetting the database}

The \texttt{rake db:reset} task will drop the database, recreate it and load the current schema into it.

This is not the same as running all the migrations – see the section on \hyperlink{schema-dumping-and-you}{schema.rb}.

\subsection{ Running specific migrations}

If you need to run a specific migration up or down, the \texttt{db:migrate:up} and \texttt{db:migrate:down} tasks will do that. Just specify the appropriate version and the corresponding migration will have its \texttt{up} or \texttt{down} method invoked, for example,
\begin{verbatim}
$ rake db:migrate:up VERSION=20080906120000
\end{verbatim}

will run the \texttt{up} method from the 20080906120000 migration. These tasks still check whether the migration has already run, so for example \textbf{db:migrate:up VERSION=20080906120000} will do nothing if Active Record believes that 20080906120000 has already been run.

\subsection{ Changing the output of running migrations}

By default migrations tell you exactly what they’re doing and how long it took. A migration creating a table and adding an index might produce output like this
\\ \\
\begin{minipage}{\textwidth}
\begin{verbatim}
==  CreateProducts: migrating =================================================
-- create_table(:products)
   -> 0.0028s
==  CreateProducts: migrated (0.0028s) ========================================
\end{verbatim}
\end{minipage}
\\ \\

Several methods are provided in migrations that allow you to control all this:
\\ \\
\begin{tabular}{l|p{150}}
\hline
\textbf{Method             } & \textbf{Purpose} \\ 
\hline
suppress\_messages     & Takes a block as an argument and suppresses any output                        generated by the block. \\ 
say                   & Takes a message argument and outputs it as is. A second                        boolean argument can be passed to specify whether to                        indent or not. \\ 
say\_with\_time         & Outputs text along with how long it took to run its                        block. If the block returns an integer it assumes it                        is the number of rows affected.
\end{tabular}
\\ \\


For example, this migration
\\ \\
\begin{minipage}{\textwidth}
\begin{verbatim}
class CreateProducts < ActiveRecord::Migration
  def change
    suppress_messages do
      create_table :products do |t|
        t.string :name
        t.text :description
        t.timestamps
      end
    end
    say "Created a table"
    suppress_messages {add_index :products, :name}
    say "and an index!", true
    say_with_time 'Waiting for a while' do
      sleep 10
      250
    end
  end
end
\end{verbatim}
\end{minipage}
\\ \\


generates the following output
\begin{verbatim}
==  CreateProducts: migrating =================================================
-- Created a table
   -> and an index!
-- Waiting for a while
   -> 10.0013s
   -> 250 rows
==  CreateProducts: migrated (10.0054s) =======================================
\end{verbatim}

If you want Active Record to not output anything, then running
\begin{verbatim}
rake db:migrate VERBOSE=false} will suppress all output.  
\end{verbatim}


\section{ Using Models in Your Migrations}

When creating or updating data in a migration it is often tempting to use one of your models. After all, they exist to provide easy access to the underlying data. This can be done, but some caution should be observed.

For example, problems occur when the model uses database columns which are (1) not currently in the database and (2) will be created by this or a subsequent migration.

Consider this example, where Alice and Bob are working on the same code base which contains a \texttt{Product} model:

Bob goes on vacation.

Alice creates a migration for the \texttt{products} table which adds a new column and initializes it.  She also adds a validation to the \texttt{Product} model for the new column.
\begin{verbatim}
# db/migrate/20100513121110_add_flag_to_product.rb
 
class AddFlagToProduct < ActiveRecord::Migration
  def change
    add_column :products, :flag, :boolean
    Product.all.each do |product|
      product.update_attributes!(:flag => 'false')
    end
  end
end
\end{verbatim}

\begin{verbatim}
# app/model/product.rb
 
class Product < ActiveRecord::Base
  validates :flag, :presence => true
end
\end{verbatim}

Alice adds a second migration which adds and initializes another column to the \texttt{products} table and also adds a validation to the \texttt{Product} model for the new column.
\\ \\
\begin{minipage}{\textwidth}
\begin{verbatim}
# db/migrate/20100515121110_add_fuzz_to_product.rb
 
class AddFuzzToProduct < ActiveRecord::Migration
  def change
    add_column :products, :fuzz, :string
    Product.all.each do |product|
      product.update_attributes! :fuzz => 'fuzzy'
    end
  end
end
\end{verbatim}

\begin{verbatim}
# app/model/product.rb
 
class Product < ActiveRecord::Base
  validates :flag, :fuzz, :presence => true
end
\end{verbatim}
\end{minipage}
\\ \\


Both migrations work for Alice.

Bob comes back from vacation and:
\begin{enumerate}
	\item Updates the source – which contains both migrations and the latests version of the Product model.
	\item Runs outstanding migrations with \texttt{rake db:migrate}, which includes the one that updates the \texttt{Product} model.
\end{enumerate}

The migration crashes because when the model attempts to save, it tries to validate the second added column, which is not in the database when the \emph{first} migration runs:
\begin{verbatim}
rake aborted!
An error has occurred, this and all later migrations canceled:
 
undefined method `fuzz' for #<Product:0x000001049b14a0>
\end{verbatim}

A fix for this is to create a local model within the migration. This keeps rails from running the validations, so that the migrations run to completion.

When using a faux model, it’s a good idea to call \texttt{Product.reset\_column\_information} to refresh the \texttt{ActiveRecord} cache for the \texttt{Product} model prior to updating data in the database.

If Alice had done this instead, there would have been no problem:
\\ \\
\begin{minipage}{\textwidth}
\begin{verbatim}
# db/migrate/20100513121110_add_flag_to_product.rb
 
class AddFlagToProduct < ActiveRecord::Migration
  class Product < ActiveRecord::Base
  end
 
  def change
    add_column :products, :flag, :integer
    Product.reset_column_information
    Product.all.each do |product|
      product.update_attributes!(:flag => false)
    end
  end
end
\end{verbatim}
\end{minipage}
\\ \\
\\ \\
\begin{minipage}{\textwidth}
\begin{verbatim}
# db/migrate/20100515121110_add_fuzz_to_product.rb
 
class AddFuzzToProduct < ActiveRecord::Migration
  class Product < ActiveRecord::Base
  end
 
  def change
    add_column :products, :fuzz, :string
    Product.reset_column_information
    Product.all.each do |product|
      product.update_attributes!(:fuzz => 'fuzzy')
    end
  end
end
\end{verbatim}
\end{minipage}
\\ \\


\section{ Schema Dumping and You}

\subsection{ What are Schema Files for?}

Migrations, mighty as they may be, are not the authoritative source for your database schema. That role falls to either \texttt{db/schema.rb} or an SQL file which Active Record generates by examining the database. They are not designed to be edited, they just represent the current state of the database.

There is no need (and it is error prone) to deploy a new instance of an app by replaying the entire migration history. It is much simpler and faster to just load into the database a description of the current schema.

For example, this is how the test database is created: the current development database is dumped (either to \texttt{db/schema.rb} or \texttt{db/structure.sql}) and then loaded into the test database.

Schema files are also useful if you want a quick look at what attributes an Active Record object has. This information is not in the model’s code and is frequently spread across several migrations, but the information is nicely summed up in the schema file. The \href{https://github.com/ctran/annotate_models}{annotate\_models} gem automatically adds and updates comments at the top of each model summarizing the schema if you desire that functionality.

\subsection{ Types of Schema Dumps}

There are two ways to dump the schema. This is set in \texttt{config/application.rb} by the \texttt{config.active\_record.schema\_format} setting, which may be either \texttt{:sql} or \texttt{:ruby}.

If \texttt{:ruby} is selected then the schema is stored in \texttt{db/schema.rb}. If you look at this file you’ll find that it looks an awful lot like one very big migration:
\begin{verbatim}
ActiveRecord::Schema.define(:version => 20080906171750) do
  create_table "authors", :force => true do |t|
    t.string   "name"
    t.datetime "created_at"
    t.datetime "updated_at"
  end
 
  create_table "products", :force => true do |t|
    t.string   "name"
    t.text "description"
    t.datetime "created_at"
    t.datetime "updated_at"
    t.string "part_number"
  end
end
\end{verbatim}

In many ways this is exactly what it is. This file is created by inspecting the database and expressing its structure using \texttt{create\_table}, \texttt{add\_index}, and so on. Because this is database-independent, it could be loaded into any database that Active Record supports. This could be very useful if you were to distribute an application that is able to run against multiple databases.

There is however a trade-off: \texttt{db/schema.rb} cannot express database specific items such as foreign key constraints, triggers, or stored procedures. While in a migration you can execute custom SQL statements, the schema dumper cannot reconstitute those statements from the database. If you are using features like this, then you should set the schema format to \texttt{:sql}.

Instead of using Active Record’s schema dumper, the database’s structure will be dumped using a tool specific to the database (via the \texttt{db:structure:dump} Rake task) into \texttt{db/structure.sql}. For example, for the PostgreSQL RDBMS, the \texttt{pg\_dump} utility is used. For MySQL, this file will contain the output of \textbf{SHOWCREATETABLE} for the various tables. Loading these schemas is simply a question of executing the SQL statements they contain. By definition, this will create a perfect copy of the database’s structure. Using the \texttt{:sql} schema format will, however, prevent loading the schema into a RDBMS other than the one used to create it.

\subsection{ Schema Dumps and Source Control}

Because schema dumps are the authoritative source for your database schema, it is strongly recommended that you check them into source control.

\section{ Active Record and Referential Integrity}

The Active Record way claims that intelligence belongs in your models, not in the database. As such, features such as triggers or foreign key constraints, which push some of that intelligence back into the database, are not heavily used.

Validations such as \texttt{validates :foreign\_key, :uniqueness =$>$ true} are one way in which models can enforce data integrity. The \texttt{:dependent} option on associations allows models to automatically destroy child objects when the parent is destroyed. Like anything which operates at the application level, these cannot guarantee referential integrity and so some people augment them with foreign key constraints in the database.

Although Active Record does not provide any tools for working directly with such features, the \texttt{execute} method can be used to execute arbitrary SQL. You could also use some plugin like \href{https://github.com/matthuhiggins/foreigner}{foreigner} which add foreign key support to Active Record (including support for dumping foreign keys in \texttt{db/schema.rb}).

\chapter{Active Record Validations and Callbacks}

This guide teaches you how to hook into the life cycle of your Active  Record objects. You will learn how to validate the state of objects  before they go into the database, and how to perform custom operations  at certain points in the object life cycle.

After reading this guide and trying out the presented concepts, we hope that you’ll be able to:
\begin{itemize}
	\item Understand the life cycle of Active Record objects
	\item Use the built-in Active Record validation helpers
	\item Create your own custom validation methods
	\item Work with the error messages generated by the validation process
	\item Create callback methods that respond to events in the object life cycle
	\item Create special classes that encapsulate common behavior for your callbacks
	\item Create Observers that respond to life cycle events outside of the original class
\end{itemize}

\section{ The Object Life Cycle}

During the normal operation of a Rails application, objects may be  created, updated, and destroyed. Active Record provides hooks into this \emph{object life cycle} so that you can control your application and its data.

Validations allow you to ensure that only valid data is stored in  your database. Callbacks and observers allow you to trigger logic before  or after an alteration of an object’s state.

\section{ Validations Overview}

Before you dive into the detail of validations in Rails, you should  understand a bit about how validations fit into the big picture.

\subsection{ Why Use Validations?}

Validations are used to ensure that only valid data is saved into  your database. For example, it may be important to your application to  ensure that every user provides a valid email address and mailing  address.

There are several ways to validate data before it is saved into your  database, including native database constraints, client-side  validations, controller-level validations, and model-level validations:
\begin{itemize}
	\item Database constraints and/or stored procedures make the validation  mechanisms database-dependent and can make testing and maintenance more  difficult. However, if your database is used by other applications, it  may be a good idea to use some constraints at the database level.  Additionally, database-level validations can safely handle some things  (such as uniqueness in heavily-used tables) that can be difficult to  implement otherwise.
	\item Client-side validations can be useful, but are generally unreliable  if used alone. If they are implemented using JavaScript, they may be  bypassed if JavaScript is turned off in the user’s browser. However, if  combined with other techniques, client-side validation can be a  convenient way to provide users with immediate feedback as they use your  site.
	\item Controller-level validations can be tempting to use, but often  become unwieldy and difficult to test and maintain. Whenever possible,  it’s a good idea to \href{http://weblog.jamisbuck.org/2006/10/18/skinny-controller-fat-model}{keep your controllers skinny}, as it will make your application a pleasure to work with in the long run.
	\item Model-level validations are the best way to ensure that only valid  data is saved into your database. They are database agnostic, cannot be  bypassed by end users, and are convenient to test and maintain. Rails  makes them easy to use, provides built-in helpers for common needs, and  allows you to create your own validation methods as well.
\end{itemize}

\subsection{ When Does Validation Happen?}

There are two kinds of Active Record objects: those that correspond  to a row inside your database and those that do not. When you create a  fresh object, for example using the \texttt{new} method, that object does not belong to the database yet. Once you call \texttt{save} upon that object it will be saved into the appropriate database table. Active Record uses the \texttt{new\_record?}  instance method to determine whether an object is already in the  database or not. Consider the following simple Active Record class:
\\ \\
\begin{minipage}{\textwidth}{\scriptsize
\begin{verbatim}
class Person < ActiveRecord::Base
end
\end{verbatim}}
\end{minipage}
\\ \\


We can see how it works by looking at some \texttt{rails console} output:
\\ \\
\begin{minipage}{\textwidth}{\scriptsize
\begin{verbatim}
>> p = Person.new(:name => "John Doe")
=> #<Person id: nil, name: "John Doe", created_at: nil, :updated_at: nil>
>> p.new_record?
=> true
>> p.save
=> true
>> p.new_record?
=> false
\end{verbatim}}
\end{minipage}
\\ \\

Creating and saving a new record will send an SQL\texttt{INSERT} operation to the database. Updating an existing record will send an SQL\texttt{UPDATE}  operation instead. Validations are typically run before these commands  are sent to the database. If any validations fail, the object will be  marked as invalid and Active Record will not perform the \texttt{INSERT} or \texttt{UPDATE}  operation. This helps to avoid storing an invalid object in the  database. You can choose to have specific validations run when an object  is created, saved, or updated.

There are many ways to change the state of an  object in the database. Some methods will trigger validations, but some  will not. This means that it’s possible to save an object in the  database in an invalid state if you aren’t careful.

The following methods trigger validations, and will save the object to the database only if the object is valid:
\begin{itemize}
	\item \texttt{create}
	\item \texttt{create!}
	\item \texttt{save}
	\item \texttt{save!}
	\item \texttt{update}
	\item \texttt{update\_attributes}
	\item \texttt{update\_attributes!}
\end{itemize}

The bang versions (e.g. \texttt{save!}) raise an exception if the record is invalid. The non-bang versions don’t: \texttt{save} and \texttt{update\_attributes} return \texttt{false}, \texttt{create} and \texttt{update} just return the objects.

\subsection{ Skipping Validations}

The following methods skip validations, and will save the object to  the database regardless of its validity. They should be used with  caution.
\begin{itemize}
	\item \texttt{decrement!}
	\item \texttt{decrement\_counter}
	\item \texttt{increment!}
	\item \texttt{increment\_counter}
	\item \texttt{toggle!}
	\item \texttt{touch}
	\item \texttt{update\_all}
	\item \texttt{update\_attribute}
	\item \texttt{update\_column}
	\item \texttt{update\_counters}
\end{itemize}

Note that \texttt{save} also has the ability to skip validations if passed \texttt{:validate =$>$ false} as argument. This technique should be used with caution.
\begin{itemize}
	\item \texttt{save(:validate =$>$ false)}
\end{itemize}

\subsection{ \texttt{valid?} and \texttt{invalid?}}

To verify whether or not an object is valid, Rails uses the \texttt{valid?} method. You can also use this method on your own. \texttt{valid?} triggers your validations and returns true if no errors were found in the object, and false otherwise.
\\ \\
\begin{minipage}{\textwidth}{\scriptsize
\begin{verbatim}
class Person < ActiveRecord::Base
  validates :name, :presence => true
end
 
Person.create(:name => "John Doe").valid? # => true
Person.create(:name => nil).valid? # => false
\end{verbatim}}
\end{minipage}
\\ \\

After Active Record has performed validations, any errors found can be accessed through the \texttt{errors}  instance method, which returns a collection of errors. By definition,  an object is valid if this collection is empty after running  validations.

Note that an object instantiated with \texttt{new} will not report errors even if it’s technically invalid, because validations are not run when using \texttt{new}.
\\ \\
\begin{minipage}{\textwidth}{\scriptsize
\begin{verbatim}
class Person < ActiveRecord::Base
  validates :name, :presence => true
end
 
>> p = Person.new
=> #<Person id: nil, name: nil>
>> p.errors
=> {}
 
>> p.valid?
=> false
>> p.errors
=> {:name=>["can't be blank"]}
 
>> p = Person.create
=> #<Person id: nil, name: nil>
>> p.errors
=> {:name=>["can't be blank"]}
 
>> p.save
=> false
 
>> p.save!
=> ActiveRecord::RecordInvalid: Validation failed: Name can't be blank
 
>> Person.create!
=> ActiveRecord::RecordInvalid: Validation failed: Name can't be blank
\end{verbatim}}
\end{minipage}
\\ \\

\texttt{invalid?} is simply the inverse of \texttt{valid?}. \texttt{invalid?} triggers your validations, returning true if any errors were found in the object, and false otherwise.

\subsection{ \texttt{errors[]}}

To verify whether or not a particular attribute of an object is valid, you can use \texttt{errors[:attribute]}. It returns an array of all the errors for \texttt{:attribute}. If there are no errors on the specified attribute, an empty array is returned.

This method is only useful \emph{after} validations have been run,  because it only inspects the errors collection and does not trigger  validations itself. It’s different from the \texttt{ActiveRecord::Base\#invalid?}  method explained above because it doesn’t verify the validity of the  object as a whole. It only checks to see whether there are errors found  on an individual attribute of the object.
\\ \\
\begin{minipage}{\textwidth}{\scriptsize
\begin{verbatim}
class Person < ActiveRecord::Base
  validates :name, :presence => true
end
 
>> Person.new.errors[:name].any? # => false
>> Person.create.errors[:name].any? # => true
\end{verbatim}}
\end{minipage}
\\ \\

We’ll cover validation errors in greater depth in the \hyperlink{working-with-validation-errors}{Working with Validation Errors} section. For now, let’s turn to the built-in validation helpers that Rails provides by default.

\section{ Validation Helpers}

Active Record offers many pre-defined validation helpers that you can  use directly inside your class definitions. These helpers provide  common validation rules. Every time a validation fails, an error message  is added to the object’s \texttt{errors} collection, and this message is associated with the attribute being validated.

Each helper accepts an arbitrary number of attribute names, so with a  single line of code you can add the same kind of validation to several  attributes.

All of them accept the \texttt{:on} and \texttt{:message} options, which define when the validation should be run and what message should be added to the \texttt{errors} collection if it fails, respectively. The \texttt{:on} option takes one of the values \texttt{:save} (the default), \texttt{:create}  or \texttt{:update}. There is a default error message for each one of the validation helpers. These messages are used when the \texttt{:message} option isn’t specified. Let’s take a look at each one of the available helpers.

\subsection{ \texttt{acceptance}}

Validates that a checkbox on the user interface was checked when a  form was submitted. This is typically used when the user needs to agree  to your application’s terms of service, confirm reading some text, or  any similar concept. This validation is very specific to web  applications and this ‘acceptance’ does not need to be recorded anywhere  in your database (if you don’t have a field for it, the helper will  just create a virtual attribute).
\\ \\
\begin{minipage}{\textwidth}{\scriptsize
\begin{verbatim}
class Person < ActiveRecord::Base
  validates :terms_of_service, :acceptance => true
end
\end{verbatim}}
\end{minipage}
\\ \\

The default error message for this helper is “\emph{must be accepted}”.

It can receive an \texttt{:accept} option, which determines the value that will be considered acceptance. It defaults to “1” and can be easily changed.
\\ \\
\begin{minipage}{\textwidth}{\scriptsize
\begin{verbatim}
class Person < ActiveRecord::Base
  validates :terms_of_service, :acceptance => { :accept => 'yes' }
end
\end{verbatim}}
\end{minipage}
\\ \\

\subsection{ \texttt{validates\_associated}}

You should use this helper when your model has associations with  other models and they also need to be validated. When you try to save  your object, \texttt{valid?} will be called upon each one of the associated objects.
\\ \\
\begin{minipage}{\textwidth}{\scriptsize
\begin{verbatim}
class Library < ActiveRecord::Base
  has_many :books
  validates_associated :books
end
\end{verbatim}}
\end{minipage}
\\ \\

This validation will work with all of the association types.

Don’t use \texttt{validates\_associated} on both ends of your associations. They would call each other in an infinite loop.

The default error message for \texttt{validates\_associated} is “\emph{is invalid}”. Note that each associated object will contain its own \texttt{errors} collection; errors do not bubble up to the calling model.

\subsection{ \texttt{confirmation}}

You should use this helper when you have two text fields that should  receive exactly the same content. For example, you may want to confirm  an email address or a password. This validation creates a virtual  attribute whose name is the name of the field that has to be confirmed  with “\_confirmation” appended.
\\ \\
\begin{minipage}{\textwidth}{\scriptsize
\begin{verbatim}
class Person < ActiveRecord::Base
  validates :email, :confirmation => true
end
\end{verbatim}}
\end{minipage}
\\ \\

In your view template you could use something like
\\ \\
\begin{minipage}{\textwidth}{\scriptsize
\begin{verbatim}
<%= text_field :person, :email %>
<%= text_field :person, :email_confirmation %>
\end{verbatim}}
\end{minipage}
\\ \\

This check is performed only if \texttt{email\_confirmation} is not \texttt{nil}. To require confirmation, make sure to add a presence check for the confirmation attribute (we’ll take a look at \texttt{presence} later on this guide):
\\ \\
\begin{minipage}{\textwidth}{\scriptsize
\begin{verbatim}
class Person < ActiveRecord::Base
  validates :email, :confirmation => true
  validates :email_confirmation, :presence => true
end
\end{verbatim}}
\end{minipage}
\\ \\

The default error message for this helper is “\emph{doesn’t match confirmation}”.

\subsection{ \texttt{exclusion}}

This helper validates that the attributes’ values are not included in  a given set. In fact, this set can be any enumerable object.
\\ \\
\begin{minipage}{\textwidth}{\scriptsize
\begin{verbatim}
class Account < ActiveRecord::Base
  validates :subdomain, :exclusion => { :in => %w(www us ca jp),
    :message => "Subdomain %{value} is reserved." }
end
\end{verbatim}}
\end{minipage}
\\ \\

The \texttt{exclusion} helper has an option \texttt{:in} that receives the set of values that will not be accepted for the validated attributes. The \texttt{:in} option has an alias called \texttt{:within} that you can use for the same purpose, if you’d like to. This example uses the \texttt{:message} option to show how you can include the attribute’s value.

The default error message is “\emph{is reserved}”.

\subsection{ \texttt{format}}

This helper validates the attributes’ values by testing whether they  match a given regular expression, which is specified using the \texttt{:with} option.
\\ \\
\begin{minipage}{\textwidth}{\scriptsize
\begin{verbatim}
class Product < ActiveRecord::Base
  validates :legacy_code, :format => { :with => /\A[a-zA-Z]+\z/,
    :message => "Only letters allowed" }
end
\end{verbatim}}
\end{minipage}
\\ \\
The default error message is “\emph{is invalid}”.

\subsection{ \texttt{inclusion}}

This helper validates that the attributes’ values are included in a given set. In fact, this set can be any enumerable object.
\\ \\
\begin{minipage}{\textwidth}{\scriptsize
\begin{verbatim}
class Coffee < ActiveRecord::Base
  validates :size, :inclusion => { :in => %w(small medium large),
    :message => "%{value} is not a valid size" }
end
\end{verbatim}}
\end{minipage}
\\ \\

The \texttt{inclusion} helper has an option \texttt{:in} that receives the set of values that will be accepted. The \texttt{:in} option has an alias called \texttt{:within} that you can use for the same purpose, if you’d like to. The previous example uses the \texttt{:message} option to show how you can include the attribute’s value.

The default error message for this helper is “\emph{is not included in the list}”.

\subsection{ \texttt{length}}

This helper validates the length of the attributes’ values. It  provides a variety of options, so you can specify length constraints in  different ways:
\\ \\
\begin{minipage}{\textwidth}{\scriptsize
\begin{verbatim}
class Person < ActiveRecord::Base
  validates :name, :length => { :minimum => 2 }
  validates :bio, :length => { :maximum => 500 }
  validates :password, :length => { :in => 6..20 }
  validates :registration_number, :length => { :is => 6 }
end
\end{verbatim}}
\end{minipage}
\\ \\

The possible length constraint options are:
\begin{itemize}
	\item \texttt{:minimum} – The attribute cannot have less than the specified length.
	\item \texttt{:maximum} – The attribute cannot have more than the specified length.
	\item \texttt{:in} (or \texttt{:within}) – The attribute length must be included in a given interval. The value for this option must be a range.
	\item \texttt{:is} – The attribute length must be equal to the given value.
\end{itemize}

The default error messages depend on the type of length validation  being performed. You can personalize these messages using the \texttt{:wrong\_length}, \texttt{:too\_long}, and \texttt{:too\_short} options and \texttt{\%\{count\}} as a placeholder for the number corresponding to the length constraint being used. You can still use the \texttt{:message} option to specify an error message.
\\ \\
\begin{minipage}{\textwidth}{\scriptsize
\begin{verbatim}
class Person < ActiveRecord::Base
  validates :bio, :length => { :maximum => 1000,
    :too_long => "%{count} characters is the maximum allowed" }
end
\end{verbatim}}
\end{minipage}
\\ \\

This helper counts characters by default, but you can split the value in a different way using the \texttt{:tokenizer} option:
\\ \\
\begin{minipage}{\textwidth}{\scriptsize
\begin{verbatim}
class Essay < ActiveRecord::Base
  validates :content, :length => {
    :minimum   => 300,
    :maximum   => 400,
    :tokenizer => lambda { |str| str.scan(/\w+/) },
    :too_short => "must have at least %{count} words",
    :too_long  => "must have at most %{count} words"
  }
end
\end{verbatim}}
\end{minipage}
\\ \\

Note that the default error messages are plural (e.g., “is too short (minimum is \%\{count\} characters)”). For this reason, when \texttt{:minimum} is 1 you should provide a personalized message or use \texttt{validates\_presence\_of} instead. When \texttt{:in} or \texttt{:within} have a lower limit of 1, you should either provide a personalized message or call \texttt{presence} prior to \texttt{length}.

The \texttt{size} helper is an alias for \texttt{length}.

\subsection{ \texttt{numericality}}

This helper validates that your attributes have only numeric values.  By default, it will match an optional sign followed by an integral or  floating point number. To specify that only integral numbers are allowed  set \texttt{:only\_integer} to true.

If you set \texttt{:only\_integer} to \texttt{true}, then it will use the
\\ \\
\begin{minipage}{\textwidth}{\scriptsize
\begin{verbatim}
/\A[+-]?\d+\Z/
\end{verbatim}}
\end{minipage}
\\ \\

regular expression to validate the attribute’s value. Otherwise, it will try to convert the value to a number using \texttt{Float}.

Note that the regular expression above allows a trailing newline character.
\\ \\
\begin{minipage}{\textwidth}{\scriptsize
\begin{verbatim}
class Player < ActiveRecord::Base
  validates :points, :numericality => true
  validates :games_played, :numericality => { :only_integer => true }
end
\end{verbatim}}
\end{minipage}
\\ \\

Besides \texttt{:only\_integer}, this helper also accepts the following options to add constraints to acceptable values:
\begin{itemize}
	\item \texttt{:greater\_than} – Specifies the value must be greater than the supplied value. The default error message for this option is “\emph{must be greater than \%\{count\}}”.
	\item \texttt{:greater\_than\_or\_equal\_to} – Specifies the value must be greater than or equal to the supplied value. The default error message for this option is “\emph{must be greater than or equal to \%\{count\}}”.
	\item \texttt{:equal\_to} – Specifies the value must be equal to the supplied value. The default error message for this option is “\emph{must be equal to \%\{count\}}”.
	\item \texttt{:less\_than} – Specifies the value must be less than the supplied value. The default error message for this option is “\emph{must be less than \%\{count\}}”.
	\item \texttt{:less\_than\_or\_equal\_to} – Specifies the value must be less than or equal the supplied value. The default error message for this option is “\emph{must be less than or equal to \%\{count\}}”.
	\item \texttt{:odd} – Specifies the value must be an odd number if set to true. The default error message for this option is “\emph{must be odd}”.
	\item \texttt{:even} – Specifies the value must be an even number if set to true. The default error message for this option is “\emph{must be even}”.
\end{itemize}

The default error message is “\emph{is not a number}”.

\subsection{ \texttt{presence}}

This helper validates that the specified attributes are not empty. It uses the \texttt{blank?} method to check if the value is either \texttt{nil} or a blank string, that is, a string that is either empty or consists of whitespace.
\\ \\
\begin{minipage}{\textwidth}{\scriptsize
\begin{verbatim}
class Person < ActiveRecord::Base
  validates :name, :login, :email, :presence => true
end
\end{verbatim}}
\end{minipage}
\\ \\

If you want to be sure that an association is present, you’ll need to  test whether the foreign key used to map the association is present,  and not the associated object itself.
\\ \\
\begin{minipage}{\textwidth}{\scriptsize
\begin{verbatim}
class LineItem < ActiveRecord::Base
  belongs_to :order
  validates :order_id, :presence => true
end
\end{verbatim}}
\end{minipage}
\\ \\

Since \texttt{false.blank?} is true, if you want to validate the presence of a boolean field you should use \texttt{validates :field\_name, :inclusion =$>$ \{ :in =$>$ [true, false] \}}.

The default error message is “\emph{can’t be empty}”.

\subsection{ \texttt{uniqueness}}

This helper validates that the attribute’s value is unique right  before the object gets saved. It does not create a uniqueness constraint  in the database, so it may happen that two different database  connections create two records with the same value for a column that you  intend to be unique. To avoid that, you must create a unique index in  your database.
\\ \\
\begin{minipage}{\textwidth}{\scriptsize
\begin{verbatim}
class Account < ActiveRecord::Base
  validates :email, :uniqueness => true
end
\end{verbatim}}
\end{minipage}
\\ \\

The validation happens by performing an SQL query into the model’s table, searching for an existing record with the same value in that attribute.

There is a \texttt{:scope} option that you can use to specify other attributes that are used to limit the uniqueness check:
\\ \\
\begin{minipage}{\textwidth}{\scriptsize
\begin{verbatim}
class Holiday < ActiveRecord::Base
  validates :name, :uniqueness => { :scope => :year,
    :message => "should happen once per year" }
end
\end{verbatim}}
\end{minipage}
\\ \\

There is also a \texttt{:case\_sensitive} option that you can use to  define whether the uniqueness constraint will be case sensitive or not.  This option defaults to true.
\\ \\
\begin{minipage}{\textwidth}{\scriptsize
\begin{verbatim}
class Person < ActiveRecord::Base
  validates :name, :uniqueness => { :case_sensitive => false }
end
\end{verbatim}}
\end{minipage}
\\ \\

Note that some databases are configured to perform case-insensitive searches anyway.

The default error message is “\emph{has already been taken}”.

\subsection{ \texttt{validates\_with}}

This helper passes the record to a separate class for validation.
\\ \\
\begin{minipage}{\textwidth}{\scriptsize
\begin{verbatim}
class Person < ActiveRecord::Base
  validates_with GoodnessValidator
end
 
class GoodnessValidator < ActiveModel::Validator
  def validate(record)
    if record.first_name == "Evil"
      record.errors[:base] << "This person is evil"
    end
  end
end
\end{verbatim}}
\end{minipage}
\\ \\

Errors added to \texttt{record.errors[:base]} relate to the state of the record as a whole, and not to a specific attribute.

The \texttt{validates\_with} helper takes a class, or a list of classes to use for validation. There is no default error message for \texttt{validates\_with}. You must manually add errors to the record’s errors collection in the validator class.

To implement the validate method, you must have a \texttt{record} parameter defined, which is the record to be validated.

Like all other validations, \texttt{validates\_with} takes the \texttt{:if}, \texttt{:unless} and \texttt{:on} options. If you pass any other options, it will send those options to the validator class as \texttt{options}:
\\ \\
\begin{minipage}{\textwidth}{\scriptsize
\begin{verbatim}
class Person < ActiveRecord::Base
  validates_with GoodnessValidator, :fields => [:first_name, :last_name]
end
 
class GoodnessValidator < ActiveModel::Validator
  def validate(record)
    if options[:fields].any?{|field| record.send(field) == "Evil" }
      record.errors[:base] << "This person is evil"
    end
  end
end
\end{verbatim}}
\end{minipage}
\\ \\
\subsection{ \texttt{validates\_each}}

This helper validates attributes against a block. It doesn’t have a  predefined validation function. You should create one using a block, and  every attribute passed to \texttt{validates\_each} will be tested against it. In the following example, we don’t want names and surnames to begin with lower case.
\\ \\
\begin{minipage}{\textwidth}{\scriptsize
\begin{verbatim}
class Person < ActiveRecord::Base
  validates_each :name, :surname do |record, attr, value|
record.errors.add(attr, 'must start with upper case') if value =~ /\A[a-z]/
  end
end
\end{verbatim}}
\end{minipage}
\\ \\

The block receives the record, the attribute’s name and the  attribute’s value. You can do anything you like to check for valid data  within the block. If your validation fails, you should add an error  message to the model, therefore making it invalid.

\section{ Common Validation Options}

These are common validation options:

\subsection{ \texttt{:allow\_nil}}

The \texttt{:allow\_nil} option skips the validation when the value being validated is \texttt{nil}.
\\ \\
\begin{minipage}{\textwidth}{\scriptsize
\begin{verbatim}
class Coffee < ActiveRecord::Base
  validates :size, :inclusion => { :in => %w(small medium large),
    :message => "%{value} is not a valid size" }, :allow_nil => true
end
\end{verbatim}}
\end{minipage}
\\ \\

\texttt{:allow\_nil} is ignored by the presence validator.

\subsection{ \texttt{:allow\_blank}}

The \texttt{:allow\_blank} option is similar to the \texttt{:allow\_nil} option. This option will let validation pass if the attribute’s value is \texttt{blank?}, like \texttt{nil} or an empty string for example.
\\ \\
\begin{minipage}{\textwidth}{\scriptsize
\begin{verbatim}
class Topic < ActiveRecord::Base
  validates :title, :length => { :is => 5 }, :allow_blank => true
end
 
Topic.create("title" => "").valid?  # => true
Topic.create("title" => nil).valid? # => true
\end{verbatim}}
\end{minipage}
\\ \\

\texttt{:allow\_blank} is ignored by the presence validator.

\subsection{ \texttt{:message}}

As you’ve already seen, the \texttt{:message} option lets you specify the message that will be added to the \texttt{errors}  collection when validation fails. When this option is not used, Active  Record will use the respective default error message for each validation  helper.

\subsection{ \texttt{:on}}

The \texttt{:on} option lets you specify when the validation should  happen. The default behavior for all the built-in validation helpers is  to be run on save (both when you’re creating a new record and when  you’re updating it). If you want to change it, you can use \texttt{:on =$>$ :create} to run the validation only when a new record is created or \texttt{:on =$>$ :update} to run the validation only when a record is updated.
\\ \\
\begin{minipage}{\textwidth}{\scriptsize
\begin{verbatim}
class Person < ActiveRecord::Base
  # it will be possible to update email with a duplicated value
  validates :email, :uniqueness => true, :on => :create
 
  # it will be possible to create the record with a non-numerical age
  validates :age, :numericality => true, :on => :update
 
  # the default (validates on both create and update)
  validates :name, :presence => true, :on => :save
end
\end{verbatim}}
\end{minipage}
\\ \\

\section{ Conditional Validation}

Sometimes it will make sense to validate an object just when a given predicate is satisfied. You can do that by using the \texttt{:if} and \texttt{:unless} options, which can take a symbol, a string or a \texttt{Proc}. You may use the \texttt{:if} option when you want to specify when the validation \textbf{should} happen. If you want to specify when the validation \textbf{should not} happen, then you may use the \texttt{:unless} option.

\subsection{ Using a Symbol with \texttt{:if} and \texttt{:unless}}

You can associate the \texttt{:if} and \texttt{:unless} options with a  symbol corresponding to the name of a method that will get called right  before validation happens. This is the most commonly used option.
\\ \\
\begin{minipage}{\textwidth}{\scriptsize
\begin{verbatim}
class Order < ActiveRecord::Base
  validates :card_number, :presence => true, :if => :paid_with_card?
 
  def paid_with_card?
    payment_type == "card"
  end
end
\end{verbatim}}
\end{minipage}
\\ \\

\subsection{ Using a String with \texttt{:if} and \texttt{:unless}}

You can also use a string that will be evaluated using \texttt{eval} and needs to contain valid Ruby code. You should use this option only when the string represents a really short condition.
\\ \\
\begin{minipage}{\textwidth}{\scriptsize
\begin{verbatim}
class Person < ActiveRecord::Base
  validates :surname, :presence => true, :if => "name.nil?"
end
\end{verbatim}}
\end{minipage}
\\ \\

\subsection{ Using a Proc with \texttt{:if} and \texttt{:unless}}

Finally, it’s possible to associate \texttt{:if} and \texttt{:unless} with a \texttt{Proc} object which will be called. Using a \texttt{Proc}  object gives you the ability to write an inline condition instead of a  separate method. This option is best suited for one-liners.
\\ \\
\begin{minipage}{\textwidth}{\scriptsize
\begin{verbatim}
class Account < ActiveRecord::Base
  validates :password, :confirmation => true,
    :unless => Proc.new { |a| a.password.blank? }
end
\end{verbatim}}
\end{minipage}
\\ \\

\subsection{ Grouping conditional validations}

Sometimes it is useful to have multiple validations use one condition, it can be easily achieved using \texttt{with\_options}.
\\ \\
\begin{minipage}{\textwidth}{\scriptsize
\begin{verbatim}
class User < ActiveRecord::Base
  with_options :if => :is_admin? do |admin|
    admin.validates :password, :length => { :minimum => 10 }
    admin.validates :email, :presence => true
  end
end
\end{verbatim}}
\end{minipage}
\\ \\

All validations inside of \texttt{with\_options} block will have automatically passed the condition \texttt{:if =$>$ :is\_admin?}

\section{ Performing Custom Validations}

When the built-in validation helpers are not enough for your needs,  you can write your own validators or validation methods as you prefer.

\subsection{ Custom Validators}

Custom validators are classes that extend \texttt{ActiveModel::Validator}. These classes must implement a \texttt{validate} method which takes a record as an argument and performs the validation on it. The custom validator is called using the \texttt{validates\_with} method.
\\ \\
\begin{minipage}{\textwidth}{\scriptsize
\begin{verbatim}
class MyValidator < ActiveModel::Validator
  def validate(record)
    unless record.name.starts_with? 'X'
      record.errors[:name] << 'Need a name starting with X please!'
    end
  end
end
 
class Person
  include ActiveModel::Validations
  validates_with MyValidator
end
\end{verbatim}}
\end{minipage}
\\ \\

The easiest way to add custom validators for validating individual attributes is with the convenient \texttt{ActiveModel::EachValidator}. In this case, the custom validator class must implement a \texttt{validate\_each}  method which takes three arguments: record, attribute and value which  correspond to the instance, the attribute to be validated and the value  of the attribute in the passed instance.
\\ \\
\begin{minipage}{\textwidth}{\scriptsize
\begin{verbatim}
class EmailValidator < ActiveModel::EachValidator
  def validate_each(record, attribute, value)
    unless value =~ /\A([^@\s]+)@((?:[-a-z0-9]+\.)+[a-z]{2,})\z/i
      record.errors[attribute] << (options[:message] || "is not an email")
    end
  end
end
 
class Person < ActiveRecord::Base
  validates :email, :presence => true, :email => true
end
\end{verbatim}}
\end{minipage}
\\ \\

As shown in the example, you can also combine standard validations with your own custom validators.

\subsection{ Custom Methods}

You can also create methods that verify the state of your models and add messages to the \texttt{errors} collection when they are invalid. You must then register these methods by using the \texttt{validate} class method, passing in the symbols for the validation methods’ names.

You can pass more than one symbol for each class method and the  respective validations will be run in the same order as they were  registered.
\\ \\
\begin{minipage}{\textwidth}{\scriptsize
\begin{verbatim}
class Invoice < ActiveRecord::Base
  validate :expiration_date_cannot_be_in_the_past,
    :discount_cannot_be_greater_than_total_value
 
  def expiration_date_cannot_be_in_the_past
    if !expiration_date.blank? and expiration_date < Date.today
      errors.add(:expiration_date, "can't be in the past")
    end
  end
 
  def discount_cannot_be_greater_than_total_value
    if discount > total_value
      errors.add(:discount, "can't be greater than total value")
    end
  end
end
\end{verbatim}}
\end{minipage}
\\ \\

By default such validations will run every time you call \texttt{valid?}. It is also possible to control when to run these custom validations by giving an \texttt{:on} option to the \texttt{validate} method, with either: \texttt{:create} or \texttt{:update}.
\\ \\
\begin{minipage}{\textwidth}{\scriptsize
\begin{verbatim}
class Invoice < ActiveRecord::Base
  validate :active_customer, :on => :create
 
  def active_customer
    errors.add(:customer_id, "is not active") unless customer.active?
  end
end
\end{verbatim}}
\end{minipage}
\\ \\

You can even create your own validation helpers and reuse them in  several different models. For example, an application that manages  surveys may find it useful to express that a certain field corresponds  to a set of choices:
\\ \\
\begin{minipage}{\textwidth}{\scriptsize
\begin{verbatim}
ActiveRecord::Base.class_eval do
  def self.validates_as_choice(attr_name, n, options={})
    validates attr_name, :inclusion => { { :in => 1..n }.merge!(options) }
  end
end
\end{verbatim}}
\end{minipage}
\\ \\

Simply reopen \texttt{ActiveRecord::Base} and define a class method like that. You’d typically put this code somewhere in \texttt{config/initializers}. You can use this helper like this:
\\ \\
\begin{minipage}{\textwidth}{\scriptsize
\begin{verbatim}
class Movie < ActiveRecord::Base
  validates_as_choice :rating, 5
end
\end{verbatim}}
\end{minipage}
\\ \\

\section{ Working with Validation Errors}

In addition to the \texttt{valid?} and \texttt{invalid?} methods covered earlier, Rails provides a number of methods for working with the \texttt{errors} collection and inquiring about the validity of objects.

The following is a list of the most commonly used methods. Please refer to the \texttt{ActiveModel::Errors} documentation for a list of all the available methods.

\subsection{ \texttt{errors}}

Returns an instance of the class \texttt{ActiveModel::Errors} (which  behaves like an ordered hash) containing all errors. Each key is the  attribute name and the value is an array of strings with all errors.
\\ \\
\begin{minipage}{\textwidth}{\scriptsize
\begin{verbatim}
class Person < ActiveRecord::Base
  validates :name, :presence => true, :length => { :minimum => 3 }
end
 
person = Person.new
person.valid? # => false
person.errors
 # => {:name => ["can't be blank", "is too short (minimum is 3 characters)"]}
 
person = Person.new(:name => "John Doe")
person.valid? # => true
person.errors # => []
\end{verbatim}}
\end{minipage}
\\ \\

\subsection{ \texttt{errors[]}}

\texttt{errors[]} is used when you want to check the error messages  for a specific attribute. It returns an array of strings with all error  messages for the given attribute, each string with one error message. If  there are no errors related to the attribute, it returns an empty  array.
\\ \\
\begin{minipage}{\textwidth}{\scriptsize
\begin{verbatim}
class Person < ActiveRecord::Base
  validates :name, :presence => true, :length => { :minimum => 3 }
end
 
person = Person.new(:name => "John Doe")
person.valid? # => true
person.errors[:name] # => []
 
person = Person.new(:name => "JD")
person.valid? # => false
person.errors[:name] # => ["is too short (minimum is 3 characters)"]
 
person = Person.new
person.valid? # => false
person.errors[:name]
 # => ["can't be blank", "is too short (minimum is 3 characters)"]
\end{verbatim}}
\end{minipage}
\\ \\

\subsection{ \texttt{errors.add}}

The \texttt{add} method lets you manually add messages that are related to particular attributes. You can use the \texttt{errors.full\_messages} or \texttt{errors.to\_a}  methods to view the messages in the form they might be displayed to a  user. Those particular messages get the attribute name prepended (and  capitalized). \texttt{add} receives the name of the attribute you want to add the message to, and the message itself.
\\ \\
\begin{minipage}{\textwidth}{\scriptsize
\begin{verbatim}
class Person < ActiveRecord::Base
  def a_method_used_for_validation_purposes
    errors.add(:name, "cannot contain the characters !@#%*()_-+=")
  end
end
 
person = Person.create(:name => "!@#")
 
person.errors[:name]
 # => ["cannot contain the characters !@#%*()_-+="]
 
person.errors.full_messages
 # => ["Name cannot contain the characters !@#%*()_-+="]
\end{verbatim}}
\end{minipage}
\\ \\

Another way to do this is using \texttt{[]=} setter
\\ \\
\begin{minipage}{\textwidth}{\scriptsize
\begin{verbatim}
class Person < ActiveRecord::Base
    def a_method_used_for_validation_purposes
      errors[:name] = "cannot contain the characters !@#%*()_-+="
    end
  end
 
  person = Person.create(:name => "!@#")
 
  person.errors[:name]
   # => ["cannot contain the characters !@#%*()_-+="]
 
  person.errors.to_a
   # => ["Name cannot contain the characters !@#%*()_-+="]
\end{verbatim}}
\end{minipage}
\\ \\

\subsection{ \texttt{errors[:base]}}

You can add error messages that are related to the object’s state as a  whole, instead of being related to a specific attribute. You can use  this method when you want to say that the object is invalid, no matter  the values of its attributes. Since \texttt{errors[:base]} is an array, you can simply add a string to it and it will be used as an error message.
\\ \\
\begin{minipage}{\textwidth}{\scriptsize
\begin{verbatim}
class Person < ActiveRecord::Base
  def a_method_used_for_validation_purposes
    errors[:base] << "This person is invalid because ..."
  end
end
\end{verbatim}}
\end{minipage}
\\ \\

\subsection{ \texttt{errors.clear}}

The \texttt{clear} method is used when you intentionally want to clear all the messages in the \texttt{errors} collection. Of course, calling \texttt{errors.clear} upon an invalid object won’t actually make it valid: the \texttt{errors} collection will now be empty, but the next time you call \texttt{valid?}  or any method that tries to save this object to the database, the  validations will run again. If any of the validations fail, the \texttt{errors} collection will be filled again.
\\ \\
\begin{minipage}{\textwidth}{\scriptsize
\begin{verbatim}
class Person < ActiveRecord::Base
  validates :name, :presence => true, :length => { :minimum => 3 }
end
 
person = Person.new
person.valid? # => false
person.errors[:name]
 # => ["can't be blank", "is too short (minimum is 3 characters)"]
 
person.errors.clear
person.errors.empty? # => true
 
p.save # => false
 
p.errors[:name]
 # => ["can't be blank", "is too short (minimum is 3 characters)"]
\end{verbatim}}
\end{minipage}
\\ \\

\subsection{ \texttt{errors.size}}

The \texttt{size} method returns the total number of error messages for the object.
\\ \\
\begin{minipage}{\textwidth}{\scriptsize
\begin{verbatim}
class Person < ActiveRecord::Base
  validates :name, :presence => true, :length => { :minimum => 3 }
end
 
person = Person.new
person.valid? # => false
person.errors.size # => 2
 
person = Person.new(:name => "Andrea", :email => "andrea@example.com")
person.valid? # => true
person.errors.size # => 0
\end{verbatim}}
\end{minipage}
\\ \\

\section{ Displaying Validation Errors in the View}

\href{https://github.com/joelmoss/dynamic_form}{DynamicForm} provides helpers to display the error messages of your models in your view templates.

You can install it as a gem by adding this line to your Gemfile:
\\ \\
\begin{minipage}{\textwidth}{\scriptsize
\begin{verbatim}
gem "dynamic_form"
\end{verbatim}}
\end{minipage}
\\ \\

Now you will have access to the two helper methods \texttt{error\_messages} and \texttt{error\_messages\_for} in your view templates.

\subsection{ \texttt{error\_messages} and \texttt{error\_messages\_for}}

When creating a form with the \texttt{form\_for} helper, you can use the \texttt{error\_messages} method on the form builder to render all failed validation messages for the current model instance.
\\ \\
\begin{minipage}{\textwidth}{\scriptsize
\begin{verbatim}
class Product < ActiveRecord::Base
  validates :description, :value, :presence => true
  validates :value, :numericality => true, :allow_nil => true
end
\end{verbatim}}
\end{minipage}
\\ \\
\\ \\
\begin{minipage}{\textwidth}{\scriptsize
\begin{verbatim}
<%= form_for(@product) do |f| %>
  <%= f.error_messages %>
  <p>
    <%= f.label :description %><br />
    <%= f.text_field :description %>
  </p>
  <p>
    <%= f.label :value %><br />
    <%= f.text_field :value %>
  </p>
  <p>
    <%= f.submit "Create" %>
  </p>
<% end %>
\end{verbatim}}
\end{minipage}
\\ \\

If you submit the form with empty fields, the result will be similar to the one shown below:


\includegraphics[width=\textwidth]{../error_messages.png}

The appearance of the generated HTML will be different from the one shown, unless you have used scaffolding. See \hyperlink{customizing-error-messages-css}{Customizing the Error Messages CSS}.

You can also use the \texttt{error\_messages\_for} helper to display  the error messages of a model assigned to a view template. It is very  similar to the previous example and will achieve exactly the same  result.
\\ \\
\begin{minipage}{\textwidth}{\scriptsize
\begin{verbatim}
<%= error_messages_for :product %>
\end{verbatim}}
\end{minipage}
\\ \\

The displayed text for each error message will always be formed by  the capitalized name of the attribute that holds the error, followed by  the error message itself.

Both the \texttt{form.error\_messages} and the \texttt{error\_messages\_for} helpers accept options that let you customize the \texttt{div}  element that holds the messages, change the header text, change the  message below the header, and specify the tag used for the header  element. For example,
\\ \\
\begin{minipage}{\textwidth}{\scriptsize
\begin{verbatim}
<%= f.error_messages :header_message => "Invalid product!",
  :message => "You'll need to fix the following fields:",
  :header_tag => :h3 %>
\end{verbatim}}
\end{minipage}
\\ \\

results in:


\includegraphics[width=\textwidth]{../customized_error_messages.png}

If you pass \texttt{nil} in any of these options, the corresponding section of the \texttt{div} will be discarded.

\subsection{ Customizing the Error Messages CSS}

The selectors used to customize the style of error messages are:
\begin{itemize}
	\item \texttt{.field\_with\_errors} – Style for the form fields and labels with errors.
	\item \texttt{\#error\_explanation} – Style for the \texttt{div} element with the error messages.
	\item \texttt{\#error\_explanation h2} – Style for the header of the \texttt{div} element.
	\item \texttt{\#error\_explanation p} – Style for the paragraph holding the message that appears right below the header of the \texttt{div} element.
	\item \texttt{\#error\_explanation ul li} – Style for the list items with individual error messages.
\end{itemize}

If scaffolding was used, file \texttt{app/assets/stylesheets/scaffolds.css.scss} will have been generated automatically. This file defines the red-based styles you saw in the examples above.

The name of the class and the id can be changed with the \texttt{:class} and \texttt{:id} options, accepted by both helpers.

\subsection{ Customizing the Error Messages HTML}

By default, form fields with errors are displayed enclosed by a \texttt{div} element with the \texttt{field\_with\_errors}CSS class. However, it’s possible to override that.

The way form fields with errors are treated is defined by \texttt{ActionView::Base.field\_error\_proc}. This is a \texttt{Proc} that receives two parameters:
\begin{itemize}
	\item A string with the HTML tag
	\item An instance of \texttt{ActionView::Helpers::InstanceTag}.
\end{itemize}

Below is a simple example where we change the Rails behavior to  always display the error messages in front of each of the form fields in  error. The error messages will be enclosed by a \texttt{span} element with a \texttt{validation-error}CSS class. There will be no \texttt{div} element enclosing the \texttt{input} element, so we get rid of that red border around the text field. You can use the \texttt{validation-error}CSS class to style it anyway you want.
\\ \\
\begin{minipage}{\textwidth}{\scriptsize
\begin{verbatim}
ActionView::Base.field_error_proc = Proc.new do |html_tag, instance|
errors = Array(instance.error_message).join(',')
%(#{html_tag}<span class="validation-error">&nbsp;#{errors}</span>).html_safe
end
\end{verbatim}}
\end{minipage}
\\ \\

The result looks like the following:


\includegraphics[width=\textwidth]{../validation_error_messages.png}

\section{ Callbacks Overview}

Callbacks are methods that get called at certain moments of an  object’s life cycle. With callbacks it is possible to write code that  will run whenever an Active Record object is created, saved, updated,  deleted, validated, or loaded from the database.

\subsection{ Callback Registration}

In order to use the available callbacks, you need to register them.  You can implement the callbacks as ordinary methods and use a  macro-style class method to register them as callbacks:
\\ \\
\begin{minipage}{\textwidth}{\scriptsize
\begin{verbatim}
class User < ActiveRecord::Base
  validates :login, :email, :presence => true
 
  before_validation :ensure_login_has_a_value
 
  protected
  def ensure_login_has_a_value
    if login.nil?
      self.login = email unless email.blank?
    end
  end
end
\end{verbatim}}
\end{minipage}
\\ \\

The macro-style class methods can also receive a block. Consider  using this style if the code inside your block is so short that it fits  in a single line:
\\ \\
\begin{minipage}{\textwidth}{\scriptsize
\begin{verbatim}
class User < ActiveRecord::Base
  validates :login, :email, :presence => true
 
  before_create do |user|
    user.name = user.login.capitalize if user.name.blank?
  end
end
\end{verbatim}}
\end{minipage}
\\ \\

It is considered good practice to declare callback methods as  protected or private. If left public, they can be called from outside of  the model and violate the principle of object encapsulation.

\section{ Available Callbacks}

Here is a list with all the available Active Record callbacks, listed  in the same order in which they will get called during the respective  operations:

\subsection{ Creating an Object}
\begin{itemize}
	\item \texttt{before\_validation}
	\item \texttt{after\_validation}
	\item \texttt{before\_save}
	\item \texttt{around\_save}
	\item \texttt{before\_create}
	\item \texttt{around\_create}
	\item \texttt{after\_create}
	\item \texttt{after\_save}
\end{itemize}

\subsection{ Updating an Object}
\begin{itemize}
	\item \texttt{before\_validation}
	\item \texttt{after\_validation}
	\item \texttt{before\_save}
	\item \texttt{around\_save}
	\item \texttt{before\_update}
	\item \texttt{around\_update}
	\item \texttt{after\_update}
	\item \texttt{after\_save}
\end{itemize}

\subsection{ Destroying an Object}
\begin{itemize}
	\item \texttt{before\_destroy}
	\item \texttt{around\_destroy}
	\item \texttt{after\_destroy}
\end{itemize}

\texttt{after\_save} runs both on create and update, but always \emph{after} the more specific callbacks \texttt{after\_create} and \texttt{after\_update}, no matter the order in which the macro calls were executed.

\subsection{ \texttt{after\_initialize} and \texttt{after\_find}}

The \texttt{after\_initialize} callback will be called whenever an Active Record object is instantiated, either by directly using \texttt{new} or when a record is loaded from the database. It can be useful to avoid the need to directly override your Active Record \texttt{initialize} method.

The \texttt{after\_find} callback will be called whenever Active Record loads a record from the database. \texttt{after\_find} is called before \texttt{after\_initialize} if both are defined.

The \texttt{after\_initialize} and \texttt{after\_find} callbacks have no \texttt{before\_*} counterparts, but they can be registered just like the other Active Record callbacks.
\\ \\
\begin{minipage}{\textwidth}{\scriptsize
\begin{verbatim}
class User < ActiveRecord::Base
  after_initialize do |user|
    puts "You have initialized an object!"
  end
 
  after_find do |user|
    puts "You have found an object!"
  end
end
 
>> User.new
You have initialized an object!
=> #<User id: nil>
 
>> User.first
You have found an object!
You have initialized an object!
=> #<User id: 1>
\end{verbatim}}
\end{minipage}
\\ \\

\section{ Running Callbacks}

The following methods trigger callbacks:
\begin{itemize}
	\item \texttt{create}
	\item \texttt{create!}
	\item \texttt{decrement!}
	\item \texttt{destroy}
	\item \texttt{destroy\_all}
	\item \texttt{increment!}
	\item \texttt{save}
	\item \texttt{save!}
	\item \texttt{save(:validate =$>$ false)}
	\item \texttt{toggle!}
	\item \texttt{update}
	\item \texttt{update\_attribute}
	\item \texttt{update\_attributes}
	\item \texttt{update\_attributes!}
	\item \texttt{valid?}
\end{itemize}

Additionally, the \texttt{after\_find} callback is triggered by the following finder methods:
\begin{itemize}
	\item \texttt{all}
	\item \texttt{first}
	\item \texttt{find}
	\item \texttt{find\_all\_by\_\emph{attribute}}
	\item \texttt{find\_by\_\emph{attribute}}
	\item \texttt{find\_by\_\emph{attribute}!}
	\item \texttt{last}
\end{itemize}

The \texttt{after\_initialize} callback is triggered every time a new object of the class is initialized.

\section{ Skipping Callbacks}

Just as with validations, it is also possible to skip callbacks.  These methods should be used with caution, however, because important  business rules and application logic may be kept in callbacks. Bypassing  them without understanding the potential implications may lead to  invalid data.
\begin{itemize}
	\item \texttt{decrement}
	\item \texttt{decrement\_counter}
	\item \texttt{delete}
	\item \texttt{delete\_all}
	\item \texttt{find\_by\_sql}
	\item \texttt{increment}
	\item \texttt{increment\_counter}
	\item \texttt{toggle}
	\item \texttt{touch}
	\item \texttt{update\_column}
	\item \texttt{update\_all}
	\item \texttt{update\_counters}
\end{itemize}

\section{ Halting Execution}

As you start registering new callbacks for your models, they will be  queued for execution. This queue will include all your model’s  validations, the registered callbacks, and the database operation to be  executed.

The whole callback chain is wrapped in a transaction. If any \emph{before} callback method returns exactly \texttt{false} or raises an exception, the execution chain gets halted and a ROLLBACK is issued; \emph{after} callbacks can only accomplish that by raising an exception.

Raising an arbitrary exception may break code that expects \texttt{save} and its friends not to fail like that. The \texttt{ActiveRecord::Rollback} exception is thought precisely to tell Active Record a rollback is going on. That one is internally captured but not reraised.

\section{ Relational Callbacks}

Callbacks work through model relationships, and can even be defined  by them. Suppose an example where a user has many posts. A user’s posts  should be destroyed if the user is destroyed. Let’s add an \texttt{after\_destroy} callback to the \texttt{User} model by way of its relationship to the \texttt{Post} model:
\\ \\
\begin{minipage}{\textwidth}{\scriptsize
\begin{verbatim}
class User < ActiveRecord::Base
  has_many :posts, :dependent => :destroy
end
 
class Post < ActiveRecord::Base
  after_destroy :log_destroy_action
 
  def log_destroy_action
    puts 'Post destroyed'
  end
end
 
>> user = User.first
=> #<User id: 1>
>> user.posts.create!
=> #<Post id: 1, user_id: 1>
>> user.destroy
Post destroyed
=> #<User id: 1>
\end{verbatim}}
\end{minipage}
\\ \\

\section{ Conditional Callbacks}

As with validations, we can also make the calling of a callback  method conditional on the satisfaction of a given predicate. We can do  this using the \texttt{:if} and \texttt{:unless} options, which can take a symbol, a string or a \texttt{Proc}. You may use the \texttt{:if} option when you want to specify under which conditions the callback \textbf{should} be called. If you want to specify the conditions under which the callback \textbf{should not} be called, then you may use the \texttt{:unless} option.

\subsection{ Using \texttt{:if} and \texttt{:unless} with a \texttt{Symbol}}

You can associate the \texttt{:if} and \texttt{:unless} options with a  symbol corresponding to the name of a predicate method that will get  called right before the callback. When using the \texttt{:if} option, the callback won’t be executed if the predicate method returns false; when using the \texttt{:unless}  option, the callback won’t be executed if the predicate method returns  true. This is the most common option. Using this form of registration it  is also possible to register several different predicates that should  be called to check if the callback should be executed.
\\ \\
\begin{minipage}{\textwidth}{\scriptsize
\begin{verbatim}
class Order < ActiveRecord::Base
  before_save :normalize_card_number, :if => :paid_with_card?
end
\end{verbatim}}
\end{minipage}
\\ \\

\subsection{ Using \texttt{:if} and \texttt{:unless} with a String}

You can also use a string that will be evaluated using \texttt{eval}  and hence needs to contain valid Ruby code. You should use this option  only when the string represents a really short condition:
\\ \\
\begin{minipage}{\textwidth}{\scriptsize
\begin{verbatim}
class Order < ActiveRecord::Base
  before_save :normalize_card_number, :if => "paid_with_card?"
end
\end{verbatim}}
\end{minipage}
\\ \\

\subsection{ Using \texttt{:if} and \texttt{:unless} with a \texttt{Proc}}

Finally, it is possible to associate \texttt{:if} and \texttt{:unless} with a \texttt{Proc} object. This option is best suited when writing short validation methods, usually one-liners:
\\ \\
\begin{minipage}{\textwidth}{\scriptsize
\begin{verbatim}
class Order < ActiveRecord::Base
  before_save :normalize_card_number,
    :if => Proc.new { |order| order.paid_with_card? }
end
\end{verbatim}}
\end{minipage}
\\ \\

\subsection{ Multiple Conditions for Callbacks}

When writing conditional callbacks, it is possible to mix both \texttt{:if} and \texttt{:unless} in the same callback declaration:
\\ \\
\begin{minipage}{\textwidth}{\scriptsize
\begin{verbatim}
class Comment < ActiveRecord::Base
  after_create :send_email_to_author, :if => :author_wants_emails?,
    :unless => Proc.new { |comment| comment.post.ignore_comments? }
end
\end{verbatim}}
\end{minipage}
\\ \\

\section{ Callback Classes}

Sometimes the callback methods that you’ll write will be useful  enough to be reused by other models. Active Record makes it possible to  create classes that encapsulate the callback methods, so it becomes very  easy to reuse them.

Here’s an example where we create a class with an \texttt{after\_destroy} callback for a \texttt{PictureFile} model:
\\ \\
\begin{minipage}{\textwidth}{\scriptsize
\begin{verbatim}
class PictureFileCallbacks
  def after_destroy(picture_file)
    if File.exists?(picture_file.filepath)
      File.delete(picture_file.filepath)
    end
  end
end
\end{verbatim}}
\end{minipage}
\\ \\

When declared inside a class, as above, the callback methods will  receive the model object as a parameter. We can now use the callback  class in the model:
\\ \\
\begin{minipage}{\textwidth}{\scriptsize
\begin{verbatim}
class PictureFile < ActiveRecord::Base
  after_destroy PictureFileCallbacks.new
end
\end{verbatim}}
\end{minipage}
\\ \\

Note that we needed to instantiate a new \texttt{PictureFileCallbacks}  object, since we declared our callback as an instance method. This is  particularly useful if the callbacks make use of the state of the  instantiated object. Often, however, it will make more sense to declare  the callbacks as class methods:
\\ \\
\begin{minipage}{\textwidth}{\scriptsize
\begin{verbatim}
class PictureFileCallbacks
  def self.after_destroy(picture_file)
    if File.exists?(picture_file.filepath)
      File.delete(picture_file.filepath)
    end
  end
end
\end{verbatim}}
\end{minipage}
\\ \\

If the callback method is declared this way, it won’t be necessary to instantiate a \texttt{PictureFileCallbacks} object.
\\ \\
\begin{minipage}{\textwidth}{\scriptsize
\begin{verbatim}
class PictureFile < ActiveRecord::Base
  after_destroy PictureFileCallbacks
end
\end{verbatim}}
\end{minipage}
\\ \\

You can declare as many callbacks as you want inside your callback classes.

\section{ Observers}

Observers are similar to callbacks, but with important differences.  Whereas callbacks can pollute a model with code that isn’t directly  related to its purpose, observers allow you to add the same  functionality without changing the code of the model. For example, it  could be argued that a \texttt{User} model should not include code to  send registration confirmation emails. Whenever you use callbacks with  code that isn’t directly related to your model, you may want to consider  creating an observer instead.

\subsection{ Creating Observers}

For example, imagine a \texttt{User} model where we want to send an  email every time a new user is created. Because sending emails is not  directly related to our model’s purpose, we should create an observer to  contain the code implementing this functionality.
\\ \\
\begin{minipage}{\textwidth}{\scriptsize
\begin{verbatim}
$ rails generate observer User
\end{verbatim}}
\end{minipage}
\\ \\

generates \texttt{app/models/user\_observer.rb} containing the observer class \texttt{UserObserver}:
\\ \\
\begin{minipage}{\textwidth}{\scriptsize
\begin{verbatim}
class UserObserver < ActiveRecord::Observer
end
\end{verbatim}}
\end{minipage}
\\ \\

You may now add methods to be called at the desired occasions:
\\ \\
\begin{minipage}{\textwidth}{\scriptsize
\begin{verbatim}
class UserObserver < ActiveRecord::Observer
  def after_create(model)
    # code to send confirmation email...
  end
end
\end{verbatim}}
\end{minipage}
\\ \\

As with callback classes, the observer’s methods receive the observed model as a parameter.

\subsection{ Registering Observers}

Observers are conventionally placed inside of your \texttt{app/models} directory and registered in your application’s \texttt{config/application.rb} file. For example, the \texttt{UserObserver} above would be saved as \texttt{app/models/user\_observer.rb} and registered in \texttt{config/application.rb} this way:
\\ \\
\begin{minipage}{\textwidth}{\scriptsize
\begin{verbatim}
# Activate observers that should always be running.
config.active_record.observers = :user_observer
\end{verbatim}}
\end{minipage}
\\ \\

As usual, settings in \texttt{config/environments} take precedence over those in \texttt{config/application.rb}.  So, if you prefer that an observer doesn’t run in all environments, you  can simply register it in a specific environment instead.

\subsection{ Sharing Observers}

By default, Rails will simply strip “Observer” from an observer’s  name to find the model it should observe. However, observers can also be  used to add behavior to more than one model, and thus it is possible to  explicitly specify the models that our observer should observe:
\\ \\
\begin{minipage}{\textwidth}{\scriptsize
\begin{verbatim}
class MailerObserver < ActiveRecord::Observer
  observe :registration, :user
 
  def after_create(model)
    # code to send confirmation email...
  end
end
\end{verbatim}}
\end{minipage}
\\ \\

In this example, the \texttt{after\_create} method will be called whenever a \texttt{Registration} or \texttt{User} is created. Note that this new \texttt{MailerObserver} would also need to be registered in \texttt{config/application.rb} in order to take effect:
\\ \\
\begin{minipage}{\textwidth}{\scriptsize
\begin{verbatim}
# Activate observers that should always be running.
config.active_record.observers = :mailer_observer
\end{verbatim}}
\end{minipage}
\\ \\

\section{ Transaction Callbacks}

There are two additional callbacks that are triggered by the completion of a database transaction: \texttt{after\_commit} and \texttt{after\_rollback}. These callbacks are very similar to the \texttt{after\_save}  callback except that they don’t execute until after database changes  have either been committed or rolled back. They are most useful when  your active record models need to interact with external systems which  are not part of the database transaction.

Consider, for example, the previous example where the \texttt{PictureFile} model needs to delete a file after the corresponding record is destroyed. If anything raises an exception after the \texttt{after\_destroy}  callback is called and the transaction rolls back, the file will have  been deleted and the model will be left in an inconsistent state. For  example, suppose that \texttt{picture\_file\_2} in the code below is not valid and the \texttt{save!} method raises an error.
\\ \\
\begin{minipage}{\textwidth}{\scriptsize
\begin{verbatim}
PictureFile.transaction do
  picture_file_1.destroy
  picture_file_2.save!
end
\end{verbatim}}
\end{minipage}
\\ \\

By using the \texttt{after\_commit} callback we can account for this case.
\\ \\
\begin{minipage}{\textwidth}{\scriptsize
\begin{verbatim}
class PictureFile < ActiveRecord::Base
  attr_accessor :delete_file
 
  after_destroy do |picture_file|
    picture_file.delete_file = picture_file.filepath
  end
 
  after_commit do |picture_file|
    if picture_file.delete_file && File.exist?(picture_file.delete_file)
      File.delete(picture_file.delete_file)
      picture_file.delete_file = nil
    end
  end
end
\end{verbatim}}
\end{minipage}
\\ \\

The \texttt{after\_commit} and \texttt{after\_rollback} callbacks are  guaranteed to be called for all models created, updated, or destroyed  within a transaction block. If any exceptions are raised within one of  these callbacks, they will be ignored so that they don’t interfere with  the other callbacks. As such, if your callback code could raise an  exception, you’ll need to rescue it and handle it appropriately within  the callback.    

\chapter{A Guide to Active Record Associations}

This guide covers the association features of Active Record. By referring to this guide, you will be able to:
\begin{itemize}
	\item Declare associations between Active Record models
	\item Understand the various types of Active Record associations
	\item Use the methods added to your models by creating associations
\end{itemize}

\section{ Why Associations?}

Why do we need associations between models? Because they make common  operations simpler and easier in your code. For example, consider a  simple Rails application that includes a model for customers and a model  for orders. Each customer can have many orders. Without associations,  the model declarations would look like this:
\\ \\
\begin{minipage}{\textwidth}{\scriptsize{
\begin{verbatim}
class Customer < ActiveRecord::Base
end
 
class Order < ActiveRecord::Base
end
\end{verbatim}}
\end{minipage}
\\ \\


Now, suppose we wanted to add a new order for an existing customer. We’d need to do something like this:
\\ \\
\begin{minipage}{\textwidth}{\scriptsize{
\begin{verbatim}
@order = Order.create(:order_date => Time.now,
  :customer_id => @customer.id)
\end{verbatim}}
\end{minipage}
\\ \\

Or consider deleting a customer, and ensuring that all of its orders get deleted as well:
\\ \\
\begin{minipage}{\textwidth}{\scriptsize
\begin{verbatim}
@orders = Order.where(:customer_id => @customer.id)
@orders.each do |order|
  order.destroy
end
@customer.destroy
\end{verbatim}}
\end{minipage}
\\ \\

With Active Record associations, we can streamline these — and other —  operations by declaratively telling Rails that there is a connection  between the two models. Here’s the revised code for setting up customers  and orders:
\\ \\
\begin{minipage}{\textwidth}{\scriptsize
\begin{verbatim}
class Customer < ActiveRecord::Base
  has_many :orders, :dependent => :destroy
end
 
class Order < ActiveRecord::Base
  belongs_to :customer
end
\end{verbatim}}
\end{minipage}
\\ \\

With this change, creating a new order for a particular customer is easier:
\\ \\
\begin{minipage}{\textwidth}{\scriptsize
\begin{verbatim}
@order = @customer.orders.create(:order_date => Time.now)
\end{verbatim}}
\end{minipage}
\\ \\

Deleting a customer and all of its orders is \emph{much} easier:
\\ \\
\begin{minipage}{\textwidth}{\scriptsize
\begin{verbatim}
@customer.destroy
\end{verbatim}}
\end{minipage}
\\ \\

To learn more about the different types of associations, read the  next section of this guide. That’s followed by some tips and tricks for  working with associations, and then by a complete reference to the  methods and options for associations in Rails.

\section{ The Types of Associations}

In Rails, an \emph{association} is a connection between two Active  Record models. Associations are implemented using macro-style calls, so  that you can declaratively add features to your models. For example, by  declaring that one model \texttt{belongs\_to} another, you instruct Rails  to maintain Primary Key–Foreign Key information between instances of  the two models, and you also get a number of utility methods added to  your model. Rails supports six types of associations:
\begin{itemize}
	\item \texttt{belongs\_to}
	\item \texttt{has\_one}
	\item \texttt{has\_many}
	\item \texttt{has\_many :through}
	\item \texttt{has\_one :through}
	\item \texttt{has\_and\_belongs\_to\_many}
\end{itemize}

In the remainder of this guide, you’ll learn how to declare and use  the various forms of associations. But first, a quick introduction to  the situations where each association type is appropriate.

\subsection{ The \texttt{belongs\_to} Association}

A \texttt{belongs\_to} association sets up a one-to-one connection  with another model, such that each instance of the declaring model  “belongs to” one instance of the other model. For example, if your  application includes customers and orders, and each order can be  assigned to exactly one customer, you’d declare the order model this  way:
\\ \\
\begin{minipage}{\textwidth}{\scriptsize
\begin{verbatim}
class Order < ActiveRecord::Base
  belongs_to :customer
end
\end{verbatim}}
\end{minipage}
\\ \\


\includegraphics[width=\textwidth]{../belongs_to.png}

\subsection{ The \texttt{has\_one} Association}

A \texttt{has\_one} association also sets up a one-to-one connection  with another model, but with somewhat different semantics (and  consequences). This association indicates that each instance of a model  contains or possesses one instance of another model. For example, if  each supplier in your application has only one account, you’d declare  the supplier model like this:
\\ \\
\begin{minipage}{\textwidth}{\scriptsize
\begin{verbatim}
class Supplier < ActiveRecord::Base
  has_one :account
end
\end{verbatim}}
\end{minipage}
\\ \\

\includegraphics[width=\textwidth]{../has_one.png}
\newpage
\subsection{ The \texttt{has\_many} Association}

A \texttt{has\_many} association indicates a one-to-many connection with another model. You’ll often find this association on the “other side” of a \texttt{belongs\_to}  association. This association indicates that each instance of the model  has zero or more instances of another model. For example, in an  application containing customers and orders, the customer model could be  declared like this:
\\ \\
\begin{minipage}{\textwidth}{\scriptsize
\begin{verbatim}
class Customer < ActiveRecord::Base
  has_many :orders
end
\end{verbatim}}
\end{minipage}
\\ \\

The name of the other model is pluralized when declaring a \texttt{has\_many} association.

\includegraphics[width=\textwidth]{../has_many.png}

\subsection{ The \texttt{has\_many :through} Association}

A \texttt{has\_many :through} association is often used to set up a  many-to-many connection with another model. This association indicates  that the declaring model can be matched with zero or more instances of  another model by proceeding \emph{through} a third model. For example,  consider a medical practice where patients make appointments to see  physicians. The relevant association declarations could look like this:
\\ \\
\begin{minipage}{\textwidth}{\scriptsize
\begin{verbatim}
class Physician < ActiveRecord::Base
  has_many :appointments
  has_many :patients, :through => :appointments
end
 
class Appointment < ActiveRecord::Base
  belongs_to :physician
  belongs_to :patient
end
 
class Patient < ActiveRecord::Base
  has_many :appointments
  has_many :physicians, :through => :appointments
end
\end{verbatim}}
\end{minipage}
\\ \\

\includegraphics[width=\textwidth]{../has_many_through.png}

The collection of join models can be managed via the API. For example, if you assign
\\ \\
\begin{minipage}{\textwidth}{\scriptsize
\begin{verbatim}
physician.patients = patients
\end{verbatim}}
\end{minipage}
\\ \\
new join models are created for newly associated objects, and if some are gone their rows are deleted.



Automatic deletion of join models is direct, no destroy callbacks are triggered.

The \texttt{has\_many :through} association is also useful for setting up “shortcuts” through nested \texttt{has\_many}  associations. For example, if a document has many sections, and a  section has many paragraphs, you may sometimes want to get a simple  collection of all paragraphs in the document. You could set that up this  way:
\\ \\
\begin{minipage}{\textwidth}{\scriptsize
\begin{verbatim}
class Document < ActiveRecord::Base
  has_many :sections
  has_many :paragraphs, :through => :sections
end
 
class Section < ActiveRecord::Base
  belongs_to :document
  has_many :paragraphs
end
 
class Paragraph < ActiveRecord::Base
  belongs_to :section
end
\end{verbatim}}
\end{minipage}
\\ \\

With \texttt{:through =$>$ :sections} specified, Rails will now understand:
\\ \\
\begin{minipage}{\textwidth}{\scriptsize
\begin{verbatim}
@document.paragraphs
\end{verbatim}}
\end{minipage}
\\ \\
\subsection{ The \texttt{has\_one :through} Association}

A \texttt{has\_one :through} association sets up a one-to-one  connection with another model. This association indicates that the  declaring model can be matched with one instance of another model by  proceeding \emph{through} a third model. For example, if each supplier  has one account, and each account is associated with one account  history, then the customer model could look like this:
\\ \\
\begin{minipage}{\textwidth}{\scriptsize
\begin{verbatim}
class Supplier < ActiveRecord::Base
  has_one :account
  has_one :account_history, :through => :account
end
 
class Account < ActiveRecord::Base
  belongs_to :supplier
  has_one :account_history
end
 
class AccountHistory < ActiveRecord::Base
  belongs_to :account
end
\end{verbatim}}
\end{minipage}
\\ \\


\includegraphics[width=\textwidth]{../has_one_through.png}

\subsection{ The \texttt{has\_and\_belongs\_to\_many} Association}

A \texttt{has\_and\_belongs\_to\_many} association creates a direct  many-to-many connection with another model, with no intervening model.  For example, if your application includes assemblies and parts, with  each assembly having many parts and each part appearing in many  assemblies, you could declare the models this way:
\\ \\
\begin{minipage}{\textwidth}{\scriptsize
\begin{verbatim}
class Assembly < ActiveRecord::Base
  has_and_belongs_to_many :parts
end
 
class Part < ActiveRecord::Base
  has_and_belongs_to_many :assemblies
end
\end{verbatim}}
\end{minipage}
\\ \\


\includegraphics[width=\textwidth]{../habtm.png}

\subsection{ Choosing Between \texttt{belongs\_to} and \texttt{has\_one}}

If you want to set up a one-to-one relationship between two models, you’ll need to add \texttt{belongs\_to} to one, and \texttt{has\_one} to the other. How do you know which is which?

The distinction is in where you place the foreign key (it goes on the table for the class declaring the \texttt{belongs\_to} association), but you should give some thought to the actual meaning of the data as well. The \texttt{has\_one}  relationship says that one of something is yours – that is, that  something points back to you. For example, it makes more sense to say  that a supplier owns an account than that an account owns a supplier.  This suggests that the correct relationships are like this:
\\ \\
\begin{minipage}{\textwidth}{\scriptsize
\begin{verbatim}
class Supplier < ActiveRecord::Base
  has_one :account
end
 
class Account < ActiveRecord::Base
  belongs_to :supplier
end
\end{verbatim}}
\end{minipage}
\\ \\

The corresponding migration might look like this:
\\ \\
\begin{minipage}{\textwidth}{\scriptsize
\begin{verbatim}
class CreateSuppliers < ActiveRecord::Migration
  def change
    create_table :suppliers do |t|
      t.string  :name
      t.timestamps
    end
 
    create_table :accounts do |t|
      t.integer :supplier_id
      t.string  :account_number
      t.timestamps
    end
  end
end
\end{verbatim}}
\end{minipage}
\\ \\

Using \texttt{t.integer :supplier\_id} makes the  foreign key naming obvious and explicit. In current versions of Rails,  you can abstract away this implementation detail by using \texttt{t.references :supplier} instead.

\subsection{ Choosing Between has\_many :through \\ and has\_and\_belongs\_to\_many}

Rails offers two different ways to declare a many-to-many relationship between models. The simpler way is to use \texttt{has\_and\_belongs\_to\_many}, which allows you to make the association directly:
\\ \\
\begin{minipage}{\textwidth}{\scriptsize
\begin{verbatim}
class Assembly < ActiveRecord::Base
  has_and_belongs_to_many :parts
end
 
class Part < ActiveRecord::Base
  has_and_belongs_to_many :assemblies
end
\end{verbatim}}
\end{minipage}
\\ \\

The second way to declare a many-to-many relationship is to use \texttt{has\_many :through}. This makes the association indirectly, through a join model:
\\ \\
\begin{minipage}{\textwidth}{\scriptsize
\begin{verbatim}
class Assembly < ActiveRecord::Base
  has_many :manifests
  has_many :parts, :through => :manifests
end
 
class Manifest < ActiveRecord::Base
  belongs_to :assembly
  belongs_to :part
end
 
class Part < ActiveRecord::Base
  has_many :manifests
  has_many :assemblies, :through => :manifests
end
\end{verbatim}}
\end{minipage}
\\ \\

The simplest rule of thumb is that you should set up a \texttt{has\_many :through}  relationship if you need to work with the relationship model as an  independent entity. If you don’t need to do anything with the  relationship model, it may be simpler to set up a \texttt{has\_and\_belongs\_to\_many} relationship (though you’ll need to remember to create the joining table in the database).

You should use \texttt{has\_many :through} if you need validations, callbacks, or extra attributes on the join model.

\subsection{ Polymorphic Associations}

A slightly more advanced twist on associations is the \emph{polymorphic association}.  With polymorphic associations, a model can belong to more than one  other model, on a single association. For example, you might have a  picture model that belongs to either an employee model or a product  model. Here’s how this could be declared:
\\ \\
\begin{minipage}{\textwidth}{\scriptsize
\begin{verbatim}
class Picture < ActiveRecord::Base
  belongs_to :imageable, :polymorphic => true
end
 
class Employee < ActiveRecord::Base
  has_many :pictures, :as => :imageable
end
 
class Product < ActiveRecord::Base
  has_many :pictures, :as => :imageable
end
\end{verbatim}}
\end{minipage}
\\ \\

You can think of a polymorphic \texttt{belongs\_to} declaration as setting up an interface that any other model can use. From an instance of the \texttt{Employee} model, you can retrieve a collection of pictures: \texttt{@employee.pictures}.

Similarly, you can retrieve \texttt{@product.pictures}.

If you have an instance of the \texttt{Picture} model, you can get to its parent via \texttt{@picture.imageable}.  To make this work, you need to declare both a foreign key column and a  type column in the model that declares the polymorphic interface:
\\ \\
\begin{minipage}{\textwidth}{\scriptsize
\begin{verbatim}
class CreatePictures < ActiveRecord::Migration
  def change
    create_table :pictures do |t|
      t.string  :name
      t.integer :imageable_id
      t.string  :imageable_type
      t.timestamps
    end
  end
end
\end{verbatim}}
\end{minipage}
\\ \\

This migration can be simplified by using the \texttt{t.references} form:
\\ \\
\begin{minipage}{\textwidth}{\scriptsize
\begin{verbatim}
class CreatePictures < ActiveRecord::Migration
  def change
    create_table :pictures do |t|
      t.string :name
      t.references :imageable, :polymorphic => true
      t.timestamps
    end
  end
end
\end{verbatim}}
\end{minipage}
\\ \\


\includegraphics[width=\textwidth]{../polymorphic.png}

\subsection{ Self Joins}

In designing a data model, you will sometimes find a model that  should have a relation to itself. For example, you may want to store all  employees in a single database model, but be able to trace  relationships such as between manager and subordinates. This situation  can be modeled with self-joining associations:
\\ \\
\begin{minipage}{\textwidth}{\scriptsize
\begin{verbatim}
class Employee < ActiveRecord::Base
  has_many :subordinates, :class_name => "Employee",
    :foreign_key => "manager_id"
  belongs_to :manager, :class_name => "Employee"
end
\end{verbatim}}
\end{minipage}
\\ \\

With this setup, you can retrieve \texttt{@employee.subordinates} and \texttt{@employee.manager}.

\section{ Tips, Tricks, and Warnings}

Here are a few things you should know to make efficient use of Active Record associations in your Rails applications:
\begin{itemize}
	\item Controlling caching
	\item Avoiding name collisions
	\item Updating the schema
	\item Controlling association scope
	\item Bi-directional associations
\end{itemize}

\subsection{ Controlling Caching}

All of the association methods are built around caching, which keeps  the result of the most recent query available for further operations.  The cache is even shared across methods. For example:
\\ \\
\begin{minipage}{\textwidth}{\scriptsize
\begin{verbatim}
customer.orders        # retrieves orders from the database
customer.orders.size   # uses the cached copy of orders
customer.orders.empty? # uses the cached copy of orders
\end{verbatim}}
\end{minipage}
\\ \\

But what if you want to reload the cache, because data might have been changed by some other part of the application? Just pass \texttt{true} to the association call:
\\ \\
\begin{minipage}{\textwidth}{\scriptsize
\begin{verbatim}
customer.orders             # retrieves orders from the database
customer.orders.size        # uses the cached copy of orders
customer.orders(true).empty?# discards the cached copy of orders
                            # and goes back to the database
\end{verbatim}}
\end{minipage}
\\ \\

\subsection{ Avoiding Name Collisions}

You are not free to use just any name for your associations. Because  creating an association adds a method with that name to the model, it is  a bad idea to give an association a name that is already used for an  instance method of \texttt{ActiveRecord::Base}. The association method would override the base method and break things. For instance, \texttt{attributes} or \texttt{connection} are bad names for associations.

\subsection{ Updating the Schema}

Associations are extremely useful, but they are not magic. You are  responsible for maintaining your database schema to match your  associations. In practice, this means two things, depending on what sort  of associations you are creating. For \texttt{belongs\_to} associations you need to create foreign keys, and for \texttt{has\_and\_belongs\_to\_many} associations you need to create the appropriate join table.

\subsubsection{ Creating Foreign Keys for \texttt{belongs\_to} Associations}

When you declare a \texttt{belongs\_to} association, you need to create foreign keys as appropriate. For example, consider this model:
\\ \\
\begin{minipage}{\textwidth}{\scriptsize
\begin{verbatim}
class Order < ActiveRecord::Base
  belongs_to :customer
end
\end{verbatim}}
\end{minipage}
\\ \\

This declaration needs to be backed up by the proper foreign key declaration on the orders table:
\\ \\
\begin{minipage}{\textwidth}{\scriptsize
\begin{verbatim}
class CreateOrders < ActiveRecord::Migration
  def change
    create_table :orders do |t|
      t.datetime :order_date
      t.string   :order_number
      t.integer  :customer_id
    end
  end
end
\end{verbatim}}
\end{minipage}
\\ \\

If you create an association some time after you build the underlying model, you need to remember to create an \texttt{add\_column} migration to provide the necessary foreign key.

\subsubsection{ Creating Join Tables for \\ \texttt{has\_and\_belongs\_to\_many} Associations}

If you create a \texttt{has\_and\_belongs\_to\_many} association, you  need to explicitly create the joining table. Unless the name of the join  table is explicitly specified by using the \texttt{:join\_table} option,  Active Record creates the name by using the lexical order of the class  names. So a join between customer and order models will give the default  join table name of “customers\_orders” because “c” outranks “o” in  lexical ordering.

The precedence between model names is calculated using the \texttt{$<$} operator for \texttt{String}.  This means that if the strings are of different lengths, and the  strings are equal when compared up to the shortest length, then the  longer string is considered of higher lexical precedence than the  shorter one. For example, one would expect the tables “paper\_boxes” and  “papers” to generate a join table name of “papers\_paper\_boxes” because  of the length of the name “paper\_boxes”, but it in fact generates a join  table name of “paper\_boxes\_papers” (because the underscore ‘\_’ is  lexicographically \emph{less} than ‘s’ in common encodings).

Whatever the name, you must manually generate the join table with an  appropriate migration. For example, consider these associations:
\\ \\
\begin{minipage}{\textwidth}{\scriptsize
\begin{verbatim}
class Assembly < ActiveRecord::Base
  has_and_belongs_to_many :parts
end
 
class Part < ActiveRecord::Base
  has_and_belongs_to_many :assemblies
end
\end{verbatim}}
\end{minipage}
\\ \\

These need to be backed up by a migration to create the \texttt{assemblies\_parts} table. This table should be created without a primary key:
\\ \\
\begin{minipage}{\textwidth}{\scriptsize
\begin{verbatim}
class CreateAssemblyPartJoinTable < ActiveRecord::Migration
  def change
    create_table :assemblies_parts, :id => false do |t|
      t.integer :assembly_id
      t.integer :part_id
    end
  end
end
\end{verbatim}}
\end{minipage}
\\ \\

We pass \texttt{:id =$>$ false} to \texttt{create\_table} because  that table does not represent a model. That’s required for the  association to work properly. If you observe any strange behavior in a \texttt{has\_and\_belongs\_to\_many} association like mangled models IDs, or exceptions about conflicting IDs chances are you forgot that bit.

\subsection{ Controlling Association Scope}

By default, associations look for objects only within the current  module’s scope. This can be important when you declare Active Record  models within a module. For example:
\\ \\
\begin{minipage}{\textwidth}{\scriptsize
\begin{verbatim}
module MyApplication
  module Business
    class Supplier < ActiveRecord::Base
       has_one :account
    end
 
    class Account < ActiveRecord::Base
       belongs_to :supplier
    end
  end
end
\end{verbatim}}
\end{minipage}
\\ \\

This will work fine, because both the \texttt{Supplier} and the \texttt{Account} class are defined within the same scope. But the following will \emph{not} work, because \texttt{Supplier} and \texttt{Account} are defined in different scopes:
\\ \\
\begin{minipage}{\textwidth}{\scriptsize
\begin{verbatim}
module MyApplication
  module Business
    class Supplier < ActiveRecord::Base
       has_one :account
    end
  end
 
  module Billing
    class Account < ActiveRecord::Base
       belongs_to :supplier
    end
  end
end
\end{verbatim}}
\end{minipage}
\\ \\

To associate a model with a model in a different namespace, you must  specify the complete class name in your association declaration:
\\ \\
\begin{minipage}{\textwidth}{\scriptsize
\begin{verbatim}
module MyApplication
  module Business
    class Supplier < ActiveRecord::Base
       has_one :account,
        :class_name => "MyApplication::Billing::Account"
    end
  end
 
  module Billing
    class Account < ActiveRecord::Base
       belongs_to :supplier,
        :class_name => "MyApplication::Business::Supplier"
    end
  end
end
\end{verbatim}}
\end{minipage}
\\ \\

\subsection{ Bi-directional Associations}

It’s normal for associations to work in two directions, requiring declaration on two different models:
\\ \\
\begin{minipage}{\textwidth}{\scriptsize
\begin{verbatim}
class Customer < ActiveRecord::Base
  has_many :orders
end
 
class Order < ActiveRecord::Base
  belongs_to :customer
end
\end{verbatim}}
\end{minipage}
\\ \\

By default, Active Record doesn’t know about the connection between  these associations. This can lead to two copies of an object getting out  of sync:
\\ \\
\begin{minipage}{\textwidth}{\scriptsize
\begin{verbatim}
c = Customer.first
o = c.orders.first
c.first_name == o.customer.first_name # => true
c.first_name = 'Manny'
c.first_name == o.customer.first_name # => false
\end{verbatim}}
\end{minipage}
\\ \\

This happens because c and o.customer are two different in-memory  representations of the same data, and neither one is automatically  refreshed from changes to the other. Active Record provides the \texttt{:inverse\_of} option so that you can inform it of these relations:
\\ \\
\begin{minipage}{\textwidth}{\scriptsize
\begin{verbatim}
class Customer < ActiveRecord::Base
  has_many :orders, :inverse_of => :customer
end
 
class Order < ActiveRecord::Base
  belongs_to :customer, :inverse_of => :orders
end
\end{verbatim}}
\end{minipage}
\\ \\

With these changes, Active Record will only load one copy of the  customer object, preventing inconsistencies and making your application  more efficient:
\\ \\
\begin{minipage}{\textwidth}{\scriptsize
\begin{verbatim}
c = Customer.first
o = c.orders.first
c.first_name == o.customer.first_name # => true
c.first_name = 'Manny'
c.first_name == o.customer.first_name # => true
\end{verbatim}}
\end{minipage}
\\ \\

There are a few limitations to \texttt{inverse\_of} support:
\begin{itemize}
	\item They do not work with \texttt{:through} associations.
	\item They do not work with \texttt{:polymorphic} associations.
	\item They do not work with \texttt{:as} associations.
	\item For \texttt{belongs\_to} associations, \texttt{has\_many} inverse associations are ignored.
\end{itemize}

\section{ Detailed Association Reference}

The following sections give the details of each type of association,  including the methods that they add and the options that you can use  when declaring an association.

\subsection{ \texttt{belongs\_to} Association Reference}

The \texttt{belongs\_to} association creates a one-to-one match with  another model. In database terms, this association says that this class  contains the foreign key. If the other class contains the foreign key,  then you should use \texttt{has\_one} instead.

\subsubsection{ Methods Added by \texttt{belongs\_to}}

When you declare a \texttt{belongs\_to} association, the declaring class automatically gains four methods related to the association:
\begin{itemize}
	\item \texttt{\emph{association}(force\_reload = false)}
	\item \texttt{\emph{association}=(associate)}
	\item \texttt{build\_\emph{association}(attributes = \{\})}
	\item \texttt{create\_\emph{association}(attributes = \{\})}
\end{itemize}

In all of these methods, \texttt{\emph{association}} is replaced with the symbol passed as the first argument to \texttt{belongs\_to}. For example, given the declaration:
\\ \\
\begin{minipage}{\textwidth}{\scriptsize
\begin{verbatim}
class Order < ActiveRecord::Base
  belongs_to :customer
end
\end{verbatim}}
\end{minipage}
\\ \\

Each instance of the order model will have these methods:
\\ \\
\begin{minipage}{\textwidth}{\scriptsize
\begin{verbatim}
customer
customer=
build_customer
create_customer
\end{verbatim}}
\end{minipage}
\\ \\

When initializing a new \texttt{has\_one} or \texttt{belongs\_to} association you must use the \texttt{build\_} prefix to build the association, rather than the \texttt{association.build} method that would be used for \texttt{has\_many} or \texttt{has\_and\_belongs\_to\_many} associations. To create one, use the \texttt{create\_} prefix.

\paragraph{4.1.1.1 \texttt{\emph{association}(force\_reload = false)}}\\ \\\\

The \texttt{\emph{association}} method returns the associated object, if any. If no associated object is found, it returns \texttt{nil}.
\\ \\
\begin{minipage}{\textwidth}{\scriptsize
\begin{verbatim}
@customer = @order.customer
\end{verbatim}}
\end{minipage}
\\ \\

If the associated object has already been retrieved from the database  for this object, the cached version will be returned. To override this  behavior (and force a database read), pass \texttt{true} as the \texttt{force\_reload} argument.

\paragraph{4.1.1.2 \texttt{\emph{association}=(associate)}}\\ \\\\

The \texttt{\emph{association}=} method assigns an associated  object to this object. Behind the scenes, this means extracting the  primary key from the associate object and setting this object’s foreign  key to the same value.
\\ \\
\begin{minipage}{\textwidth}{\scriptsize
\begin{verbatim}
@order.customer = @customer
\end{verbatim}}
\end{minipage}
\\ \\

\paragraph{4.1.1.3 \texttt{build\_\emph{association}(attributes = \{\})}}
\indent
The \texttt{build\_\emph{association}} method returns a new object  of the associated type. This object will be instantiated from the passed  attributes, and the link through this object’s foreign key will be set,  but the associated object will \emph{not} yet be saved.
\\ \\
\begin{minipage}{\textwidth}{\scriptsize
\begin{verbatim}
@customer = @order.build_customer(:customer_number => 123,
  :customer_name => "John Doe")
\end{verbatim}}
\end{minipage}
\\ \\

\paragraph{4.1.1.4 \texttt{create\_\emph{association}(attributes = \{\})}}\\ \\\\

The \texttt{create\_\emph{association}} method returns a new object  of the associated type. This object will be instantiated from the passed  attributes, the link through this object’s foreign key will be set,  and, once it passes all of the validations specified on the associated  model, the associated object \emph{will} be saved.
\\ \\
\begin{minipage}{\textwidth}{\scriptsize
\begin{verbatim}
@customer = @order.create_customer(:customer_number => 123,
  :customer_name => "John Doe")
\end{verbatim}}
\end{minipage}
\\ \\

\subsubsection{ Options for \texttt{belongs\_to}}

While Rails uses intelligent defaults that will work well in most  situations, there may be times when you want to customize the behavior  of the \texttt{belongs\_to} association reference. Such customizations  can easily be accomplished by passing options when you create the  association. For example, this assocation uses two such options:
\\ \\
\begin{minipage}{\textwidth}{\scriptsize
\begin{verbatim}
class Order < ActiveRecord::Base
  belongs_to :customer, :counter_cache => true,
    :conditions => "active = 1"
end
\end{verbatim}}
\end{minipage}
\\ \\

The \texttt{belongs\_to} association supports these options:
\begin{itemize}
	\item \texttt{:autosave}
	\item \texttt{:class\_name}
	\item \texttt{:conditions}
	\item \texttt{:counter\_cache}
	\item \texttt{:dependent}
	\item \texttt{:foreign\_key}
	\item \texttt{:include}
	\item \texttt{:inverse\_of}
	\item \texttt{:polymorphic}
	\item \texttt{:readonly}
	\item \texttt{:select}
	\item \texttt{:touch}
	\item \texttt{:validate}
\end{itemize}

\paragraph{4.1.2.1 \texttt{:autosave}}\\ \\\\

If you set the \texttt{:autosave} option to \texttt{true}, Rails will save any loaded members and destroy members that are marked for destruction whenever you save the parent object.

\paragraph{4.1.2.2 \texttt{:class\_name}}\\ \\\\

If the name of the other model cannot be derived from the association name, you can use the \texttt{:class\_name}  option to supply the model name. For example, if an order belongs to a  customer, but the actual name of the model containing customers is \texttt{Patron}, you’d set things up this way:
\\ \\
\begin{minipage}{\textwidth}{\scriptsize
\begin{verbatim}
class Order < ActiveRecord::Base
  belongs_to :customer, :class_name => "Patron"
end
\end{verbatim}}
\end{minipage}
\\ \\

\paragraph{4.1.2.3 \texttt{:conditions}}\\ \\\\

The \texttt{:conditions} option lets you specify the conditions that the associated object must meet (in the syntax used by an SQL\texttt{WHERE} clause).
\\ \\
\begin{minipage}{\textwidth}{\scriptsize
\begin{verbatim}
class Order < ActiveRecord::Base
  belongs_to :customer, :conditions => "active = 1"
end
\end{verbatim}}
\end{minipage}
\\ \\

\paragraph{4.1.2.4 \texttt{:counter\_cache}}\\ \\\\

The \texttt{:counter\_cache} option can be used to make finding the number of belonging objects more efficient. Consider these models:
\\ \\
\begin{minipage}{\textwidth}{\scriptsize
\begin{verbatim}
class Order < ActiveRecord::Base
  belongs_to :customer
end
class Customer < ActiveRecord::Base
  has_many :orders
end
\end{verbatim}}
\end{minipage}
\\ \\

With these declarations, asking for the value of \texttt{@customer.orders.size} requires making a call to the database to perform a \texttt{COUNT(*)} query. To avoid this call, you can add a counter cache to the \emph{belonging} model:
\\ \\
\begin{minipage}{\textwidth}{\scriptsize
\begin{verbatim}
class Order < ActiveRecord::Base
  belongs_to :customer, :counter_cache => true
end
class Customer < ActiveRecord::Base
  has_many :orders
end
\end{verbatim}}
\end{minipage}
\\ \\

With this declaration, Rails will keep the cache value up to date, and then return that value in response to the \texttt{size} method.

Although the \texttt{:counter\_cache} option is specified on the model that includes the \texttt{belongs\_to} declaration, the actual column must be added to the \emph{associated} model. In the case above, you would need to add a column named \texttt{orders\_count} to the \texttt{Customer} model. You can override the default column name if you need to:
\\ \\
\begin{minipage}{\textwidth}{\scriptsize
\begin{verbatim}
class Order < ActiveRecord::Base
  belongs_to :customer, :counter_cache => :count_of_orders
end
class Customer < ActiveRecord::Base
  has_many :orders
end
\end{verbatim}}
\end{minipage}
\\ \\

Counter cache columns are added to the containing model’s list of read-only attributes through \texttt{attr\_readonly}.

\paragraph{4.1.2.5 \texttt{:dependent}}\\ \\\\

If you set the \texttt{:dependent} option to \texttt{:destroy}, then deleting this object will call the \texttt{destroy} method on the associated object to delete that object. If you set the \texttt{:dependent} option to \texttt{:delete}, then deleting this object will delete the associated object \emph{without} calling its \texttt{destroy} method.

You should not specify this option on a \texttt{belongs\_to} association that is connected with a \texttt{has\_many} association on the other class. Doing so can lead to orphaned records in your database.

\paragraph{4.1.2.6 \texttt{:foreign\_key}}\\ \\\\

By convention, Rails assumes that the column used to hold the foreign  key on this model is the name of the association with the suffix \texttt{\_id} added. The \texttt{:foreign\_key} option lets you set the name of the foreign key directly:
\\ \\
\begin{minipage}{\textwidth}{\scriptsize
\begin{verbatim}
class Order < ActiveRecord::Base
  belongs_to :customer, :class_name => "Patron",
    :foreign_key => "patron_id"
end
\end{verbatim}}
\end{minipage}
\\ \\

In any case, Rails will not create foreign key columns for you. You need to explicitly define them as part of your migrations.

\paragraph{4.1.2.7 \texttt{:include}}\\ \\\\

You can use the \texttt{:include} option to specify second-order  associations that should be eager-loaded when this association is used.  For example, consider these models:
\\ \\
\begin{minipage}{\textwidth}{\scriptsize
\begin{verbatim}
class LineItem < ActiveRecord::Base
  belongs_to :order
end
 
class Order < ActiveRecord::Base
  belongs_to :customer
  has_many :line_items
end
 
class Customer < ActiveRecord::Base
  has_many :orders
end
\end{verbatim}}
\end{minipage}
\\ \\

If you frequently retrieve customers directly from line items (\texttt{@line\_item.order.customer}), then you can make your code somewhat more efficient by including customers in the association from line items to orders:
\\ \\
\begin{minipage}{\textwidth}{\scriptsize
\begin{verbatim}
class LineItem < ActiveRecord::Base
  belongs_to :order, :include => :customer
end
 
class Order < ActiveRecord::Base
  belongs_to :customer
  has_many :line_items
end
 
class Customer < ActiveRecord::Base
  has_many :orders
end

\end{verbatim}}
\end{minipage}
\\ \\

There’s no need to use \texttt{:include} for immediate associations – that is, if you have \texttt{Order belongs\_to :customer}, then the customer is eager-loaded automatically when it’s needed.

\paragraph{4.1.2.8 \texttt{:inverse\_of}}\\ \\\\

The \texttt{:inverse\_of} option specifies the name of the \texttt{has\_many} or \texttt{has\_one} association that is the inverse of this association. Does not work in combination with the \texttt{:polymorphic} options.
\\ \\
\begin{minipage}{\textwidth}{\scriptsize
\begin{verbatim}
class Customer < ActiveRecord::Base
  has_many :orders, :inverse_of => :customer
end
 
class Order < ActiveRecord::Base
  belongs_to :customer, :inverse_of => :orders
end
\end{verbatim}}
\end{minipage}
\\ \\

\paragraph{4.1.2.9 \texttt{:polymorphic}}\\ \\\\

Passing \texttt{true} to the \texttt{:polymorphic} option indicates that this is a polymorphic association. Polymorphic associations were discussed in detail \hyperlink{polymorphic-associations}{earlier in this guide}.

\paragraph{4.1.2.10 \texttt{:readonly}}\\ \\\\

If you set the \texttt{:readonly} option to \texttt{true}, then the associated object will be read-only when retrieved via the association.

\paragraph{4.1.2.11 \texttt{:select}}\\ \\\\

The \texttt{:select} option lets you override the SQL\texttt{SELECT} clause that is used to retrieve data about the associated object. By default, Rails retrieves all columns.

If you set the \texttt{:select} option on a \texttt{belongs\_to} association, you should also set the \texttt{foreign\_key} option to guarantee the correct results.

\paragraph{4.1.2.12 \texttt{:touch}}\\ \\\\

If you set the \texttt{:touch} option to \texttt{:true}, then the \texttt{updated\_at} or \texttt{updated\_on} timestamp on the associated object will be set to the current time whenever this object is saved or destroyed:
\\ \\
\begin{minipage}{\textwidth}{\scriptsize
\begin{verbatim}
class Order < ActiveRecord::Base
  belongs_to :customer, :touch => true
end
 
class Customer < ActiveRecord::Base
  has_many :orders
end
\end{verbatim}}
\end{minipage}
\\ \\

In this case, saving or destroying an order will update the timestamp  on the associated customer. You can also specify a particular timestamp  attribute to update:
\\ \\
\begin{minipage}{\textwidth}{\scriptsize
\begin{verbatim}
class Order < ActiveRecord::Base
  belongs_to :customer, :touch => :orders_updated_at
end
\end{verbatim}}
\end{minipage}
\\ \\

\paragraph{4.1.2.13 \texttt{:validate}}\\ \\\\

If you set the \texttt{:validate} option to \texttt{true}, then associated objects will be validated whenever you save this object. By default, this is \texttt{false}: associated objects will not be validated when this object is saved.

\subsubsection{ Do Any Associated Objects Exist?}

You can see if any associated objects exist by using the \emph{association}.nil? method:
\\ \\
\begin{minipage}{\textwidth}{\scriptsize
\begin{verbatim}
if @order.customer.nil?
  @msg = "No customer found for this order"
end
\end{verbatim}}
\end{minipage}
\\ \\

\subsubsection{ When are Objects Saved?}

Assigning an object to a \texttt{belongs\_to} association does \emph{not} automatically save the object. It does not save the associated object either.

\subsection{ \texttt{has\_one} Association Reference}

The \texttt{has\_one} association creates a one-to-one match with  another model. In database terms, this association says that the other  class contains the foreign key. If this class contains the foreign key,  then you should use \texttt{belongs\_to} instead.

\subsubsection{ Methods Added by \texttt{has\_one}}

When you declare a \texttt{has\_one} association, the declaring class automatically gains four methods related to the association:
\begin{itemize}
	\item \texttt{\emph{association}(force\_reload = false)}
	\item \texttt{\emph{association}=(associate)}
	\item \texttt{build\_\emph{association}(attributes = \{\})}
	\item \texttt{create\_\emph{association}(attributes = \{\})}
\end{itemize}

In all of these methods, \texttt{\emph{association}} is replaced with the symbol passed as the first argument to \texttt{has\_one}. For example, given the declaration:
\\ \\
\begin{minipage}{\textwidth}{\scriptsize
\begin{verbatim}
class Supplier < ActiveRecord::Base
  has_one :account
end
\end{verbatim}}
\end{minipage}
\\ \\

Each instance of the \texttt{Supplier} model will have these methods:
\\ \\
\begin{minipage}{\textwidth}{\scriptsize
\begin{verbatim}
account
account=
build_account
create_account
\end{verbatim}}
\end{minipage}
\\ \\

When initializing a new \texttt{has\_one} or \texttt{belongs\_to} association you must use the \texttt{build\_} prefix to build the association, rather than the \texttt{association.build} method that would be used for \texttt{has\_many} or \texttt{has\_and\_belongs\_to\_many} associations. To create one, use the \texttt{create\_} prefix.

\paragraph{4.2.1.1 \texttt{\emph{association}(force\_reload = false)}}\\ \\\\

The \texttt{\emph{association}} method returns the associated object, if any. If no associated object is found, it returns \texttt{nil}.
\\ \\
\begin{minipage}{\textwidth}{\scriptsize
\begin{verbatim}
@account = @supplier.account
\end{verbatim}}
\end{minipage}
\\ \\

If the associated object has already been retrieved from the database  for this object, the cached version will be returned. To override this  behavior (and force a database read), pass \texttt{true} as the \texttt{force\_reload} argument.

\paragraph{4.2.1.2 \texttt{\emph{association}=(associate)}}\\ \\\\

The \texttt{\emph{association}=} method assigns an associated  object to this object. Behind the scenes, this means extracting the  primary key from this object and setting the associate object’s foreign  key to the same value.
\\ \\
\begin{minipage}{\textwidth}{\scriptsize
\begin{verbatim}
@supplier.account = @account
\end{verbatim}}
\end{minipage}
\\ \\

\paragraph{4.2.1.3 \texttt{build\_\emph{association}(attributes = \{\})}}\\ \\\\

The \texttt{build\_\emph{association}} method returns a new object  of the associated type. This object will be instantiated from the passed  attributes, and the link through its foreign key will be set, but the  associated object will \emph{not} yet be saved.
\\ \\
\begin{minipage}{\textwidth}{\scriptsize
\begin{verbatim}
@account = @supplier.build_account(:terms => "Net 30")
\end{verbatim}}
\end{minipage}
\\ \\

\paragraph{4.2.1.4 \texttt{create\_\emph{association}(attributes = \{\})}}\\ \\\\

The \texttt{create\_\emph{association}} method returns a new object  of the associated type. This object will be instantiated from the passed  attributes, the link through its foreign key will be set, and, once it  passes all of the validations specified on the associated model, the  associated object \emph{will} be saved.
\\ \\
\begin{minipage}{\textwidth}{\scriptsize
\begin{verbatim}
@account = @supplier.create_account(:terms => "Net 30")
\end{verbatim}}
\end{minipage}
\\ \\

\subsubsection{ Options for \texttt{has\_one}}

While Rails uses intelligent defaults that will work well in most  situations, there may be times when you want to customize the behavior  of the \texttt{has\_one} association reference. Such customizations can  easily be accomplished by passing options when you create the  association. For example, this assocation uses two such options:
\\ \\
\begin{minipage}{\textwidth}{\scriptsize
\begin{verbatim}
class Supplier < ActiveRecord::Base
  has_one :account, :class_name => "Billing",
          :dependent => :nullify
end
\end{verbatim}}
\end{minipage}
\\ \\

The \texttt{has\_one} association supports these options:
\begin{itemize}
	\item \texttt{:as}
	\item \texttt{:autosave}
	\item \texttt{:class\_name}
	\item \texttt{:conditions}
	\item \texttt{:dependent}
	\item \texttt{:foreign\_key}
	\item \texttt{:include}
	\item \texttt{:inverse\_of}
	\item \texttt{:order}
	\item \texttt{:primary\_key}
	\item \texttt{:readonly}
	\item \texttt{:select}
	\item \texttt{:source}
	\item \texttt{:source\_type}
	\item \texttt{:through}
	\item \texttt{:validate}
\end{itemize}

\paragraph{4.2.2.1 \texttt{:as}}\\ \\\\

Setting the \texttt{:as} option indicates that this is a polymorphic association. Polymorphic associations were discussed in detail \hyperlink{polymorphic-associations}{earlier in this guide}.

\paragraph{4.2.2.2 \texttt{:autosave}}\\ \\\\

If you set the \texttt{:autosave} option to \texttt{true}, Rails will save any loaded members and destroy members that are marked for destruction whenever you save the parent object.

\paragraph{4.2.2.3 \texttt{:class\_name}}\\ \\\\

If the name of the other model cannot be derived from the association name, you can use the \texttt{:class\_name}  option to supply the model name. For example, if a supplier has an  account, but the actual name of the model containing accounts is \texttt{Billing}, you’d set things up this way:
\\ \\
\begin{minipage}{\textwidth}{\scriptsize
\begin{verbatim}
class Supplier < ActiveRecord::Base
  has_one :account, :class_name => "Billing"
end
\end{verbatim}}
\end{minipage}
\\ \\

\paragraph{4.2.2.4 \texttt{:conditions}}\\ \\\\

The \texttt{:conditions} option lets you specify the conditions that the associated object must meet (in the syntax used by an SQL\texttt{WHERE} clause).
\\ \\
\begin{minipage}{\textwidth}{\scriptsize
\begin{verbatim}
class Supplier < ActiveRecord::Base
  has_one :account, :conditions => "confirmed = 1"
end
\end{verbatim}}
\end{minipage}
\\ \\

\paragraph{4.2.2.5 \texttt{:dependent}}\\ \\\\

If you set the \texttt{:dependent} option to \texttt{:destroy}, then deleting this object will call the \texttt{destroy} method on the associated object to delete that object. If you set the \texttt{:dependent} option to \texttt{:delete}, then deleting this object will delete the associated object \emph{without} calling its \texttt{destroy} method. If you set the \texttt{:dependent} option to \texttt{:nullify}, then deleting this object will set the foreign key in the association object to \texttt{NULL}.

\paragraph{4.2.2.6 \texttt{:foreign\_key}}\\ \\\\

By convention, Rails assumes that the column used to hold the foreign  key on the other model is the name of this model with the suffix \texttt{\_id} added. The \texttt{:foreign\_key} option lets you set the name of the foreign key directly:
\\ \\
\begin{minipage}{\textwidth}{\scriptsize
\begin{verbatim}
class Supplier < ActiveRecord::Base
  has_one :account, :foreign_key => "supp_id"
end
\end{verbatim}}
\end{minipage}
\\ \\

In any case, Rails will not create foreign key columns for you. You need to explicitly define them as part of your migrations.

\paragraph{4.2.2.7 \texttt{:include}}\\ \\\\

You can use the \texttt{:include} option to specify second-order  associations that should be eager-loaded when this association is used.  For example, consider these models:
\\ \\
\begin{minipage}{\textwidth}{\scriptsize
\begin{verbatim}
class Supplier < ActiveRecord::Base
  has_one :account
end
 
class Account < ActiveRecord::Base
  belongs_to :supplier
  belongs_to :representative
end
 
class Representative < ActiveRecord::Base
  has_many :accounts
end
\end{verbatim}}
\end{minipage}
\\ \\

If you frequently retrieve representatives directly from suppliers (\texttt{@supplier.account.representative}),  then you can make your code somewhat more efficient by including  representatives in the association from suppliers to accounts:
\\ \\
\begin{minipage}{\textwidth}{\scriptsize
\begin{verbatim}
class Supplier < ActiveRecord::Base
  has_one :account, :include => :representative
end
 
class Account < ActiveRecord::Base
  belongs_to :supplier
  belongs_to :representative
end
 
class Representative < ActiveRecord::Base
  has_many :accounts
end
\end{verbatim}}
\end{minipage}
\\ \\

\paragraph{4.2.2.8 \texttt{:inverse\_of}}\\ \\\\

The \texttt{:inverse\_of} option specifies the name of the \texttt{belongs\_to} association that is the inverse of this association. Does not work in combination with the \texttt{:through} or \texttt{:as} options.
\\ \\
\begin{minipage}{\textwidth}{\scriptsize
\begin{verbatim}
class Supplier < ActiveRecord::Base
  has_one :account, :inverse_of => :supplier
end
 
class Account < ActiveRecord::Base
  belongs_to :supplier, :inverse_of => :account
end
\end{verbatim}}
\end{minipage}
\\ \\

\paragraph{4.2.2.9 \texttt{:order}}\\ \\\\

The \texttt{:order} option dictates the order in which associated objects will be received (in the syntax used by an SQL\texttt{ORDER BY} clause). Because a \texttt{has\_one} association will only retrieve a single associated object, this option should not be needed.

\paragraph{4.2.2.10 \texttt{:primary\_key}}\\ \\\\

By convention, Rails assumes that the column used to hold the primary key of this model is \texttt{id}. You can override this and explicitly specify the primary key with the \texttt{:primary\_key} option.

\paragraph{4.2.2.11 \texttt{:readonly}}\\ \\\\

If you set the \texttt{:readonly} option to \texttt{true}, then the associated object will be read-only when retrieved via the association.

\paragraph{4.2.2.12 \texttt{:select}}\\ \\\\

The \texttt{:select} option lets you override the SQL\texttt{SELECT} clause that is used to retrieve data about the associated object. By default, Rails retrieves all columns.

\paragraph{4.2.2.13 \texttt{:source}}\\ \\\\

The \texttt{:source} option specifies the source association name for a \texttt{has\_one :through} association.

\paragraph{4.2.2.14 \texttt{:source\_type}}\\ \\\\

The \texttt{:source\_type} option specifies the source association type for a \texttt{has\_one :through} association that proceeds through a polymorphic association.

\paragraph{4.2.2.15 \texttt{:through}}\\ \\\\

The \texttt{:through} option specifies a join model through which to perform the query. \texttt{has\_one :through} associations were discussed in detail \hyperlink{the-has_one-through-association}{earlier in this guide}.

\paragraph{4.2.2.16 \texttt{:validate}}\\ \\\\

If you set the \texttt{:validate} option to \texttt{true}, then associated objects will be validated whenever you save this object. By default, this is \texttt{false}: associated objects will not be validated when this object is saved.

\subsubsection{ Do Any Associated Objects Exist?}

You can see if any associated objects exist by using the \emph{association}.nil? method:
\\ \\
\begin{minipage}{\textwidth}{\scriptsize
\begin{verbatim}
if @supplier.account.nil?
  @msg = "No account found for this supplier"
end
\end{verbatim}}
\end{minipage}
\\ \\

\subsubsection{ When are Objects Saved?}

When you assign an object to a \texttt{has\_one} association, that  object is automatically saved (in order to update its foreign key). In  addition, any object being replaced is also automatically saved, because  its foreign key will change too.

If either of these saves fails due to validation errors, then the assignment statement returns \texttt{false} and the assignment itself is cancelled.

If the parent object (the one declaring the \texttt{has\_one} association) is unsaved (that is, \texttt{new\_record?} returns \texttt{true}) then the child objects are not saved. They will automatically when the parent object is saved.

If you want to assign an object to a \texttt{has\_one} association without saving the object, use the \texttt{\emph{association}.build} method.

\subsection{ \texttt{has\_many} Association Reference}

The \texttt{has\_many} association creates a one-to-many relationship  with another model. In database terms, this association says that the  other class will have a foreign key that refers to instances of this  class.

\subsubsection{ Methods Added by \texttt{has\_many}}

When you declare a \texttt{has\_many} association, the declaring class automatically gains 13 methods related to the association:
\begin{itemize}
	\item \texttt{\emph{collection}(force\_reload = false)}
	\item \texttt{\emph{collection}$<$$<$(object, …)}
	\item \texttt{\emph{collection}.delete(object, …)}
	\item \texttt{\emph{collection}=objects}
	\item \texttt{\emph{collection\_singular}\_ids}
	\item \texttt{\emph{collection\_singular}\_ids=ids}
	\item \texttt{\emph{collection}.clear}
	\item \texttt{\emph{collection}.empty?}
	\item \texttt{\emph{collection}.size}
	\item \texttt{\emph{collection}.find(…)}
	\item \texttt{\emph{collection}.where(…)}
	\item \texttt{\emph{collection}.exists?(…)}
	\item \texttt{\emph{collection}.build(attributes = \{\}, …)}
	\item \texttt{\emph{collection}.create(attributes = \{\})}
\end{itemize}

In all of these methods, \texttt{\emph{collection}} is replaced with the symbol passed as the first argument to \texttt{has\_many}, and \texttt{\emph{collection\_singular}} is replaced with the singularized version of that symbol.. For example, given the declaration:
\\ \\
\begin{minipage}{\textwidth}{\scriptsize
\begin{verbatim}
class Customer < ActiveRecord::Base
  has_many :orders
end
\end{verbatim}}
\end{minipage}
\\ \\

Each instance of the customer model will have these methods:
\\ \\
\begin{minipage}{\textwidth}{\scriptsize
\begin{verbatim}
orders(force_reload = false)
orders<<(object, ...)
orders.delete(object, ...)
orders=objects
order_ids
order_ids=ids
orders.clear
orders.empty?
orders.size
orders.find(...)
orders.where(...)
orders.exists?(...)
orders.build(attributes = {}, ...)
orders.create(attributes = {})
\end{verbatim}}
\end{minipage}
\\ \\

\paragraph{4.3.1.1 \texttt{\emph{collection}(force\_reload = false)}}\\ \\\\

The \texttt{\emph{collection}} method returns an array of all of the associated objects. If there are no associated objects, it returns an empty array.
\\ \\
\begin{minipage}{\textwidth}{\scriptsize
\begin{verbatim}
@orders = @customer.orders
\end{verbatim}}
\end{minipage}
\\ \\

\paragraph{4.3.1.2 \texttt{\emph{collection}$<$$<$(object, …)}}\\ \\\\

The \texttt{\emph{collection}$<$$<$} method adds one or more objects to the collection by setting their foreign keys to the primary key of the calling model.
\\ \\
\begin{minipage}{\textwidth}{\scriptsize
\begin{verbatim}
@customer.orders << @order1
\end{verbatim}}
\end{minipage}
\\ \\

\paragraph{4.3.1.3 \texttt{\emph{collection}.delete(object, …)}}\\ \\\\

The \texttt{\emph{collection}.delete} method removes one or more objects from the collection by setting their foreign keys to \texttt{NULL}.
\\ \\
\begin{minipage}{\textwidth}{\scriptsize
\begin{verbatim}
@customer.orders.delete(@order1)
\end{verbatim}}
\end{minipage}
\\ \\

Additionally, objects will be destroyed if they’re associated with \texttt{:dependent =$>$ :destroy}, and deleted if they’re associated with \texttt{:dependent =$>$ :delete\_all}.

\paragraph{4.3.1.4 \texttt{\emph{collection}=objects}}\\ \\\\

The \texttt{\emph{collection}=} method makes the collection contain only the supplied objects, by adding and deleting as appropriate.

\paragraph{4.3.1.5 \texttt{\emph{collection\_singular}\_ids}}\\ \\\\

The \texttt{\emph{collection\_singular}\_ids} method returns an array of the ids of the objects in the collection.
\\ \\
\begin{minipage}{\textwidth}{\scriptsize
\begin{verbatim}
@order_ids = @customer.order_ids
\end{verbatim}}
\end{minipage}
\\ \\

\paragraph{4.3.1.6 \texttt{\emph{collection\_singular}\_ids=ids}}\\ \\\\

The \texttt{\emph{collection\_singular}\_ids=} method makes the  collection contain only the objects identified by the supplied primary  key values, by adding and deleting as appropriate.

\paragraph{4.3.1.7 \texttt{\emph{collection}.clear}}\\ \\\\

The \texttt{\emph{collection}.clear} method removes every object from the collection. This destroys the associated objects if they are associated with \texttt{:dependent =$>$ :destroy}, deletes them directly from the database if \texttt{:dependent =$>$ :delete\_all}, and otherwise sets their foreign keys to \texttt{NULL}.

\paragraph{4.3.1.8 \texttt{\emph{collection}.empty?}}\\ \\\\

The \texttt{\emph{collection}.empty?} method returns \texttt{true} if the collection does not contain any associated objects.
\\ \\
\begin{minipage}{\textwidth}{\scriptsize
\begin{verbatim}
<% if @customer.orders.empty? %>
  No Orders Found
<% end %>
\end{verbatim}}
\end{minipage}
\\ \\

\paragraph{4.3.1.9 \texttt{\emph{collection}.size}}\\ \\\\

The \texttt{\emph{collection}.size} method returns the number of objects in the collection.
\\ \\
\begin{minipage}{\textwidth}{\scriptsize
\begin{verbatim}
@order_count = @customer.orders.size
\end{verbatim}}
\end{minipage}
\\ \\

\paragraph{4.3.1.10 \texttt{\emph{collection}.find(…)}}\\ \\\\

The \texttt{\emph{collection}.find} method finds objects within the collection.

\noindent It uses the same syntax and options as \texttt{ActiveRecord::Base.find}.
\\ \\
\begin{minipage}{\textwidth}{\scriptsize
\begin{verbatim}
@open_orders = @customer.orders.where(:open => 1)
\end{verbatim}}
\end{minipage}
\\ \\

\paragraph{4.3.1.11 \texttt{\emph{collection}.where(…)}}\\ \\\\

The \texttt{\emph{collection}.where} method finds objects within  the collection based on the conditions supplied but the objects are  loaded lazily meaning that the database is queried only when the  object(s) are accessed.
\\ \\
\begin{minipage}{\textwidth}{\scriptsize
\begin{verbatim}
@open_orders = @customer.orders.where(:open => true)
# No query yet
@open_order = @open_orders.first 
# Now the database will be queried
\end{verbatim}}
\end{minipage}
\\ \\

\paragraph{4.3.1.12 \texttt{\emph{collection}.exists?(…)}}\\ \\\\

The \texttt{\emph{collection}.exists?} method checks whether an  object meeting the supplied conditions exists in the collection. It uses  the same syntax and options as \texttt{ActiveRecord::Base.exists?}.

\paragraph{4.3.1.13 \texttt{\emph{collection}.build(attributes = \{\}, …)}}\\ \\\\

The \texttt{\emph{collection}.build} method returns one or more new  objects of the associated type. These objects will be instantiated from  the passed attributes, and the link through their foreign key will be  created, but the associated objects will \emph{not} yet be saved.
\\ \\
\begin{minipage}{\textwidth}{\scriptsize
\begin{verbatim}
@order = @customer.orders.build(:order_date => Time.now,
  :order_number => "A12345")
\end{verbatim}}
\end{minipage}
\\ \\

\paragraph{4.3.1.14 \texttt{\emph{collection}.create(attributes = \{\})}}\\ \\\\

The \texttt{\emph{collection}.create} method returns a new object  of the associated type. This object will be instantiated from the passed  attributes, the link through its foreign key will be created, and, once  it passes all of the validations specified on the associated model, the  associated object \emph{will} be saved.
\\ \\
\begin{minipage}{\textwidth}{\scriptsize
\begin{verbatim}
@order = @customer.orders.create(:order_date => Time.now,
  :order_number => "A12345")
\end{verbatim}}
\end{minipage}
\\ \\

\subsubsection{ Options for \texttt{has\_many}}

While Rails uses intelligent defaults that will work well in most  situations, there may be times when you want to customize the behavior  of the \texttt{has\_many} association reference. Such customizations can  easily be accomplished by passing options when you create the  association. For example, this assocation uses two such options:
\\ \\
\begin{minipage}{\textwidth}{\scriptsize
\begin{verbatim}
class Customer < ActiveRecord::Base
has_many :orders, :dependent => :delete_all, :validate => :false
end
\end{verbatim}}
\end{minipage}
\\ \\

The \texttt{has\_many} association supports these options:
\begin{itemize}
	\item \texttt{:as}
	\item \texttt{:autosave}
	\item \texttt{:class\_name}
	\item \texttt{:conditions}
	\item \texttt{:counter\_sql}
	\item \texttt{:dependent}
	\item \texttt{:extend}
	\item \texttt{:finder\_sql}
	\item \texttt{:foreign\_key}
	\item \texttt{:group}
	\item \texttt{:include}
	\item \texttt{:inverse\_of}
	\item \texttt{:limit}
	\item \texttt{:offset}
	\item \texttt{:order}
	\item \texttt{:primary\_key}
	\item \texttt{:readonly}
	\item \texttt{:select}
	\item \texttt{:source}
	\item \texttt{:source\_type}
	\item \texttt{:through}
	\item \texttt{:uniq}
	\item \texttt{:validate}
\end{itemize}

\paragraph{4.3.2.1 \texttt{:as}}\\ \\\\

Setting the \texttt{:as} option indicates that this is a polymorphic association, as discussed \hyperlink{polymorphic-associations}{earlier in this guide}.

\paragraph{4.3.2.2 \texttt{:autosave}}\\ \\\\

If you set the \texttt{:autosave} option to \texttt{true}, Rails will save any loaded members and destroy members that are marked for destruction whenever you save the parent object.

\paragraph{4.3.2.3 \texttt{:class\_name}}\\ \\\\

If the name of the other model cannot be derived from the association name, you can use the \texttt{:class\_name}  option to supply the model name. For example, if a customer has many  orders, but the actual name of the model containing orders is \texttt{Transaction}, you’d set things up this way:
\\ \\
\begin{minipage}{\textwidth}{\scriptsize
\begin{verbatim}
class Customer < ActiveRecord::Base
  has_many :orders, :class_name => "Transaction"
end
\end{verbatim}}
\end{minipage}
\\ \\

\paragraph{4.3.2.4 \texttt{:conditions}}\\ \\\\

The \texttt{:conditions} option lets you specify the conditions that the associated object must meet (in the syntax used by an SQL\texttt{WHERE} clause).
\\ \\
\begin{minipage}{\textwidth}{\scriptsize
\begin{verbatim}
class Customer < ActiveRecord::Base
  has_many :confirmed_orders, :class_name => "Order",
    :conditions => "confirmed = 1"
end
\end{verbatim}}
\end{minipage}
\\ \\

You can also set conditions via a hash:
\\ \\
\begin{minipage}{\textwidth}{\scriptsize
\begin{verbatim}
class Customer < ActiveRecord::Base
  has_many :confirmed_orders, :class_name => "Order",
    :conditions => { :confirmed => true }
end
\end{verbatim}}
\end{minipage}
\\ \\

If you use a hash-style \texttt{:conditions} option, then record creation via this association will be automatically scoped using the hash. In this case, using \texttt{@customer.confirmed\_orders.create} or \texttt{@customer.confirmed\_orders.build} will create orders where the confirmed column has the value \texttt{true}.

If you need to evaluate conditions dynamically at runtime, use a proc:
\\ \\
\begin{minipage}{\textwidth}{\scriptsize
\begin{verbatim}
class Customer < ActiveRecord::Base
  has_many :latest_orders, :class_name => "Order",
  :conditions => proc { ["orders.created_at > ?", 10.hours.ago] }
end
\end{verbatim}}
\end{minipage}
\\ \\

\paragraph{4.3.2.5 \texttt{:counter\_sql}}\\ \\\\

Normally Rails automatically generates the proper SQL to count the association members. With the \texttt{:counter\_sql} option, you can specify a complete SQL statement to count them yourself.

If you specify \texttt{:finder\_sql} but not \texttt{:counter\_sql}, then the counter SQL will be generated by substituting \texttt{SELECT COUNT(*) FROM} for the \texttt{SELECT ... FROM} clause of your \texttt{:finder\_sql} statement.

\paragraph{4.3.2.6 \texttt{:dependent}}\\ \\\\

If you set the \texttt{:dependent} option to \texttt{:destroy}, then deleting this object will call the \texttt{destroy} method on the associated objects to delete those objects. If you set the \texttt{:dependent} option to \texttt{:delete\_all}, then deleting this object will delete the associated objects \emph{without} calling their \texttt{destroy} method. If you set the \texttt{:dependent} option to \texttt{:nullify}, then deleting this object will set the foreign key in the associated objects to \texttt{NULL}.

This option is ignored when you use the \texttt{:through} option on the association.

\paragraph{4.3.2.7 \texttt{:extend}}\\ \\\\

The \texttt{:extend} option specifies a named module to extend the association proxy. Association extensions are discussed in detail \hyperlink{association-extensions}{later in this guide}.

\paragraph{4.3.2.8 \texttt{:finder\_sql}}\\ \\\\

Normally Rails automatically generates the proper SQL to fetch the association members. With the \texttt{:finder\_sql} option, you can specify a complete SQL statement to fetch them yourself. If fetching objects requires complex multi-table SQL, this may be necessary.

\paragraph{4.3.2.9 \texttt{:foreign\_key}}\\ \\\\

By convention, Rails assumes that the column used to hold the foreign  key on the other model is the name of this model with the suffix \texttt{\_id} added. The \texttt{:foreign\_key} option lets you set the name of the foreign key directly:
\\ \\
\begin{minipage}{\textwidth}{\scriptsize
\begin{verbatim}
class Customer < ActiveRecord::Base
  has_many :orders, :foreign_key => "cust_id"
end
\end{verbatim}}
\end{minipage}
\\ \\

In any case, Rails will not create foreign key columns for you. You need to explicitly define them as part of your migrations.

\paragraph{4.3.2.10 \texttt{:group}}\\ \\\\

The \texttt{:group} option supplies an attribute name to group the result set by, using a \texttt{GROUP BY} clause in the finder SQL.
\\ \\
\begin{minipage}{\textwidth}{\scriptsize
\begin{verbatim}
class Customer < ActiveRecord::Base
has_many :line_items, :through => :orders, :group => "orders.id"
end
\end{verbatim}}
\end{minipage}
\\ \\

\paragraph{4.3.2.11 \texttt{:include}}\\ \\\\

You can use the \texttt{:include} option to specify second-order  associations that should be eager-loaded when this association is used.  For example, consider these models:
\\ \\
\begin{minipage}{\textwidth}{\scriptsize
\begin{verbatim}
class Customer < ActiveRecord::Base
  has_many :orders
end
 
class Order < ActiveRecord::Base
  belongs_to :customer
  has_many :line_items
end
 
class LineItem < ActiveRecord::Base
  belongs_to :order
end
\end{verbatim}}
\end{minipage}
\\ \\

If you frequently retrieve line items directly from customers (\texttt{@customer.orders.line\_items}), then you can make your code somewhat more efficient by including line items in the association from customers to orders:
\\ \\
\begin{minipage}{\textwidth}{\scriptsize
\begin{verbatim}
class Customer < ActiveRecord::Base
  has_many :orders, :include => :line_items
end
 
class Order < ActiveRecord::Base
  belongs_to :customer
  has_many :line_items
end
 
class LineItem < ActiveRecord::Base
  belongs_to :order
end
\end{verbatim}}
\end{minipage}
\\ \\

\paragraph{4.3.2.12 \texttt{:inverse\_of}}\\ \\\\

The \texttt{:inverse\_of} option specifies the name of the \texttt{belongs\_to} association that is the inverse of this association. Does not work in combination with the \texttt{:through} or \texttt{:as} options.
\\ \\
\begin{minipage}{\textwidth}{\scriptsize
\begin{verbatim}
class Customer < ActiveRecord::Base
  has_many :orders, :inverse_of => :customer
end
 
class Order < ActiveRecord::Base
  belongs_to :customer, :inverse_of => :orders
end
\end{verbatim}}
\end{minipage}
\\ \\

\paragraph{4.3.2.13 \texttt{:limit}}\\ \\\\

The \texttt{:limit} option lets you restrict the total number of objects that will be fetched through an association.
\\ \\
\begin{minipage}{\textwidth}{\scriptsize
\begin{verbatim}
class Customer < ActiveRecord::Base
  has_many :recent_orders, :class_name => "Order",
    :order => "order_date DESC", :limit => 100
end
\end{verbatim}}
\end{minipage}
\\ \\

\paragraph{4.3.2.14 \texttt{:offset}}\\ \\\\

The \texttt{:offset} option lets you specify the starting offset for fetching objects via an association. For example, if you set \texttt{:offset =$>$ 11}, it will skip the first 11 records.

\paragraph{4.3.2.15 \texttt{:order}}\\ \\\\

The \texttt{:order} option dictates the order in which associated objects will be received (in the syntax used by an SQL\texttt{ORDER BY} clause).
\\ \\
\begin{minipage}{\textwidth}{\scriptsize
\begin{verbatim}
class Customer < ActiveRecord::Base
  has_many :orders, :order => "date_confirmed DESC"
end
\end{verbatim}}
\end{minipage}
\\ \\

\paragraph{4.3.2.16 \texttt{:primary\_key}}\\ \\\\

By convention, Rails assumes that the column used to hold the primary key of the association is \texttt{id}. You can override this and explicitly specify the primary key with the \texttt{:primary\_key} option.

\paragraph{4.3.2.17 \texttt{:readonly}}\\ \\\\

If you set the \texttt{:readonly} option to \texttt{true}, then the associated objects will be read-only when retrieved via the association.

\paragraph{4.3.2.18 \texttt{:select}}\\ \\\\

The \texttt{:select} option lets you override the SQL\texttt{SELECT} clause that is used to retrieve data about the associated objects. By default, Rails retrieves all columns.

If you specify your own \texttt{:select}, be sure to include the primary key and foreign key columns of the associated model. If you do not, Rails will throw an error.

\paragraph{4.3.2.19 \texttt{:source}}\\ \\\\

The \texttt{:source} option specifies the source association name for a \texttt{has\_many :through}  association. You only need to use this option if the name of the source  association cannot be automatically inferred from the association name.

\paragraph{4.3.2.20 \texttt{:source\_type}}\\ \\\\

The \texttt{:source\_type} option specifies the source association type for a \texttt{has\_many :through} association that proceeds through a polymorphic association.

\paragraph{4.3.2.21 \texttt{:through}}\\ \\\\

The \texttt{:through} option specifies a join model through which to perform the query. \texttt{has\_many :through} associations provide a way to implement many-to-many relationships, as discussed \hyperlink{the-has_many-through-association}{earlier in this guide}.

\paragraph{4.3.2.22 \texttt{:uniq}}\\ \\\\

Set the \texttt{:uniq} option to true to keep the collection free of duplicates. This is mostly useful together with the \texttt{:through} option.
\\ \\
\begin{minipage}{\textwidth}{\scriptsize
\begin{verbatim}
class Person < ActiveRecord::Base
  has_many :readings
  has_many :posts, :through => :readings
end
 
person = Person.create(:name => 'john')
post   = Post.create(:name => 'a1')
person.posts << post
person.posts << post
person.posts.inspect 
# => [#<Post id: 5, name: "a1">, #<Post id: 5, name: "a1">]
Reading.all.inspect  
# => [#<Reading id: 12, person_id: 5, post_id: 5>,
      #<Reading id: 13, person_id: 5, post_id: 5>]
\end{verbatim}}
\end{minipage}
\\ \\

In the above case there are two readings and \texttt{person.posts} brings out both of them even though these records are pointing to the same post.

Now let’s set \texttt{:uniq} to true:
\\ \\
\begin{minipage}{\textwidth}{\scriptsize
\begin{verbatim}
class Person
  has_many :readings
  has_many :posts, :through => :readings, :uniq => true
end
 
person = Person.create(:name => 'honda')
post   = Post.create(:name => 'a1')
person.posts << post
person.posts << post
person.posts.inspect 
# => [#<Post id: 7, name: "a1">]
Reading.all.inspect  
# => [#<Reading id: 16, person_id: 7, post_id: 7>, 
      #<Reading id: 17, person_id: 7, post_id: 7>]
\end{verbatim}}
\end{minipage}
\\ \\

In the above case there are still two readings. However \texttt{person.posts} shows only one post because the collection loads only unique records.

\paragraph{4.3.2.23 \texttt{:validate}}\\ \\\\

If you set the \texttt{:validate} option to \texttt{false}, then associated objects will not be validated whenever you save this object. By default, this is \texttt{true}: associated objects will be validated when this object is saved.

\subsubsection{ When are Objects Saved?}

When you assign an object to a \texttt{has\_many} association, that  object is automatically saved (in order to update its foreign key). If  you assign multiple objects in one statement, then they are all saved.

If any of these saves fails due to validation errors, then the assignment statement returns \texttt{false} and the assignment itself is cancelled.

If the parent object (the one declaring the \texttt{has\_many} association) is unsaved (that is, \texttt{new\_record?} returns \texttt{true})  then the child objects are not saved when they are added. All unsaved  members of the association will automatically be saved when the parent  is saved.

If you want to assign an object to a \texttt{has\_many} association without saving the object, use the \texttt{\emph{collection}.build} method.

\subsection{ \texttt{has\_and\_belongs\_to\_many} Association Reference}

The \texttt{has\_and\_belongs\_to\_many} association creates a  many-to-many relationship with another model. In database terms, this  associates two classes via an intermediate join table that includes  foreign keys referring to each of the classes.

\subsubsection{ Methods Added by \texttt{has\_and\_belongs\_to\_many}}

When you declare a \texttt{has\_and\_belongs\_to\_many} association, the declaring class automatically gains 13 methods related to the association:
\begin{itemize}
	\item \texttt{\emph{collection}(force\_reload = false)}
	\item \texttt{\emph{collection}$<$$<$(object, …)}
	\item \texttt{\emph{collection}.delete(object, …)}
	\item \texttt{\emph{collection}=objects}
	\item \texttt{\emph{collection\_singular}\_ids}
	\item \texttt{\emph{collection\_singular}\_ids=ids}
	\item \texttt{\emph{collection}.clear}
	\item \texttt{\emph{collection}.empty?}
	\item \texttt{\emph{collection}.size}
	\item \texttt{\emph{collection}.find(…)}
	\item \texttt{\emph{collection}.where(…)}
	\item \texttt{\emph{collection}.exists?(…)}
	\item \texttt{\emph{collection}.build(attributes = \{\})}
	\item \texttt{\emph{collection}.create(attributes = \{\})}
\end{itemize}

In all of these methods, \texttt{\emph{collection}} is replaced with the symbol passed as the first argument to \texttt{has\_and\_belongs\_to\_many}, and \texttt{\emph{collection\_singular}} is replaced with the singularized version of that symbol. For example, given the declaration:
\\ \\
\begin{minipage}{\textwidth}{\scriptsize
\begin{verbatim}
class Part < ActiveRecord::Base
  has_and_belongs_to_many :assemblies
end
\end{verbatim}}
\end{minipage}
\\ \\

Each instance of the part model will have these methods:
\\ \\
\begin{minipage}{\textwidth}{\scriptsize
\begin{verbatim}
assemblies(force_reload = false)
assemblies<<(object, ...)
assemblies.delete(object, ...)
assemblies=objects
assembly_ids
assembly_ids=ids
assemblies.clear
assemblies.empty?
assemblies.size
assemblies.find(...)
assemblies.where(...)
assemblies.exists?(...)
assemblies.build(attributes = {}, ...)
assemblies.create(attributes = {})
\end{verbatim}}
\end{minipage}
\\ \\

\paragraph{4.4.1.1 Additional Column Methods}\\ \\\\

If the join table for a \texttt{has\_and\_belongs\_to\_many} association  has additional columns beyond the two foreign keys, these columns will  be added as attributes to records retrieved via that association.  Records returned with additional attributes will always be read-only,  because Rails cannot save changes to those attributes.

The use of extra attributes on the join table in a \\ has\_and\_belongs\_to\_many  association is deprecated. If you require this sort of complex behavior  on the table that joins two models in a many-to-many relationship, you  should use a \texttt{has\_many :through} association instead of \texttt{has\_and\_belongs\_to\_many}.

\paragraph{4.4.1.2 \texttt{\emph{collection}(force\_reload = false)}}\\ \\\\

The \texttt{\emph{collection}} method returns an array of all of the associated objects. If there are no associated objects, it returns an empty array.
\\ \\
\begin{minipage}{\textwidth}{\scriptsize
\begin{verbatim}
@assemblies = @part.assemblies
\end{verbatim}}
\end{minipage}
\\ \\

\paragraph{4.4.1.3 \texttt{\emph{collection}$<$$<$(object, …)}}\\ \\\\

The \texttt{\emph{collection}$<$$<$} method adds one or more objects to the collection by creating records in the join table.
\\ \\
\begin{minipage}{\textwidth}{\scriptsize
\begin{verbatim}
@part.assemblies << @assembly1
\end{verbatim}}
\end{minipage}
\\ \\

This method is aliased as \emph{collection}.concat and \emph{collection}.push.

\paragraph{4.4.1.4 \texttt{\emph{collection}.delete(object, …)}}\\ \\\\

The \texttt{\emph{collection}.delete} method removes one or more  objects from the collection by deleting records in the join table. This  does not destroy the objects.
\\ \\
\begin{minipage}{\textwidth}{\scriptsize
\begin{verbatim}
@part.assemblies.delete(@assembly1)
\end{verbatim}}
\end{minipage}
\\ \\

\paragraph{4.4.1.5 \texttt{\emph{collection}=objects}}\\ \\\\

The \texttt{\emph{collection}=} method makes the collection contain only the supplied objects, by adding and deleting as appropriate.

\paragraph{4.4.1.6 \texttt{\emph{collection\_singular}\_ids}}\\ \\\\

The \texttt{\emph{collection\_singular}\_ids} method returns an array of the ids of the objects in the collection.
\\ \\
\begin{minipage}{\textwidth}{\scriptsize
\begin{verbatim}
@assembly_ids = @part.assembly_ids
\end{verbatim}}
\end{minipage}
\\ \\

\paragraph{4.4.1.7 \texttt{\emph{collection\_singular}\_ids=ids}}\\ \\\\

The \texttt{\emph{collection\_singular}\_ids=} method makes the  collection contain only the objects identified by the supplied primary  key values, by adding and deleting as appropriate.

\paragraph{4.4.1.8 \texttt{\emph{collection}.clear}}\\ \\\\

The \texttt{\emph{collection}.clear} method removes every object  from the collection by deleting the rows from the joining table. This  does not destroy the associated objects.

\paragraph{4.4.1.9 \texttt{\emph{collection}.empty?}}\\ \\\\

The \texttt{\emph{collection}.empty?} method returns \texttt{true} if the collection does not contain any associated objects.
\\ \\
\begin{minipage}{\textwidth}{\scriptsize
\begin{verbatim}
<% if @part.assemblies.empty? %>
  This part is not used in any assemblies
<% end %>
\end{verbatim}}
\end{minipage}
\\ \\

\paragraph{4.4.1.10 \texttt{\emph{collection}.size}}\\ \\\\

The \texttt{\emph{collection}.size} method returns the number of objects in the collection.
\\ \\
\begin{minipage}{\textwidth}{\scriptsize
\begin{verbatim}
@assembly_count = @part.assemblies.size
\end{verbatim}}
\end{minipage}
\\ \\

\paragraph{4.4.1.11 \texttt{\emph{collection}.find(…)}}\\ \\\\

The \texttt{\emph{collection}.find} method finds objects within the collection. It uses the same syntax and options as \texttt{ActiveRecord::Base.find}. It also adds the additional condition that the object must be in the collection.
\\ \\
\begin{minipage}{\textwidth}{\scriptsize
\begin{verbatim}
@new_assemblies = @part.assemblies.all(
  :conditions => ["created_at > ?", 2.days.ago])
\end{verbatim}}
\end{minipage}
\\ \\
Beginning with Rails 3, supplying options to the ActiveRecord::Base.find method is discouraged. Use \emph{collection}.where instead when you need to pass conditions.

\paragraph{4.4.1.12 \texttt{\emph{collection}.where(…)}}\\ \\\\

The \texttt{\emph{collection}.where} method finds objects within  the collection based on the conditions supplied but the objects are  loaded lazily meaning that the database is queried only when the  object(s) are accessed. It also adds the additional condition that the  object must be in the collection.
\\ \\
\begin{minipage}{\textwidth}{\scriptsize
\begin{verbatim}
@new_assemblies = @part.assemblies.where("created_at > ?", 2.days.ago)
\end{verbatim}}
\end{minipage}
\\ \\

\paragraph{4.4.1.13 \texttt{\emph{collection}.exists?(…)}}\\ \\\\

The \texttt{\emph{collection}.exists?} method checks whether an  object meeting the supplied conditions exists in the collection. It uses  the same syntax and options as \texttt{ActiveRecord::Base.exists?}.

\paragraph{4.4.1.14 \texttt{\emph{collection}.build(attributes = \{\})}}\\ \\\\

The \texttt{\emph{collection}.build} method returns a new object of  the associated type. This object will be instantiated from the passed  attributes, and the link through the join table will be created, but the  associated object will \emph{not} yet be saved.
\\ \\
\begin{minipage}{\textwidth}{\scriptsize
\begin{verbatim}
@assembly = @part.assemblies.build(
  {:assembly_name => "Transmission housing"})
\end{verbatim}}
\end{minipage}
\\ \\

\paragraph{4.4.1.15 \texttt{\emph{collection}.create(attributes = \{\})}}\\ \\\\

The \texttt{\emph{collection}.create} method returns a new object  of the associated type. This object will be instantiated from the passed  attributes, the link through the join table will be created, and, once  it passes all of the validations specified on the associated model, the  associated object \emph{will} be saved.
\\ \\
\begin{minipage}{\textwidth}{\scriptsize
\begin{verbatim}
@assembly = @part.assemblies.create(
  {:assembly_name => "Transmission housing"})
\end{verbatim}}
\end{minipage}
\\ \\

\subsubsection{ Options for \texttt{has\_and\_belongs\_to\_many}}

While Rails uses intelligent defaults that will work well in most  situations, there may be times when you want to customize the behavior  of the \texttt{has\_and\_belongs\_to\_many} association reference. Such  customizations can easily be accomplished by passing options when you  create the association. For example, this assocation uses two such  options:
\\ \\
\begin{minipage}{\textwidth}{\scriptsize
\begin{verbatim}
class Parts < ActiveRecord::Base
  has_and_belongs_to_many :assemblies, :uniq => true,
    :read_only => true
end
\end{verbatim}}
\end{minipage}
\\ \\

The \texttt{has\_and\_belongs\_to\_many} association supports these options:
\begin{itemize}
	\item \texttt{:association\_foreign\_key}
	\item \texttt{:autosave}
	\item \texttt{:class\_name}
	\item \texttt{:conditions}
	\item \texttt{:counter\_sql}
	\item \texttt{:delete\_sql}
	\item \texttt{:extend}
	\item \texttt{:finder\_sql}
	\item \texttt{:foreign\_key}
	\item \texttt{:group}
	\item \texttt{:include}
	\item \texttt{:insert\_sql}
	\item \texttt{:join\_table}
	\item \texttt{:limit}
	\item \texttt{:offset}
	\item \texttt{:order}
	\item \texttt{:readonly}
	\item \texttt{:select}
	\item \texttt{:uniq}
	\item \texttt{:validate}
\end{itemize}

\paragraph{4.4.2.1 \texttt{:association\_foreign\_key}}\\ \\\\

By convention, Rails assumes that the column in the join table used  to hold the foreign key pointing to the other model is the name of that  model with the suffix \texttt{\_id} added. The \texttt{:association\_foreign\_key} option lets you set the name of the foreign key directly:

The \texttt{:foreign\_key} and \texttt{:association\_foreign\_key} options are useful when setting up a many-to-many self-join. For example:
\\ \\
\begin{minipage}{\textwidth}{\scriptsize
\begin{verbatim}
class User < ActiveRecord::Base
  has_and_belongs_to_many :friends, :class_name => "User",
    :foreign_key => "this_user_id",
    :association_foreign_key => "other_user_id"
end
\end{verbatim}}
\end{minipage}
\\ \\

\paragraph{4.4.2.2 \texttt{:autosave}}\\ \\\\

If you set the \texttt{:autosave} option to \texttt{true}, Rails will save any loaded members and destroy members that are marked for destruction whenever you save the parent object.

\paragraph{4.4.2.3 \texttt{:class\_name}}\\ \\\\

If the name of the other model cannot be derived from the association name, you can use the \texttt{:class\_name}  option to supply the model name. For example, if a part has many  assemblies, but the actual name of the model containing assemblies is \texttt{Gadget}, you’d set things up this way:
\\ \\
\begin{minipage}{\textwidth}{\scriptsize
\begin{verbatim}
class Parts < ActiveRecord::Base
  has_and_belongs_to_many :assemblies, :class_name => "Gadget"
end
\end{verbatim}}
\end{minipage}
\\ \\

\paragraph{4.4.2.4 \texttt{:conditions}}\\ \\\\

The \texttt{:conditions} option lets you specify the conditions that the associated object must meet (in the syntax used by an SQL\texttt{WHERE} clause).
\\ \\
\begin{minipage}{\textwidth}{\scriptsize
\begin{verbatim}
class Parts < ActiveRecord::Base
  has_and_belongs_to_many :assemblies,
    :conditions => "factory = 'Seattle'"
end
\end{verbatim}}
\end{minipage}
\\ \\

You can also set conditions via a hash:
\\ \\
\begin{minipage}{\textwidth}{\scriptsize
\begin{verbatim}
class Parts < ActiveRecord::Base
  has_and_belongs_to_many :assemblies,
    :conditions => { :factory => 'Seattle' }
end
\end{verbatim}}
\end{minipage}
\\ \\

If you use a hash-style \texttt{:conditions} option, then record creation via this association will be automatically scoped using the hash. In this case, using \texttt{@parts.assemblies.create} or \texttt{@parts.assemblies.build} will create orders where the \texttt{factory} column has the value “Seattle”.

\paragraph{4.4.2.5 \texttt{:counter\_sql}}\\ \\\\

Normally Rails automatically generates the proper SQL to count the association members. With the \texttt{:counter\_sql} option, you can specify a complete SQL statement to count them yourself.

If you specify \texttt{:finder\_sql} but not \texttt{:counter\_sql}, then the counter SQL will be generated by substituting \texttt{SELECT COUNT(*) FROM} for the \texttt{SELECT ... FROM} clause of your \texttt{:finder\_sql} statement.

\paragraph{4.4.2.6 \texttt{:delete\_sql}}\\ \\\\

Normally Rails automatically generates the proper SQL to remove links between the associated classes. With the \texttt{:delete\_sql} option, you can specify a complete SQL statement to delete them yourself.

\paragraph{4.4.2.7 \texttt{:extend}}\\ \\\\

The \texttt{:extend} option specifies a named module to extend the association proxy. Association extensions are discussed in detail \hyperlink{association-extensions}{later in this guide}.

\paragraph{4.4.2.8 \texttt{:finder\_sql}}\\ \\\\

Normally Rails automatically generates the proper SQL to fetch the association members. With the \texttt{:finder\_sql} option, you can specify a complete SQL statement to fetch them yourself. If fetching objects requires complex multi-table SQL, this may be necessary.

\paragraph{4.4.2.9 \texttt{:foreign\_key}}\\ \\\\

By convention, Rails assumes that the column in the join table used  to hold the foreign key pointing to this model is the name of this model  with the suffix \texttt{\_id} added. The \texttt{:foreign\_key} option lets you set the name of the foreign key directly:
\\ \\
\begin{minipage}{\textwidth}{\scriptsize
\begin{verbatim}
class User < ActiveRecord::Base
  has_and_belongs_to_many :friends, :class_name => "User",
    :foreign_key => "this_user_id",
    :association_foreign_key => "other_user_id"
end
\end{verbatim}}
\end{minipage}
\\ \\

\paragraph{4.4.2.10 \texttt{:group}}\\ \\\\

The \texttt{:group} option supplies an attribute name to group the result set by, using a \texttt{GROUP BY} clause in the finder SQL.
\\ \\
\begin{minipage}{\textwidth}{\scriptsize
\begin{verbatim}
class Parts < ActiveRecord::Base
  has_and_belongs_to_many :assemblies, :group => "factory"
end
\end{verbatim}}
\end{minipage}
\\ \\

\paragraph{4.4.2.11 \texttt{:include}}\\ \\\\

You can use the \texttt{:include} option to specify second-order associations that should be eager-loaded when this association is used.

\paragraph{4.4.2.12 \texttt{:insert\_sql}}\\ \\\\

Normally Rails automatically generates the proper SQL to create links between the associated classes. With the \texttt{:insert\_sql} option, you can specify a complete SQL statement to insert them yourself.

\paragraph{4.4.2.13 \texttt{:join\_table}}\\ \\\\

If the default name of the join table, based on lexical ordering, is not what you want, you can use the \texttt{:join\_table} option to override the default.

\paragraph{4.4.2.14 \texttt{:limit}}\\ \\\\

The \texttt{:limit} option lets you restrict the total number of objects that will be fetched through an association.
\\ \\
\begin{minipage}{\textwidth}{\scriptsize
\begin{verbatim}
class Parts < ActiveRecord::Base
  has_and_belongs_to_many :assemblies, :order => "created_at DESC",
    :limit => 50
end
\end{verbatim}}
\end{minipage}
\\ \\

\paragraph{4.4.2.15 \texttt{:offset}}\\ \\\\

The \texttt{:offset} option lets you specify the starting offset for fetching objects via an association. For example, if you set \texttt{:offset =$>$ 11}, it will skip the first 11 records.

\paragraph{4.4.2.16 \texttt{:order}}\\ \\\\

The \texttt{:order} option dictates the order in which associated objects will be received (in the syntax used by an SQL\texttt{ORDER BY} clause).
\\ \\
\begin{minipage}{\textwidth}{\scriptsize
\begin{verbatim}
class Parts < ActiveRecord::Base
  has_and_belongs_to_many :assemblies, :order => "assembly_name ASC"
end
\end{verbatim}}
\end{minipage}
\\ \\

\paragraph{4.4.2.17 \texttt{:readonly}}\\ \\\\

If you set the \texttt{:readonly} option to \texttt{true}, then the associated objects will be read-only when retrieved via the association.

\paragraph{4.4.2.18 \texttt{:select}}\\ \\\\

The \texttt{:select} option lets you override the SQL\texttt{SELECT} clause that is used to retrieve data about the associated objects. By default, Rails retrieves all columns.

\paragraph{4.4.2.19 \texttt{:uniq}}\\ \\\\

Specify the \texttt{:uniq =$>$ true} option to remove duplicates from the collection.

\paragraph{4.4.2.20 \texttt{:validate}}\\ \\\\

If you set the \texttt{:validate} option to \texttt{false}, then associated objects will not be validated whenever you save this object. By default, this is \texttt{true}: associated objects will be validated when this object is saved.

\subsubsection{ When are Objects Saved?}

When you assign an object to a \texttt{has\_and\_belongs\_to\_many}  association, that object is automatically saved (in order to update the  join table). If you assign multiple objects in one statement, then they  are all saved.

If any of these saves fails due to validation errors, then the assignment statement returns \texttt{false} and the assignment itself is cancelled.

If the parent object (the one declaring \\ the has\_and\_belongs\_to\_many association) is unsaved (that is, \texttt{new\_record?} returns \texttt{true})  then the child objects are not saved when they are added. All unsaved  members of the association will automatically be saved when the parent  is saved.

If you want to assign an object to a \texttt{has\_and\_belongs\_to\_many} association without saving the object, use the \texttt{\emph{collection}.build} method.

\subsection{ Association Callbacks}

Normal callbacks hook into the life cycle of Active Record objects,  allowing you to work with those objects at various points. For example,  you can use a \texttt{:before\_save} callback to cause something to happen just before an object is saved.

Association callbacks are similar to normal callbacks, but they are  triggered by events in the life cycle of a collection. There are four  available association callbacks:
\begin{itemize}
	\item \texttt{before\_add}
	\item \texttt{after\_add}
	\item \texttt{before\_remove}
	\item \texttt{after\_remove}
\end{itemize}

You define association callbacks by adding options to the association declaration. For example:
\\ \\
\begin{minipage}{\textwidth}{\scriptsize
\begin{verbatim}
class Customer < ActiveRecord::Base
  has_many :orders, :before_add => :check_credit_limit
 
  def check_credit_limit(order)
    ...
  end
end
\end{verbatim}}
\end{minipage}
\\ \\

Rails passes the object being added or removed to the callback.

You can stack callbacks on a single event by passing them as an array:
\\ \\
\begin{minipage}{\textwidth}{\scriptsize
\begin{verbatim}
class Customer < ActiveRecord::Base
  has_many :orders,
    :before_add => [:check_credit_limit, 
                    :calculate_shipping_charges]
 
  def check_credit_limit(order)
    ...
  end
 
  def calculate_shipping_charges(order)
    ...
  end
end
\end{verbatim}}
\end{minipage}
\\ \\

If a \texttt{before\_add} callback throws an exception, the object does not get added to the collection. Similarly, if a \texttt{before\_remove} callback throws an exception, the object does not get removed from the collection.

\subsection{ Association Extensions}

You’re not limited to the functionality that Rails automatically  builds into association proxy objects. You can also extend these objects  through anonymous modules, adding new finders, creators, or other  methods. For example:
\\ \\
\begin{minipage}{\textwidth}{\scriptsize
\begin{verbatim}
class Customer < ActiveRecord::Base
  has_many :orders do
    def find_by_order_prefix(order_number)
      find_by_region_id(order_number[0..2])
    end
  end
end
\end{verbatim}}
\end{minipage}
\\ \\

If you have an extension that should be shared by many associations, you can use a named extension module. For example:
\\ \\
\begin{minipage}{\textwidth}{\scriptsize
\begin{verbatim}
module FindRecentExtension
  def find_recent
    where("created_at > ?", 5.days.ago)
  end
end
 
class Customer < ActiveRecord::Base
  has_many :orders, :extend => FindRecentExtension
end
 
class Supplier < ActiveRecord::Base
  has_many :deliveries, :extend => FindRecentExtension
end
\end{verbatim}}
\end{minipage}
\\ \\

To include more than one extension module in a single association, specify an array of modules:
\\ \\
\begin{minipage}{\textwidth}{\scriptsize
\begin{verbatim}
class Customer < ActiveRecord::Base
  has_many :orders,
    :extend => [FindRecentExtension, FindActiveExtension]
end
\end{verbatim}}
\end{minipage}
\\ \\

Extensions can refer to the internals of the association proxy using these three attributes of the \texttt{proxy\_association} accessor:
\begin{itemize}
	\item \texttt{proxy\_association.owner} returns the object that the association is a part of.
	\item \texttt{proxy\_association.reflection} returns the reflection object that describes the association.
	\item \texttt{proxy\_association.target} returns the associated object for \texttt{belongs\_to} or \texttt{has\_one}, or the collection of associated objects for \texttt{has\_many} or \texttt{has\_and\_belongs\_to\_many}.
\end{itemize}

\chapter{Active Record Query Interface}

This guide covers different ways to retrieve data from the database  using Active Record. By referring to this guide, you will be able to:
\begin{itemize}
	\item Find records using a variety of methods and conditions
	\item Specify the order, retrieved attributes, grouping, and other properties of the found records
	\item Use eager loading to reduce the number of database queries needed for data retrieval
	\item Use dynamic finders methods
	\item Check for the existence of particular records
	\item Perform various calculations on Active Record models
	\item Run EXPLAIN on relations
\end{itemize}

This Guide is based on Rails 3.0. Some of the code shown here will not work in other versions of Rails.

If you’re used to using raw SQL to find  database records, then you will generally find that there are better  ways to carry out the same operations in Rails. Active Record insulates  you from the need to use SQL in most cases.

Code examples throughout this guide will refer to one or more of the following models:

All of the following models use id as the primary key, unless specified otherwise.
\begin{code}
class Client < ActiveRecord::Base
  has_one :address
  has_many :orders
  has_and_belongs_to_many :roles
end
\end{code}
\begin{code}
class Address < ActiveRecord::Base
  belongs_to :client
end
\end{code}
\begin{code}
class Order < ActiveRecord::Base
  belongs_to :client, :counter_cache => true
end
\end{code}
\begin{code}
class Role < ActiveRecord::Base
  has_and_belongs_to_many :clients
end
\end{code}


Active Record will perform queries on the database for you and is  compatible with most database systems (MySQL, PostgreSQL and SQLite to  name a few). Regardless of which database system you’re using, the  Active Record method format will always be the same.

\section{ Retrieving Objects from the Database}

To retrieve objects from the database, Active Record provides several  finder methods. Each finder method allows you to pass arguments into it  to perform certain queries on your database without writing raw SQL.

The methods are:
\begin{itemize}
	\item where
	\item select
	\item group
	\item order
	\item reorder
	\item reverse\_order
	\item limit
	\item offset
	\item joins
	\item includes
	\item lock
	\item readonly
	\item from
	\item having
\end{itemize}

All of the above methods return an instance of ActiveRecord::Relation.

The primary operation of Model.find(options) can be summarized as:
\begin{itemize}
	\item Convert the supplied options to an equivalent SQL query.
	\item Fire the SQL query and retrieve the corresponding results from the database.
	\item Instantiate the equivalent Ruby object of the appropriate model for every resulting row.
	\item Run after\_find callbacks, if any.
\end{itemize}

\subsection{ Retrieving a Single Object}

Active Record provides five different ways of retrieving a single object.

\subsubsection{ Using a Primary Key}

Using Model.find(primary\_key), you can retrieve the object corresponding to the specified \emph{primary key} that matches any supplied options. For example:
\begin{code}
# Find the client with primary key (id) 10.
client = Client.find(10)
# => #<Client id: 10, first_name: "Ryan">
\end{code}


The SQL equivalent of the above is:
\begin{code}
SELECT * FROM clients WHERE (clients.id = 10) LIMIT 1
\end{code}

Model.find(primary\_key) will raise an ActiveRecord::RecordNotFound exception if no matching record is found.

\subsubsection{ first}

Model.first finds the first record matched by the supplied options, if any. For example:
\begin{code}
client = Client.first
# => #<Client id: 1, first_name: "Lifo">
\end{code}

The SQL equivalent of the above is:
\begin{code}
SELECT * FROM clients LIMIT 1
\end{code}

Model.first returns nil if no matching record is found. No exception will be raised.

\subsubsection{ last}

Model.last finds the last record matched by the supplied options. For example:
\begin{code}
client = Client.last
# => #<Client id: 221, first_name: "Russel">
\end{code}

The SQL equivalent of the above is:
\begin{code}
SELECT * FROM clients ORDER BY clients.id DESC LIMIT 1
\end{code}

Model.last returns nil if no matching record is found. No exception will be raised.

\subsubsection{ first!}

Model.first! finds the first record. For example:
\begin{code}
client = Client.first!
# => #<Client id: 1, first_name: "Lifo">
\end{code}

The SQL equivalent of the above is:
\begin{code}
SELECT * FROM clients LIMIT 1
\end{code}

Model.first! raises RecordNotFound if no matching record is found.

\subsubsection{ last!}

Model.last! finds the last record. For example:
\begin{code}
client = Client.last!
# => #<Client id: 221, first_name: "Russel">
\end{code}

The SQL equivalent of the above is:
\begin{code}
SELECT * FROM clients ORDER BY clients.id DESC LIMIT 1
\end{code}

Model.last! raises RecordNotFound if no matching record is found.

\subsection{ Retrieving Multiple Objects}

\subsubsection{ Using Multiple Primary Keys}

Model.find(array\_of\_primary\_key) accepts an array of \emph{primary keys}, returning an array containing all of the matching records for the supplied \emph{primary keys}. For example:
\begin{code}
# Find the clients with primary keys 1 and 10.
client = Client.find([1, 10]) # Or even Client.find(1, 10)
# => [#<Client id: 1, first_name: "Lifo">, 
      #<Client id: 10, first_name: "Ryan">]
\end{code}

The SQL equivalent of the above is:
\begin{code}
SELECT * FROM clients WHERE (clients.id IN (1,10))
\end{code}

Model.find(array\_of\_primary\_key) will raise an \\ ActiveRecord::RecordNotFound exception unless a matching record is found for \textbf{all} of the supplied primary keys.

\subsection{ Retrieving Multiple Objects in Batches}

We often need to iterate over a large set of records, as when we send  a newsletter to a large set of users, or when we export data.

This may appear straightforward:
\begin{code}
# This is very inefficient when the 
# users table has thousands of rows.
User.all.each do |user|
  NewsLetter.weekly_deliver(user)
end
\end{code}

But this approach becomes increasingly impractical as the table size increases, since User.all.each instructs Active Record to fetch \emph{the entire table}  in a single pass, build a model object per row, and then keep the  entire array of model objects in memory. Indeed, if we have a large  number of records, the entire collection may exceed the amount of memory  available.

Rails provides two methods that address this problem by dividing  records into memory-friendly batches for processing. The first method, find\_each, retrieves a batch of records and then yields \emph{each} record to the block individually as a model. The second method, find\_in\_batches, retrieves a batch of records and then yields \emph{the entire batch} to the block as an array of models.

The find\_each and find\_in\_batches  methods are intended for use in the batch processing of a large number  of records that wouldn’t fit in memory all at once. If you just need to  loop over a thousand records the regular find methods are the preferred  option.

\subsubsection{ find\_each}

The find\_each method retrieves a batch of records and then yields \emph{each} record to the block individually as a model. In the following example, find\_each will retrieve 1000 records (the current default for both find\_each and find\_in\_batches)  and then yield each record individually to the block as a model. This  process is repeated until all of the records have been processed:
\begin{code}
User.find_each do |user|
  NewsLetter.weekly_deliver(user)
end
\end{code}

\paragraph{1.3.1.1 Options for find\_each}

The find\_each method accepts most of the options allowed by the regular find method, except for :order and :limit, which are reserved for internal use by find\_each.

Two additional options, :batch\_size and :start, are available as well.

\textbf{:batch\_size}

The :batch\_size option allows you to specify the number of  records to be retrieved in each batch, before being passed individually  to the block. For example, to retrieve records in batches of 5000:
\begin{code}
User.find_each(:batch_size => 5000) do |user|
  NewsLetter.weekly_deliver(user)
end
\end{code}

\textbf{:start}

By default, records are fetched in ascending order of the primary key, which must be an integer. The :start  option allows you to configure the first ID of the sequence whenever  the lowest ID is not the one you need. This would be useful, for  example, if you wanted to resume an interrupted batch process, provided  you saved the last processed ID as a checkpoint.

For example, to send newsletters only to users with the primary key starting from 2000, and to retrieve them in batches of 5000:
\begin{code}
User.find_each(:start => 2000, :batch_size => 5000) do |user|
  NewsLetter.weekly_deliver(user)
end
\end{code}

Another example would be if you wanted multiple workers handling the  same processing queue. You could have each worker handle 10000 records  by setting the appropriate :start option on each worker.

The :include option allows you to name associations that should be loaded alongside with the models.

\subsubsection{ find\_in\_batches}

The find\_in\_batches method is similar to find\_each, since both retrieve batches of records. The difference is that find\_in\_batches yields \emph{batches}  to the block as an array of models, instead of individually. The  following example will yield to the supplied block an array of up to  1000 invoices at a time, with the final block containing any remaining  invoices:
\begin{code}
# Give add_invoices an array of 1000 invoices at a time
Invoice.find_in_batches(:include => :invoice_lines) do |invoices|
  export.add_invoices(invoices)
end
\end{code}

The :include option allows you to name associations that should be loaded alongside with the models.

\paragraph{1.3.2.1 Options for find\_in\_batches}

The find\_in\_batches method accepts the same :batch\_size and :start options as find\_each, as well as most of the options allowed by the regular find method, except for :order and :limit, which are reserved for internal use by find\_in\_batches.

\section{ Conditions}

The where method allows you to specify conditions to limit the records returned, representing the WHERE-part of the SQL statement. Conditions can either be specified as a string, array, or hash.

\subsection{ Pure String Conditions}

If you’d like to add conditions to your find, you could just specify them in there, just like Client.where("orders\_count = '2'"). This will find all clients where the orders\_count field’s value is 2.

Building your own conditions as pure strings can leave you vulnerable to SQL injection exploits. For example, Client.where("first\_name LIKE '\%\#\{params[:first\_name]\\%'")} is not safe. See the next section for the preferred way to handle conditions using an array.

\subsection{ Array Conditions}

Now what if that number could vary, say as an argument from somewhere? The find would then take the form:
\begin{code}
Client.where("orders_count = ?", params[:orders])
\end{code}

Active Record will go through the first element in the conditions  value and any additional elements will replace the question marks (?) in the first element.

If you want to specify multiple conditions:
\begin{code}
Client.where("orders_count = ? AND locked = ?", 
params[:orders], false)
\end{code}

In this example, the first question mark will be replaced with the value in params[:orders] and the second will be replaced with the SQL representation of false, which depends on the adapter.

This code is highly preferable:
\begin{code}
Client.where("orders_count = ?", params[:orders])
\end{code}

to this code:
\begin{code}
Client.where("orders_count = #{params[:orders]}")
\end{code}

because of argument safety. Putting the variable directly into the conditions string will pass the variable to the database \textbf{as-is}.  This means that it will be an unescaped variable directly from a user  who may have malicious intent. If you do this, you put your entire  database at risk because once a user finds out he or she can exploit  your database they can do just about anything to it. Never ever put your  arguments directly inside the conditions string.

For more information on the dangers of SQL injection, see the \href{http://guides.rubyonrails.org/security.html#sql-injection}{Ruby on Rails Security Guide}.

\subsubsection{ Placeholder Conditions}

Similar to the (?) replacement style of params, you can also specify keys/values hash in your array conditions:
\begin{code}
Client.where("created_at >= :start_date AND created_at <= :end_date",
  {:start_date => params[:start_date], :end_date => params[:end_date]})
\end{code}

This makes for clearer readability if you have a large number of variable conditions.

\subsubsection{ Range Conditions}

If you’re looking for a range inside of a table (for example, users  created in a certain timeframe) you can use the conditions option  coupled with the INSQL statement for this. If you had two dates coming in from a controller you could do something like this to look for a range:
\begin{code}
Client.where(:created_at => 
(params[:start_date].to_date)..(params[:end_date].to_date)
)
\end{code}

This query will generate something similar to the following SQL:
\begin{code}
SELECT "clients".* FROM "clients" WHERE (
"clients"."created_at" BETWEEN '2010-09-29' AND '2010-11-30'
)
\end{code}

\subsection{ Hash Conditions}

Active Record also allows you to pass in hash conditions which can  increase the readability of your conditions syntax. With hash  conditions, you pass in a hash with keys of the fields you want  conditionalised and the values of how you want to conditionalise them:

Only equality, range and subset checking are possible with Hash conditions.

\subsubsection{ Equality Conditions}
\begin{code}
Client.where(:locked => true)
\end{code}

The field name can also be a string:
\begin{code}
Client.where('locked' => true)
\end{code}

\subsubsection{ Range Conditions}

The good thing about this is that we can pass in a range for our  fields without it generating a large query as shown in the preamble of  this section.
\begin{code}
Client.where(:created_at => 
(Time.now.midnight - 1.day)..Time.now.midnight
)
\end{code}

This will find all clients created yesterday by using a BETWEENSQL statement:
\begin{code}
SELECT * FROM clients WHERE (
clients.created_at BETWEEN '2008-12-21 00:00:00' AND '2008-12-22 00:00:00'
)
\end{code}

This demonstrates a shorter syntax for the examples in \hyperlink{array-conditions}{Array Conditions}

\subsubsection{ Subset Conditions}

If you want to find records using the IN expression you can pass an array to the conditions hash:
\begin{code}
Client.where(:orders_count => [1,3,5])
\end{code}

This code will generate SQL like this:
\begin{code}
SELECT * FROM clients WHERE (clients.orders_count IN (1,3,5))
\end{code}

\section{ Ordering}

To retrieve records from the database in a specific order, you can use the order method.

For example, if you’re getting a set of records and want to order them in ascending order by the created\_at field in your table:
\begin{code}
Client.order("created_at")
\end{code}

You could specify ASC or DESC as well:
\begin{code}
Client.order("created_at DESC")
# OR
Client.order("created_at ASC")
\end{code}

Or ordering by multiple fields:
\begin{code}
Client.order("orders_count ASC, created_at DESC")
\end{code}

\section{ Selecting Specific Fields}

By default, Model.find selects all the fields from the result set using select *.

To select only a subset of fields from the result set, you can specify the subset via the select method.

If the select method is used, all the returning objects will be \hyperlink{readonly-objects}{read only}.


\\

For example, to select only viewable\_by and locked columns:
\begin{code}
Client.select("viewable_by, locked")
\end{code}

The SQL query used by this find call will be somewhat like:
\begin{code}
SELECT viewable_by, locked FROM clients
\end{code}

Be careful because this also means you’re initializing a model object  with only the fields that you’ve selected. If you attempt to access a  field that is not in the initialized record you’ll receive:
\begin{code}
ActiveModel::MissingAttributeError: missing attribute: <attribute>
\end{code}

Where $<$attribute$>$ is the attribute you asked for. The id method will not raise the ActiveRecord::MissingAttributeError, so just be careful when working with associations because they need the id method to function properly.

If you would like to only grab a single record per unique value in a certain field, you can use uniq:
\begin{code}
Client.select(:name).uniq
\end{code}

This would generate SQL like:
\begin{code}
SELECT DISTINCT name FROM clients
\end{code}

You can also remove the uniqueness constraint:
\begin{code}
query = Client.select(:name).uniq
# => Returns unique names
 
query.uniq(false)
# => Returns all names, even if there are duplicates
\end{code}

\section{ Limit and Offset}

To apply LIMIT to the SQL fired by the Model.find, you can specify the LIMIT using limit and offset methods on the relation.

You can use limit to specify the number of records to be retrieved, and use offset to specify the number of records to skip before starting to return the records. For example
\begin{code}
Client.limit(5)
\end{code}

will return a maximum of 5 clients and because it specifies no offset it will return the first 5 in the table. The SQL it executes looks like this:
\begin{code}
SELECT * FROM clients LIMIT 5
\end{code}

Adding offset to that
\begin{code}
Client.limit(5).offset(30)
\end{code}

will return instead a maximum of 5 clients beginning with the 31st. The SQL looks like:
\begin{code}
SELECT * FROM clients LIMIT 5 OFFSET 30
\end{code}

\section{ Group}

To apply a GROUP BY clause to the SQL fired by the finder, you can specify the group method on the find.

For example, if you want to find a collection of the dates orders were created on:
\begin{code}
Order.select("date(created_at) as ordered_date,
sum(price) as total_price").group("date(created_at)")
\end{code}

And this will give you a single Order object for each date where there are orders in the database.

The SQL that would be executed would be something like this:
\begin{code}
SELECT date(created_at) as ordered_date, sum(price) as total_price 
FROM orders GROUP BY date(created_at)
\end{code}

\section{ Having}

SQL uses the HAVING clause to specify conditions on the GROUP BY fields. You can add the HAVING clause to the SQL fired by the Model.find by adding the :having option to the find.

For example:
\begin{code}
Order.select("date(created_at) as ordered_date, sum(price) as total_price")
.group("date(created_at)").having("sum(price) > ?", 100)
\end{code}

The SQL that would be executed would be something like this:
\begin{code}
SELECT date(created_at) as ordered_date, sum(price) as total_price 
FROM orders GROUP BY date(created_at) HAVING sum(price) > 100
\end{code}

This will return single order objects for each day, but only those that are ordered more than \$100 in a day.

\section{ Overriding Conditions}

\subsection{ except}

You can specify certain conditions to be excepted by using the except method. For example:
\begin{code}
Post.where('id > 10').limit(20).order('id asc').except(:order)
\end{code}

The SQL that would be executed:
\begin{code}
SELECT * FROM posts WHERE id > 10 LIMIT 20
\end{code}

\subsection{ only}

You can also override conditions using the only method. For example:
\begin{code}
Post.where('id > 10').limit(20).order('id desc').only(:order, :where)
\end{code}

The SQL that would be executed:
\begin{code}
SELECT * FROM posts WHERE id > 10 ORDER BY id DESC
\end{code}

\subsection{ reorder}

The reorder method overrides the default scope order. For example:
\begin{code}
class Post < ActiveRecord::Base
  ..
  ..
  has_many :comments, :order => 'posted_at DESC'
end
 
Post.find(10).comments.reorder('name')
\end{code}

The SQL that would be executed:
\begin{code}
SELECT * FROM posts WHERE id = 10 ORDER BY name
\end{code}

In case the reorder clause is not used, the SQL executed would be:
\begin{code}
SELECT * FROM posts WHERE id = 10 ORDER BY posted_at DESC
\end{code}

\subsection{ reverse\_order}

The reverse\_order method reverses the ordering clause if specified.
\begin{code}
Client.where("orders_count > 10").order(:name).reverse_order
\end{code}

The SQL that would be executed:
\begin{code}
SELECT * FROM clients WHERE orders_count > 10 ORDER BY name DESC
\end{code}

If no ordering clause is specified in the query, the reverse\_order orders by the primary key in reverse order.
\begin{code}
Client.where("orders_count > 10").reverse_order
\end{code}

The SQL that would be executed:
\begin{code}
SELECT * FROM clients WHERE orders_count > 10 ORDER BY clients.id DESC
\end{code}

This method accepts \textbf{no} arguments.

\section{ Readonly Objects}

Active Record provides readonly method on a relation to  explicitly disallow modification or deletion of any of the returned  object. Any attempt to alter or destroy a readonly record will not  succeed, raising an ActiveRecord::ReadOnlyRecord exception.
\begin{code}
client = Client.readonly.first
client.visits += 1
client.save
\end{code}

As client is explicitly set to be a readonly object, the above code will raise an \\ ActiveRecord::ReadOnlyRecord exception when calling client.save with an updated value of \emph{visits}.

\section{ Locking Records for Update}

Locking is helpful for preventing race conditions when updating records in the database and ensuring atomic updates.

Active Record provides two locking mechanisms:
\begin{itemize}
	\item Optimistic Locking
	\item Pessimistic Locking
\end{itemize}

\subsection{ Optimistic Locking}

Optimistic locking allows multiple users to access the same record  for edits, and assumes a minimum of conflicts with the data. It does  this by checking whether another process has made changes to a record  since it was opened. An ActiveRecord::StaleObjectError exception is thrown if that has occurred and the update is ignored.

\textbf{Optimistic locking column}

In order to use optimistic locking, the table needs to have a column called lock\_version. Each time the record is updated, Active Record increments the lock\_version column. If an update request is made with a lower value in the lock\_version field than is currently in the lock\_version column in the database, the update request will fail with an \\ ActiveRecord::StaleObjectError. Example:
\begin{code}
c1 = Client.find(1)
c2 = Client.find(1)
 
c1.first_name = "Michael"
c1.save
 
c2.name = "should fail"
c2.save # Raises an ActiveRecord::StaleObjectError
\end{code}

You’re then responsible for dealing with the conflict by rescuing the  exception and either rolling back, merging, or otherwise apply the  business logic needed to resolve the conflict.

You must ensure that your database schema defaults the lock\_version column to 0.

This behavior can be turned off by setting \\ ActiveRecord::Base.lock\_optimistically = false.

To override the name of the lock\_version column, ActiveRecord::Base provides a class method called set\_locking\_column:
\begin{code}
class Client < ActiveRecord::Base
  set_locking_column :lock_client_column
end
\end{code}

\subsection{ Pessimistic Locking}

Pessimistic locking uses a locking mechanism provided by the underlying database. Using lock when building a relation obtains an exclusive lock on the selected rows. Relations using lock are usually wrapped inside a transaction for preventing deadlock conditions.

For example:
\begin{code}
Item.transaction do
  i = Item.lock.first
  i.name = 'Jones'
  i.save
end
\end{code}

The above session produces the following SQL for a MySQL backend:
\begin{code}
SQL (0.2ms)   BEGIN
Item Load (0.3ms) 
SELECT * FROM `items` LIMIT 1 FOR UPDATE
Item Update(0.4ms)
UPDATE `items` 
SET `updated_at`='2009-02-07 18:05:56', `name`='Jones' 
WHERE `id`=1
SQL (0.8ms)   COMMIT
\end{code}

You can also pass raw SQL to the lock method for allowing different types of locks. For example, MySQL has an expression called LOCK IN SHARE MODE  where you can lock a record but still allow other queries to read it.  To specify this expression just pass it in as the lock option:
\begin{code}
Item.transaction do
  i = Item.lock("LOCK IN SHARE MODE").find(1)
  i.increment!(:views)
end
\end{code}

If you already have an instance of your model, you can start a  transaction and acquire the lock in one go using the following code:
\begin{code}
item = Item.first
item.with_lock do
  # This block is called within a transaction,
  # item is already locked.
  item.increment!(:views)
end
\end{code}

\section{ Joining Tables}

Active Record provides a finder method called joins for specifying JOIN clauses on the resulting SQL. There are multiple ways to use the joins method.

\subsection{ Using a String SQL Fragment}

You can just supply the raw SQL specifying the JOIN clause to joins:
\begin{code}
Client.joins(
'LEFT OUTER JOIN addresses ON addresses.client_id = clients.id'
)
\end{code}

This will result in the following SQL:
\begin{code}
SELECT clients.* FROM clients 
LEFT OUTER JOIN addresses 
ON addresses.client_id = clients.id
\end{code}

\subsection{ Using Array/Hash of Named Associations}

This method only works with INNER JOIN.

Active Record lets you use the names of the \href{http://guides.rubyonrails.org/association_basics.html}{associations} defined on the model as a shortcut for specifying JOIN clause for those associations when using the joins method.

For example, consider the following Category, Post, Comments and Guest models:
\begin{code}
class Category < ActiveRecord::Base
  has_many :posts
end
 
class Post < ActiveRecord::Base
  belongs_to :category
  has_many :comments
  has_many :tags
end
 
class Comment < ActiveRecord::Base
  belongs_to :post
  has_one :guest
end
 
class Guest < ActiveRecord::Base
  belongs_to :comment
end
 
class Tag < ActiveRecord::Base
  belongs_to :post
end
\end{code}

Now all of the following will produce the expected join queries using INNER JOIN:

\subsubsection{ Joining a Single Association}
\begin{code}
Category.joins(:posts)
\end{code}

This produces:
\begin{code}
SELECT categories.* FROM categories
  INNER JOIN posts ON posts.category_id = categories.id
\end{code}

Or, in English: “return a Category object for all categories with  posts”. Note that you will see duplicate categories if more than one  post has the same category. If you want unique categories, you can use  Category.joins(:post).select(“distinct(categories.id)”).

\subsubsection{ Joining Multiple Associations}
\begin{code}
Post.joins(:category, :comments)
\end{code}

This produces:
\begin{code}
SELECT posts.* FROM posts
  INNER JOIN categories ON posts.category_id = categories.id
  INNER JOIN comments ON comments.post_id = posts.id
\end{code}

Or, in English: “return all posts that have a category and at least  one comment”. Note again that posts with multiple comments will show up  multiple times.

\subsubsection{ Joining Nested Associations (Single Level)}
\begin{code}
Post.joins(:comments => :guest)
\end{code}

This produces:
\begin{code}
SELECT posts.* FROM posts
  INNER JOIN comments ON comments.post_id = posts.id
  INNER JOIN guests ON guests.comment_id = comments.id
\end{code}

Or, in English: “return all posts that have a comment made by a guest.”

\subsubsection{ Joining Nested Associations (Multiple Level)}
\begin{code}
Category.joins(:posts => [{:comments => :guest}, :tags])
\end{code}

This produces:
\begin{code}
SELECT categories.* FROM categories
  INNER JOIN posts ON posts.category_id = categories.id
  INNER JOIN comments ON comments.post_id = posts.id
  INNER JOIN guests ON guests.comment_id = comments.id
  INNER JOIN tags ON tags.post_id = posts.id
\end{code}

\subsection{ Specifying Conditions on the Joined Tables}

You can specify conditions on the joined tables using the regular \hyperlink{array-conditions}{Array} and \hyperlink{pure-string-conditions}{String} conditions. \hyperlink{hash-conditions}{Hash conditions} provides a special syntax for specifying conditions for the joined tables:
\begin{code}
time_range = (Time.now.midnight - 1.day)..Time.now.midnight
Client.joins(:orders).where('orders.created_at' => time_range)
\end{code}

An alternative and cleaner syntax is to nest the hash conditions:
\begin{code}
time_range = (Time.now.midnight - 1.day)..Time.now.midnight
Client.joins(:orders).where(:orders => {:created_at => time_range})
\end{code}

This will find all clients who have orders that were created yesterday, again using a BETWEENSQL expression.

\section{ Eager Loading Associations}

Eager loading is the mechanism for loading the associated records of the objects returned by Model.find using as few queries as possible.

\textbf{N + 1 queries problem}

Consider the following code, which finds 10 clients and prints their postcodes:
\begin{code}
clients = Client.limit(10)
 
clients.each do |client|
  puts client.address.postcode
end
\end{code}

This code looks fine at the first sight. But the problem lies within  the total number of queries executed. The above code executes 1 ( to  find 10 clients ) + 10 ( one per each client to load the address ) = \textbf{11} queries in total.

\textbf{Solution to N + 1 queries problem}

Active Record lets you specify in advance all the associations that are going to be loaded. This is possible by specifying the includes method of the Model.find call. With includes, Active Record ensures that all of the specified associations are loaded using the minimum possible number of queries.

Revisiting the above case, we could rewrite Client.all to use eager load addresses:
\begin{code}
clients = Client.includes(:address).limit(10)
 
clients.each do |client|
  puts client.address.postcode
end
\end{code}

The above code will execute just \textbf{2} queries, as opposed to \textbf{11} queries in the previous case:
\begin{code}
SELECT * FROM clients LIMIT 10
SELECT addresses.* FROM addresses
  WHERE (addresses.client_id IN (1,2,3,4,5,6,7,8,9,10))
\end{code}

\subsection{ Eager Loading Multiple Associations}

Active Record lets you eager load any number of associations with a single Model.find call by using an array, hash, or a nested hash of array/hash with the includes method.

\subsubsection{ Array of Multiple Associations}
\begin{code}
Post.includes(:category, :comments)
\end{code}

This loads all the posts and the associated category and comments for each post.

\subsubsection{ Nested Associations Hash}
\begin{code}
Category.includes(:posts => [{:comments => :guest}, :tags]).find(1)
\end{code}

This will find the category with id 1 and eager load all of the  associated posts, the associated posts’ tags and comments, and every  comment’s guest association.

\subsection{ Specifying Conditions on Eager Loaded Associations}

Even though Active Record lets you specify conditions on the eager loaded associations just like joins, the recommended way is to use \hyperlink{joining-tables}{joins} instead.

However if you must do this, you may use where as you would normally.
\begin{code}
Post.includes(:comments).where("comments.visible", true)
\end{code}

This would generate a query which contains a LEFT OUTER JOIN whereas the joins method would generate one using the INNER JOIN function instead.
\begin{code}
SELECT "posts"."id" AS t0_r0, ... "comments"."updated_at" AS t1_r5 
FROM "posts" 
LEFT OUTER JOIN "comments" ON "comments"."post_id" = "posts"."id" 
WHERE (comments.visible = 1)
\end{code}

If there was no where condition, this would generate the normal set of two queries.

If, in the case of this includes query, there were no comments for any posts, all the posts would still be loaded. By using joins (an INNERJOIN), the join conditions \textbf{must} match, otherwise no records will be returned.

\section{ Scopes}

Scoping allows you to specify commonly-used ARel queries which can be  referenced as method calls on the association objects or models. With  these scopes, you can use every method previously covered such as where, joins and includes. All scope methods will return an ActiveRecord::Relation object which will allow for further methods (such as other scopes) to be called on it.

To define a simple scope, we use the scope method inside the class, passing the ARel query that we’d like run when this scope is called:
\begin{code}
class Post < ActiveRecord::Base
  scope :published, where(:published => true)
end
\end{code}

Just like before, these methods are also chainable:
\begin{code}
class Post < ActiveRecord::Base
  scope :published, where(:published => true).joins(:category)
end
\end{code}

Scopes are also chainable within scopes:
\begin{code}
class Post < ActiveRecord::Base
scope :published, where(:published => true)
scope :published_and_commented, 
      published.and(self.arel_table[:comments_count].gt(0))
end
\end{code}

To call this published scope we can call it on either the class:
\begin{code}
Post.published # => [published posts]
\end{code}

Or on an association consisting of Post objects:
\begin{code}
category = Category.first
category.posts.published # => [published posts belonging to this category]
\end{code}

\subsection{ Working with times}

If you’re working with dates or times within scopes, due to how they  are evaluated, you will need to use a lambda so that the scope is  evaluated every time.
\begin{code}
class Post < ActiveRecord::Base
  scope :last_week, lambda { where("created_at < ?", Time.zone.now ) }
end
\end{code}

Without the lambda, this Time.zone.now will only be called once.

\subsection{ Passing in arguments}

When a lambda is used for a scope, it can take arguments:
\begin{code}
class Post < ActiveRecord::Base
  scope :1_week_before, lambda { |time| where("created_at < ?", time) }
end
\end{code}

This may then be called using this:
\begin{code}
Post.1_week_before(Time.zone.now)
\end{code}

However, this is just duplicating the functionality that would be provided to you by a class method.
\begin{code}
class Post < ActiveRecord::Base
  def self.1_week_before(time)
    where("created_at < ?", time)
  end
end
\end{code}

Using a class method is the preferred way to accept arguments for  scopes. These methods will still be accessible on the association  objects:
\begin{code}
category.posts.1_week_before(time)
\end{code}

\subsection{ Working with scopes}

Where a relational object is required, the scoped method may come in handy. This will return an ActiveRecord::Relation object which can have further scoping applied to it afterwards. A place where this may come in handy is on associations
\begin{code}
client = Client.find_by_first_name("Ryan")
orders = client.orders.scoped
\end{code}

With this new orders object, we are able to ascertain that  this object can have more scopes applied to it. For instance, if we  wanted to return orders only in the last 30 days at a later point.
\begin{code}
orders.where("created_at > ?", 30.days.ago)
\end{code}

\subsection{ Applying a default scope}

If we wish for a scope to be applied across all queries to the model we can use the default\_scope method within the model itself.
\begin{code}
class Client < ActiveRecord::Base
  default_scope where("removed_at IS NULL")
end
\end{code}

When queries are executed on this model, the SQL query will now look something like this:
\begin{code}
SELECT * FROM clients WHERE removed_at IS NULL
\end{code}

\subsection{ Removing all scoping}

If we wish to remove scoping for any reason we can use the unscoped method. This is especially useful if a default\_scope is specified in the model and should not be applied for this particular query.
\begin{code}
Client.unscoped.all
\end{code}

This method removes all scoping and will do a normal query on the table.

\section{ Dynamic Finders}

For every field (also known as an attribute) you define in your  table, Active Record provides a finder method. If you have a field  called first\_name on your Client model for example, you get find\_by\_first\_name and find\_all\_by\_first\_name for free from Active Record. If you have a locked field on the Client model, you also get find\_by\_locked and find\_all\_by\_locked methods.

You can also use find\_last\_by\_* methods which will find the last record matching your argument.

You can specify an exclamation point (!) on the end of the dynamic finders to get them to raise an ActiveRecord::RecordNotFound error if they do not return any records, like Client.find\_by\_name!("Ryan")

If you want to find both by name and locked, you can chain these finders together by simply typing “and” between the fields. For example, Client.find\_by\_first\_name\_and\_locked("Ryan", true).

Up to and including Rails 3.1, when the number  of arguments passed to a dynamic finder method is lesser than the number  of fields, say Client.find\_by\_name\_and\_locked(“Ryan”), the behavior is to pass nil as the missing argument. This is \textbf{unintentional} and this behavior will be changed in Rails 3.2 to throw an ArgumentError.

\section{ Find or build a new object}

It’s common that you need to find a record or create it if it doesn’t exist. You can do that with the first\_or\_create and first\_or\_create! methods.

\subsection{ first\_or\_create}

The first\_or\_create method checks whether first returns nil or not. If it does return nil, then create is called. This is very powerful when coupled with the where method. Let’s see an example.

Suppose you want to find a client named ‘Andy’, and if there’s none, create one and additionally set his locked attribute to false. You can do so by running:
\begin{code}
Client.where(:first_name => 'Andy').first_or_create(:locked => false)
#
 => #<Client id: 1, first_name: "Andy", orders_count: 0, locked: 
false, created_at: "2011-08-30 06:09:27", updated_at: "2011-08-30 
06:09:27">
\end{code}

The SQL generated by this method looks like this:
\begin{code}
SELECT * FROM clients WHERE (clients.first_name = 'Andy') LIMIT 1
BEGIN
INSERT INTO clients 
(created_at, first_name, locked, orders_count, updated_at) 
VALUES ('2011-08-30 05:22:57', 'Andy', 0, NULL, '2011-08-30 05:22:57')
COMMIT
\end{code}

first\_or\_create returns either the record that already  exists or the new record. In our case, we didn’t already have a client  named Andy so the record is created and returned.

The new record might not be saved to the database; that depends on whether validations passed or not (just like create).

It’s also worth noting that first\_or\_create takes into account the arguments of the where method. In the example above we didn’t explicitly pass a :first\_name =$>$ 'Andy' argument to first\_or\_create. However, that was used when creating the new record because it was already passed before to the where method.

You can do the same with the find\_or\_create\_by method:
\begin{code}
Client.find_or_create_by_first_name(
:first_name => "Andy", :locked => false
)
\end{code}

This method still works, but it’s encouraged to use first\_or\_create because it’s more explicit on which arguments are used to \emph{find} the record and which are used to \emph{create}, resulting in less confusion overall.

\subsection{ first\_or\_create!}

You can also use first\_or\_create! to raise an exception if  the new record is invalid. Validations are not covered on this guide,  but let’s assume for a moment that you temporarily add
\begin{code}
validates :orders_count, :presence => true
\end{code}

to your Client model. If you try to create a new Client without passing an orders\_count, the record will be invalid and an exception will be raised:
\begin{code}
Client.where(:first_name => 'Andy').first_or_create!(:locked => false)
# => ActiveRecord::RecordInvalid: Validation failed: 
Orders count can't be blank
\end{code}

As with first\_or\_create there is a find\_or\_create\_by! method but the first\_or\_create! method is preferred for clarity.

\subsection{ first\_or\_initialize}

The first\_or\_initialize method will work just like first\_or\_create but it will not call create but new. This means that a new model instance will be created in memory but won’t be saved to the database. Continuing with the first\_or\_create example, we now want the client named ‘Nick’:
\begin{code}
nick = Client.where(:first_name => 'Nick')
      .first_or_initialize(:locked => false)
#
 => <Client id: nil, first_name: "Nick", orders_count: 0, locked: 
false, created_at: "2011-08-30 06:09:27", updated_at: "2011-08-30 
06:09:27">
 
nick.persisted?
# => false
 
nick.new_record?
# => true
\end{code}

Because the object is not yet stored in the database, the SQL generated looks like this:
\begin{code}
SELECT * FROM clients WHERE (clients.first_name = 'Nick') LIMIT 1
\end{code}

When you want to save it to the database, just call save:
\begin{code}
nick.save
# => true
\end{code}

\section{ Finding by SQL}

If you’d like to use your own SQL to find records in a table you can use find\_by\_sql. The find\_by\_sql  method will return an array of objects even if the underlying query  returns just a single record. For example you could run this query:
\begin{code}
Client.find_by_sql("SELECT * FROM clients
  INNER JOIN orders ON clients.id = orders.client_id
  ORDER clients.created_at desc")
\end{code}

find\_by\_sql provides you with a simple way of making custom calls to the database and retrieving instantiated objects.

\section{ select\_all}

find\_by\_sql has a close relative called connection\#select\_all. select\_all will retrieve objects from the database using custom SQL just like find\_by\_sql but will not instantiate them. Instead, you will get an array of hashes where each hash indicates a record.
\begin{code}
Client.connection.select_all("SELECT * FROM clients WHERE id = '1'")
\end{code}

\section{ pluck}

pluck can be used to query a single column from the  underlying table of a model. It accepts a column name as argument and  returns an array of values of the specified column with the  corresponding data type.
\begin{code}
Client.where(:active => true).pluck(:id)
# SELECT id FROM clients WHERE active = 1
 
Client.uniq.pluck(:role)
# SELECT DISTINCT role FROM clients
\end{code}

pluck makes it possible to replace code like
\begin{code}
Client.select(:id).map { |c| c.id }
\end{code}

with
\begin{code}
Client.pluck(:id)
\end{code}

\section{ Existence of Objects}

If you simply want to check for the existence of the object there’s a method called exists?. This method will query the database using the same query as find, but instead of returning an object or collection of objects it will return either true or false.
\begin{code}
Client.exists?(1
\end{code}

The exists? method also takes multiple ids, but the catch is that it will return true if any one of those records exists.
\begin{code}
Client.exists?(1,2,3)
# or
Client.exists?([1,2,3])
\end{code}

It’s even possible to use exists? without any arguments on a model or a relation.
\begin{code}
Client.where(:first_name => 'Ryan').exists?
\end{code}

The above returns true if there is at least one client with the first\_name ‘Ryan’ and false otherwise.
\begin{code}
Client.exists?
\end{code}

The above returns false if the clients table is empty and true otherwise.

You can also use any? and many? to check for existence on a model or relation.
\begin{code}
# via a model
Post.any?
Post.many?
 
# via a named scope
Post.recent.any?
Post.recent.many?
 
# via a relation
Post.where(:published => true).any?
Post.where(:published => true).many?
 
# via an association
Post.first.categories.any?
Post.first.categories.many?
\end{code}

\section{ Calculations}

This section uses count as an example method in this preamble, but the options described apply to all sub-sections.

All calculation methods work directly on a model:
\begin{code}
Client.count
# SELECT count(*) AS count_all FROM clients
\end{code}

Or on a relation:
\begin{code}
Client.where(:first_name => 'Ryan').count
# SELECT count(*) AS count_all FROM clients WHERE (first_name = 'Ryan')
\end{code}

You can also use various finder methods on a relation for performing complex calculations:
\begin{code}
Client.includes("orders").where(
:first_name => 'Ryan', :orders => {:status => 'received'}
).count
\end{code}

Which will execute:
\begin{code}
SELECT count(DISTINCT clients.id) AS count_all FROM clients
  LEFT OUTER JOIN orders ON orders.client_id = client.id WHERE
  (clients.first_name = 'Ryan' AND orders.status = 'received')
\end{code}

\subsection{ Count}

If you want to see how many records are in your model’s table you could call Client.count  and that will return the number. If you want to be more specific and  find all the clients with their age present in the database you can use Client.count(:age).

For options, please see the parent section, \hyperlink{calculations}{Calculations}.

\subsection{ Average}

If you want to see the average of a certain number in one of your tables you can call the average method on the class that relates to the table. This method call will look something like this:
\begin{code}
Client.average("orders_count")
\end{code}

This will return a number (possibly a floating point number such as 3.14159265) representing the average value in the field.

For options, please see the parent section, \hyperlink{calculations}{Calculations}.

\subsection{ Minimum}

If you want to find the minimum value of a field in your table you can call the minimum method on the class that relates to the table. This method call will look something like this:
\begin{code}
Client.minimum("age")
\end{code}

For options, please see the parent section, \hyperlink{calculations}{Calculations}.

\subsection{ Maximum}

If you want to find the maximum value of a field in your table you can call the maximum method on the class that relates to the table. This method call will look something like this:
\begin{code}
Client.maximum("age")
\end{code}

For options, please see the parent section, \hyperlink{calculations}{Calculations}.

\subsection{ Sum}

If you want to find the sum of a field for all records in your table you can call the sum method on the class that relates to the table. This method call will look something like this:
\begin{code}
Client.sum("orders_count")
\end{code}

For options, please see the parent section, \hyperlink{calculations}{Calculations}.

\newpage
\section{ Running EXPLAIN}

You can run EXPLAIN on the queries triggered by relations. For example,
\begin{code}
User.where(:id => 1).joins(:posts).explain
\end{code}

may yield
{\tiny
\begin{verbatim}
EXPLAIN for: 
SELECT `users`.* FROM `users` 
INNER JOIN `posts` ON `posts`.`user_id` = `users`.`id` 
WHERE `users`.`id` = 1

+----+-------------+-------+-------+---------------+---------+---------+-------+------+-------------+
|
 id | select_type | table | type  | possible_keys | 
key     | key_len | ref   | rows | 
Extra       |
+----+-------------+-------+-------+---------------+---------+---------+-------+------+-------------+
| 
 1 | SIMPLE      | users | const | 
PRIMARY       | PRIMARY | 
4       | const |    1 
|            
 |
| 
 1 | SIMPLE      | posts | ALL   | 
NULL          | 
NULL    | NULL    | NULL  
|    1 | Using where |
+----+-------------+-------+-------+---------------+---------+---------+-------+------+-------------+
2 rows in set (0.00 sec)
\end{verbatim}
}

under MySQL.


Active Record performs a pretty printing that emulates the one of the database shells. So, the same query running with the PostgreSQL adapter would yield instead
{\tiny
\begin{verbatim}
EXPLAIN for: 
SELECT "users".* FROM "users" 
INNER JOIN "posts" ON "posts"."user_id" = "users"."id" 
WHERE "users"."id" = 1
                                  QUERY PLAN
------------------------------------------------------------------------------
 Nested Loop Left Join  (cost=0.00..37.24 rows=8 width=0)
   Join Filter: (posts.user_id = users.id)
   ->  Index Scan using users_pkey on users  (cost=0.00..8.27 rows=1 width=4)
         Index Cond: (id = 1)
   ->  Seq Scan on posts  (cost=0.00..28.88 rows=8 width=4)
         Filter: (posts.user_id = 1)
(6 rows)
\end{verbatim}
}

Eager loading may trigger more than one query under the hood, and some queries may need the results of previous ones. Because of that, explain actually executes the query, and then asks for the query plans. For example,
\begin{code}
User.where(:id => 1).includes(:posts).explain
\end{code}

yields
{\tiny
\begin{verbatim}
EXPLAIN for: SELECT `users`.* FROM `users`  WHERE `users`.`id` = 1
+----+-------------+-------+-------+---------------+---------+---------+-------+------+-------+
|
 id | select_type | table | type  | possible_keys | 
key     | key_len | ref   | rows | Extra |
+----+-------------+-------+-------+---------------+---------+---------+-------+------+-------+
| 
 1 | SIMPLE      | users | const | 
PRIMARY       | PRIMARY | 
4       | const |    1 
|       |
+----+-------------+-------+-------+---------------+---------+---------+-------+------+-------+
1 row in set (0.00 sec)
 
EXPLAIN for: SELECT `posts`.* FROM `posts`  WHERE `posts`.`user_id` IN (1)
+----+-------------+-------+------+---------------+------+---------+------+------+-------------+
|
 id | select_type | table | type | possible_keys | key  | key_len |
 ref  | rows | Extra       |
+----+-------------+-------+------+---------------+------+---------+------+------+-------------+
| 
 1 | SIMPLE      | posts | ALL  | 
NULL          | NULL | 
NULL    | NULL |    1 | Using where |
+----+-------------+-------+------+---------------+------+---------+------+------+-------------+
1 row in set (0.00 sec)
\end{verbatim}
}

under MySQL.

\subsection{ Automatic EXPLAIN}

Active Record is able to run EXPLAIN automatically on slow queries and log its output. This feature is controlled by the configuration parameter
\begin{code}
config.active_record.auto_explain_threshold_in_seconds
\end{code}

If set to a number, any query exceeding those many seconds will have its EXPLAIN automatically triggered and logged. In the case of relations, the threshold is compared to the total time needed to fetch records. So, a relation is seen as a unit of work, no matter whether the implementation of eager loading involves several queries under the hood.

A threshold of nil disables automatic EXPLAINs.

The default threshold in development mode is 0.5 seconds, and nil in test and production modes.

Automatic EXPLAIN gets disabled if Active Record has no logger, regardless of the value of the threshold.

\subsubsection{ Disabling Automatic EXPLAIN}

Automatic EXPLAIN can be selectively silenced with ActiveRecord::Base.silence\_auto\_explain:
\begin{code}
ActiveRecord::Base.silence_auto_explain do
  # no automatic EXPLAIN is triggered here
end
\end{code}

That may be useful for queries you know are slow but fine, like a heavyweight report of an admin interface.

As its name suggests, silence\_auto\_explain only silences automatic EXPLAINs. Explicit calls to ActiveRecord::Relation\#explain run.

\subsection{ Interpreting EXPLAIN}

Interpretation of the output of EXPLAIN is beyond the scope of this guide. The following pointers may be helpful:
\begin{itemize}
	\item SQLite3: \href{http://www.sqlite.org/eqp.html}{EXPLAINQUERYPLAN}
\end{itemize}
\begin{itemize}
	\item MySQL: \href{http://dev.mysql.com/doc/refman/5.6/en/explain-output.html}{EXPLAIN Output Format}
\end{itemize}
\begin{itemize}
	\item PostgreSQL: \href{http://www.postgresql.org/docs/current/static/using-explain.html}{Using EXPLAIN}
\end{itemize}

\chapter{Layouts and Rendering in Rails}

This guide covers the basic layout features of Action Controller and  Action View. By referring to this guide, you will be able to:
\begin{itemize}
	\item Use the various rendering methods built into Rails
	\item Create layouts with multiple content sections
	\item Use partials to DRY up your views
	\item Use nested layouts (sub-templates)
\end{itemize}

\section{ Overview: How the Pieces Fit Together}

This guide focuses on the interaction between Controller and View in  the Model-View-Controller triangle. As you know, the Controller is  responsible for orchestrating the whole process of handling a request in  Rails, though it normally hands off any heavy code to the Model. But  then, when it’s time to send a response back to the user, the Controller  hands things off to the View. It’s that handoff that is the subject of  this guide.

In broad strokes, this involves deciding what should be sent as the  response and calling an appropriate method to create that response. If  the response is a full-blown view, Rails also does some extra work to  wrap the view in a layout and possibly to pull in partial views. You’ll  see all of those paths later in this guide.

\section{ Creating Responses}

From the controller’s point of view, there are three ways to create an HTTP response:
\begin{itemize}
	\item Call render to create a full response to send back to the browser
	\item Call redirect\_to to send an HTTP redirect status code to the browser
	\item Call head to create a response consisting solely of HTTP headers to send back to the browser
\end{itemize}

I’ll cover each of these methods in turn. But first, a few words  about the very easiest thing that the controller can do to create a  response: nothing at all.

\subsection{ Rendering by Default: Convention Over Configuration in Action}

You’ve heard that Rails promotes “convention over configuration”.  Default rendering is an excellent example of this. By default,  controllers in Rails automatically render views with names that  correspond to valid routes. For example, if you have this code in your BooksController class:
\begin{code}
class BooksController < ApplicationController
end
\end{code}

And the following in your routes file:
\begin{code}
resources :books
\end{code}

And you have a view file app/views/books/index.html.erb:
\begin{code}
<h1>Books are coming soon!</h1>
\end{code}

Rails will automatically render app/views/books/index.html.erb when you navigate to /books and you will see “Books are coming soon!” on your screen.

However a coming soon screen is only minimally useful, so you will soon create your Book model and add the index action to BooksController:
\begin{code}
class BooksController < ApplicationController
  def index
    @books = Book.all
  end
end
\end{code}

Note that we don’t have explicit render at the end of the index  action in accordance with “convention over configuration” principle. The  rule is that if you do not explicitly render something at the end of a  controller action, Rails will automatically look for the action\_name.html.erb template in the controller’s view path and render it. So in this case, Rails will render the app/views/books/index.html.erb file.

If we want to display the properties of all the books in our view, we can do so with an ERB template like this:
\begin{code}
<h1>Listing Books</h1>
 
<table>
  <tr>
    <th>Title</th>
    <th>Summary</th>
    <th></th>
    <th></th>
    <th></th>
  </tr>
 
<% @books.each do |book| %>
  <tr>
    <td><%= book.title %></td>
    <td><%= book.content %></td>
    <td><%= link_to 'Show', book %></td>
    <td><%= link_to 'Edit', edit_book_path(book) %></td>
    <td><%= link_to 'Remove', book, :confirm => 
        'Are you sure?', :method => :delete %></td>
  </tr>
<% end %>
</table>
 
<br />
 
<%= link_to 'New book', new_book_path %>
\end{code}

The actual rendering is done by subclasses of \\ ActionView::TemplateHandlers.  This guide does not dig into that process, but it’s important to know  that the file extension on your view controls the choice of template  handler. Beginning with Rails 2, the standard extensions are .erb for ERB (HTML with embedded Ruby), and .builder for Builder (XML generator).

\subsection{ Using render}

In most cases, the ActionController::Base\#render method does  the heavy lifting of rendering your application’s content for use by a  browser. There are a variety of ways to customize the behaviour of render.  You can render the default view for a Rails template, or a specific  template, or a file, or inline code, or nothing at all. You can render  text, JSON, or XML. You can specify the content type or HTTP status of the rendered response as well.

If you want to see the exact results of a call to render without needing to inspect it in a browser, you can call render\_to\_string. This method takes exactly the same options as render, but it returns a string instead of sending a response back to the browser.

\subsubsection{ Rendering Nothing}

Perhaps the simplest thing you can do with render is to render nothing at all:
\begin{code}
render :nothing => true
\end{code}

If you look at the response for this using cURL, you will see the following:
\begin{code}
$ curl -i 127.0.0.1:3000/books
HTTP/1.1 200 OK
Connection: close
Date: Sun, 24 Jan 2010 09:25:18 GMT
Transfer-Encoding: chunked
Content-Type: */*; charset=utf-8
X-Runtime: 0.014297
Set-Cookie: _blog_session=...snip...; path=/; HttpOnly
Cache-Control: no-cache
 
$
\end{code}

We see there is an empty response (no data after the Cache-Control line), but the request was successful because Rails has set the response to 200 OK. You can set the :status option on render to change this response. Rendering nothing can be useful for AJAX requests where all you want to send back to the browser is an acknowledgment that the request was completed.

You should probably be using the head method, discussed later in this guide, instead of render :nothing. This provides additional flexibility and makes it explicit that you’re only generating HTTP headers.

\subsubsection{ Rendering an Action’s View}

If you want to render the view that corresponds to a different action within the same template, you can use render with the name of the view:
\begin{code}
def update
  @book = Book.find(params[:id])
  if @book.update_attributes(params[:book])
    redirect_to(@book)
  else
    render "edit"
  end
end
\end{code}

If the call to update\_attributes fails, calling the update action in this controller will render the edit.html.erb template belonging to the same controller.

If you prefer, you can use a symbol instead of a string to specify the action to render:
\begin{code}
def update
  @book = Book.find(params[:id])
  if @book.update_attributes(params[:book])
    redirect_to(@book)
  else
    render :edit
  end
end
\end{code}

To be explicit, you can use render with the :action option (though this is no longer necessary in Rails 3.0):
\begin{code}
def update
  @book = Book.find(params[:id])
  if @book.update_attributes(params[:book])
    redirect_to(@book)
  else
    render :action => "edit"
  end
end
\end{code}

Using render with :action is a  frequent source of confusion for Rails newcomers. The specified action  is used to determine which view to render, but Rails does \emph{not}  run any of the code for that action in the controller. Any instance  variables that you require in the view must be set up in the current  action before calling render.

\subsubsection{ Rendering an Action’s Template from Another Controller}

What if you want to render a template from an entirely different  controller from the one that contains the action code? You can also do  that with render, which accepts the full path (relative to app/views) of the template to render. For example, if you’re running code in an AdminProductsController that lives in app/controllers/admin, you can render the results of an action to a template in app/views/products this way:
\begin{code}
render 'products/show'
\end{code}

Rails knows that this view belongs to a different controller because  of the embedded slash character in the string. If you want to be  explicit, you can use the :template option (which was required on Rails 2.2 and earlier):
\begin{code}
render :template => 'products/show
\end{code}

\subsubsection{ Rendering an Arbitrary File}

The render method can also use a view that’s entirely  outside of your application (perhaps you’re sharing views between two  Rails applications):
\begin{code}
render "/u/apps/warehouse_app/current/app/views/products/show"
\end{code}

Rails determines that this is a file render because of the leading slash character. To be explicit, you can use the :file option (which was required on Rails 2.2 and earlier):
\begin{code}
render :file =>
  "/u/apps/warehouse_app/current/app/views/products/show"
\end{code}

The :file option takes an absolute file-system path. Of  course, you need to have rights to the view that you’re using to render  the content.

By default, the file is rendered without using the  current layout. If you want Rails to put the file into the current  layout, you need to add the :layout =$>$ true option.

If you’re running Rails on Microsoft Windows, you should use the :file option to render a file, because Windows filenames do not have the same format as Unix filenames.

\subsubsection{ Wrapping it up}

The above three ways of rendering (rendering another template within  the controller, rendering a template within another controller and  rendering an arbitrary file on the file system) are actually variants of  the same action.

In fact, in the BooksController class, inside of the update action  where we want to render the edit template if the book does not update  successfully, all of the following render calls would all render the edit.html.erb template in the views/books directory:
\begin{code}
render :edit
render :action => :edit
render 'edit'
render 'edit.html.erb'
render :action => 'edit'
render :action => 'edit.html.erb'
render 'books/edit'
render 'books/edit.html.erb'
render :template => 'books/edit'
render :template => 'books/edit.html.erb'
render '/path/to/rails/app/views/books/edit'
render '/path/to/rails/app/views/books/edit.html.erb'
render :file => '/path/to/rails/app/views/books/edit'
render :file => '/path/to/rails/app/views/books/edit.html.erb'
\end{code}

Which one you use is really a matter of style and convention, but the  rule of thumb is to use the simplest one that makes sense for the code  you are writing.

\subsubsection{ Using render with :inline}

The render method can do without a view completely, if you’re willing to use the :inline option to supply ERB as part of the method call. This is perfectly valid:
\begin{code}
render :inline =>
  "<% products.each do |p| %><p><%= p.name %></p><% end %>"
\end{code}

There is seldom any good reason to use this option. Mixing ERB into your controllers defeats the MVC  orientation of Rails and will make it harder for other developers to  follow the logic of your project. Use a separate erb view instead.

By default, inline rendering uses ERB. You can force it to use Builder instead with the :type option:
\begin{code}
render :inline =>
  "xml.p {'Horrid coding practice!'}", :type => :builder
\end{code}

\subsubsection{ Rendering Text}

You can send plain text – with no markup at all – back to the browser by using the :text option to render:
\begin{code}
render :text => "OK"
\end{code}

Rendering pure text is most useful when you’re responding to AJAX or web service requests that are expecting something other than proper HTML.

By default, if you use the :text option,  the text is rendered without using the current layout. If you want Rails  to put the text into the current layout, you need to add the :layout =$>$ true option.

\subsubsection{ Rendering JSON}

JSON is a JavaScript data format used by many AJAX libraries. Rails has built-in support for converting objects to JSON and rendering that JSON back to the browser:
\begin{code}
render :json => @product
\end{code}

You don’t need to call to\_json on the object that you want to render. If you use the :json option, render will automatically call to\_json for you.

\subsubsection{ Rendering XML}

Rails also has built-in support for converting objects to XML and rendering that XML back to the caller:
\begin{code}
render :xml => @product
\end{code}

You don’t need to call to\_xml on the object that you want to render. If you use the :xml option, render will automatically call to\_xml for you.

\subsubsection{ Rendering Vanilla JavaScript}

Rails can render vanilla JavaScript:
\begin{code}
render :js => "alert('Hello Rails');"
\end{code}

This will send the supplied string to the browser with a MIME type of text/javascript.

\subsubsection{ Options for render}

Calls to the render method generally accept four options:
\begin{itemize}
	\item :content\_type
	\item :layout
	\item :status
	\item :location
\end{itemize}

\paragraph{2.2.11.1 The :content\_type Option}

By default, Rails will serve the results of a rendering operation with the MIME content-type of text/html (or application/json if you use the :json option, or application/xml for the :xml option.). There are times when you might like to change this, and you can do so by setting the :content\_type option:
\begin{code}
render :file => filename, :content_type => 'application/rss'
\end{code}

\paragraph{2.2.11.2 The :layout Option}

With most of the options to render, the rendered content is  displayed as part of the current layout. You’ll learn more about layouts  and how to use them later in this guide.

You can use the :layout option to tell Rails to use a specific file as the layout for the current action:
\begin{code}
render :layout => 'special_layout'
\end{code}

You can also tell Rails to render with no layout at all:
\begin{code}
render :layout => false
\end{code}

\paragraph{2.2.11.3 The :status Option}

Rails will automatically generate a response with the correct HTTP status code (in most cases, this is 200 OK). You can use the :status option to change this:
\begin{code}
render :status => 500
render :status => :forbidden
\end{code}

Rails understands both numeric and symbolic status codes.

\paragraph{2.2.11.4 The :location Option}

You can use the :location option to set the HTTPLocation header:
\begin{code}
render :xml => photo, :location => photo_url(photo)
\end{code}

\subsubsection{ Finding Layouts}

To find the current layout, Rails first looks for a file in app/views/layouts with the same base name as the controller. For example, rendering actions from the PhotosController class will use \\ app/views/layouts/photos.html.erb (or app/views/layouts/photos.builder). If there is no such controller-specific layout, Rails will use \\ app/views/layouts/application.html.erb or \\ app/views/layouts/application.builder. If there is no .erb layout, Rails will use a .builder  layout if one exists. Rails also provides several ways to more  precisely assign specific layouts to individual controllers and actions.

\paragraph{2.2.12.1 Specifying Layouts for Controllers}

You can override the default layout conventions in your controllers by using the layout declaration. For example:
\begin{code}
class ProductsController < ApplicationController
  layout "inventory"
  #...
end
\end{code}

With this declaration, all of the methods within ProductsController will use app/views/layouts/inventory.html.erb for their layout.

To assign a specific layout for the entire application, use a layout declaration in your ApplicationController class:
\begin{code}
class ApplicationController < ActionController::Base
  layout "main"
  #...
end
\end{code}

With this declaration, all of the views in the entire application will use app/views/layouts/main.html.erb for their layout.

\paragraph{2.2.12.2 Choosing Layouts at Runtime}

You can use a symbol to defer the choice of layout until a request is processed:
\begin{code}
class ProductsController < ApplicationController
  layout :products_layout
 
  def show
    @product = Product.find(params[:id])
  end
 
  private
    def products_layout
      @current_user.special? ? "special" : "products"
    end
 
end
\end{code}

Now, if the current user is a special user, they’ll get a special layout when viewing a product.

You can even use an inline method, such as a Proc, to determine the  layout. For example, if you pass a Proc object, the block you give the  Proc will be given the controller instance, so the layout can be determined based on the current request:
\begin{code}
class ProductsController < ApplicationController
layout Proc.new { 
|controller| controller.request.xhr? ? 'popup' : 'application'
}
end
\end{code}

\paragraph{2.2.12.3 Conditional Layouts}

Layouts specified at the controller level support the :only and :except  options. These options take either a method name, or an array of method  names, corresponding to method names within the controller:
\begin{code}
class ProductsController < ApplicationController
  layout "product", :except => [:index, :rss]
end
\end{code}

With this declaration, the product layout would be used for everything but the rss and index methods.

\paragraph{2.2.12.4 Layout Inheritance}

Layout declarations cascade downward in the hierarchy, and more  specific layout declarations always override more general ones. For  example:
\begin{itemize}
	\item application\_controller.rb
\end{itemize}
\begin{code}
class ApplicationController < ActionController::Base
  layout "main"
end
\end{code}
\begin{itemize}
	\item posts\_controller.rb
\end{itemize}
\begin{code}
class PostsController < ApplicationController
end
\end{code}
\begin{itemize}
	\item special\_posts\_controller.rb
\end{itemize}
\begin{code}
class SpecialPostsController < PostsController
  layout "special"
end
\end{code}
\begin{itemize}
	\item old\_posts\_controller.rb
\end{itemize}
\begin{code}
class OldPostsController < SpecialPostsController
  layout nil
 
  def show
    @post = Post.find(params[:id])
  end
 
  def index
    @old_posts = Post.older
    render :layout => "old"
  end
  # ...
end
\end{code}

In this application:
\begin{itemize}
	\item In general, views will be rendered in the main layout
	\item PostsController\#index will use the main layout
	\item SpecialPostsController\#index will use the special layout
	\item OldPostsController\#show will use no layout at all
	\item OldPostsController\#index will use the old layout
\end{itemize}

\subsubsection{ Avoiding Double Render Errors}

Sooner or later, most Rails developers will see the error message  “Can only render or redirect once per action”. While this is annoying,  it’s relatively easy to fix. Usually it happens because of a fundamental  misunderstanding of the way that render works.

For example, here’s some code that will trigger this error:
\begin{code}
def show
  @book = Book.find(params[:id])
  if @book.special?
    render :action => "special_show"
  end
  render :action => "regular_show"
end
\end{code}

If @book.special? evaluates to true, Rails will start the rendering process to dump the @book variable into the special\_show view. But this will \emph{not} stop the rest of the code in the show action from running, and when Rails hits the end of the action, it will start to render the regular\_show view – and throw an error. The solution is simple: make sure that you have only one call to render or redirect in a single code path. One thing that can help is and return. Here’s a patched version of the method:
\begin{code}
def show
  @book = Book.find(params[:id])
  if @book.special?
    render :action => "special_show" and return
  end
  render :action => "regular_show"
end
\end{code}

Make sure to use and return instead of \&\& return because \&\& return will not work due to the operator precedence in the Ruby Language.

Note that the implicit render done by ActionController detects if render has been called, so the following will work without errors:
\begin{code}
def show
  @book = Book.find(params[:id])
  if @book.special?
    render :action => "special_show"
  end
end
\end{code}

This will render a book with special? set with the special\_show template, while other books will render with the default show template.

\subsection{ Using redirect\_to}

Another way to handle returning responses to an HTTP request is with redirect\_to. As you’ve seen, render tells Rails which view (or other asset) to use in constructing a response. The redirect\_to method does something completely different: it tells the browser to send a new request for a different URL. For example, you could redirect from wherever you are in your code to the index of photos in your application with this call:
\begin{code}
redirect_to photos_url
\end{code}

You can use redirect\_to with any arguments that you could use with link\_to or url\_for. There’s also a special redirect that sends the user back to the page they just came from:
\begin{code}
redirect_to :back
\end{code}

\subsubsection{ Getting a Different Redirect Status Code}

Rails uses HTTP status code 302, a temporary redirect, when you call redirect\_to. If you’d like to use a different status code, perhaps 301, a permanent redirect, you can use the :status option:
\begin{code}
redirect_to photos_path, :status => 301
\end{code}

Just like the :status option for render, :status for redirect\_to accepts both numeric and symbolic header designations.

\subsubsection{ The Difference Between render and redirect\_to}

Sometimes inexperienced developers think of redirect\_to as a sort of goto command, moving execution from one place to another in your Rails code. This is \emph{not}  correct. Your code stops running and waits for a new request for the  browser. It just happens that you’ve told the browser what request it  should make next, by sending back an HTTP 302 status code.

Consider these actions to see the difference:
\begin{code}
def index
  @books = Book.all
end
 
def show
  @book = Book.find_by_id(params[:id])
  if @book.nil?
    render :action => "index"
  end
end
\end{code}

With the code in this form, there will likely be a problem if the @book variable is nil. Remember, a render :action doesn’t run any code in the target action, so nothing will set up the @books variable that the index view will probably require. One way to fix this is to redirect instead of rendering:
\begin{code}
def index
  @books = Book.all
end
 
def show
  @book = Book.find_by_id(params[:id])
  if @book.nil?
    redirect_to :action => :index
  end
end
\end{code}

With this code, the browser will make a fresh request for the index page, the code in the index method will run, and all will be well.

The only downside to this code is that it requires a round trip to the browser: the browser requested the show action with /books/1  and the controller finds that there are no books, so the controller  sends out a 302 redirect response to the browser telling it to go to /books/, the browser complies and sends a new request back to the controller asking now for the index  action, the controller then gets all the books in the database and  renders the index template, sending it back down to the browser which  then shows it on your screen.

While in a small application, this added latency might not be a  problem, it is something to think about if response time is a concern.  We can demonstrate one way to handle this with a contrived example:
\begin{code}
def index
  @books = Book.all
end
 
def show
  @book = Book.find_by_id(params[:id])
  if @book.nil?
    @books = Book.all
    render "index", :alert => 'Your book was not found!'
  end
end
\end{code}

This would detect that there are no books with the specified ID, populate the @books instance variable with all the books in the model, and then directly render the index.html.erb template, returning it to the browser with a flash alert message to tell the user what happened.

\subsection{ Using head To Build Header-Only Responses}

The head method can be used to send responses with only headers to the browser. It provides a more obvious alternative to calling render :nothing. The head  method takes one parameter, which is interpreted as a hash of header  names and values. For example, you can return only an error header:
\begin{code}
head :bad_request
\end{code}

This would produce the following header:
\begin{code}
HTTP/1.1 400 Bad Request
Connection: close
Date: Sun, 24 Jan 2010 12:15:53 GMT
Transfer-Encoding: chunked
Content-Type: text/html; charset=utf-8
X-Runtime: 0.013483
Set-Cookie: _blog_session=...snip...; path=/; HttpOnly
Cache-Control: no-cache
\end{code}

Or you can use other HTTP headers to convey other information:
\begin{code}
head :created, :location => photo_path(@photo)
\end{code}

Which would produce:
\begin{code}
HTTP/1.1 201 Created
Connection: close
Date: Sun, 24 Jan 2010 12:16:44 GMT
Transfer-Encoding: chunked
Location: /photos/1
Content-Type: text/html; charset=utf-8
X-Runtime: 0.083496
Set-Cookie: _blog_session=...snip...; path=/; HttpOnly
Cache-Control: no-cache
\end{code}

\section{ Structuring Layouts}

When Rails renders a view as a response, it does so by combining the  view with the current layout, using the rules for finding the current  layout that were covered earlier in this guide. Within a layout, you  have access to three tools for combining different bits of output to  form the overall response:
\begin{itemize}
	\item Asset tags
	\item yield and content\_for
	\item Partials
\end{itemize}

\subsection{ Asset Tag Helpers}

Asset tag helpers provide methods for generating HTML  that link views to feeds, JavaScript, stylesheets, images, videos and  audios. There are six asset tag helpers available in Rails:
\begin{itemize}
	\item auto\_discovery\_link\_tag
	\item javascript\_include\_tag
	\item stylesheet\_link\_tag
	\item image\_tag
	\item video\_tag
	\item audio\_tag
\end{itemize}

You can use these tags in layouts or other views, although the auto\_discovery\_link\_tag, javascript\_include\_tag, and stylesheet\_link\_tag, are most commonly used in the $<$head$>$ section of a layout.

The asset tag helpers do \emph{not} verify the  existence of the assets at the specified locations; they simply assume  that you know what you’re doing and generate the link.

\subsubsection{ Linking to Feeds with the auto\_discovery\_link\_tag}

The auto\_discovery\_link\_tag helper builds HTML that most browsers and newsreaders can use to detect the presences of RSS or ATOM feeds. It takes the type of the link (:rss or :atom), a hash of options that are passed through to url\_for, and a hash of options for the tag:
\begin{code}
<%= auto_discovery_link_tag(:rss, {:action => "feed"},
  {:title => "RSS Feed"}) %>
\end{code}

There are three tag options available for the auto\_discovery\_link\_tag:
\begin{itemize}
	\item :rel specifies the rel value in the link. The default value is “alternate”.
	\item :type specifies an explicit MIME type. Rails will generate an appropriate MIME type automatically.
	\item :title specifies the title of the link. The default value is the upshifted :type value, for example, “ATOM” or “RSS”.
\end{itemize}

\subsubsection{ Linking to JavaScript Files with the \\ javascript\_include\_tag}

The javascript\_include\_tag helper returns an HTMLscript tag for each source provided.

If you are using Rails with the \href{http://guides.rubyonrails.org/asset_pipeline.html}{Asset Pipeline} enabled, this helper will generate a link to /assets/javascripts/ rather than public/javascripts which was used in earlier versions of Rails. This link is then served by the Sprockets gem, which was introduced in Rails 3.1.

A JavaScript file within a Rails application or Rails engine goes in one of three locations: app/assets, lib/assets or vendor/assets. These locations are explained in detail in the \href{http://guides.rubyonrails.org/asset_pipeline.html#asset-organization}{Asset Organization section in the Asset Pipeline Guide}

You can specify a full path relative to the document root, or a URL, if you prefer. For example, to link to a JavaScript file that is inside a directory called javascripts inside of one of app/assets, lib/assets or vendor/assets, you would do this:
\begin{code}
<%= javascript_include_tag "main" %>
\end{code}

Rails will then output a script tag such as this:
\begin{code}
<script src='/assets/main.js' type="text/javascript"></script>
\end{code}

The request to this asset is then served by the Sprockets gem.

To include multiple files such as app/assets/javascripts/main.js and app/assets/javascripts/columns.js at the same time:
\begin{code}
<%= javascript_include_tag "main", "columns" %>
\end{code}

To include app/assets/javascripts/main.js \\ and app/assets/javascripts/photos/columns.js:
\begin{code}
<%= javascript_include_tag "main", "/photos/columns" %>
\end{code}

To include http://example.com/main.js:
\begin{code}
<%= javascript_include_tag "http://example.com/main.js" %>
\end{code}

If the application does not use the asset pipeline, the :defaults option loads jQuery by default:
\begin{code}
<%= javascript_include_tag :defaults %>
\end{code}

Outputting script tags such as this:
\begin{code}
<script src="/javascripts/jquery.js" type="text/javascript"></script>
<script src="/javascripts/jquery_ujs.js" type="text/javascript"></script>
\end{code}

These two files for jQuery, jquery.js and jquery\_ujs.js must be placed inside public/javascripts if the application doesn’t use the asset pipeline. These files can be downloaded from the \href{https://github.com/indirect/jquery-rails/tree/master/vendor/assets/javascripts}{jquery-rails repository on GitHub}

If you are using the asset pipeline, this tag will render a script tag for an asset called defaults.js, which would not exist in your application unless you’ve explicitly defined it to be.

And you can in any case override the :defaults expansion in config/application.rb:
\begin{code}
config.action_view.javascript_expansions[:defaults] = %w(foo.js bar.js)
\end{code}

You can also define new defaults:
\begin{code}
config.action_view.javascript_expansions[:projects] 
  = %w(projects.js tickets.js)
\end{code}

And use them by referencing them exactly like :defaults:
\begin{code}
<%= javascript_include_tag :projects %>
\end{code}

When using :defaults, if an application.js file exists in public/javascripts it will be included as well at the end.

Also, if the asset pipeline is disabled, the :all expansion loads every JavaScript file in public/javascripts:
\begin{code}
<%= javascript_include_tag :all %>
\end{code}

Note that your defaults of choice will be included first, so they will be available to all subsequently included files.

You can supply the :recursive option to load files in subfolders of public/javascripts as well:
\begin{code}
<%= javascript_include_tag :all, :recursive => true %>
\end{code}

If you’re loading multiple JavaScript files, you can create a better  user experience by combining multiple files into a single download. To  make this happen in production, specify :cache =$>$ true in your javascript\_include\_tag:
\begin{code}
<%= javascript_include_tag "main", "columns", :cache => true %>
\end{code}

By default, the combined file will be delivered as javascripts/all.js. You can specify a location for the cached asset file instead:
\begin{code}
<%= javascript_include_tag "main", "columns",
  :cache => 'cache/main/display' %>
\end{code}

You can even use dynamic paths such as \\ cache/\#\{current\_site\/main/display}.

\subsubsection{ Linking to CSS Files with the stylesheet\_link\_tag}

The stylesheet\_link\_tag helper returns an HTML$<$link$>$ tag for each source provided.

If you are using Rails with the “Asset Pipeline” enabled, this helper will generate a link to /assets/stylesheets/. This link is then processed by the Sprockets gem. A stylesheet file can be stored in one of three locations: app/assets, lib/assets or vendor/assets.

You can specify a full path relative to the document root, or a URL. For example, to link to a stylesheet file that is inside a directory called stylesheets inside of one of app/assets, lib/assets or vendor/assets, you would do this:
\begin{code}
<%= stylesheet_link_tag "main" %>
\end{code}

To include app/assets/stylesheets/main.css \\ and app/assets/stylesheets/columns.css:
\begin{code}
<%= stylesheet_link_tag "main", "columns" %>
\end{code}

To include app/assets/stylesheets/main.css \\ and app/assets/stylesheets/photos/columns.css:
\begin{code}
<%= stylesheet_link_tag "main", "/photos/columns" %>
\end{code}

To include http://example.com/main.css:
\begin{code}
<%= stylesheet_link_tag "http://example.com/main.css" %>
\end{code}

By default, the stylesheet\_link\_tag creates links with media="screen" rel="stylesheet" type="text/css". You can override any of these defaults by specifying an appropriate option (:media, :rel, or :type):
\begin{code}
<%= stylesheet_link_tag "main_print", :media => "print" %>
\end{code}

If the asset pipeline is disabled, the all option links every CSS file in public/stylesheets:
\begin{code}
<%= stylesheet_link_tag :all %>
\end{code}

You can supply the :recursive option to link files in subfolders of public/stylesheets as well:
\begin{code}
<%= stylesheet_link_tag :all, :recursive => true %>
\end{code}

If you’re loading multiple CSS files, you  can create a better user experience by combining multiple files into a  single download. To make this happen in production, specify :cache =$>$ true in your stylesheet\_link\_tag:
\begin{code}
<%= stylesheet_link_tag "main", "columns", :cache => true %>
\end{code}

By default, the combined file will be delivered as stylesheets/all.css. You can specify a location for the cached asset file instead:
\begin{code}
<%= stylesheet_link_tag "main", "columns",
  :cache => 'cache/main/display' %>
\end{code}

You can even use dynamic paths such as \\ cache/\#\{current\_site\/main/display}.

\subsubsection{ Linking to Images with the image\_tag}

The image\_tag helper builds an HTML$<$img /$>$ tag to the specified file. By default, files are loaded from public/images.

Note that you must specify the extension of the  image. Previous versions of Rails would allow you to just use the image  name and would append .png if no extension was given but Rails 3.0 does not.
\begin{code}
<%= image_tag "header.png" %>
\end{code}

You can supply a path to the image if you like:
\begin{code}
<%= image_tag "icons/delete.gif" %>
\end{code}

You can supply a hash of additional HTML options:
\begin{code}
<%= image_tag "icons/delete.gif", {:height => 45} %>
\end{code}

You can also supply an alternate image to show on mouseover:
\begin{code}
<%= image_tag "home.gif", :onmouseover => "menu/home_highlight.gif" %>
\end{code}

You can supply alternate text for the image which will be used if the  user has images turned off in their browser. If you do not specify an  alt text explicitly, it defaults to the file name of the file,  capitalized and with no extension. For example, these two image tags  would return the same code:
\begin{code}
<%= image_tag "home.gif" %>
<%= image_tag "home.gif", :alt => "Home" %>
\end{code}

You can also specify a special size tag, \\ in the format “\{width\}x\{height\}”:
\begin{code}
<%= image_tag "home.gif", :size => "50x20" %>
\end{code}

In addition to the above special tags, you can supply a final hash of standard HTML options, such as :class, :id or :name:
\begin{code}
<%= image_tag "home.gif", :alt => "Go Home",
                          :id => "HomeImage",
                          :class => 'nav_bar' %>
\end{code}

\subsubsection{ Linking to Videos with the video\_tag}

The video\_tag helper builds an HTML 5 $<$video$>$ tag to the specified file. By default, files are loaded from public/videos.
\begin{code}
<%= video_tag "movie.ogg" %>
\end{code}

Produces
\begin{code}
<video src="/videos/movie.ogg" />
\end{code}

Like an image\_tag you can supply a path, either absolute, or relative to the public/videos directory. Additionally you can specify the :size =$>$ "\#\{width\x\#\{height\}"} option just like an image\_tag. Video tags can also have any of the HTML options specified at the end (id, class et al).

The video tag also supports all of the $<$video$>$HTML options through the HTML options hash, including:
\begin{itemize}
	\item :poster =$>$ 'image\_name.png', provides an image to put in place of the video before it starts playing.
	\item :autoplay =$>$ true, starts playing the video on page load.
	\item :loop =$>$ true, loops the video once it gets to the end.
	\item :controls =$>$ true, provides browser supplied controls for the user to interact with the video.
	\item :autobuffer =$>$ true, the video will pre load the file for the user on page load.
\end{itemize}

You can also specify multiple videos to play by passing an array of videos to the video\_tag:
\begin{code}
<%= video_tag ["trailer.ogg", "movie.ogg"] %>
\end{code}

This will produce:
\begin{code}
<video><source src="trailer.ogg" /><source src="movie.ogg" /></video>
\end{code}

\subsubsection{ Linking to Audio Files with the audio\_tag}

The audio\_tag helper builds an HTML 5 $<$audio$>$ tag to the specified file. By default, files are loaded from public/audios.
\begin{code}
<%= audio_tag "music.mp3" %>
\end{code}

You can supply a path to the audio file if you like:
\begin{code}
<%= audio_tag "music/first_song.mp3" %>
\end{code}

You can also supply a hash of additional options, such as :id, :class etc.

Like the video\_tag, the audio\_tag has special options:
\begin{itemize}
	\item :autoplay =$>$ true, starts playing the audio on page load
	\item :controls =$>$ true, provides browser supplied controls for the user to interact with the audio.
	\item :autobuffer =$>$ true, the audio will pre load the file for the user on page load.
\end{itemize}

\subsection{ Understanding yield}

Within the context of a layout, yield identifies a section where content from the view should be inserted. The simplest way to use this is to have a single yield, into which the entire contents of the view currently being rendered is inserted:
\begin{code}
<html>
  <head>
  </head>
  <body>
  <%= yield %>
  </body>
</html>
\end{code}

You can also create a layout with multiple yielding regions:
\begin{code}
<html>
  <head>
  <%= yield :head %>
  </head>
  <body>
  <%= yield %>
  </body>
</html>
\end{code}

The main body of the view will always render into the unnamed yield. To render content into a named yield, you use the content\_for method.

\subsection{ Using the content\_for Method}

The content\_for method allows you to insert content into a named yield block in your layout. For example, this view would work with the layout that you just saw:
\begin{code}
<% content_for :head do %>
  <title>A simple page</title>
<% end %>
 
<p>Hello, Rails!</p>
\end{code}

The result of rendering this page into the supplied layout would be this HTML:
\begin{code}
<html>
  <head>
  <title>A simple page</title>
  </head>
  <body>
  <p>Hello, Rails!</p>
  </body>
</html>
\end{code}

The content\_for method is very helpful when your layout  contains distinct regions such as sidebars and footers that should get  their own blocks of content inserted. It’s also useful for inserting  tags that load page-specific JavaScript or css files into the header of  an otherwise generic layout.

\subsection{ Using Partials}

Partial templates – usually just called “partials” – are another  device for breaking the rendering process into more manageable chunks.  With a partial, you can move the code for rendering a particular piece  of a response to its own file.

\subsubsection{ Naming Partials}

To render a partial as part of a view, you use the render method within the view:
\begin{code}
<%= render "menu" %>
\end{code}

This will render a file named \_menu.html.erb at that point  within the view being rendered. Note the leading underscore character:  partials are named with a leading underscore to distinguish them from  regular views, even though they are referred to without the underscore.  This holds true even when you’re pulling in a partial from another  folder:
\begin{code}
<%= render "shared/menu" %>
\end{code}

That code will pull in the partial \\ from app/views/shared/\_menu.html.erb.

\subsubsection{ Using Partials to Simplify Views}

One way to use partials is to treat them as the equivalent of  subroutines: as a way to move details out of a view so that you can  grasp what’s going on more easily. For example, you might have a view  that looked like this:
\begin{code}
<%= render "shared/ad_banner" %>
 
<h1>Products</h1>
 
<p>Here are a few of our fine products:</p>
...
 
<%= render "shared/footer" %>
\end{code}

Here, the \_ad\_banner.html.erb and \_footer.html.erb  partials could contain content that is shared among many pages in your  application. You don’t need to see the details of these sections when  you’re concentrating on a particular page.

For content that is shared among all pages in your application, you can use partials directly from layouts.

\subsubsection{ Partial Layouts}

A partial can use its own layout file, just as a view can use a layout. For example, you might call a partial like this:
\begin{code}
<%= render :partial => "link_area", :layout => "graybar" %>
\end{code}

This would look for a partial named \_link\_area.html.erb and render it using the layout \_graybar.html.erb.  Note that layouts for partials follow the same leading-underscore  naming as regular partials, and are placed in the same folder with the  partial that they belong to (not in the master layouts folder).

Also note that explicitly specifying :partial is required when passing additional options such as :layout.

\subsubsection{ Passing Local Variables}

You can also pass local variables into partials, making them even  more powerful and flexible. For example, you can use this technique to  reduce duplication between new and edit pages, while still keeping a bit  of distinct content:
\begin{itemize}
	\item new.html.erb
\end{itemize}
\begin{code}
<h1>New zone</h1>
<%= error_messages_for :zone %>
<%= render :partial => "form", :locals => { :zone => @zone } %>
\end{code}
\begin{itemize}
	\item edit.html.erb
\end{itemize}
\begin{code}
<h1>Editing zone</h1>
<%= error_messages_for :zone %>
<%= render :partial => "form", :locals => { :zone => @zone } %>
\end{code}
\begin{itemize}
	\item \_form.html.erb
\end{itemize}
\begin{code}
<%= form_for(zone) do |f| %>
  <p>
    <b>Zone name</b><br />
    <%= f.text_field :name %>
  </p>
  <p>
    <%= f.submit %>
  </p>
<% end %>
\end{code}

Although the same partial will be rendered into both views, Action  View’s submit helper will return “Create Zone” for the new action and  “Update Zone” for the edit action.

Every partial also has a local variable with the same name as the  partial (minus the underscore). You can pass an object in to this local  variable via the :object option:
\begin{code}
<%= render :partial => "customer", :object => @new_customer %>
\end{code}

Within the customer partial, the customer variable will refer to @new\_customer from the parent view.

In previous versions of Rails, the default local  variable would look for an instance variable with the same name as the  partial in the parent. This behavior was deprecated in 2.3 and has been  removed in Rails 3.0.

If you have an instance of a model to render into a partial, you can use a shorthand syntax:
\begin{code}
<%= render @customer %>
\end{code}

Assuming that the @customer instance variable contains an instance of the Customer model, this will use \_customer.html.erb to render it and will pass the local variable customer into the partial which will refer to the @customer instance variable in the parent view.

\subsubsection{ Rendering Collections}

Partials are very useful in rendering collections. When you pass a collection to a partial via the :collection option, the partial will be inserted once for each member in the collection:
\begin{itemize}
	\item index.html.erb
\end{itemize}
\begin{code}
<h1>Products</h1>
<%= render :partial => "product", :collection => @products %>
\end{code}
\begin{itemize}
	\item \_product.html.erb
\end{itemize}
\begin{code}
<p>Product Name: <%= product.name %></p>
\end{code}

When a partial is called with a pluralized collection, then the  individual instances of the partial have access to the member of the  collection being rendered via a variable named after the partial. In  this case, the partial is \_product, and within the \_product partial, you can refer to product to get the instance that is being rendered.

In Rails 3.0, there is also a shorthand for this. Assuming @products is a collection of product instances, you can simply write this in the index.html.erb to produce the same result:
\begin{code}
<h1>Products</h1>
<%= render @products %>
\end{code}

Rails determines the name of the partial to use by looking at the  model name in the collection. In fact, you can even create a  heterogeneous collection and render it this way, and Rails will choose  the proper partial for each member of the collection:

In the event that the collection is empty, render will return nil, so it should be fairly simple to provide alternative content.
\begin{code}
<h1>Products</h1>
<%= render(@products) || 'There are no products available.' %>
\end{code}
\begin{itemize}
	\item index.html.erb
\end{itemize}
\begin{code}
<h1>Contacts</h1>
<%= render [customer1, employee1, customer2, employee2] %>
\end{code}
\begin{itemize}
	\item customers/\_customer.html.erb
\end{itemize}
\begin{code}
<p>Customer: <%= customer.name %></p>
\end{code}
\begin{itemize}
	\item employees/\_employee.html.erb
\end{itemize}
\begin{code}
<p>Employee: <%= employee.name %></p>
\end{code}

In this case, Rails will use the customer or employee partials as appropriate for each member of the collection.

\subsubsection{ Local Variables}

To use a custom local variable name within the partial, specify the :as option in the call to the partial:
\begin{code}
<%= render :partial => "product", 
  :collection => @products, :as => :item %>
\end{code}

With this change, you can access an instance of the @products collection as the item local variable within the partial.

You can also pass in arbitrary local variables to any partial you are rendering with the :locals =$>$ \{\} option:
\begin{code}
<%= render :partial => 'products', :collection => @products,
           :as => :item, :locals => {:title => "Products Page"} %>
\end{code}

Would render a partial \_products.html.erb once for each instance of product in the @products instance variable passing the instance to the partial as a local variable called item and to each partial, make the local variable title available with the value Products Page.

Rails also makes a counter variable available  within a partial called by the collection, named after the member of the  collection followed by \_counter. For example, if you’re rendering @products, within the partial you can refer to product\_counter to tell you how many times the partial has been rendered. This does not work in conjunction with the :as =$>$ :value option.

You can also specify a second partial to be rendered between instances of the main partial by using the :spacer\_template option:

\subsubsection{ Spacer Templates}
\begin{code}
<%= render :partial => @products, :spacer_template => "product_ruler" %>
\end{code}

Rails will render the \_product\_ruler partial (with no data passed in to it) between each pair of \_product partials.

\subsection{ Using Nested Layouts}

You may find that your application requires a layout that differs  slightly from your regular application layout to support one particular  controller. Rather than repeating the main layout and editing it, you  can accomplish this by using nested layouts (sometimes called  sub-templates). Here’s an example:

Suppose you have the following ApplicationController layout:
\begin{itemize}
	\item app/views/layouts/application.html.erb
\end{itemize}
\begin{code}
<html>
<head>
  <title><%= @page_title or 'Page Title' %></title>
  <%= stylesheet_link_tag 'layout' %>
  <style type="text/css"><%= yield :stylesheets %></style>
</head>
<body>
  <div id="top_menu">Top menu items here</div>
  <div id="menu">Menu items here</div>
  <div id="content">
    <%=content_for?(:content)?yield(:content):yield %>
  </div>
</body>
</html>
\end{code}

On pages generated by NewsController, you want to hide the top menu and add a right menu:
\begin{itemize}
	\item app/views/layouts/news.html.erb
\end{itemize}
\begin{code}
<% content_for :stylesheets do %>
  #top_menu {display: none}
  #right_menu {float: right; background-color: yellow; color: black}
<% end %>
<% content_for :content do %>
  <div id="right_menu">Right menu items here</div>
  <%= content_for?(:news_content) ? yield(:news_content) : yield %>
<% end %>
<%= render :template => 'layouts/application' %>
\end{code}

That’s it. The News views will use the new layout, hiding the top menu and adding a new right menu inside the “content” div.

There are several ways of getting similar results with different  sub-templating schemes using this technique. Note that there is no limit  in nesting levels. One can use the ActionView::render method via render :template =$>$ 'layouts/news' to base a new layout on the News layout. If you are sure you will not subtemplate the News layout, you can replace the content\_for?(:news\_content) ? yield(:news\_content) : yield with simply yield.

\chapter{Rails Form helpers}

Forms in web applications are an essential interface for user input.  However, form markup can quickly become tedious to write and maintain  because of form control naming and their numerous attributes. Rails  deals away with these complexities by providing view helpers for  generating form markup. However, since they have different use-cases,  developers are required to know all the differences between similar  helper methods before putting them to use.

In this guide you will:
\begin{itemize}
	\item Create search forms and similar kind of generic forms not representing any specific model in your application
	\item Make model-centric forms for creation and editing of specific database records
	\item Generate select boxes from multiple types of data
	\item Understand the date and time helpers Rails provides
	\item Learn what makes a file upload form different
	\item Learn some cases of building forms to external resources
	\item Find out where to look for complex forms
\end{itemize}

This guide is not intended to be a complete documentation of available form helpers and their arguments. Please visit \href{http://api.rubyonrails.org/}{the Rails API documentation} for a complete reference.

\section{ Dealing with Basic Forms}

The most basic form helper is form\_tag.
\begin{code}
<%= form_tag do %>
  Form contents
<% end %>
\end{code}

When called without arguments like this, it creates a $<$form$>$ tag which, when submitted, will POST to the current page. For instance, assuming the current page is /home/index, the generated HTML will look like this (some line breaks added for readability):
\begin{code}
<form accept-charset="UTF-8" action="/home/index" method="post">
  <div style="margin:0;padding:0">
    <input name="utf8" type="hidden" value="&#x2713;" />
    <input name="authenticity_token" type="hidden" 
           value="f755bb0ed134b76c432144748a6d4b7a7ddf2b71" />
  </div>
  Form contents
</form>
\end{code}

Now, you’ll notice that the HTML contains something extra: a div  element with two hidden input elements inside. This div is important,  because the form cannot be successfully submitted without it. The first  input element with name utf8 enforces browsers to properly respect your form’s character encoding and is generated for all forms whether their actions are “GET” or “POST”. The second input element with name authenticity\_token is a security feature of Rails called \textbf{cross-site request forgery protection}, and form helpers generate it for every non-GET form (provided that this security feature is enabled). You can read more about this in the \href{http://guides.rubyonrails.org/security.html#_cross_site_reference_forgery_csrf}{Security Guide}.

Throughout this guide, the div with the hidden input elements will be excluded from code samples for brevity.

\subsection{ A Generic Search Form}

One of the most basic forms you see on the web is a search form. This form contains:
\begin{enumerate}
	\item a form element with “GET” method,
	\item a label for the input,
	\item a text input element, and
	\item a submit element.
\end{enumerate}

To create this form you will use form\_tag, label\_tag, text\_field\_tag, and submit\_tag, respectively. Like this:
\begin{code}
<%= form_tag("/search", :method => "get") do %>
  <%= label_tag(:q, "Search for:") %>
  <%= text_field_tag(:q) %>
  <%= submit_tag("Search") %>
<% end %>
\end{code}

This will generate the following HTML:
\begin{code}
<form accept-charset="UTF-8" action="/search" method="get">
  <label for="q">Search for:</label>
  <input id="q" name="q" type="text" />
  <input name="commit" type="submit" value="Search" />
</form>
\end{code}

For every form input, an ID attribute is generated from its name (“q” in the example). These IDs can be very useful for CSS styling or manipulation of form controls with JavaScript.

Besides text\_field\_tag and submit\_tag, there is a similar helper for \emph{every} form control in HTML.

Always use “GET” as  the method for search forms. This allows users to bookmark a specific  search and get back to it. More generally Rails encourages you to use  the right HTTP verb for an action.

\subsection{ Multiple Hashes in Form Helper Calls}

The form\_tag helper accepts 2 arguments: the path for the action and an options hash. This hash specifies the method of form submission and HTML options such as the form element’s class.

As with the link\_to helper, the path argument doesn’t have to be given a string; it can be a hash of URL parameters recognizable by Rails’ routing mechanism, which will turn the hash into a valid URL. However, since both arguments to form\_tag are hashes, you can easily run into a problem if you would like to specify both. For instance, let’s say you write this:
\begin{code}
form_tag(:controller => "people", :action => "search", 
         :method => "get", :class => "nifty_form")
# => '<form accept-charset="UTF-8" 
            action="/people/search?method=get&class=nifty_form" 
            method="post">'
\end{code}

Here, method and class are appended to the query string of the generated URL  because you even though you mean to write two hashes, you really only  specified one. So you need to tell Ruby which is which by delimiting the  first hash (or both) with curly brackets. This will generate the HTML you expect:
\begin{code}
form_tag({:controller => "people", :action => "search"},
          :method => "get", :class => "nifty_form")
# => '<form accept-charset="UTF-8" action="/people/search"
            method="get" class="nifty_form">'
\end{code}

\subsection{ Helpers for Generating Form Elements}

Rails provides a series of helpers for generating form elements such  as checkboxes, text fields, and radio buttons. These basic helpers, with  names ending in “\_tag” (such as text\_field\_tag and check\_box\_tag), generate just a single $<$input$>$  element. The first parameter to these is always the name of the input.  When the form is submitted, the name will be passed along with the form  data, and will make its way to the params hash in the controller with the value entered by the user for that field. For example, if the form contains $<$\%= text\_field\_tag(:query) \%$>$, then you would be able to get the value of this field in the controller with params[:query].

When naming inputs, Rails uses certain conventions that make it  possible to submit parameters with non-scalar values such as arrays or  hashes, which will also be accessible in params. You can read more about them in \hyperlink{understanding-parameter-naming-conventions}{chapter 7 of this guide}. For details on the precise usage of these helpers, please refer to the \href{http://api.rubyonrails.org/classes/ActionView/Helpers/FormTagHelper.html}{API documentation}.

\subsubsection{ Checkboxes}

Checkboxes are form controls that give the user a set of options they can enable or disable:
\begin{code}
<%= check_box_tag(:pet_dog) %>
<%= label_tag(:pet_dog, "I own a dog") %>
<%= check_box_tag(:pet_cat) %>
<%= label_tag(:pet_cat, "I own a cat") %>
\end{code}

This generates the following:
\begin{code}
<input id="pet_dog" name="pet_dog" type="checkbox" value="1" />
<label for="pet_dog">I own a dog</label>
<input id="pet_cat" name="pet_cat" type="checkbox" value="1" />
<label for="pet_cat">I own a cat</label>
\end{code}

The first parameter to check\_box\_tag, of course, is the name  of the input. The second parameter, naturally, is the value of the  input. This value will be included in the form data (and be present in params) when the checkbox is checked.

\subsubsection{ Radio Buttons}

Radio buttons, while similar to checkboxes, are controls that specify  a set of options in which they are mutually exclusive (i.e., the user  can only pick one):
\begin{code}
<%= radio_button_tag(:age, "child") %>
<%= label_tag(:age_child, "I am younger than 21") %>
<%= radio_button_tag(:age, "adult") %>
<%= label_tag(:age_adult, "I'm over 21") %>
\end{code}

Output:
\begin{code}
<input id="age_child" name="age" type="radio" value="child" />
<label for="age_child">I am younger than 21</label>
<input id="age_adult" name="age" type="radio" value="adult" />
<label for="age_adult">I'm over 21</label>
\end{code}

As with check\_box\_tag, the second parameter to radio\_button\_tag  is the value of the input. Because these two radio buttons share the  same name (age) the user will only be able to select one, and params[:age] will contain either “child” or “adult”.

Always use labels for checkbox and radio buttons.  They associate text with a specific option and make it easier for users  to click the inputs by expanding the clickable region.

\subsection{ Other Helpers of Interest}

Other form controls worth mentioning are textareas, password fields, hidden fields, search fields, telephone fields, URL fields and email fields:
\begin{code}
<%= text_area_tag(:message, "Hi, nice site", :size => "24x6") %>
<%= password_field_tag(:password) %>
<%= hidden_field_tag(:parent_id, "5") %>
<%= search_field(:user, :name) %>
<%= telephone_field(:user, :phone) %>
<%= url_field(:user, :homepage) %>
<%= email_field(:user, :address) %>
\end{code}

Output:
\begin{code}
<textarea id="message" name="message" cols="24" rows="6">
  Hi, nice site
</textarea>
<input id="password" name="password" type="password" />
<input id="parent_id" name="parent_id" type="hidden" value="5" />
<input id="user_name" name="user[name]" size="30" type="search" />
<input id="user_phone" name="user[phone]" size="30" type="tel" />
<input id="user_homepage" size="30" name="user[homepage]" type="url" />
<input id="user_address" size="30" name="user[address]" type="email" />
\end{code}

Hidden inputs are not shown to the user but instead hold data like  any textual input. Values inside them can be changed with JavaScript.

The search, telephone, URL,  and email inputs are HTML5 controls. If you require your app to have a  consistent experience in older browsers, you will need an HTML5 polyfill  (provided by CSS and/or JavaScript). There is definitely \href{https://github.com/Modernizr/Modernizr/wiki/HTML5-Cross-Browser-Polyfills}{no shortage of solutions for this}, although a couple of popular tools at the moment are \href{http://www.modernizr.com/}{Modernizr} and \href{http://yepnopejs.com/}{yepnope}, which provide a simple way to add functionality based on the presence of detected HTML5 features.

If you’re using password input fields (for any  purpose), you might want to configure your application to prevent those  parameters from being logged. You can learn about this in the \href{http://guides.rubyonrails.org/security.html#logging}{Security Guide}.

\section{ Dealing with Model Objects}

\subsection{ Model Object Helpers}

A particularly common task for a form is editing or creating a model object. While the *\_tag  helpers can certainly be used for this task they are somewhat verbose  as for each tag you would have to ensure the correct parameter name is  used and set the default value of the input appropriately. Rails  provides helpers tailored to this task. These helpers lack the \_tag  suffix, for example text\_field, text\_area.

For these helpers the first argument is the name of an instance  variable and the second is the name of a method (usually an attribute)  to call on that object. Rails will set the value of the input control to  the return value of that method for the object and set an appropriate  input name. If your controller has defined @person and that person’s name is Henry then a form containing:
\begin{code}
<%= text_field(:person, :name) %>
\end{code}

will produce output similar to
\begin{code}
<input id="person_name" name="person[name]" type="text" value="Henry"/>
\end{code}

Upon form submission the value entered by the user will be stored in params[:person][:name]. The params[:person] hash is suitable for passing to Person.new or, if @person is an instance of Person, @person.update\_attributes.  While the name of an attribute is the most common second parameter to  these helpers this is not compulsory. In the example above, as long as  person objects have a name and a name= method Rails will be happy.

You must pass the name of an instance variable, i.e. :person or "person", not an actual instance of your model object.

Rails provides helpers for displaying the validation errors associated with a model object. These are covered in detail by the \href{http://guides.rubyonrails.org/active_record_validations_callbacks.html#displaying-validation-errors-in-the-view}{Active Record Validations and Callbacks} guide.

\subsection{ Binding a Form to an Object}

While this is an increase in comfort it is far from perfect. If  Person has many attributes to edit then we would be repeating the name  of the edited object many times. What we want to do is somehow bind a  form to a model object, which is exactly what form\_for does.

Assume we have a controller for dealing with articles \\ app/controllers/articles\_controller.rb:
\begin{code}
def new
  @article = Article.new
end
\end{code}

The corresponding view app/views/articles/new.html.erb using form\_for looks like this:
\begin{code}
<%= form_for @article, :url => { :action => "create" }, 
    :html => {:class => "nifty_form"} do |f| %>
  <%= f.text_field :title %>
  <%= f.text_area :body, :size => "60x12" %>
  <%= f.submit "Create" %>
<% end %>
\end{code}

There are a few things to note here:
\begin{enumerate}
	\item @article is the actual object being edited.
	\item There is a single hash of options. Routing options are passed in the :url hash, HTML options are passed in the :html hash. Also you can provide a :namespace  option for your form to ensure uniqueness of id attributes on form  elements. The namespace attribute will be prefixed with underscore on  the generated HTML id.
	\item The form\_for method yields a \textbf{form builder} object (the f variable).
	\item Methods to create form controls are called \textbf{on} the form builder object f
\end{enumerate}

The resulting HTML is:
\begin{code}
<form accept-charset="UTF-8" action="/articles/create" 
                      method="post" class="nifty_form">
<input id="article_title" name="article[title]" size="30" type="text" />
<textarea id="article_body" name="article[body]" cols="60" rows="12">
</textarea>
<input name="commit" type="submit" value="Create" />
</form>
\end{code}

The name passed to form\_for controls the key used in params to access the form’s values. Here the name is article and so all the inputs have names of the form article[\emph{attribute\_name]}. Accordingly, in the create action params[:article] will be a hash with keys :title and :body. You can read more about the significance of input names in the parameter\_names section.

The helper methods called on the form builder are identical to the  model object helpers except that it is not necessary to specify which  object is being edited since this is already managed by the form  builder.

You can create a similar binding without actually creating $<$form$>$ tags with the fields\_for  helper. This is useful for editing additional model objects with the  same form. For example if you had a Person model with an associated  ContactDetail model you could create a form for creating both like so:
\begin{code}
<%= form_for @person, :url => { :action => "create" } do |person_form| %>
  <%= person_form.text_field :name %>
  <%= fields_for @person.contact_detail do |contact_details_form| %>
    <%= contact_details_form.text_field :phone_number %>
  <% end %>
<% end %>
\end{code}

which produces the following output:
\begin{code}
<form accept-charset="UTF-8" action="/people/create" 
    class="new_person" id="new_person" method="post">
  <input id="person_name" name="person[name]" size="30" type="text" />
  <input id="contact_detail_phone_number" 
      name="contact_detail[phone_number]" size="30" type="text" />
</form>
\end{code}

The object yielded by fields\_for is a form builder like the one yielded by form\_for (in fact form\_for calls fields\_for internally).

\subsection{ Relying on Record Identification}

The Article model is directly available to users of the application,  so — following the best practices for developing with Rails — you should  declare it \textbf{a resource}:
\begin{code}
resources :articles
\end{code}

Declaring a resource has a number of side-affects. See \href{http://guides.rubyonrails.org/routing.html#resource-routing-the-rails-default}{Rails Routing From the Outside In} for more information on setting up and using resources.

When dealing with RESTful resources, calls to form\_for can get significantly easier if you rely on \textbf{record identification}. In short, you can just pass the model instance and have Rails figure out model name and the rest:
\begin{code}
## Creating a new article
# long-style:
form_for(@article, :url => articles_path)
# same thing, short-style (record identification gets used):
form_for(@article)
 
## Editing an existing article
# long-style:
form_for(@article, :url => article_path(@article),
                   :html => { :method => "put" })
# short-style:
form_for(@article)
\end{code}

Notice how the short-style form\_for invocation is  conveniently the same, regardless of the record being new or existing.  Record identification is smart enough to figure out if the record is new  by asking record.new\_record?. It also selects the correct path to submit to and the name based on the class of the object.

Rails will also automatically set the class and id of the form appropriately: a form creating an article would have id and classnew\_article. If you were editing the article with id 23, the class would be set to edit\_article and the id to edit\_article\_23. These attributes will be omitted for brevity in the rest of this guide.

When you’re using STI  (single-table inheritance) with your models, you can’t rely on record  identification on a subclass if only their parent class is declared a  resource. You will have to specify the model name, :url, and :method explicitly.

\subsubsection{ Dealing with Namespaces}

If you have created namespaced routes, form\_for has a nifty shorthand for that too. If your application has an admin namespace then
\begin{code}
form_for [:admin, @article]
\end{code}

will create a form that submits to the articles controller inside the admin namespace (submitting to admin\_article\_path(@article) in the case of an update). If you have several levels of namespacing then the syntax is similar:
\begin{code}
form_for [:admin, :management, @article]
\end{code}

For more information on Rails’ routing system and the associated conventions, please see the \href{http://guides.rubyonrails.org/routing.html}{routing guide}.

\subsection{ How do forms with PUT or DELETE methods work?}

The Rails framework encourages RESTful design of your applications, which means you’ll be making a lot of “PUT” and “DELETE” requests (besides “GET” and “POST”). However, most browsers \emph{don’t support} methods other than “GET” and “POST” when it comes to submitting forms.

Rails works around this issue by emulating other methods over POST with a hidden input named "\_method", which is set to reflect the desired method:
\begin{code}
form_tag(search_path, :method => "put")
\end{code}

output:
\begin{code}
<form accept-charset="UTF-8" action="/search" method="post">
  <div style="margin:0;padding:0">
    <input name="_method" type="hidden" value="put" />
    <input name="utf8" type="hidden" value="&#x2713;" />
    <input name="authenticity_token" type="hidden" 
           value="f755bb0ed134b76c432144748a6d4b7a7ddf2b71" />
  </div>
  ...
\end{code}

When parsing POSTed data, Rails will take into account the special \_method parameter and acts as if the HTTP method was the one specified inside it (“PUT” in this example).

\section{ Making Select Boxes with Ease}

Select boxes in HTML require a significant amount of markup (one OPTION element for each option to choose from), therefore it makes the most sense for them to be dynamically generated.

Here is what the markup might look like:
\begin{code}
<select name="city_id" id="city_id">
  <option value="1">Lisbon</option>
  <option value="2">Madrid</option>
  ...
  <option value="12">Berlin</option>
</select>
\end{code}

Here you have a list of cities whose names are presented to the user.  Internally the application only wants to handle their IDs so they are  used as the options’ value attribute. Let’s see how Rails can help out  here.

\subsection{ The Select and Option Tags}

The most generic helper is select\_tag, which — as the name implies — simply generates the SELECT tag that encapsulates an options string:
\begin{code}
<%= select_tag(:city_id, '<option value="1">Lisbon</option>...') %>
\end{code}

This is a start, but it doesn’t dynamically create the option tags. You can generate option tags with the options\_for\_select helper:
\begin{code}
<%= options_for_select([['Lisbon', 1], ['Madrid', 2], ...]) %>
 
output:
 
<option value="1">Lisbon</option>
<option value="2">Madrid</option>
...
\end{code}

The first argument to options\_for\_select is a nested array  where each element has two elements: option text (city name) and option  value (city id). The option value is what will be submitted to your  controller. Often this will be the id of a corresponding database object  but this does not have to be the case.

Knowing this, you can combine select\_tag and options\_for\_select to achieve the desired, complete markup:
\begin{code}
<%= select_tag(:city_id, options_for_select(...)) %>
\end{code}

options\_for\_select allows you to pre-select an option by passing its value.
\begin{code}
<%= options_for_select([['Lisbon', 1], ['Madrid', 2], ...], 2) %>
 
output:
 
<option value="1">Lisbon</option>
<option value="2" selected="selected">Madrid</option>
...
\end{code}

Whenever Rails sees that the internal value of an option being generated matches this value, it will add the selected attribute to that option.

The second argument to options\_for\_select must be exactly equal to the desired internal value. In particular if the value is the integer 2 you cannot pass “2” to options\_for\_select — you must pass 2. Be aware of values extracted from the params hash as they are all strings.

\subsection{ Select Boxes for Dealing with Models}

In most cases form controls will be tied to a specific database model  and as you might expect Rails provides helpers tailored for that  purpose. Consistent with other form helpers, when dealing with models  you drop the \_tag suffix from select\_tag:
\begin{code}
# controller:
@person = Person.new(:city_id => 2)
\end{code}
\begin{code}
# view:
<%= select(:person, :city_id, [['Lisbon', 1], ['Madrid', 2], ...]) %>
\end{code}

Notice that the third parameter, the options array, is the same kind of argument you pass to options\_for\_select.  One advantage here is that you don’t have to worry about pre-selecting  the correct city if the user already has one — Rails will do this for  you by reading from the @person.city\_id attribute.

As with other helpers, if you were to use the select helper on a form builder scoped to the @person object, the syntax would be:
\begin{code}
# select on a form builder
<%= f.select(:city_id, ...) %>
\end{code}

If you are using select (or similar helpers such as collection\_select, select\_tag) to set a belongs\_to association you must pass the name of the foreign key (in the example above city\_id), not the name of association itself. If you specify city instead of city\_id Active Record will raise an error along the lines of  ActiveRecord::AssociationTypeMismatch: City(\#17815740) expected, got String(\#1138750)  when you pass the params hash to Person.new or update\_attributes.  Another way of looking at this is that form helpers only edit  attributes. You should also be aware of the potential security  ramifications of allowing users to edit foreign keys directly. You may  wish to consider the use of attr\_protected and attr\_accessible. For further details on this, see the \href{http://guides.rubyonrails.org/security.html#_mass_assignment}{Ruby On Rails Security Guide}.

\subsection{ Option Tags from a Collection of Arbitrary Objects}

Generating options tags with options\_for\_select requires  that you create an array containing the text and value for each option.  But what if you had a City model (perhaps an Active Record one) and you  wanted to generate option tags from a collection of those objects? One  solution would be to make a nested array by iterating over them:
\begin{code}
<% cities_array = City.all.map { |city| [city.name, city.id] } %>
<%= options_for_select(cities_array) %>
\end{code}

This is a perfectly valid solution, but Rails provides a less verbose alternative: options\_from\_collection\_for\_select.  This helper expects a collection of arbitrary objects and two  additional arguments: the names of the methods to read the option \textbf{value} and \textbf{text} from, respectively:
\begin{code}
<%= options_from_collection_for_select(City.all, :id, :name) %>
\end{code}

As the name implies, this only generates option tags. To generate a  working select box you would need to use it in conjunction with select\_tag, just as you would with options\_for\_select. When working with model objects, just as select combines select\_tag and options\_for\_select, collection\_select combines select\_tag with \\ options\_from\_collection\_for\_select.
\begin{code}
<%= collection_select(:person, :city_id, City.all, :id, :name) %>
\end{code}

To recap, options\_from\_collection\_for\_select is to collection\_select what options\_for\_select is to select.

Pairs passed to options\_for\_select should have the name first and the id second, however with options\_from\_collection\_for\_select the first argument is the value method and the second the text method.

\subsection{ Time Zone and Country Select}

To leverage time zone support in Rails, you have to ask your users  what time zone they are in. Doing so would require generating select  options from a list of pre-defined TimeZone objects using collection\_select, but you can simply use the time\_zone\_select helper that already wraps this:
\begin{code}
<%= time_zone_select(:person, :time_zone) %>
\end{code}

There is also time\_zone\_options\_for\_select helper for a more manual (therefore more customizable) way of doing this. Read the API documentation to learn about the possible arguments for these two methods.

Rails \emph{used} to have a country\_select helper for choosing countries, but this has been extracted to the \href{https://github.com/chrislerum/country_select}{country\_select plugin}.  When using this, be aware that the exclusion or inclusion of certain  names from the list can be somewhat controversial (and was the reason  this functionality was extracted from Rails).

\section{ Using Date and Time Form Helpers}

The date and time helpers differ from all the other form helpers in two important respects:
\begin{enumerate}
	\item Dates and times are not representable by a single input element.  Instead you have several, one for each component (year, month, day etc.)  and so there is no single value in your params hash with your date or time.
	\item Other helpers use the \_tag suffix to indicate whether a helper is a barebones helper or one that operates on model objects. With dates and times, select\_date, select\_time and select\_datetime are the barebones helpers, date\_select, time\_select and datetime\_select are the equivalent model object helpers.
\end{enumerate}

Both of these families of helpers will create a series of select boxes for the different components (year, month, day etc.).

\subsection{ Barebones Helpers}

The select\_* family of helpers take as their first argument  an instance of Date, Time or DateTime that is used as the currently  selected value. You may omit this parameter, in which case the current  date is used. For example
\begin{code}
<%= select_date Date.today, :prefix => :start_date %>
\end{code}

outputs (with actual option values omitted for brevity)
\begin{code}
<select id="start_date_year" name="start_date[year]"> ... </select>
<select id="start_date_month" name="start_date[month]"> ... </select>
<select id="start_date_day" name="start_date[day]"> ... </select>
\end{code}

The above inputs would result in params[:start\_date] being a hash with keys :year, :month, :day.  To get an actual Time or Date object you would have to extract these  values and pass them to the appropriate constructor, for example
\begin{code}
Date.civil(params[:start_date][:year].to_i,
           params[:start_date][:month].to_i, 
           params[:start_date][:day].to_i)
\end{code}

The :prefix option is the key used to retrieve the hash of date components from the params hash. Here it was set to start\_date, if omitted it will default to date.

\subsection{ Model Object Helpers}

select\_date does not work well with forms that update or create Active Record objects as Active Record expects each element of the params  hash to correspond to one attribute. The model object helpers for dates and times submit parameters with  special names, when Active Record sees parameters with such names it  knows they must be combined with the other parameters and given to a  constructor appropriate to the column type. For example:
\begin{code}
<%= date_select :person, :birth_date %>
\end{code}

outputs (with actual option values omitted for brevity)
\begin{code}
<select id="person_birth_date_1i" name="person[birth_date(1i)]">
</select>
<select id="person_birth_date_2i" name="person[birth_date(2i)]">
</select>
<select id="person_birth_date_3i" name="person[birth_date(3i)]">
</select>
\end{code}

which results in a params hash like
\begin{code}
{:person => {'birth_date(1i)' => '2008', 
'birth_date(2i)' => '11', 'birth_date(3i)' => '22'}}
\end{code}

When this is passed to Person.new (or update\_attributes), Active Record spots that these parameters should all be used to construct the birth\_date attribute and uses the suffixed information to determine in which order it should pass these parameters to functions such as Date.civil.

\subsection{ Common Options}

Both families of helpers use the same core set of functions to  generate the individual select tags and so both accept largely the same  options. In particular, by default Rails will generate year options 5  years either side of the current year. If this is not an appropriate  range, the :start\_year and :end\_year options override this. For an exhaustive list of the available options, refer to the \href{http://api.rubyonrails.org/classes/ActionView/Helpers/DateHelper.html}{API documentation}.

As a rule of thumb you should be using date\_select when working with model objects and select\_date in other cases, such as a search form which filters results by date.

In many cases the built-in date pickers are clumsy  as they do not aid the user in working out the relationship between the  date and the day of the week.

\subsection{ Individual Components}

Occasionally you need to display just a single date component such as  a year or a month. Rails provides a series of helpers for this, one for  each component select\_year, select\_month, select\_day, select\_hour, select\_minute, select\_second.  These helpers are fairly straightforward. By default they will generate  an input field named after the time component (for example “year” for select\_year, “month” for select\_month etc.) although this can be overridden with the  :field\_name option. The :prefix option works in the same way that it does for select\_date and select\_time and has the same default value.

The first parameter specifies which value should be selected and can  either be an instance of a Date, Time or DateTime, in which case the  relevant component will be extracted, or a numerical value. For example
\begin{code}
<%= select_year(2009) %>
<%= select_year(Time.now) %>
\end{code}

will produce the same output if the current year is 2009 and the value chosen by the user can be retrieved by params[:date][:year].

\section{ Uploading Files}

A common task is uploading some sort of file, whether it’s a picture of a person or a CSV file containing data to process. The most important thing to remember with file uploads is that the rendered form’s encoding \textbf{MUST} be set to “multipart/form-data”. If you use form\_for, this is done automatically. If you use form\_tag, you must set it yourself, as per the following example.

The following two forms both upload a file.
\begin{code}
<%= form_tag({:action => :upload}, :multipart => true) do %>
  <%= file_field_tag 'picture' %>
<% end %>
 
<%= form_for @person do |f| %>
  <%= f.file_field :picture %>
<% end %>
\end{code}

Since Rails 3.1, forms rendered using form\_for have their encoding set to multipart/form-data automatically once a file\_field is used inside the block. Previous versions required you to set this explicitly.

Rails provides the usual pair of helpers: the barebones file\_field\_tag and the model oriented file\_field.  The only difference with other helpers is that you cannot set a default  value for file inputs as this would have no meaning. As you would  expect in the first case the uploaded file is in params[:picture] and in the second case in params[:person][:picture].

\subsection{ What Gets Uploaded}

The object in the params hash is an instance of a subclass  of IO. Depending on the size of the uploaded file it may in fact be a  StringIO or an instance of File backed by a temporary file. In both  cases the object will have an original\_filename attribute containing the name the file had on the user’s computer and a content\_type attribute containing the MIME type of the uploaded file. The following snippet saves the uploaded content in \#\{Rails.root\/public/uploads} under the same name as the original file (assuming the form was the one in the previous example).
\begin{code}
def upload
  uploaded_io = params[:person][:picture]
  File.open(Rails.root.join('public', 'uploads',
       uploaded_io.original_filename), 'w') do |file|
    file.write(uploaded_io.read)
  end
end
\end{code}

Once a file has been uploaded, there are a multitude of potential  tasks, ranging from where to store the files (on disk, Amazon S3, etc)  and associating them with models to resizing image files and generating  thumbnails. The intricacies of this are beyond the scope of this guide,  but there are several libraries designed to assist with these. Two of  the better known ones are \href{https://github.com/jnicklas/carrierwave}{CarrierWave} and \href{http://www.thoughtbot.com/projects/paperclip}{Paperclip}.

If the user has not selected a file the corresponding parameter will be an empty string.

\subsection{ Dealing with Ajax}

Unlike other forms making an asynchronous file upload form is not as simple as providing form\_for with :remote =$>$ true.  With an Ajax form the serialization is done by JavaScript running  inside the browser and since JavaScript cannot read files from your hard  drive the file cannot be uploaded. The most common workaround is to use  an invisible iframe that serves as the target for the form submission.

\section{ Customizing Form Builders}

As mentioned previously the object yielded by form\_for and fields\_for  is an instance of FormBuilder (or a subclass thereof). Form builders  encapsulate the notion of displaying form elements for a single object.  While you can of course write helpers for your forms in the usual way  you can also subclass FormBuilder and add the helpers there. For example
\begin{code}
<%= form_for @person do |f| %>
  <%= text_field_with_label f, :first_name %>
<% end %>
\end{code}

can be replaced with
\begin{code}
<%= form_for @person, :builder => LabellingFormBuilder do |f| %>
  <%= f.text_field :first_name %>
<% end %>
\end{code}

by defining a LabellingFormBuilder class similar to the following:
\begin{code}
class LabellingFormBuilder < ActionView::Helpers::FormBuilder
  def text_field(attribute, options={})
    label(attribute) + super
  end
end
\end{code}

If you reuse this frequently you could define a labeled\_form\_for helper that automatically applies the :builder =$>$ LabellingFormBuilder option.

The form builder used also determines what happens when you do
\begin{code}
<%= render :partial => f %>
\end{code}

If f is an instance of FormBuilder then this will render the form partial, setting the partial’s object to the form builder. If the form builder is of class LabellingFormBuilder then the labelling\_form partial would be rendered instead.

\section{ Understanding Parameter Naming Conventions}

As you’ve seen in the previous sections, values from forms can be at the top level of the params hash or nested in another hash. For example in a standard create action for a Person model, params[:model] would usually be a hash of all the attributes for the person to create. The params hash can also contain arrays, arrays of hashes and so on.

Fundamentally HTML forms don’t know about  any sort of structured data, all they generate is name–value pairs,  where pairs are just plain strings. The arrays and hashes you see in  your application are the result of some parameter naming conventions  that Rails uses.

You may find you can try out examples in this  section faster by using the console to directly invoke Racks’ parameter  parser. For example,
\begin{code}
Rack::Utils.parse_query "name=fred&phone=0123456789"
# => {"name"=>"fred", "phone"=>"0123456789"}
\end{code}

\subsection{ Basic Structures}

The two basic structures are arrays and hashes. Hashes mirror the syntax used for accessing the value in params. For example if a form contains
\begin{code}
<input id="person_name" name="person[name]" type="text" value="Henry"/>
\end{code}

the params hash will contain
\begin{code}
{'person' => {'name' => 'Henry'}}
\end{code}

and params[:person][:name] will retrieve the submitted value in the controller.

Hashes can be nested as many levels as required, for example
\begin{code}
<input id="person_address_city" name="person[address][city]" 
                              type="text" value="New York"/>
\end{code}

will result in the params hash being
\begin{code}
{'person' => {'address' => {'city' => 'New York'}}}
\end{code}

Normally Rails ignores duplicate parameter names. If the parameter  name contains an empty set of square brackets [] then they will be  accumulated in an array. If you wanted people to be able to input  multiple phone numbers, you could place this in the form:
\begin{code}
<input name="person[phone_number][]" type="text"/>
<input name="person[phone_number][]" type="text"/>
<input name="person[phone_number][]" type="text"/>
\end{code}

This would result in params[:person][:phone\_number] being an array.

\subsection{ Combining Them}

We can mix and match these two concepts. For example, one element of a  hash might be an array as in the previous example, or you can have an  array of hashes. For example a form might let you create any number of  addresses by repeating the following form fragment
\begin{code}
<input name="addresses[][line1]" type="text"/>
<input name="addresses[][line2]" type="text"/>
<input name="addresses[][city]" type="text"/>
\end{code}

This would result in params[:addresses] being an array of hashes with keys line1, line2 and city.  Rails decides to start accumulating values in a new hash whenever it  encounters an input name that already exists in the current hash.

There’s a restriction, however, while hashes can be nested  arbitrarily, only one level of “arrayness” is allowed. Arrays can be  usually replaced by hashes, for example instead of having an array of  model objects one can have a hash of model objects keyed by their id, an  array index or some other parameter.

Array parameters do not play well with the check\_box helper. According to the HTML  specification unchecked checkboxes submit no value. However it is often  convenient for a checkbox to always submit a value. The check\_box  helper fakes this by creating an auxiliary hidden input with the same  name. If the checkbox is unchecked only the hidden input is submitted  and if it is checked then both are submitted but the value submitted by  the checkbox takes precedence. When working with array parameters this  duplicate submission will confuse Rails since duplicate input names are  how it decides when to start a new array element. It is preferable to  either use check\_box\_tag or to use hashes instead of arrays.

\subsection{ Using Form Helpers}

The previous sections did not use the Rails form helpers at all.  While you can craft the input names yourself and pass them directly to  helpers such as text\_field\_tag Rails also provides higher level support. The two tools at your disposal here are the name parameter to form\_for and fields\_for and the :index option that helpers take.

You might want to render a form with a set of edit fields for each of a person’s addresses. For example:
\begin{code}
<%= form_for @person do |person_form| %>
  <%= person_form.text_field :name %>
  <% @person.addresses.each do |address| %>
  <%= person_form.fields_for address, :index => address do |address_form|%>
      <%= address_form.text_field :city %>
  <% end %>
  <% end %>
<% end %>
\end{code}

Assuming the person had two addresses, with ids 23 and 45 this would create output similar to this:
\begin{code}
<form accept-charset="UTF-8" action="/people/1" 
      class="edit_person" id="edit_person_1" method="post">
<input id="person_name" name="person[name]" size="30" type="text" />
<input id="person_address_23_city" 
  name="person[address][23][city]" size="30" type="text" />
<input id="person_address_45_city" 
  name="person[address][45][city]" size="30" type="text" />
</form>
\end{code}

This will result in a params hash that looks like
\begin{code}
{'person' => {'name' => 'Bob', 'address' => 
    {'23' => {'city' => 'Paris'}, '45' => {'city' => 'London'}}}}
\end{code}

Rails knows that all these inputs should be part of the person hash because you called fields\_for on the first form builder. By specifying an :index option you’re telling Rails that instead of naming the inputs person[address][city]  it should insert that index surrounded by [] between the address and  the city. If you pass an Active Record object as we did then Rails will  call to\_param on it, which by default returns the database id.  This is often useful as it is then easy to locate which Address record  should be modified. You can pass numbers with some other significance,  strings or even nil (which will result in an array parameter being created).

To create more intricate nestings, you can specify the first part of the input name (person[address] in the previous example) explicitly, for example
\begin{code}
<%= fields_for 'person[address][primary]', address,
               :index => address do |address_form| %>
  <%= address_form.text_field :city %>
<% end %>
\end{code}

will create inputs like
\begin{code}
<input id="person_address_primary_1_city" 
       name="person[address][primary][1][city]"
       size="30" type="text" value="bologna" />
\end{code}

As a general rule the final input name is the concatenation of the name given to fields\_for/form\_for, the index value and the name of the attribute. You can also pass an :index option directly to helpers such as text\_field, but it is usually less repetitive to specify this at the form builder level rather than on individual input controls.

As a shortcut you can append [] to the name and omit the :index option. This is the same as specifying :index =$>$ address so
\begin{code}
<%= fields_for 'person[address][primary][]', address do |address_form| %>
  <%= address_form.text_field :city %>
<% end %>
\end{code}

produces exactly the same output as the previous example.

\section{ Forms to external resources}

If you need to post some data to an external resource it is still  great to build your from using rails form helpers. But sometimes you  need to set an authenticity\_token for this resource. You can do it by passing an :authenticity\_token =$>$ 'your\_external\_token' parameter to the form\_tag options:
\begin{code}
<%= form_tag 'http://farfar.away/form', 
    :authenticity_token => 'external_token') do %>
  Form contents
<% end %>
\end{code}

Sometimes when you submit data to an external resource, like payment  gateway, fields you can use in your form are limited by an external API. So you may want not to generate an authenticity\_token hidden field at all. For doing this just pass false to the :authenticity\_token option:
\begin{code}
<%= form_tag 'http://farfar.away/form', 
    :authenticity_token => false) do %>
  Form contents
<% end %>
\end{code}

The same technique is available for the form\_for too:
\begin{code}
<%= form_for @invoice, :url => external_url, 
    :authenticity_token => 'external_token' do |f|
  Form contents
<% end %>
\end{code}

Or if you don’t want to render an authenticity\_token field:
\begin{code}
<%= form_for @invoice, :url => external_url, 
    :authenticity_token => false do |f|
  Form contents
<% end %>
\end{code}

\section{ Building Complex Forms}

Many apps grow beyond simple forms editing a single object. For  example when creating a Person you might want to allow the user to (on  the same form) create multiple address records (home, work, etc.). When  later editing that person the user should be able to add, remove or  amend addresses as necessary. While this guide has shown you all the  pieces necessary to handle this, Rails does not yet have a standard  end-to-end way of accomplishing this, but many have come up with viable  approaches. These include:
\begin{itemize}
	\item As of Rails 2.3, Rails includes \href{http://guides.rubyonrails.org/2_3_release_notes.html#nested-attributes}{Nested Attributes} and \href{http://guides.rubyonrails.org/2_3_release_notes.html#nested-object-forms}{Nested Object Forms}
	\item Ryan Bates’ series of Railscasts on \href{http://railscasts.com/episodes/75}{complex forms}
	\item Handle Multiple Models in One Form from \href{http://media.pragprog.com/titles/fr_arr/multiple_models_one_form.pdf}{Advanced Rails Recipes}
	\item Eloy Duran’s \href{https://github.com/alloy/complex-form-examples/}{complex-forms-examples} application
	\item Lance Ivy’s \href{https://github.com/cainlevy/nested_assignment/tree/master}{nested\_assignment} plugin and \href{https://github.com/cainlevy/complex-form-examples/tree/cainlevy}{sample application}
	\item James Golick’s \href{https://github.com/jamesgolick/attribute_fu}{attribute\_fu} plugin
\end{itemize}

\chapter{Action Controller Overview}

In this guide you will learn how controllers work and how they fit  into the request cycle in your application. After reading this guide,  you will be able to:
\begin{itemize}
	\item Follow the flow of a request through a controller
	\item Understand why and how to store data in the session or cookies
	\item Work with filters to execute code during request processing
	\item Use Action Controller’s built-in HTTP authentication
	\item Stream data directly to the user’s browser
	\item Filter sensitive parameters so they do not appear in the application’s log
	\item Deal with exceptions that may be raised during request processing
\end{itemize}

\section{ What Does a Controller Do?}

Action Controller is the C in MVC. After  routing has determined which controller to use for a request, your  controller is responsible for making sense of the request and producing  the appropriate output. Luckily, Action Controller does most of the  groundwork for you and uses smart conventions to make this as  straightforward as possible.

For most conventional \href{http://en.wikipedia.org/wiki/Representational_state_transfer}{RESTful}  applications, the controller will receive the request (this is  invisible to you as the developer), fetch or save data from a model and  use a view to create HTML output. If your  controller needs to do things a little differently, that’s not a  problem, this is just the most common way for a controller to work.

A controller can thus be thought of as a middle man between models  and views. It makes the model data available to the view so it can  display that data to the user, and it saves or updates data from the  user to the model.

For more details on the routing process, see \href{http://guides.rubyonrails.org/routing.html}{Rails Routing from the Outside In}.

\section{ Methods and Actions}

A controller is a Ruby class which inherits from ApplicationController  and has methods just like any other class. When your application  receives a request, the routing will determine which controller and  action to run, then Rails creates an instance of that controller and  runs the method with the same name as the action.
\begin{code}
class ClientsController < ApplicationController
  def new
  end
end
\end{code}

As an example, if a user goes to /clients/new in your application to add a new client, Rails will create an instance of ClientsController and run the new method. Note that the empty method from the example above could work just fine because Rails will by default render the new.html.erb view unless the action says otherwise. The new method could make available to the view a @client instance variable by creating a new Client:
\begin{code}
def new
  @client = Client.new
end
\end{code}

The \href{http://guides.rubyonrails.org/layouts_and_rendering.html}{Layouts \& Rendering Guide} explains this in more detail.

ApplicationController inherits from ActionController::Base,  which defines a number of helpful methods. This guide will cover some  of these, but if you’re curious to see what’s in there, you can see all  of them in the API documentation or in the source itself.

Only public methods are callable as actions. It is a best practice to  lower the visibility of methods which are not intended to be actions,  like auxiliary methods or filters.

\section{ Parameters}

You will probably want to access data sent in by the user or other  parameters in your controller actions. There are two kinds of parameters  possible in a web application. The first are parameters that are sent  as part of the URL, called query string parameters. The query string is everything after "?" in the URL. The second type of parameter is usually referred to as POST data. This information usually comes from an HTML form which has been filled in by the user. It’s called POST data because it can only be sent as part of an HTTPPOST request. Rails does not make any distinction between query string parameters and POST parameters, and both are available in the params hash in your controller:
\begin{code}
class ClientsController < ActionController::Base
  # This action uses query string parameters because it gets run
  # by an HTTP GET request, but this does not make any difference
  # to the way in which the parameters are accessed. The URL for
  # this action would look like this in order to list activated
  # clients: /clients?status=activated
  def index
    if params[:status] == "activated"
      @clients = Client.activated
    else
      @clients = Client.unactivated
    end
  end
 
  # This action uses POST parameters. They are most likely coming
  # from an HTML form which the user has submitted. The URL for
  # this RESTful request will be "/clients", and the data will be
  # sent as part of the request body.
  def create
    @client = Client.new(params[:client])
    if @client.save
      redirect_to @client
    else
      # This line overrides the default rendering behavior, which
      # would have been to render the "create" view.
      render :action => "new"
    end
  end
end
\end{code}

\subsection{ Hash and Array Parameters}

The params hash is not limited to one-dimensional keys and  values. It can contain arrays and (nested) hashes. To send an array of  values, append an empty pair of square brackets "[]" to the key name:
\begin{code}
GET /clients?ids[]=1&ids[]=2&ids[]=3
\end{code}

The actual URL in this  example will be encoded as \\ "/clients?ids\%5b\%5d=1\&ids\%5b\%5d=2\&ids\%5b\%5d=3" as "[" and "]"  are not allowed in URLs. Most of the time you don’t have to worry about  this because the browser will take care of it for you, and Rails will  decode it back when it receives it, but if you ever find yourself having  to send those requests to the server manually you have to keep this in  mind.

The value of params[:ids] will now be ["1", "2", "3"]. Note that parameter values are always strings; Rails makes no attempt to guess or cast the type.

To send a hash you include the key name inside the brackets:
\begin{code}
<form accept-charset="UTF-8" action="/clients" method="post">
  <input type="text" name="client[name]" value="Acme" />
  <input type="text" name="client[phone]" value="12345" />
  <input type="text" name="client[address][postcode]" value="12345" />
  <input type="text" name="client[address][city]" value="Carrot City" />
</form>
\end{code}

When this form is submitted, the value of params[:client] will be \{"name" =$>$ "Acme", "phone" =$>$ "12345", "address" =$>$ \{"postcode" =$>$ "12345", "city" =$>$ "Carrot City"\\}}. Note the nested hash in params[:client][:address].

Note that the params hash is actually an instance of HashWithIndifferentAccess from Active Support, which acts like a hash that lets you use symbols and strings interchangeably as keys.

\subsection{ JSON/XML parameters}

If you’re writing a web service application, you might find yourself more comfortable on accepting parameters in JSON or XML format. Rails will automatically convert your parameters into params hash, which you’ll be able to access like you would normally do with form data.

So for example, if you are sending this JSON parameter:
\begin{code}
{"company": { "name": "acme", "address": "123 Carrot Street" } }
\end{code}

You’ll get params[:company] as \{:name =$>$ "acme", "address" =$>$ "123 Carrot Street"\}.

Also, if you’ve turned on config.wrap\_parameters in your initializer or calling wrap\_parameters in your controller, you can safely omit the root element in the JSON/XML  parameter. The parameters will be cloned and wrapped in the key  according to your controller’s name by default. So the above parameter  can be written as:
\begin{code}
{ "name": "acme", "address": "123 Carrot Street" }
\end{code}

And assume that you’re sending the data to CompaniesController, it would then be wrapped in :company key like this:
\begin{code}
{ :name => "acme", :address => "123 Carrot Street", 
:company => { :name => "acme", :address => "123 Carrot Street" }}
\end{code}

You can customize the name of the key or specific parameters you want to wrap by consulting the \href{http://api.rubyonrails.org/classes/ActionController/ParamsWrapper.html}{API documentation}

\subsection{ Routing Parameters}

The params hash will always contain the :controller and :action keys, but you should use the methods controller\_name and action\_name instead to access these values. Any other parameters defined by the routing, such as :id  will also be available. As an example, consider a listing of clients  where the list can show either active or inactive clients. We can add a  route which captures the :status parameter in a "pretty" URL:
\begin{code}
match '/clients/:status' => 'clients#index', :foo => "bar"
\end{code}

In this case, when a user opens the URL/clients/active, params[:status] will be set to "active". When this route is used, params[:foo] will also be set to "bar" just like it was passed in the query string. In the same way params[:action] will contain "index".

\subsection{ default\_url\_options}

You can set global default parameters for URL generation by defining a method called default\_url\_options in your controller. Such a method must return a hash with the desired defaults, whose keys must be symbols:
\begin{code}
class ApplicationController < ActionController::Base
  def default_url_options
    {:locale => I18n.locale}
  end
end
\end{code}

These options will be used as a starting point when generating URLs,  so it’s possible they’ll be overridden by the options passed in url\_for calls.

If you define default\_url\_options in ApplicationController, as in the example above, it would be used for all URL generation. The method can also be defined in one specific controller, in which case it only affects URLs generated there.

\section{ Session}

Your application has a session for each user in which you can store  small amounts of data that will be persisted between requests. The  session is only available in the controller and the view and can use one  of a number of different storage mechanisms:
\begin{itemize}
	\item ActionDispatch::Session::CookieStore – Stores everything on the client.
	\item ActiveRecord::SessionStore – Stores the data in a database using Active Record.
	\item ActionDispatch::Session::CacheStore – Stores the data in the Rails cache.
	\item ActionDispatch::Session::MemCacheStore – Stores the data in a  memcached cluster (this is a legacy implementation; consider using  CacheStore instead).
\end{itemize}

All session stores use a cookie to store a unique ID for each session  (you must use a cookie, Rails will not allow you to pass the session ID  in the URL as this is less secure).

For most stores this ID is used to look up the session data on the  server, e.g. in a database table. There is one exception, and that is  the default and recommended session store – the CookieStore – which  stores all session data in the cookie itself (the ID is still available  to you if you need it). This has the advantage of being very lightweight  and it requires zero setup in a new application in order to use the  session. The cookie data is cryptographically signed to make it  tamper-proof, but it is not encrypted, so anyone with access to it can  read its contents but not edit it (Rails will not accept it if it has  been edited).

The CookieStore can store around 4kB of data — much less than the  others — but this is usually enough. Storing large amounts of data in  the session is discouraged no matter which session store your  application uses. You should especially avoid storing complex objects  (anything other than basic Ruby objects, the most common example being  model instances) in the session, as the server might not be able to  reassemble them between requests, which will result in an error.

If your user sessions don’t store critical data or don’t need to be  around for long periods (for instance if you just use the flash for  messaging), you can consider using ActionDispatch::Session::CacheStore.  This will store sessions using the cache implementation you have  configured for your application. The advantage of this is that you can  use your existing cache infrastructure for storing sessions without  requiring any additional setup or administration. The downside, of  course, is that the sessions will be ephemeral and could disappear at  any time.

Read more about session storage in the \href{http://guides.rubyonrails.org/security.html}{Security Guide}.

If you need a different session storage mechanism, you can change it in the \\ config/initializers/session\_store.rb file:
\begin{code}
# Use the database for sessions instead of the cookie-based default,
# which shouldn't be used to store highly confidential information
# (create the session table with "script/rails g session_migration")
# YourApp::Application.config.session_store :active_record_store
\end{code}

Rails sets up a session key (the name of the cookie) when signing the session data. These can also be changed in \\ config/initializers/session\_store.rb:
\begin{code}
# Be sure to restart your server when you modify this file.
 
YourApp::Application.config.session_store :cookie_store,
                                          :key => '_your_app_session'
\end{code}

You can also pass a :domain key and specify the domain name for the cookie:
\begin{code}
# Be sure to restart your server when you modify this file.
 
YourApp::Application.config.session_store :cookie_store, 
:key => '_your_app_session', :domain => ".example.com"
\end{code}

Rails sets up (for the CookieStore) a secret key used for signing the session data. This can be changed in config/initializers/secret\_token.rb
\begin{code}
# Be sure to restart your server when you modify this file.
 
# Your secret key for verifying the integrity of signed cookies.
# If you change this key, all old signed cookies will become invalid!
# Make sure the secret is at least 30 characters and all random,
# no regular words or you'll be exposed to dictionary attacks.
YourApp::Application.config.secret_token = '49d3f3de9ed86c74b94ad6bd0...'
\end{code}

Changing the secret when using the CookieStore will invalidate all existing sessions.

\subsection{ Accessing the Session}

In your controller you can access the session through the session instance method.

Sessions are lazily loaded. If you don’t access  sessions in your action’s code, they will not be loaded. Hence you will  never need to disable sessions, just not accessing them will do the job.

Session values are stored using key/value pairs like a hash:
\begin{code}
class ApplicationController < ActionController::Base
 
  private
 
  # Finds the User with the ID stored in the session with the key
  # :current_user_id This is a common way to handle user login in
  # a Rails application; logging in sets the session value and
  # logging out removes it.
  def current_user
    @_current_user ||= session[:current_user_id] &&
      User.find_by_id(session[:current_user_id])
  end
end
\end{code}

To store something in the session, just assign it to the key like a hash:
\begin{code}
class LoginsController < ApplicationController
  # "Create" a login, aka "log the user in"
  def create
    if user = User.authenticate(params[:username], params[:password])
      # Save the user ID in the session so it can be used in
      # subsequent requests
      session[:current_user_id] = user.id
      redirect_to root_url
    end
  end
end
\end{code}

To remove something from the session, assign that key to be nil:
\begin{code}
class LoginsController < ApplicationController
  # "Delete" a login, aka "log the user out"
  def destroy
    # Remove the user id from the session
    @_current_user = session[:current_user_id] = nil
    redirect_to root_url
  end
end
\end{code}

To reset the entire session, use reset\_session.

\subsection{ The Flash}

The flash is a special part of the session which is cleared with each  request. This means that values stored there will only be available in  the next request, which is useful for storing error messages etc. It is  accessed in much the same way as the session, like a hash. Let’s use the  act of logging out as an example. The controller can send a message  which will be displayed to the user on the next request:
\begin{code}
class LoginsController < ApplicationController
  def destroy
    session[:current_user_id] = nil
    flash[:notice] = "You have successfully logged out"
    redirect_to root_url
  end
end
\end{code}

Note it is also possible to assign a flash message as part of the redirection.
\begin{code}
redirect_to root_url, :notice => "You have successfully logged out"
\end{code}

The destroy action redirects to the application’s root\_url,  where the message will be displayed. Note that it’s entirely up to the  next action to decide what, if anything, it will do with what the  previous action put in the flash. It’s conventional to display eventual  errors or notices from the flash in the application’s layout:
\begin{code}
<html>
  <!-- <head/> -->
  <body>
    <% if flash[:notice] %>
      <p class="notice"><%= flash[:notice] %></p>
    <% end %>
    <% if flash[:error] %>
      <p class="error"><%= flash[:error] %></p>
    <% end %>
    <!-- more content -->
  </body>
</html>
\end{code}

This way, if an action sets an error or a notice message, the layout will display it automatically.

If you want a flash value to be carried over to another request, use the keep method:
\begin{code}
class MainController < ApplicationController
  # Let's say this action corresponds to root_url, but you want
  # all requests here to be redirected to UsersController#index.
  # If an action sets the flash and redirects here, the values
  # would normally be lost when another redirect happens, but you
  # can use 'keep' to make it persist for another request.
  def index
    # Will persist all flash values.
    flash.keep
 
    # You can also use a key to keep only some kind of value.
    # flash.keep(:notice)
    redirect_to users_url
  end
end
\end{code}

\subsubsection{ flash.now}

By default, adding values to the flash will make them available to  the next request, but sometimes you may want to access those values in  the same request. For example, if the create action fails to save a resource and you render the new  template directly, that’s not going to result in a new request, but you  may still want to display a message using the flash. To do this, you  can use flash.now in the same way you use the normal flash:
\begin{code}
class ClientsController < ApplicationController
  def create
    @client = Client.new(params[:client])
    if @client.save
      # ...
    else
      flash.now[:error] = "Could not save client"
      render :action => "new"
    end
  end
end
\end{code}

\section{ Cookies}

Your application can store small amounts of data on the client —  called cookies — that will be persisted across requests and even  sessions. Rails provides easy access to cookies via the cookies method, which — much like the session — works like a hash:
\begin{code}
class CommentsController < ApplicationController
  def new
    # Auto-fill the commenter's name if it has been stored in a cookie
    @comment = Comment.new(:name => cookies[:commenter_name])
  end
 
  def create
    @comment = Comment.new(params[:comment])
    if @comment.save
      flash[:notice] = "Thanks for your comment!"
      if params[:remember_name]
        # Remember the commenter's name.
        cookies[:commenter_name] = @comment.name
      else
        # Delete cookie for the commenter's name cookie, if any.
        cookies.delete(:commenter_name)
      end
      redirect_to @comment.article
    else
      render :action => "new"
    end
  end
end
\end{code}

Note that while for session values you set the key to nil, to delete a cookie value you should use cookies.delete(:key).

\section{ Rendering xml and json data}

ActionController makes it extremely easy to render xml or json data. If you generate a controller using scaffold then your controller would look something like this.
\begin{code}
class UsersController < ApplicationController
  def index
    @users = User.all
    respond_to do |format|
      format.html # index.html.erb
      format.xml  { render :xml => @users}
      format.json { render :json => @users}
    end
  end
end
\end{code}

Notice that in the above case code is render :xml =$>$ @users and not render :xml =$>$ @users.to\_xml. That is because if the input is not string then rails automatically invokes to\_xml .

\section{ Filters}

Filters are methods that are run before, after or "around" a controller action.

Filters are inherited, so if you set a filter on ApplicationController, it will be run on every controller in your application.

Before filters may halt the request cycle. A common before filter is  one which requires that a user is logged in for an action to be run. You  can define the filter method this way:
\begin{code}
class ApplicationController < ActionController::Base
  before_filter :require_login
 
  private
 
  def require_login
    unless logged_in?
      flash[:error] = "You must be logged in to access this section"
      redirect_to new_login_url # halts request cycle
    end
  end
 
  # The logged_in? method simply returns true if the user is logged
  # in and false otherwise. It does this by "booleanizing" the
  # current_user method we created previously using a double ! operator.
  # Note that this is not common in Ruby and is discouraged unless you
  # really mean to convert something into true or false.
  def logged_in?
    !!current_user
  end
end
\end{code}

The method simply stores an error message in the flash and redirects  to the login form if the user is not logged in. If a before filter  renders or redirects, the action will not run. If there are additional  filters scheduled to run after that filter they are also cancelled.

In this example the filter is added to ApplicationController  and thus all controllers in the application inherit it. This will make  everything in the application require the user to be logged in in order  to use it. For obvious reasons (the user wouldn’t be able to log in in  the first place!), not all controllers or actions should require this.  You can prevent this filter from running before particular actions with skip\_before\_filter:
\begin{code}
class LoginsController < ApplicationController
  skip_before_filter :require_login, :only => [:new, :create]
end
\end{code}

Now, the LoginsController’s new and create actions will work as before without requiring the user to be logged in. The :only option is used to only skip this filter for these actions, and there is also an :except  option which works the other way. These options can be used when adding  filters too, so you can add a filter which only runs for selected  actions in the first place.

\subsection{ After Filters and Around Filters}

In addition to before filters, you can also run filters after an action has been executed, or both before and after.

After filters are similar to before filters, but because the action  has already been run they have access to the response data that’s about  to be sent to the client. Obviously, after filters cannot stop the  action from running.

Around filters are responsible for running their associated actions by yielding, similar to how Rack middlewares work.

For example, in a website where changes have an approval workflow an  administrator could be able to preview them easily, just apply them  within a transaction:
\begin{code}
class ChangesController < ActionController::Base
  around_filter :wrap_in_transaction, :only => :show
 
  private
 
  def wrap_in_transaction
    ActiveRecord::Base.transaction do
      begin
        yield
      ensure
        raise ActiveRecord::Rollback
      end
    end
  end
end
\end{code}

Note that an around filter wraps also rendering. In particular, if in  the example above the view itself reads from the database via a scope  or whatever, it will do so within the transaction and thus present the  data to preview.

They can choose not to yield and build the response themselves, in which case the action is not run.

\subsection{ Other Ways to Use Filters}

While the most common way to use filters is by creating private  methods and using *\_filter to add them, there are two other ways to do  the same thing.

The first is to use a block directly with the *\_filter methods. The block receives the controller as an argument, and the require\_login filter from above could be rewritten to use a block:
\begin{code}
class ApplicationController < ActionController::Base
  before_filter do |controller|
    redirect_to new_login_url unless controller.send(:logged_in?)
  end
end
\end{code}

Note that the filter in this case uses send because the logged\_in?  method is private and the filter is not run in the scope of the  controller. This is not the recommended way to implement this particular  filter, but in more simple cases it might be useful.

The second way is to use a class (actually, any object that responds  to the right methods will do) to handle the filtering. This is useful in  cases that are more complex and can not be implemented in a readable  and reusable way using the two other methods. As an example, you could  rewrite the login filter again to use a class:
\begin{code}
class ApplicationController < ActionController::Base
  before_filter LoginFilter
end
 
class LoginFilter
  def self.filter(controller)
    unless controller.send(:logged_in?)
      controller.flash[:error] = "You must be logged in"
      controller.redirect_to controller.new_login_url
    end
  end
end
\end{code}

Again, this is not an ideal example for this filter, because it’s not  run in the scope of the controller but gets the controller passed as an  argument. The filter class has a class method filter which  gets run before or after the action, depending on if it’s a before or  after filter. Classes used as around filters can also use the same filter method, which will get run in the same way. The method must yield to execute the action. Alternatively, it can have both a before and an after method that are run before and after the action.

\section{ Request Forgery Protection}

Cross-site request forgery is a type of attack in which a site tricks  a user into making requests on another site, possibly adding, modifying  or deleting data on that site without the user’s knowledge or  permission.

The first step to avoid this is to make sure all "destructive"  actions (create, update and destroy) can only be accessed with non-GET requests. If you’re following RESTful conventions you’re already doing this. However, a malicious site can still send a non-GET  request to your site quite easily, and that’s where the request forgery  protection comes in. As the name says, it protects from forged  requests.

The way this is done is to add a non-guessable token which is only  known to your server to each request. This way, if a request comes in  without the proper token, it will be denied access.

If you generate a form like this:
\begin{code}
<%= form_for @user do |f| %>
  <%= f.text_field :username %>
  <%= f.text_field :password %>
<% end %>
\end{code}

You will see how the token gets added as a hidden field:
\begin{code}
<form accept-charset="UTF-8" action="/users/1" method="post">
<input type="hidden"
       value="67250ab105eb5ad10851c00a5621854a23af5489"
       name="authenticity_token"/>
<!-- fields -->
</form>
\end{code}

Rails adds this token to every form that’s generated using the \href{http://guides.rubyonrails.org/form_helpers.html}{form helpers},  so most of the time you don’t have to worry about it. If you’re writing  a form manually or need to add the token for another reason, it’s  available through the method form\_authenticity\_token:

The form\_authenticity\_token generates a valid authentication  token. That’s useful in places where Rails does not add it  automatically, like in custom Ajax calls.

The \href{http://guides.rubyonrails.org/security.html}{Security Guide} has more about this and a lot of other security-related issues that you should be aware of when developing a web application.

\section{ The Request and Response Objects}

In every controller there are two accessor methods pointing to the  request and the response objects associated with the request cycle that  is currently in execution. The request method contains an instance of AbstractRequest and the response method returns a response object representing what is going to be sent back to the client.

\subsection{ The request Object}

The request object contains a lot of useful information about the  request coming in from the client. To get a full list of the available  methods, refer to the \href{http://api.rubyonrails.org/classes/ActionDispatch/Request.html}{API documentation}. Among the properties that you can access on this object are:
\\

\noindent
\begin{tabular}{p{\textwidth/3}|p{\textwidth/2+30}}
\hline
\textbf{Property of request} & \textbf{Purpose} \\ 
\hline
host & The hostname used for this request. \\ 
domain(n=2) & The hostname’s first n segments, starting from the right (the TLD). \\ 
format & The content type requested by the client. \\ 
method & The HTTP method used for the request. \\ 
get?, post?, put?, delete?, head? & Returns true if the HTTP method is GET/POST/PUT/DELETE/HEAD. \\ 
headers & Returns a hash containing the headers associated with the request. \\ 
port & The port number (integer) used for the request. \\ 
protocol & Returns a string containing the protocol used plus "://", for example "http://". \\ 
query\_string & The query string part of the URL, i.e., everything after "?". \\ 
remote\_ip & The IP address of the client. \\ 
url & The entire URL used for the request.
\end{tabular}

\subsubsection{ path\_parameters, query\_parameters, \\ and request\_parameters}

Rails collects all of the parameters sent along with the request in the params  hash, whether they are sent as part of the query string or the post  body. The request object has three accessors that give you access to  these parameters depending on where they came from. The query\_parameters hash contains parameters that were sent as part of the query string while the request\_parameters hash contains parameters sent as part of the post body. The path\_parameters  hash contains parameters that were recognized by the routing as being  part of the path leading to this particular controller and action.

\subsection{ The response Object}

The response object is not usually used directly, but is built up  during the execution of the action and rendering of the data that is  being sent back to the user, but sometimes – like in an after filter –  it can be useful to access the response directly. Some of these accessor  methods also have setters, allowing you to change their values.
\\

\noindent
\begin{tabular}{l|p{\textwidth/2}}
\hline
\textbf{Property of response} & \textbf{Purpose} \\ 
\hline
body & This is the string of data being sent back to the client. This is most often HTML. \\ 
status & The HTTP status code for the response, like 200 for a successful request or 404 for file not found. \\ 
location & The URL the client is being redirected to, if any. \\ 
content\_type & The content type of the response. \\ 
charset & The character set being used for the response. Default is "utf-8". \\ 
headers & Headers used for the response.
\end{tabular}

\subsubsection{ Setting Custom Headers}

If you want to set custom headers for a response then response.headers  is the place to do it. The headers attribute is a hash which maps  header names to their values, and Rails will set some of them  automatically. If you want to add or change a header, just assign it to response.headers this way:
\begin{code}
response.headers["Content-Type"] = "application/pdf"
\end{code}

\section{ HTTP Authentications}

Rails comes with two built-in HTTP authentication mechanisms:
\begin{itemize}
	\item Basic Authentication
	\item Digest Authentication
\end{itemize}

\subsection{ HTTP Basic Authentication}

HTTP basic authentication is an authentication scheme that is supported by the majority of browsers and other HTTP  clients. As an example, consider an administration section which will  only be available by entering a username and a password into the  browser’s HTTP basic dialog window. Using the built-in authentication is quite easy and only requires you to use one method, \\ http\_basic\_authenticate\_with.
\begin{code}
class AdminController < ApplicationController
  http_basic_authenticate_with :name => "humbaba", :password => "5baa61e4"
end
\end{code}

With this in place, you can create namespaced controllers that inherit from AdminController. The filter will thus be run for all actions in those controllers, protecting them with HTTP basic authentication.

\subsection{ HTTP Digest Authentication}

HTTP digest authentication is superior to  the basic authentication as it does not require the client to send an  unencrypted password over the network (though HTTP basic authentication is safe over HTTPS). Using digest authentication with Rails is quite easy and only requires using one method, authenticate\_or\_request\_with\_http\_digest.
\begin{code}
class AdminController < ApplicationController
  USERS = { "lifo" => "world" }
 
  before_filter :authenticate
 
  private
 
  def authenticate
    authenticate_or_request_with_http_digest do |username|
      USERS[username]
    end
  end
end
\end{code}

As seen in the example above, the \\ authenticate\_or\_request\_with\_http\_digest block takes only one argument – the username. And the block returns the password. Returning false or nil from the authenticate\_or\_request\_with\_http\_digest will cause authentication failure.

\section{ Streaming and File Downloads}

Sometimes you may want to send a file to the user instead of rendering an HTML page. All controllers in Rails have the send\_data and the send\_file methods, which will both stream data to the client. send\_file  is a convenience method that lets you provide the name of a file on the  disk and it will stream the contents of that file for you.

To stream data to the client, use send\_data:
\begin{code}
require "prawn"
class ClientsController < ApplicationController
  # Generates a PDF document with information on the client and
  # returns it. The user will get the PDF as a file download.
  def download_pdf
    client = Client.find(params[:id])
    send_data generate_pdf(client),
              :filename => "#{client.name}.pdf",
              :type => "application/pdf"
  end
 
  private
 
  def generate_pdf(client)
    Prawn::Document.new do
      text client.name, :align => :center
      text "Address: #{client.address}"
      text "Email: #{client.email}"
    end.render
  end
end
\end{code}

The download\_pdf action in the example above will call a private method which actually generates the PDF  document and returns it as a string. This string will then be streamed  to the client as a file download and a filename will be suggested to the  user. Sometimes when streaming files to the user, you may not want them  to download the file. Take images, for example, which can be embedded  into HTML pages. To tell the browser a file is not meant to be downloaded, you can set the :disposition option to "inline". The opposite and default value for this option is "attachment".

\subsection{ Sending Files}

If you want to send a file that already exists on disk, use the send\_file method.
\begin{code}
class ClientsController < ApplicationController
  # Stream a file that has already been generated and stored on disk.
  def download_pdf
    client = Client.find(params[:id])
    send_file("#{Rails.root}/files/clients/#{client.id}.pdf",
              :filename => "#{client.name}.pdf",
              :type => "application/pdf")
  end
end
\end{code}

This will read and stream the file 4kB at the time, avoiding loading  the entire file into memory at once. You can turn off streaming with the  :stream option or adjust the block size with the :buffer\_size option.

If :type is not specified, it will be guessed from the file extension specified in :filename. If the content type is not registered for the extension, application/octet-stream will be used.

Be careful when using data coming from the  client (params, cookies, etc.) to locate the file on disk, as this is a  security risk that might allow someone to gain access to files they are  not meant to see.

It is not recommended that you stream static files  through Rails if you can instead keep them in a public folder on your  web server. It is much more efficient to let the user download the file  directly using Apache or another web server, keeping the request from  unnecessarily going through the whole Rails stack.

\subsection{ RESTful Downloads}

While send\_data works just fine, if you are creating a  RESTful application having separate actions for file downloads is  usually not necessary. In REST terminology, the PDF  file from the example above can be considered just another  representation of the client resource. Rails provides an easy and quite  sleek way of doing "RESTful downloads". Here’s how you can rewrite the  example so that the PDF download is a part of the show action, without any streaming:
\begin{code}
class ClientsController < ApplicationController
  # The user can request to receive this resource as HTML or PDF.
  def show
    @client = Client.find(params[:id])
 
    respond_to do |format|
      format.html
      format.pdf { render :pdf => generate_pdf(@client) }
    end
  end
end
\end{code}

In order for this example to work, you have to add the PDFMIME type to Rails. This can be done by adding the following line to the file config/initializers/mime\_types.rb:
\begin{code}
Mime::Type.register "application/pdf", :pdf
\end{code}

Configuration files are not reloaded on each  request, so you have to restart the server in order for their changes to  take effect.

Now the user can request to get a PDF version of a client just by adding ".pdf" to the URL:
\begin{code}
GET /clients/1.pdf
\end{code}

\section{ Parameter Filtering}

Rails keeps a log file for each environment in the log  folder. These are extremely useful when debugging what’s actually going  on in your application, but in a live application you may not want every  bit of information to be stored in the log file. You can filter certain  request parameters from your log files by appending them to config.filter\_parameters in the application configuration. These parameters will be marked [FILTERED] in the log.
\begin{code}
config.filter_parameters << :password
\end{code}

\section{ Rescue}

Most likely your application is going to contain bugs or otherwise  throw an exception that needs to be handled. For example, if the user  follows a link to a resource that no longer exists in the database,  Active Record will throw the ActiveRecord::RecordNotFound exception.

Rails’ default exception handling displays a "500 Server Error"  message for all exceptions. If the request was made locally, a nice  traceback and some added information gets displayed so you can figure  out what went wrong and deal with it. If the request was remote Rails  will just display a simple "500 Server Error" message to the user, or a  "404 Not Found" if there was a routing error or a record could not be  found. Sometimes you might want to customize how these errors are caught  and how they’re displayed to the user. There are several levels of  exception handling available in a Rails application:

\subsection{ The Default 500 and 404 Templates}

By default a production application will render either a 404 or a 500 error message. These messages are contained in static HTML files in the public folder, in 404.html and 500.html  respectively. You can customize these files to add some extra  information and layout, but remember that they are static; i.e. you  can’t use RHTML or layouts in them, just plain HTML.

\subsection{ rescue\_from}

If you want to do something a bit more elaborate when catching errors, you can use rescue\_from, which handles exceptions of a certain type (or multiple types) in an entire controller and its subclasses.

When an exception occurs which is caught by a rescue\_from directive, the exception object is passed to the handler. The handler can be a method or a Proc object passed to the :with option. You can also use a block directly instead of an explicit Proc object.

Here’s how you can use rescue\_from to intercept all ActiveRecord::RecordNotFound errors and do something with them.
\begin{code}
class ApplicationController < ActionController::Base
  rescue_from ActiveRecord::RecordNotFound, :with => :record_not_found
 
  private
 
  def record_not_found
    render :text => "404 Not Found", :status => 404
  end
end
\end{code}

Of course, this example is anything but elaborate and doesn’t improve  on the default exception handling at all, but once you can catch all  those exceptions you’re free to do whatever you want with them. For  example, you could create custom exception classes that will be thrown  when a user doesn’t have access to a certain section of your  application:
\begin{code}
class ApplicationController < ActionController::Base
  rescue_from User::NotAuthorized, :with => :user_not_authorized
 
  private
 
  def user_not_authorized
    flash[:error] = "You don't have access to this section."
    redirect_to :back
  end
end
 
class ClientsController < ApplicationController
  # Check that the user has the right authorization to access clients.
  before_filter :check_authorization
 
  # Note how the actions don't have to worry about all the auth stuff.
  def edit
    @client = Client.find(params[:id])
  end
 
  private
 
  # If the user is not authorized, just throw the exception.
  def check_authorization
    raise User::NotAuthorized unless current_user.admin?
  end
end
\end{code}

Certain exceptions are only rescuable from the ApplicationController class, as they are raised before the controller gets initialized and the action gets executed. See Pratik Naik’s \href{http://m.onkey.org/2008/7/20/rescue-from-dispatching}{article} on the subject for more information.

\section{ Force HTTPS protocol}

Sometime you might want to force a particular controller to only be accessible via an HTTPS protocol for security reasons. Since Rails 3.1 you can now use force\_ssl method in your controller to enforce that:
\begin{code}
class DinnerController
  force_ssl
end
\end{code}

Just like the filter, you could also passing :only and :except to enforce the secure connection only to specific actions.
\begin{code}
class DinnerController
  force_ssl :only => :cheeseburger
  # or
  force_ssl :except => :cheeseburger
end
\end{code}

Please note that if you found yourself adding force\_ssl to many controllers, you may found yourself wanting to force the whole application to use HTTPS instead. In that case, you can set the config.force\_ssl in your environment file.

\chapter{Rails Routing from the Outside In}

This guide covers the user-facing features of Rails routing. By referring to this guide, you will be able to:
\begin{itemize}
	\item Understand the code in routes.rb
	\item Construct your own routes, using either the preferred resourceful style or the match method
	\item Identify what parameters to expect an action to receive
	\item Automatically create paths and URLs using route helpers
	\item Use advanced techniques such as constraints and Rack endpoints
\end{itemize}

\section{ The Purpose of the Rails Router}

The Rails router recognizes URLs and dispatches them to a  controller’s action. It can also generate paths and URLs, avoiding the  need to hardcode strings in your views.

\subsection{ Connecting URLs to Code}

When your Rails application receives an incoming request
\begin{code}
GET /patients/17
\end{code}

it asks the router to match it to a controller action. If the first matching route is
\begin{code}
match "/patients/:id" => "patients#show"
\end{code}

the request is dispatched to the patients controller’s show action with \{ :id =$>$ "17" \} in params.

\subsection{ Generating Paths and URLs from Code}

You can also generate paths and URLs. If your application contains this code:
\begin{code}
@patient = Patient.find(17)
\end{code}
\begin{code}
<%= link_to "Patient Record", patient_path(@patient) %>
\end{code}

The router will generate the path /patients/17. This reduces  the brittleness of your view and makes your code easier to understand.  Note that the id does not need to be specified in the route helper.

\section{ Resource Routing: the Rails Default}

Resource routing allows you to quickly declare all of the common  routes for a given resourceful controller. Instead of declaring separate  routes for your index, show, new, edit, create, update and destroy actions, a resourceful route declares them in a single line of code.

\subsection{ Resources on the Web}

Browsers request pages from Rails by making a request for a URL using a specific HTTP method, such as GET, POST, PUT and DELETE.  Each method is a request to perform an operation on the resource. A  resource route maps a number of related requests to actions in a single  controller.

When your Rails application receives an incoming request for
\begin{code}
DELETE /photos/17
\end{code}

it asks the router to map it to a controller action. If the first matching route is
\begin{code}
resources :photos
\end{code}

Rails would dispatch that request to the destroy method on the photos controller with \{ :id =$>$ "17" \} in params.

\subsection{ CRUD, Verbs, and Actions}

In Rails, a resourceful route provides a mapping between HTTP verbs and URLs to controller actions. By convention, each action also maps to particular CRUD operations in a database. A single entry in the routing file, such as
\begin{code}
resources :photos
\end{code}

creates seven different routes in your application, all mapping to the Photos controller:
\\


\noindent
\begin{tabular}{l|p{\textwidth/4}|l|p{\textwidth/4+10}}
\hline
\textbf{Verb} & \textbf{Path            } & \textbf{action } & \textbf{used for} \\ 
\hline
GET & /photos            & index     & display a list of all photos                  \\ 
GET & /photos/new        & new       & return an HTML form for creating a new photo  \\ 
POST & /photos            & create    & create a new photo                            \\ 
GET & /photos/:id        & show      & display a specific photo                      \\ 
GET & /photos/:id/edit   & edit      & return an HTML form for editing a photo       \\ 
PUT & /photos/:id        & update    & update a specific photo                       \\ 
DELETE & /photos/:id        & destroy   & delete a specific photo                      
\end{tabular}
\\

Rails routes are matched in the order they are specified, so if you have a resources :photos above a get 'photos/poll' the show action’s route for the resources line will be matched before the get line. To fix this, move the get line \textbf{above} the resources line so that it is matched first.

\subsection{ Paths and URLs}

Creating a resourceful route will also expose a number of helpers to the controllers in your application. In the case of resources :photos:
\begin{itemize}
	\item photos\_path returns /photos
	\item new\_photo\_path returns /photos/new
	\item edit\_photo\_path(:id) returns /photos/:id/edit \\ (for instance, edit\_photo\_path(10) returns /photos/10/edit)
	\item photo\_path(:id) returns /photos/:id (for instance, \\ photo\_path(10) returns /photos/10)
\end{itemize}

Each of these helpers has a corresponding \_url helper (such as photos\_url) which returns the same path prefixed with the current host, port and path prefix.

Because the router uses the HTTP verb and URL to match inbound requests, four URLs map to seven different actions.

\subsection{ Defining Multiple Resources at the Same Time}

If you need to create routes for more than one resource, you can save  a bit of typing by defining them all with a single call to resources:
\begin{code}
resources :photos, :books, :videos
\end{code}

This works exactly the same as
\begin{code}
resources :photos
resources :books
resources :videos
\end{code}

\subsection{ Singular Resources}

Sometimes, you have a resource that clients always look up without referencing an ID. For example, you would like /profile to always show the profile of the currently logged in user. In this case, you can use a singular resource to map /profile (rather than /profile/:id) to the show action.
\begin{code}
match "profile" => "users#show"
\end{code}

This resourceful route
\begin{code}
resource :geocoder
\end{code}

creates six different routes in your application, all mapping to the Geocoders controller:
\\

\noindent
\begin{tabular}{p{\textwidth/6-10}|p{\textwidth/4}|p{\textwidth/6}|p{\textwidth/4+20}}
\hline
\textbf{Verb } & \textbf{Path} & \textbf{action } & \textbf{used for} \\ 
\hline
GET & /geocoder/new   & new       & return an HTML form for creating the geocoder  \\ 
POST & /geocoder       & create    & create the new geocoder                        \\ 
GET & /geocoder       & show      & display the one and only geocoder resource     \\ 
GET & /geocoder/edit  & edit      & return an HTML form for editing the geocoder   \\ 
PUT & /geocoder       & update    & update the one and only geocoder resource      \\ 
DELETE & /geocoder       & destroy   & delete the geocoder resource                  
\end{tabular}
\\


Because you might want to use the same controller for a singular route (/account) and a plural route (/accounts/45), singular resources map to plural controllers.

A singular resourceful route generates these helpers:
\begin{itemize}
	\item new\_geocoder\_path returns /geocoder/new
	\item edit\_geocoder\_path returns /geocoder/edit
	\item geocoder\_path returns /geocoder
\end{itemize}

As with plural resources, the same helpers ending in \_url will also include the host, port and path prefix.

\subsection{ Controller Namespaces and Routing}

You may wish to organize groups of controllers under a namespace.  Most commonly, you might group a number of administrative controllers  under an Admin:: namespace. You would place these controllers under the app/controllers/admin directory, and you can group them together in your router:
\begin{code}
namespace :admin do
  resources :posts, :comments
end
\end{code}

This will create a number of routes for each of the posts and comments controller. For Admin::PostsController, Rails will create: \\

\noindent
{
\begin{tabular}{l|l|l|p{\textwidth/4}}
\hline
\textbf{Verb } & \textbf{Path} & \textbf{action } & \textbf{named helper} \\ 
\hline
GET & /admin/posts           & index     &  admin\_posts\_path           \\ 
GET & /admin/posts/new       & new       &  new\_admin\_post\_path        \\ 
POST & /admin/posts           & create    &  admin\_posts\_path           \\ 
GET & /admin/posts/:id       & show      &  admin\_post\_path(:id)       \\ 
GET & {\tiny /admin/posts/:id/edit}  & edit      & {\tiny edit\_admin\_post\_path(:id)}  \\ 
PUT & /admin/posts/:id       & update    &  admin\_post\_path(:id)       \\ 
DELETE & /admin/posts/:id       & destroy   &  admin\_post\_path(:id)      
\end{tabular}\\
}

If you want to route /posts (without the prefix /admin) to Admin::PostsController, you could use
\begin{code}
scope :module => "admin" do
  resources :posts, :comments
end
\end{code}

or, for a single case
\begin{code}
resources :posts, :module => "admin"
\end{code}

If you want to route /admin/posts to PostsController (without the Admin:: module prefix), you could use
\begin{code}
scope "/admin" do
  resources :posts, :comments
end
\end{code}

or, for a single case
\begin{code}
resources :posts, :path => "/admin/posts"
\end{code}

In each of these cases, the named routes remain the same as if you did not use scope. In the last case, the following paths map to PostsController: \\

\noindent
\begin{tabular}{l|p{\textwidth/3+5}|l|p{\textwidth/4}}
\hline
\textbf{Verb} & \textbf{Path} & \textbf{action } & \textbf{named helper} \\ 
\hline
GET & /admin/posts          & index     &  posts\_path          \\ 
GET & /admin/posts/new      & new       &  new\_post\_path       \\ 
POST & /admin/posts          & create    &  posts\_path          \\ 
GET & /admin/posts/:id      & show      &  post\_path(:id)      \\ 
GET & /admin/posts/:id/edit & edit      &  edit\_post\_path(:id) \\ 
PUT & /admin/posts/:id      & update    &  post\_path(:id)      \\ 
DELETE & /admin/posts/:id      & destroy   &  post\_path(:id)     
\end{tabular}\\


\subsection{ Nested Resources}

It’s common to have resources that are logically children of other  resources. For example, suppose your application includes these models:
\begin{code}
class Magazine < ActiveRecord::Base
  has_many :ads
end
 
class Ad < ActiveRecord::Base
  belongs_to :magazine
end
\end{code}

Nested routes allow you to capture this relationship in your routing. In this case, you could include this route declaration:
\begin{code}
resources :magazines do
  resources :ads
end
\end{code}

In addition to the routes for magazines, this declaration will also route ads to an AdsController. The ad URLs require a magazine:
\\
{\tiny
\noindent
\begin{tabular}{l|p{\textwidth/3}|l|p{\textwidth/3}}
\hline
\textbf{Verb} & \textbf{Path} & \textbf{action } & \textbf{used for} \\ 
\hline
GET &  /magazines /:magazine\_id /ads           & index     & display a list of all ads for a specific magazine                           \\ 
GET & /magazines /:magazine\_id /ads /new       & new       & return an HTML form for creating a new ad belonging to a specific magazine  \\ 
POST & /magazines /:magazine\_id /ads           & create    & create a new ad belonging to a specific magazine                            \\ 
GET & /magazines /:magazine\_id /ads /:id       & show      & display a specific ad belonging to a specific magazine                      \\ 
GET & /magazines /:magazine\_id /ads /:id /edit  & edit      & return an HTML form for editing an ad belonging to a specific magazine      \\ 
PUT & /magazines /:magazine\_id /ads /:id       & update    & update a specific ad belonging to a specific magazine                       \\ 
DELETE & /magazines /:magazine\_id /ads /:id       & destroy   & delete a specific ad belonging to a specific magazine                      
\end{tabular}\\
}

This will also create routing helpers such as magazine\_ads\_url and edit\_magazine\_ad\_path. These helpers take an instance of Magazine as the first parameter (magazine\_ads\_url(@magazine)).

\subsubsection{ Limits to Nesting}

You can nest resources within other nested resources if you like. For example:
\begin{code}
resources :publishers do
  resources :magazines do
    resources :photos
  end
end
\end{code}

Deeply-nested resources quickly become cumbersome. In this case, for example, the application would recognize paths such as
\begin{code}
/publishers/1/magazines/2/photos/3
\end{code}

The corresponding route helper would be \\ publisher\_magazine\_photo\_url, requiring you to specify objects at all three levels. Indeed, this situation is confusing enough that a popular \href{http://weblog.jamisbuck.org/2007/2/5/nesting-resources}{article} by Jamis Buck proposes a rule of thumb for good Rails design:

\emph{Resources should never be nested more than 1 level deep.}

\subsection{ Creating Paths and URLs From Objects}

In addition to using the routing helpers, Rails can also create paths  and URLs from an array of parameters. For example, suppose you have  this set of routes:
\begin{code}
resources :magazines do
  resources :ads
end
\end{code}

When using magazine\_ad\_path, you can pass in instances of Magazine and Ad instead of the numeric IDs.
\begin{code}
<%= link_to "Ad details", magazine_ad_path(@magazine, @ad) %>
\end{code}

You can also use url\_for with a set of objects, and Rails will automatically determine which route you want:
\begin{code}
<%= link_to "Ad details", url_for([@magazine, @ad]) %>
\end{code}

In this case, Rails will see that @magazine is a Magazine and @ad is an Ad and will therefore use the magazine\_ad\_path helper. In helpers like link\_to, you can specify just the object in place of the full url\_for call:
\begin{code}
<%= link_to "Ad details", [@magazine, @ad] %>
\end{code}

If you wanted to link to just a magazine, you could leave out the Array:
\begin{code}
<%= link_to "Magazine details", @magazine %>
\end{code}

This allows you to treat instances of your models as URLs, and is a key advantage to using the resourceful style.

\subsection{ Adding More RESTful Actions}

You are not limited to the seven routes that RESTful routing creates  by default. If you like, you may add additional routes that apply to the  collection or individual members of the collection.

\subsubsection{ Adding Member Routes}

To add a member route, just add a member block into the resource block:
\begin{code}
resources :photos do
  member do
    get 'preview'
  end
end
\end{code}

This will recognize /photos/1/preview with GET, and route to the preview action of PhotosController. It will also create the preview\_photo\_url and preview\_photo\_path helpers.

Within the block of member routes, each route name specifies the HTTP verb that it will recognize. You can use get, put, post, or delete here. If you don’t have multiple member routes, you can also pass :on to a route, eliminating the block:
\begin{code}
resources :photos do
  get 'preview', :on => :member
end
\end{code}

\subsubsection{ Adding Collection Routes}

To add a route to the collection:
\begin{code}
resources :photos do
  collection do
    get 'search'
  end
end
\end{code}

This will enable Rails to recognize paths such as /photos/search with GET, and route to the search action of PhotosController. It will also create the search\_photos\_url and search\_photos\_path route helpers.

Just as with member routes, you can pass :on to a route:
\begin{code}
resources :photos do
  get 'search', :on => :collection
end
\end{code}

\subsubsection{ A Note of Caution}

If you find yourself adding many extra actions to a resourceful  route, it’s time to stop and ask yourself whether you’re disguising the  presence of another resource.

\section{ Non-Resourceful Routes}

In addition to resource routing, Rails has powerful support for  routing arbitrary URLs to actions. Here, you don’t get groups of routes  automatically generated by resourceful routing. Instead, you set up each  route within your application separately.

While you should usually use resourceful routing, there are still  many places where the simpler routing is more appropriate. There’s no  need to try to shoehorn every last piece of your application into a  resourceful framework if that’s not a good fit.

In particular, simple routing makes it very easy to map legacy URLs to new Rails actions.

\subsection{ Bound Parameters}

When you set up a regular route, you supply a series of symbols that Rails maps to parts of an incoming HTTP request. Two of these symbols are special: :controller maps to the name of a controller in your application, and :action maps to the name of an action within that controller. For example, consider one of the default Rails routes:
\begin{code}
match ':controller(/:action(/:id))'
\end{code}

If an incoming request of /photos/show/1 is processed by this route (because it hasn’t matched any previous route in the file), then the result will be to invoke the show action of the PhotosController, and to make the final parameter "1" available as params[:id]. This route will also route the incoming request of /photos to PhotosController\#index, since :action and :id are optional parameters, denoted by parentheses.

\subsection{ Dynamic Segments}

You can set up as many dynamic segments within a regular route as you like. Anything other than :controller or :action will be available to the action as part of params. If you set up this route:
\begin{code}
match ':controller/:action/:id/:user_id'
\end{code}

An incoming path of /photos/show/1/2 will be dispatched to the show action of the PhotosController. params[:id] will be "1", and params[:user\_id] will be "2".

You can’t use :namespace or :module with a :controller path segment. If you need to do this then use a constraint on :controller that matches the namespace you require. e.g:
\begin{code}
match ':controller(/:action(/:id))', 
      :controller => /admin\/[^\/]+/
\end{code}

By default dynamic segments don’t accept dots –  this is because the dot is used as a separator for formatted routes. If  you need to use a dot within a dynamic segment add a constraint which  overrides this – for example :id =$>$ /[\textasciicircum$\backslash$/]+/ allows anything except a slash.

\subsection{ Static Segments}

You can specify static segments when creating a route:
\begin{code}
match ':controller/:action/:id/with_user/:user_id'
\end{code}

This route would respond to paths such as \\ /photos/show/1/with\_user/2. In this case, params would be \{ :controller =$>$ "photos", :action =$>$ "show", :id =$>$ "1", :user\_id =$>$ "2" \}.

\subsection{ The Query String}

The params will also include any parameters from the query string. For example, with this route:
\begin{code}
match ':controller/:action/:id'
\end{code}

An incoming path of /photos/show/1?user\_id=2 will be dispatched to the show action of the Photos controller. params will be \{ :controller =$>$ "photos", :action =$>$ "show", :id =$>$ "1", :user\_id =$>$ "2" \}.

\subsection{ Defining Defaults}

You do not need to explicitly use the :controller and :action symbols within a route. You can supply them as defaults:
\begin{code}
match 'photos/:id' => 'photos#show'
\end{code}

With this route, Rails will match an incoming path of /photos/12 to the show action of PhotosController.

You can also define other defaults in a route by supplying a hash for the :defaults option. This even applies to parameters that you do not specify as dynamic segments. For example:
\begin{code}
match 'photos/:id' => 'photos#show', 
      :defaults => { :format => 'jpg' }
\end{code}

Rails would match photos/12 to the show action of PhotosController, and set params[:format] to "jpg".

\subsection{ Naming Routes}

You can specify a name for any route using the :as option.
\begin{code}
match 'exit' => 'sessions#destroy', :as => :logout
\end{code}

This will create logout\_path and logout\_url as named helpers in your application. Calling logout\_path will return /exit

\subsection{ HTTP Verb Constraints}

You can use the :via option to constrain the request to one or more HTTP methods:
\begin{code}
match 'photos/show' => 'photos#show', :via => :get
\end{code}

There is a shorthand version of this as well:
\begin{code}
get 'photos/show'
\end{code}

You can also permit more than one verb to a single route:
\begin{code}
match 'photos/show' => 'photos#show', :via => [:get, :post]
\end{code}

\subsection{ Segment Constraints}

You can use the :constraints option to enforce a format for a dynamic segment:
\begin{code}
match 'photos/:id' => 'photos#show', 
:constraints => { :id => /[A-Z]\d{5}/ }
\end{code}

This route would match paths such as /photos/A12345. You can more succinctly express the same route this way:
\begin{code}
match 'photos/:id' => 'photos#show', :id => /[A-Z]\d{5}/
\end{code}

:constraints takes regular expressions with the restriction  that regexp anchors can’t be used. For example, the following route will  not work:
\begin{code}
match '/:id' => 'posts#show', :constraints => {:id => /^\d/}
\end{code}

However, note that you don’t need to use anchors because all routes are anchored at the start.

For example, the following routes would allow for posts with to\_param values like 1-hello-world that always begin with a number and users with to\_param values like david that never begin with a number to share the root namespace:
\begin{code}
match '/:id' => 'posts#show', :constraints => { :id => /\d.+/ }
match '/:username' => 'users#show'
\end{code}

\subsection{ Request-Based Constraints}

You can also constrain a route based on any method on the \href{http://guides.rubyonrails.org/action_controller_overview.html#the-request-object}{Request} object that returns a String.

You specify a request-based constraint the same way that you specify a segment constraint:
\begin{code}
match "photos", :constraints => {:subdomain => "admin"}
\end{code}

You can also specify constraints in a block form:
\begin{code}
namespace :admin do
  constraints :subdomain => "admin" do
    resources :photos
  end
end
\end{code}

\subsection{ Advanced Constraints}

If you have a more advanced constraint, you can provide an object that responds to matches? that Rails should use. Let’s say you wanted to route all users on a blacklist to the BlacklistController. You could do:
\begin{code}
class BlacklistConstraint
  def initialize
    @ips = Blacklist.retrieve_ips
  end
 
  def matches?(request)
    @ips.include?(request.remote_ip)
  end
end
 
TwitterClone::Application.routes.draw do
  match "*path" => "blacklist#index",
    :constraints => BlacklistConstraint.new
end
\end{code}

\subsection{ Route Globbing}

Route globbing is a way to specify that a particular parameter should  be matched to all the remaining parts of a route. For example
\begin{code}
match 'photos/*other' => 'photos#unknown'
\end{code}

This route would match photos/12 or /photos/long/path/to/12, setting params[:other] to "12" or "long/path/to/12".

Wildcard segments can occur anywhere in a route. For example,
\begin{code}
match 'books/*section/:title' => 'books#show'
\end{code}

would match books/some/section/last-words-a-memoir with params[:section] equals "some/section", and params[:title] equals "last-words-a-memoir".

Technically a route can have even more than one wildcard segment. The  matcher assigns segments to parameters in an intuitive way. For  example,
\begin{code}
match '*a/foo/*b' => 'test#index'
\end{code}

would match zoo/woo/foo/bar/baz with params[:a] equals "zoo/woo", and params[:b] equals "bar/baz".

Starting from Rails 3.1, wildcard routes will  always match the optional format segment by default. For example if you  have this route:
\begin{code}
match '*pages' => 'pages#show'
\end{code}

By requesting "/foo/bar.json", your params[:pages] will be equals to "foo/bar" with the request format of JSON. If you want the old 3.0.x behavior back, you could supply :format =$>$ false like this:
\begin{code}
match '*pages' => 'pages#show', :format => false
\end{code}

If you want to make the format segment mandatory, so it cannot be omitted, you can supply :format =$>$ true like this:
\begin{code}
match '*pages' => 'pages#show', :format => true
\end{code}

\subsection{ Redirection}

You can redirect any path to another path using the redirect helper in your router:
\begin{code}
match "/stories" => redirect("/posts")
\end{code}

You can also reuse dynamic segments from the match in the path to redirect to:
\begin{code}
match "/stories/:name" => redirect("/posts/%{name}")
\end{code}

You can also provide a block to redirect, which receives the params and (optionally) the request object:
\begin{code}
match "/stories/:name" 
=> redirect {|params| "/posts/#{params[:name].pluralize}" }

match "/stories" 
=> redirect {|p, req| "/posts/#{req.subdomain}" }
\end{code}

Please note that this redirection is a 301 "Moved Permanently"  redirect. Keep in mind that some web browsers or proxy servers will  cache this type of redirect, making the old page inaccessible.

In all of these cases, if you don’t provide the leading host (http://www.example.com), Rails will take those details from the current request.

\subsection{ Routing to Rack Applications}

Instead of a String, like "posts\#index", which corresponds to the index action in the PostsController, you can specify any \href{http://guides.rubyonrails.org/rails_on_rack.html}{Rack application} as the endpoint for a matcher.
\begin{code}
match "/application.js" => Sprockets
\end{code}

As long as Sprockets responds to call and returns a [status, headers, body], the router won’t know the difference between the Rack application and an action.

For the curious, "posts\#index" actually expands out to PostsController.action(:index), which returns a valid Rack application.

\subsection{ Using root}

You can specify what Rails should route "/" to with the root method:
\begin{code}
root :to => 'pages#main'
\end{code}

You should put the root route at the top of the file, because it is the most popular route and should be matched first. You also need to delete the public/index.html file for the root route to take effect.

\section{ Customizing Resourceful Routes}

While the default routes and helpers generated by resources :posts  will usually serve you well, you may want to customize them in some  way. Rails allows you to customize virtually any generic part of the  resourceful helpers.

\subsection{ Specifying a Controller to Use}

The :controller option lets you explicitly specify a controller to use for the resource. For example:
\begin{code}
resources :photos, :controller => "images"
\end{code}

will recognize incoming paths beginning with /photos but route to the Images controller:\\

\noindent
\begin{tabular}{l|p{\textwidth/4}|l|p{\textwidth/4}}
\hline
\textbf{Verb } & \textbf{Path} & \textbf{action } & \textbf{named helper} \\ 
\hline
GET & /photos           & index     &  photos\_path           \\ 
GET & /photos/new       & new       &  new\_photo\_path        \\ 
POST & /photos           & create    &  photos\_path           \\ 
GET & /photos/:id       & show      &  photo\_path(:id)       \\ 
GET & /photos/:id/edit  & edit      &  edit\_photo\_path(:id)  \\ 
PUT & /photos/:id       & update    &  photo\_path(:id)       \\ 
DELETE & /photos/:id       & destroy   &  photo\_path(:id)      
\end{tabular}\\


Use photos\_path, new\_photo\_path, etc. to generate paths for this resource.

\subsection{ Specifying Constraints}

You can use the :constraints option to specify a required format on the implicit id. For example:
\begin{code}
resources :photos, :constraints => {:id => /[A-Z][A-Z][0-9]+/}
\end{code}

This declaration constrains the :id parameter to match the supplied regular expression. So, in this case, the router would no longer match /photos/1 to this route. Instead, /photos/RR27 would match.

You can specify a single constraint to apply to a number of routes by using the block form:
\begin{code}
constraints(:id => /[A-Z][A-Z][0-9]+/) do
  resources :photos
  resources :accounts
end
\end{code}

Of course, you can use the more advanced constraints available in non-resourceful routes in this context.

By default the :id parameter doesn’t  accept dots – this is because the dot is used as a separator for  formatted routes. If you need to use a dot within an :id add a constraint which overrides this – for example :id =$>$ /[\textasciicircum$\backslash$/]+/ allows anything except a slash.

\subsection{ Overriding the Named Helpers}

The :as option lets you override the normal naming for the named route helpers. For example:
\begin{code}
resources :photos, :as => "images"
\end{code}

will recognize incoming paths beginning with /photos and route the requests to PhotosController, but use the value of the :as option to name the helpers.\\

\noindent
\begin{tabular}{l|p{\textwidth/4}|l|p{\textwidth/4}}
\hline
\textbf{Verb} & \textbf{Path            } & \textbf{action } & \textbf{named helper        } \\ 
\hline
GET & /photos            & index     &  images\_path           \\ 
GET & /photos/new        & new       &  new\_image\_path        \\ 
POST & /photos            & create    &  images\_path           \\ 
GET & /photos/:id        & show      &  image\_path(:id)       \\ 
GET & /photos/:id/edit   & edit      &  edit\_image\_path(:id)  \\ 
PUT & /photos/:id        & update    &  image\_path(:id)       \\ 
DELETE & /photos/:id        & destroy   &  image\_path(:id)      
\end{tabular}\\


\subsection{ Overriding the new and edit Segments}

The :path\_names option lets you override the automatically-generated "new" and "edit" segments in paths:
\begin{code}
resources :photos, :path_names=>{:new=>'make',:edit=>'change'}
\end{code}

This would cause the routing to recognize paths such as
\begin{code}
/photos/make
/photos/1/change
\end{code}

The actual action names aren’t changed by this option. The two paths shown would still route to the new and edit actions.

If you find yourself wanting to change this option uniformly for all of your routes, you can use a scope.
\begin{code}
scope :path_names => { :new => "make" } do
  # rest of your routes
end
\end{code}

\subsection{ Prefixing the Named Route Helpers}

You can use the :as option to prefix the named route helpers  that Rails generates for a route. Use this option to prevent name  collisions between routes using a path scope.
\begin{code}
scope "admin" do
  resources :photos, :as => "admin_photos"
end
 
resources :photos
\end{code}

This will provide route helpers such as admin\_photos\_path, new\_admin\_photo\_path etc.

To prefix a group of route helpers, use :as with scope:
\begin{code}
scope "admin", :as => "admin" do
  resources :photos, :accounts
end
 
resources :photos, :accounts
\end{code}

This will generate routes such as admin\_photos\_path and admin\_accounts\_path which map to /admin/photos and /admin/accounts respectively.

The namespace scope will automatically add :as as well as :module and :path prefixes.

You can prefix routes with a named parameter also:
\begin{code}
scope ":username" do
  resources :posts
end
\end{code}

This will provide you with URLs such as /bob/posts/1 and will allow you to reference the username part of the path as params[:username] in controllers, helpers and views.

\subsection{ Restricting the Routes Created}

By default, Rails creates routes for the seven default actions  (index, show, new, create, edit, update, and destroy) for every RESTful  route in your application. You can use the :only and :except options to fine-tune this behavior. The :only option tells Rails to create only the specified routes:
\begin{code}
resources :photos, :only => [:index, :show]
\end{code}

Now, a GET request to /photos would succeed, but a POST request to /photos (which would ordinarily be routed to the create action) will fail.

The :except option specifies a route or list of routes that Rails should \emph{not} create:
\begin{code}
resources :photos, :except => :destroy
\end{code}

In this case, Rails will create all of the normal routes except the route for destroy (a DELETE request to /photos/:id).

If your application has many RESTful routes, using :only and :except to generate only the routes that you actually need can cut down on memory use and speed up the routing process.

\subsection{ Translated Paths}

Using scope, we can alter path names generated by resources:
\begin{code}
scope(:path_names=>{:new=>"neu",:edit=>"bearbeiten"}) do
  resources :categories, :path => "kategorien"
end
\end{code}

Rails now creates routes to the CategoriesController.\\

\noindent
\begin{tabular}{l|p{\textwidth/3}|l|p{\textwidth/4}}
\hline
\textbf{Verb} & \textbf{Path                     } & \textbf{action } & \textbf{named helper           } \\ 
\hline
GET & /kategorien                 & index     &  categories\_path          \\ 
GET & /kategorien/neu             & new       &  new\_category\_path        \\ 
POST & /kategorien                 & create    &  categories\_path          \\ 
GET & /kategorien/:id             & show      &  category\_path(:id)       \\ 
GET & {\tiny/kategorien/:id/bearbeiten}  & edit      & {\tiny edit\_category\_path(:id)}  \\ 
PUT & /kategorien/:id             & update    &  category\_path(:id)       \\ 
DELETE & /kategorien/:id             & destroy   &  category\_path(:id)      
\end{tabular}

\subsection{ Overriding the Singular Form}

If you want to define the singular form of a resource, you should add additional rules to the Inflector.
\begin{code}
ActiveSupport::Inflector.inflections do |inflect|
  inflect.irregular 'tooth', 'teeth'
end
\end{code}

\subsection{ Using :as in Nested Resources}

The :as option overrides the automatically-generated name for the resource in nested route helpers. For example,
\begin{code}
resources :magazines do
  resources :ads, :as => 'periodical_ads'
end
\end{code}

This will create routing helpers such as \\ magazine\_periodical\_ads\_url and edit\_magazine\_periodical\_ad\_path.

\section{ Inspecting and Testing Routes}

Rails offers facilities for inspecting and testing your routes.

\subsection{ Seeing Existing Routes with rake}

If you want a complete list of all of the available routes in your application, run rake routes command. This will print all of your routes, in the same order that they appear in routes.rb. For each route, you’ll see:
\begin{itemize}
	\item The route name (if any)
	\item The HTTP verb used (if the route doesn’t respond to all verbs)
	\item The URL pattern to match
	\item The routing parameters for the route
\end{itemize}

For example, here’s a small section of the rake routes output for a RESTful route:\\

\noindent
\begin{tabular}{rlp{\textwidth/3+5}p{\textwidth/4}}
    users & GET  &  /users(.:format)    &      users\#index \\
         & POST &  /users(.:format)     &     users\#create \\
 new\_user &  GET  &  /users/new(.:format)  &    users\#new \\
edit\_user & GET  &  /users/:id/edit(.:format) & users\#edit \\
\end{tabular}\\

You may restrict the listing to the routes that map to a particular controller setting the CONTROLLER environment variable:
\begin{code}
CONTROLLER=users rake routes
\end{code}

You’ll find that the output from rake routes is much more readable if you widen your terminal window until the output lines don’t wrap.

\subsection{ Testing Routes}

Routes should be included in your testing strategy (just like the rest of your application). Rails offers three \href{http://api.rubyonrails.org/classes/ActionDispatch/Assertions/RoutingAssertions.html}{built-in assertions} designed to make testing routes simpler:
\begin{itemize}
	\item assert\_generates
	\item assert\_recognizes
	\item assert\_routing
\end{itemize}

\subsubsection{ The assert\_generates Assertion}

assert\_generates asserts that a particular set of options generate a particular path and can be used with default routes or custom routes.
{\scriptsize
\begin{code}
assert_generates "/photos/1", 
{ :controller => "photos", :action => "show", :id => "1" }

assert_generates "/about", 
:controller => "pages", :action => "about"
\end{code}
}
\subsubsection{ The assert\_recognizes Assertion}

assert\_recognizes is the inverse of assert\_generates. It asserts that a given path is recognized and routes it to a particular spot in your application.
{\scriptsize
\begin{code}
assert_recognizes({ :controller => "photos", 
      :action => "show", :id => "1" }, "/photos/1")
\end{code}
}

You can supply a :method argument to specify the HTTP verb:
{\scriptsize
\begin{code}
assert_recognizes({ :controller => "photos", 
:action => "create" }, { :path => "photos", :method => :post })
\end{code}
}

\subsubsection{ The assert\_routing Assertion}

The assert\_routing assertion checks the route both ways: it  tests that the path generates the options, and that the options generate  the path. Thus, it combines the functions of assert\_generates and assert\_recognizes.
{\scriptsize
\begin{code}
assert_routing({ :path => "photos", :method => :post }, 
{ :controller => "photos", :action => "create" })
\end{code}
}

\chapter{License}

\section*{Creative Commons Legal Code}

\section*{Attribution-ShareAlike 3.0 Unported}

\begin{quotation}           CREATIVE COMMONS CORPORATION IS NOT A LAW FIRM AND DOES           NOT PROVIDE LEGAL SERVICES. DISTRIBUTION OF THIS LICENSE           DOES NOT CREATE AN ATTORNEY-CLIENT RELATIONSHIP. CREATIVE           COMMONS PROVIDES THIS INFORMATION ON AN "AS-IS" BASIS.           CREATIVE COMMONS MAKES NO WARRANTIES REGARDING THE           INFORMATION PROVIDED, AND DISCLAIMS LIABILITY FOR DAMAGES           RESULTING FROM ITS USE.         
\end{quotation}

\section*{\emph{License}}

THE WORK (AS DEFINED BELOW) IS PROVIDED UNDER THE TERMS         OF THIS CREATIVE COMMONS PUBLIC LICENSE ("CCPL" OR         "LICENSE"). THE WORK IS PROTECTED BY COPYRIGHT AND/OR OTHER         APPLICABLE LAW. ANY USE OF THE WORK OTHER THAN AS         AUTHORIZED UNDER THIS LICENSE OR COPYRIGHT LAW IS         PROHIBITED.

BY EXERCISING ANY RIGHTS TO THE WORK PROVIDED HERE, YOU         ACCEPT AND AGREE TO BE BOUND BY THE TERMS OF THIS LICENSE.         TO THE EXTENT THIS LICENSE MAY BE CONSIDERED TO BE A         CONTRACT, THE LICENSOR GRANTS YOU THE RIGHTS CONTAINED HERE         IN CONSIDERATION OF YOUR ACCEPTANCE OF SUCH TERMS AND         CONDITIONS.

\section{ Definitions}
\begin{enumerate}
	\item \textbf{"Adaptation"} means a work based upon           the Work, or upon the Work and other pre-existing works,           such as a translation, adaptation, derivative work,           arrangement of music or other alterations of a literary           or artistic work, or phonogram or performance and           includes cinematographic adaptations or any other form in           which the Work may be recast, transformed, or adapted           including in any form recognizably derived from the           original, except that a work that constitutes a           Collection will not be considered an Adaptation for the           purpose of this License. For the avoidance of doubt,           where the Work is a musical work, performance or           phonogram, the synchronization of the Work in           timed-relation with a moving image ("synching") will be           considered an Adaptation for the purpose of this           License.
	\item \textbf{"Collection"} means a collection of           literary or artistic works, such as encyclopedias and           anthologies, or performances, phonograms or broadcasts,           or other works or subject matter other than works listed           in Section 1(f) below, which, by reason of the selection           and arrangement of their contents, constitute           intellectual creations, in which the Work is included in           its entirety in unmodified form along with one or more           other contributions, each constituting separate and           independent works in themselves, which together are           assembled into a collective whole. A work that           constitutes a Collection will not be considered an           Adaptation (as defined below) for the purposes of this           License.
	\item \textbf{"Creative Commons Compatible           License"} means a license that is listed at           http://creativecommons.org/compatiblelicenses that has           been approved by Creative Commons as being essentially           equivalent to this License, including, at a minimum,           because that license: (i) contains terms that have the           same purpose, meaning and effect as the License Elements           of this License; and, (ii) explicitly permits the           relicensing of adaptations of works made available under           that license under this License or a Creative Commons           jurisdiction license with the same License Elements as           this License.
	\item \textbf{"Distribute"} means to make available           to the public the original and copies of the Work or           Adaptation, as appropriate, through sale or other           transfer of ownership.
	\item \textbf{"License Elements"} means the           following high-level license attributes as selected by           Licensor and indicated in the title of this License:           Attribution, ShareAlike.
	\item \textbf{"Licensor"} means the individual,           individuals, entity or entities that offer(s) the Work           under the terms of this License.
	\item \textbf{"Original Author"} means, in the case           of a literary or artistic work, the individual,           individuals, entity or entities who created the Work or           if no individual or entity can be identified, the           publisher; and in addition (i) in the case of a           performance the actors, singers, musicians, dancers, and           other persons who act, sing, deliver, declaim, play in,           interpret or otherwise perform literary or artistic works           or expressions of folklore; (ii) in the case of a           phonogram the producer being the person or legal entity           who first fixes the sounds of a performance or other           sounds; and, (iii) in the case of broadcasts, the           organization that transmits the broadcast.
	\item \textbf{"Work"} means the literary and/or           artistic work offered under the terms of this License           including without limitation any production in the           literary, scientific and artistic domain, whatever may be           the mode or form of its expression including digital           form, such as a book, pamphlet and other writing; a           lecture, address, sermon or other work of the same           nature; a dramatic or dramatico-musical work; a           choreographic work or entertainment in dumb show; a           musical composition with or without words; a           cinematographic work to which are assimilated works           expressed by a process analogous to cinematography; a           work of drawing, painting, architecture, sculpture,           engraving or lithography; a photographic work to which           are assimilated works expressed by a process analogous to           photography; a work of applied art; an illustration, map,           plan, sketch or three-dimensional work relative to           geography, topography, architecture or science; a           performance; a broadcast; a phonogram; a compilation of           data to the extent it is protected as a copyrightable           work; or a work performed by a variety or circus           performer to the extent it is not otherwise considered a           literary or artistic work.
	\item \textbf{"You"} means an individual or entity           exercising rights under this License who has not           previously violated the terms of this License with           respect to the Work, or who has received express           permission from the Licensor to exercise rights under           this License despite a previous violation.
	\item \textbf{"Publicly Perform"} means to perform           public recitations of the Work and to communicate to the           public those public recitations, by any means or process,           including by wire or wireless means or public digital           performances; to make available to the public Works in           such a way that members of the public may access these           Works from a place and at a place individually chosen by           them; to perform the Work to the public by any means or           process and the communication to the public of the           performances of the Work, including by public digital           performance; to broadcast and rebroadcast the Work by any           means including signs, sounds or images.
	\item \textbf{"Reproduce"} means to make copies of           the Work by any means including without limitation by           sound or visual recordings and the right of fixation and           reproducing fixations of the Work, including storage of a           protected performance or phonogram in digital form or           other electronic medium.
\end{enumerate}

\section{ Fair Dealing Rights.} Nothing in this         License is intended to reduce, limit, or restrict any uses         free from copyright or rights arising from limitations or         exceptions that are provided for in connection with the         copyright protection under copyright law or other         applicable laws.

\section{ License Grant.} Subject to the terms         and conditions of this License, Licensor hereby grants You         a worldwide, royalty-free, non-exclusive, perpetual (for         the duration of the applicable copyright) license to         exercise the rights in the Work as stated below:
\begin{enumerate}
	\item to Reproduce the Work, to incorporate the Work into           one or more Collections, and to Reproduce the Work as           incorporated in the Collections;
	\item to create and Reproduce Adaptations provided that any           such Adaptation, including any translation in any medium,           takes reasonable steps to clearly label, demarcate or           otherwise identify that changes were made to the original           Work. For example, a translation could be marked "The           original work was translated from English to Spanish," or           a modification could indicate "The original work has been           modified.";
	\item to Distribute and Publicly Perform the Work including           as incorporated in Collections; and,
	\item to Distribute and Publicly Perform Adaptations.
	\item 

For the avoidance of doubt:
\begin{enumerate}
	\item \textbf{Non-waivable Compulsory License               Schemes}. In those jurisdictions in which the               right to collect royalties through any statutory or               compulsory licensing scheme cannot be waived, the               Licensor reserves the exclusive right to collect such               royalties for any exercise by You of the rights               granted under this License;
	\item \textbf{Waivable Compulsory License               Schemes}. In those jurisdictions in which the               right to collect royalties through any statutory or               compulsory licensing scheme can be waived, the               Licensor waives the exclusive right to collect such               royalties for any exercise by You of the rights               granted under this License; and,
	\item \textbf{Voluntary License Schemes}. The               Licensor waives the right to collect royalties,               whether individually or, in the event that the               Licensor is a member of a collecting society that               administers voluntary licensing schemes, via that               society, from any exercise by You of the rights               granted under this License.
\end{enumerate}
\end{enumerate}

The above rights may be exercised in all media and         formats whether now known or hereafter devised. The above         rights include the right to make such modifications as are         technically necessary to exercise the rights in other media         and formats. Subject to Section 8(f), all rights not         expressly granted by Licensor are hereby reserved.

\section{ Restrictions.} The license granted in         Section 3 above is expressly made subject to and limited by         the following restrictions:
\begin{enumerate}
	\item You may Distribute or Publicly Perform the Work only           under the terms of this License. You must include a copy           of, or the Uniform Resource Identifier (URI) for, this           License with every copy of the Work You Distribute or           Publicly Perform. You may not offer or impose any terms           on the Work that restrict the terms of this License or           the ability of the recipient of the Work to exercise the           rights granted to that recipient under the terms of the           License. You may not sublicense the Work. You must keep           intact all notices that refer to this License and to the           disclaimer of warranties with every copy of the Work You           Distribute or Publicly Perform. When You Distribute or           Publicly Perform the Work, You may not impose any           effective technological measures on the Work that           restrict the ability of a recipient of the Work from You           to exercise the rights granted to that recipient under           the terms of the License. This Section 4(a) applies to           the Work as incorporated in a Collection, but this does           not require the Collection apart from the Work itself to           be made subject to the terms of this License. If You           create a Collection, upon notice from any Licensor You           must, to the extent practicable, remove from the           Collection any credit as required by Section 4(c), as           requested. If You create an Adaptation, upon notice from           any Licensor You must, to the extent practicable, remove           from the Adaptation any credit as required by Section           4(c), as requested.
	\item You may Distribute or Publicly Perform an Adaptation           only under the terms of: (i) this License; (ii) a later           version of this License with the same License Elements as           this License; (iii) a Creative Commons jurisdiction           license (either this or a later license version) that           contains the same License Elements as this License (e.g.,           Attribution-ShareAlike 3.0 US)); (iv) a Creative Commons           Compatible License. If you license the Adaptation under           one of the licenses mentioned in (iv), you must comply           with the terms of that license. If you license the           Adaptation under the terms of any of the licenses           mentioned in (i), (ii) or (iii) (the "Applicable           License"), you must comply with the terms of the           Applicable License generally and the following           provisions: (I) You must include a copy of, or the URI           for, the Applicable License with every copy of each           Adaptation You Distribute or Publicly Perform; (II) You           may not offer or impose any terms on the Adaptation that           restrict the terms of the Applicable License or the           ability of the recipient of the Adaptation to exercise           the rights granted to that recipient under the terms of           the Applicable License; (III) You must keep intact all           notices that refer to the Applicable License and to the           disclaimer of warranties with every copy of the Work as           included in the Adaptation You Distribute or Publicly           Perform; (IV) when You Distribute or Publicly Perform the           Adaptation, You may not impose any effective           technological measures on the Adaptation that restrict           the ability of a recipient of the Adaptation from You to           exercise the rights granted to that recipient under the           terms of the Applicable License. This Section 4(b)           applies to the Adaptation as incorporated in a           Collection, but this does not require the Collection           apart from the Adaptation itself to be made subject to           the terms of the Applicable License.
	\item If You Distribute, or Publicly Perform the Work or           any Adaptations or Collections, You must, unless a           request has been made pursuant to Section 4(a), keep           intact all copyright notices for the Work and provide,           reasonable to the medium or means You are utilizing: (i)           the name of the Original Author (or pseudonym, if           applicable) if supplied, and/or if the Original Author           and/or Licensor designate another party or parties (e.g.,           a sponsor institute, publishing entity, journal) for           attribution ("Attribution Parties") in Licensor's           copyright notice, terms of service or by other reasonable           means, the name of such party or parties; (ii) the title           of the Work if supplied; (iii) to the extent reasonably           practicable, the URI, if any, that Licensor specifies to           be associated with the Work, unless such URI does not           refer to the copyright notice or licensing information           for the Work; and (iv) , consistent with Ssection 3(b),           in the case of an Adaptation, a credit identifying the           use of the Work in the Adaptation (e.g., "French           translation of the Work by Original Author," or           "Screenplay based on original Work by Original Author").           The credit required by this Section 4(c) may be           implemented in any reasonable manner; provided, however,           that in the case of a Adaptation or Collection, at a           minimum such credit will appear, if a credit for all           contributing authors of the Adaptation or Collection           appears, then as part of these credits and in a manner at           least as prominent as the credits for the other           contributing authors. For the avoidance of doubt, You may           only use the credit required by this Section for the           purpose of attribution in the manner set out above and,           by exercising Your rights under this License, You may not           implicitly or explicitly assert or imply any connection           with, sponsorship or endorsement by the Original Author,           Licensor and/or Attribution Parties, as appropriate, of           You or Your use of the Work, without the separate,           express prior written permission of the Original Author,           Licensor and/or Attribution Parties.
	\item Except as otherwise agreed in writing by the Licensor           or as may be otherwise permitted by applicable law, if           You Reproduce, Distribute or Publicly Perform the Work           either by itself or as part of any Adaptations or           Collections, You must not distort, mutilate, modify or           take other derogatory action in relation to the Work           which would be prejudicial to the Original Author's honor           or reputation. Licensor agrees that in those           jurisdictions (e.g. Japan), in which any exercise of the           right granted in Section 3(b) of this License (the right           to make Adaptations) would be deemed to be a distortion,           mutilation, modification or other derogatory action           prejudicial to the Original Author's honor and           reputation, the Licensor will waive or not assert, as           appropriate, this Section, to the fullest extent           permitted by the applicable national law, to enable You           to reasonably exercise Your right under Section 3(b) of           this License (right to make Adaptations) but not           otherwise.
\end{enumerate}

\section{ Representations, Warranties and         Disclaimer}

UNLESS OTHERWISE MUTUALLY AGREED TO BY THE PARTIES IN         WRITING, LICENSOR OFFERS THE WORK AS-IS AND MAKES NO         REPRESENTATIONS OR WARRANTIES OF ANY KIND CONCERNING THE         WORK, EXPRESS, IMPLIED, STATUTORY OR OTHERWISE, INCLUDING,         WITHOUT LIMITATION, WARRANTIES OF TITLE, MERCHANTIBILITY,         FITNESS FOR A PARTICULAR PURPOSE, NONINFRINGEMENT, OR THE         ABSENCE OF LATENT OR OTHER DEFECTS, ACCURACY, OR THE         PRESENCE OF ABSENCE OF ERRORS, WHETHER OR NOT DISCOVERABLE.         SOME JURISDICTIONS DO NOT ALLOW THE EXCLUSION OF IMPLIED         WARRANTIES, SO SUCH EXCLUSION MAY NOT APPLY TO YOU.

\section{ Limitation on Liability.} EXCEPT TO         THE EXTENT REQUIRED BY APPLICABLE LAW, IN NO EVENT WILL         LICENSOR BE LIABLE TO YOU ON ANY LEGAL THEORY FOR ANY         SPECIAL, INCIDENTAL, CONSEQUENTIAL, PUNITIVE OR EXEMPLARY         DAMAGES ARISING OUT OF THIS LICENSE OR THE USE OF THE WORK,         EVEN IF LICENSOR HAS BEEN ADVISED OF THE POSSIBILITY OF         SUCH DAMAGES.

\section{ Termination}
\begin{enumerate}
	\item This License and the rights granted hereunder will           terminate automatically upon any breach by You of the           terms of this License. Individuals or entities who have           received Adaptations or Collections from You under this           License, however, will not have their licenses terminated           provided such individuals or entities remain in full           compliance with those licenses. Sections 1, 2, 5, 6, 7,           and 8 will survive any termination of this License.
	\item Subject to the above terms and conditions, the           license granted here is perpetual (for the duration of           the applicable copyright in the Work). Notwithstanding           the above, Licensor reserves the right to release the           Work under different license terms or to stop           distributing the Work at any time; provided, however that           any such election will not serve to withdraw this License           (or any other license that has been, or is required to           be, granted under the terms of this License), and this           License will continue in full force and effect unless           terminated as stated above.
\end{enumerate}

\section{ Miscellaneous}
\begin{enumerate}
	\item Each time You Distribute or Publicly Perform the Work           or a Collection, the Licensor offers to the recipient a           license to the Work on the same terms and conditions as           the license granted to You under this License.
	\item Each time You Distribute or Publicly Perform an           Adaptation, Licensor offers to the recipient a license to           the original Work on the same terms and conditions as the           license granted to You under this License.
	\item If any provision of this License is invalid or           unenforceable under applicable law, it shall not affect           the validity or enforceability of the remainder of the           terms of this License, and without further action by the           parties to this agreement, such provision shall be           reformed to the minimum extent necessary to make such           provision valid and enforceable.
	\item No term or provision of this License shall be deemed           waived and no breach consented to unless such waiver or           consent shall be in writing and signed by the party to be           charged with such waiver or consent.
	\item This License constitutes the entire agreement between           the parties with respect to the Work licensed here. There           are no understandings, agreements or representations with           respect to the Work not specified here. Licensor shall           not be bound by any additional provisions that may appear           in any communication from You. This License may not be           modified without the mutual written agreement of the           Licensor and You.
	\item The rights granted under, and the subject matter           referenced, in this License were drafted utilizing the           terminology of the Berne Convention for the Protection of           Literary and Artistic Works (as amended on September 28,           1979), the Rome Convention of 1961, the WIPO Copyright           Treaty of 1996, the WIPO Performances and Phonograms           Treaty of 1996 and the Universal Copyright Convention (as           revised on July 24, 1971). These rights and subject           matter take effect in the relevant jurisdiction in which           the License terms are sought to be enforced according to           the corresponding provisions of the implementation of           those treaty provisions in the applicable national law.           If the standard suite of rights granted under applicable           copyright law includes additional rights not granted           under this License, such additional rights are deemed to           be included in the License; this License is not intended           to restrict the license of any rights under applicable           law.
\end{enumerate}
%  BREAKOUT FOR CC NOTICE.  NOT A PART OF THE LICENSE 

\begin{quotation}

\section{Creative Commons Notice}

Creative Commons is not a party to this License, and           makes no warranty whatsoever in connection with the Work.           Creative Commons will not be liable to You or any party           on any legal theory for any damages whatsoever, including           without limitation any general, special, incidental or           consequential damages arising in connection to this           license. Notwithstanding the foregoing two (2) sentences,           if Creative Commons has expressly identified itself as           the Licensor hereunder, it shall have all rights and           obligations of Licensor.

Except for the limited purpose of indicating to the           public that the Work is licensed under the CCPL, Creative           Commons does not authorize the use by either party of the           trademark "Creative Commons" or any related trademark or           logo of Creative Commons without the prior written           consent of Creative Commons. Any permitted use will be in           compliance with Creative Commons' then-current trademark           usage guidelines, as may be published on its website or           otherwise made available upon request from time to time.           For the avoidance of doubt, this trademark restriction           does not form part of the License.

Creative Commons may be contacted at \href{http://creativecommons.org/}{http://creativecommons.org/}.
\end{quotation}
%  END CC NOTICE 

\chapter{Feedback}

           You're encouraged to help improve the quality of this guide.         

           If you see any typos or factual errors you are confident to           patch, please clone \href{https://github.com/lifo/docrails}{docrails}           and push the change yourself. That branch of Rails has public write access.           Commits are still reviewed, but that happens after you've submitted your           contribution. \href{https://github.com/lifo/docrails}{docrails} is           cross-merged with master periodically.         

           You may also find incomplete content, or stuff that is not up to date.           Please do add any missing documentation for master. Check the           \href{http://guides.rubyonrails.org/ruby_on_rails_guides_guidelines.html}{Ruby on Rails Guides Guidelines}           for style and conventions.         

           If for whatever reason you spot something to fix but cannot patch it yourself, please           \href{https://github.com/rails/rails/issues}{open an issue}.         

And last but not least, any kind of discussion regarding Ruby on Rails           documentation is very welcome in the \href{http://groups.google.com/group/rubyonrails-docs}{rubyonrails-docs mailing list}.


\tableofcontents
\end{document}
